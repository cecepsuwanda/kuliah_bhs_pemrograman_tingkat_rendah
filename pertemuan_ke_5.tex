\chapter{Instruksi Logika (AND, OR, XOR, NOT); Output Teks ke Layar (INT 10h fungsi 02h)}

\section{Tujuan Pembelajaran}
Setelah mengikuti pertemuan ini, mahasiswa diharapkan dapat:
\begin{itemize}
\item Menggunakan instruksi logika AND, OR, XOR, NOT
\item Memahami operasi bitwise dan penggunaannya
\item Menggunakan interupsi INT 10h untuk output teks
\item Mampu menampilkan teks di layar dengan kontrol posisi
\end{itemize}

\section{Materi Pembelajaran}

\subsection{Instruksi Logika}
\begin{itemize}
\item Instruksi AND (operasi logika AND)
\item Instruksi OR (operasi logika OR)
\item Instruksi XOR (operasi logika XOR)
\item Instruksi NOT (operasi logika NOT)
\item Operasi bitwise dan aplikasinya
\item Pengaruh terhadap flag register
\end{itemize}

\subsection{Aplikasi Instruksi Logika}
\begin{itemize}
\item Manipulasi bit
\item Masking dan unmasking
\item Enkripsi sederhana
\item Operasi set dan clear bit
\item Perbandingan bit
\end{itemize}

\subsection{Interupsi BIOS INT 10h}
\begin{itemize}
\item Fungsi 02h: Set Cursor Position
\item Fungsi 09h: Write Character and Attribute
\item Fungsi 0Ah: Write Character Only
\item Fungsi 0Eh: Write Character in TTY Mode
\item Parameter dan penggunaan
\end{itemize}

\subsection{Output Teks ke Layar}
\begin{itemize}
\item Kontrol posisi kursor
\item Atribut karakter (warna, background)
\item Mode teks dan grafik
\item Penanganan karakter khusus
\end{itemize}

\section{Praktikum}
\begin{enumerate}
\item Program operasi logika dasar
\item Program manipulasi bit
\item Program enkripsi sederhana dengan XOR
\item Program output teks dengan kontrol posisi
\item Program menampilkan teks berwarna
\end{enumerate}

\section{Contoh Kode}
\begin{verbatim}
; Program operasi logika dan output teks
TITLE Operasi Logika dan Output
.MODEL SMALL
.STACK 100h

.DATA
    pesan DB 'Hello Assembly!$'
    nilai1 DW 0F0F0h
    nilai2 DW 0FF00h

.CODE
START:
    MOV AX, @DATA
    MOV DS, AX
    
    ; Operasi logika
    MOV AX, nilai1
    AND AX, nilai2    ; Masking
    OR AX, 000Fh      ; Set bit terakhir
    XOR AX, 0FFFFh    ; Inversi
    NOT AX            ; Komplemen
    
    ; Output teks ke layar
    MOV AH, 02h       ; Set cursor position
    MOV BH, 00h       ; Page 0
    MOV DH, 10        ; Row 10
    MOV DL, 20        ; Column 20
    INT 10h
    
    ; Tampilkan karakter
    MOV AH, 09h       ; Write character
    MOV AL, 'A'       ; Character 'A'
    MOV BH, 00h       ; Page 0
    MOV BL, 1Eh       ; Yellow on blue
    MOV CX, 1         ; Count
    INT 10h
    
    MOV AH, 4Ch
    INT 21h
END START
\end{verbatim}

\section{Latihan}
\begin{enumerate}
\item Buat program yang melakukan masking pada 8 bit terakhir
\item Buat program enkripsi sederhana menggunakan XOR
\item Buat program yang menampilkan teks di berbagai posisi layar
\item Buat program yang menampilkan teks dengan berbagai warna
\end{enumerate}

\section{Tugas}
\begin{itemize}
\item Implementasikan operasi logika untuk manipulasi data
\item Buat program yang menampilkan menu dengan kontrol posisi
\item Buat program enkripsi/dekripsi teks menggunakan XOR
\item Dokumentasikan penggunaan interupsi INT 10h
\end{itemize}

\section{Referensi}
\begin{itemize}
\item Hyde, Randall. \textit{The Art of Assembly Language}, 2nd ed., No Starch Press, 2010.
\item Partoharsojo, Hartono. \textit{Tuntunan Praktis Pemrograman Assembly}, Penerbit Informatika.
\end{itemize}

