\chapter{Pemrograman Modular: PROC/ENDP, Penggunaan Makro Sederhana (MACRO)}

\section{Tujuan Pembelajaran}
Setelah mengikuti pertemuan ini, mahasiswa diharapkan dapat:
\begin{itemize}
\item Memahami konsep pemrograman modular
\item Menggunakan PROC/ENDP untuk membuat prosedur
\item Membuat dan menggunakan makro sederhana
\item Membedakan antara prosedur dan makro
\item Mampu merancang program modular
\end{itemize}

\section{Materi Pembelajaran}

\subsection{Konsep Pemrograman Modular}
\begin{itemize}
\item Definisi pemrograman modular
\item Keuntungan pemrograman modular
\item Prinsip modularitas
\item Reusability dan maintainability
\item Organisasi kode program
\end{itemize}

\subsection{Prosedur dengan PROC/ENDP}
\begin{itemize}
\item Sintaks PROC/ENDP
\item Deklarasi prosedur
\item Parameter prosedur
\item Local variables dalam prosedur
\item Return value dari prosedur
\item Konvensi pemanggilan prosedur
\end{itemize}

\subsection{Makro (MACRO)}
\begin{itemize}
\item Definisi makro
\item Sintaks MACRO/ENDM
\item Parameter makro
\item Makro dengan multiple parameters
\item Makro bersyarat
\item Perbedaan makro dan prosedur
\end{itemize}

\subsection{Organisasi Program Modular}
\begin{itemize}
\item Struktur file program
\item Include files
\item Library routines
\item Module dependencies
\item Code organization
\end{itemize}

\section{Praktikum}
\begin{enumerate}
\item Program dengan prosedur sederhana
\item Program dengan prosedur berparameter
\item Program dengan makro sederhana
\item Program dengan makro berparameter
\item Program modular lengkap
\end{enumerate}

\section{Contoh Kode}
\begin{verbatim}
; Program demonstrasi prosedur dan makro
TITLE Pemrograman Modular
.MODEL SMALL
.STACK 100h

.DATA
    pesan1 DB 'Hello from procedure!$'
    pesan2 DB 'Hello from macro!$'
    nilai1 DW 15
    nilai2 DW 25
    hasil DW ?

; Makro untuk menampilkan pesan
tampilkan_pesan MACRO pesan
    PUSH AX
    PUSH DX
    MOV AH, 09h
    MOV DX, OFFSET pesan
    INT 21h
    POP DX
    POP AX
ENDM

; Makro untuk pertukaran nilai
tukar_nilai MACRO var1, var2
    PUSH AX
    MOV AX, var1
    XCHG AX, var2
    MOV var1, AX
    POP AX
ENDM

.CODE
START:
    MOV AX, @DATA
    MOV DS, AX
    
    ; Panggil prosedur
    CALL tampilkan_hello
    
    ; Gunakan makro
    tampilkan_pesan pesan2
    
    ; Pertukaran nilai menggunakan makro
    tukar_nilai nilai1, nilai2
    
    ; Panggil prosedur dengan parameter
    PUSH nilai1
    PUSH nilai2
    CALL tambahkan
    ADD SP, 4
    
    MOV AH, 4Ch
    INT 21h

; Prosedur tanpa parameter
tampilkan_hello PROC
    PUSH AX
    PUSH DX
    
    MOV AH, 09h
    MOV DX, OFFSET pesan1
    INT 21h
    
    POP DX
    POP AX
    RET
tampilkan_hello ENDP

; Prosedur dengan parameter
tambahkan PROC
    PUSH BP
    MOV BP, SP
    
    PUSH AX
    PUSH BX
    
    MOV AX, [BP+6]  ; Parameter kedua
    MOV BX, [BP+4]  ; Parameter pertama
    ADD AX, BX
    MOV hasil, AX
    
    POP BX
    POP AX
    POP BP
    RET
tambahkan ENDP

END START
\end{verbatim}

\section{Latihan}
\begin{enumerate}
\item Buat prosedur untuk menghitung faktorial
\item Buat makro untuk menampilkan karakter
\item Buat prosedur untuk mengurutkan array
\item Buat makro untuk operasi aritmatika
\end{enumerate}

\section{Tugas}
\begin{itemize}
\item Implementasikan library prosedur untuk operasi string
\item Buat makro untuk debugging dan logging
\item Implementasikan program kalkulator modular
\item Dokumentasikan perbedaan antara prosedur dan makro
\end{itemize}

\section{Referensi}
\begin{itemize}
\item Hyde, Randall. \textit{The Art of Assembly Language}, 2nd ed., No Starch Press, 2010.
\item Borland. \textit{Turbo Assembler (TASM) User's Guide}, Borland International.
\end{itemize}

