\documentclass[12pt,a4paper]{article}
\usepackage[utf8]{inputenc}
\usepackage[T1]{fontenc}
\usepackage[indonesian]{babel}
\usepackage{geometry}
\usepackage{longtable}
\usepackage{array}
\usepackage{booktabs}
\usepackage{enumitem}
\usepackage[hidelinks,breaklinks]{hyperref}
\geometry{margin=2.5cm}

\begin{document}

\begin{center}
\textbf{\Large RENCANA PEMBELAJARAN SEMESTER (RPS)}\\
\textbf{Berbasis Outcome-Based Education (OBE)}\\[0.5cm]
\textbf{PROGRAM STUDI TEKNIK INFORMATIKA}\\
\textbf{FAKULTAS TEKNIK}\\
\textbf{UNIVERSITAS LOREM IPSUM}\\
\end{center}

\vspace{1cm}

% ============================================================
% 1. IDENTITAS MATA KULIAH
% ============================================================
\section*{1. Identitas Mata Kuliah}
\begin{tabular}{ll}
Nama Program Studi & : Teknik Informatika \\
Nama Mata Kuliah & : Pemrograman Bahasa Tingkat Rendah \\
Kode Mata Kuliah & : TIF-201 \\
Semester & : 2 (Dua) \\
SKS / Bobot Kredit & : 3 SKS (1 Teori, 2 Praktikum) \\
Dosen Pengampu & : Dr. [Nama Dosen], M.Kom. \\
Tanggal Penyusunan & : 31 Januari 2026 \\
\end{tabular}

\vspace{0.5cm}

% ============================================================
% 2. CAPAIAN PEMBELAJARAN LULUSAN (CPL)
% ============================================================
\section*{2. Capaian Pembelajaran Lulusan (CPL)}

CPL yang dibebankan pada mata kuliah ini mencakup kompetensi lulusan dalam aspek pengetahuan, keterampilan, dan sikap:

\begin{itemize}[leftmargin=*]
  \item \textbf{CPL-1 (Pengetahuan):} Menguasai konsep arsitektur komputer, organisasi memori, dan bahasa assembly tingkat rendah serta mampu menganalisis operasi hardware pada level mesin.
  
  \item \textbf{CPL-2 (Keterampilan Umum):} Mampu menerapkan pemikiran logis, kritis, dan sistematis dalam mengembangkan program assembly serta memecahkan masalah pada level instruksi mesin.
  
  \item \textbf{CPL-3 (Keterampilan Khusus):} Mampu mengimplementasikan program bahasa assembly untuk Intel 8086, melakukan debugging, dan mengoptimasi kinerja program pada level hardware.
  
  \item \textbf{CPL-4 (Sikap):} Menunjukkan sikap teliti, sabar, dan bertanggung jawab dalam pengembangan program tingkat rendah serta mampu bekerja sama dalam tim.
\end{itemize}

\vspace{0.5cm}

% ============================================================
% 3. CAPAIAN PEMBELAJARAN MATA KULIAH (CPMK)
% ============================================================
\section*{3. Capaian Pembelajaran Mata Kuliah (CPMK)}

Kemampuan atau kompetensi spesifik yang diharapkan mahasiswa kuasai setelah menyelesaikan mata kuliah:

\begin{itemize}[leftmargin=*]
  \item \textbf{CPMK-1:} Mahasiswa mampu memahami dan menjelaskan arsitektur prosesor Intel 8086, register, dan organisasi memori.
  
  \item \textbf{CPMK-2:} Mahasiswa mampu menulis program assembly menggunakan sintaks Turbo Assembler (TASM) dengan benar.
  
  \item \textbf{CPMK-3:} Mahasiswa mampu mengimplementasikan operasi aritmatika, logika, dan kontrol dalam bahasa assembly.
  
  \item \textbf{CPMK-4:} Mahasiswa mampu mengembangkan prosedur, mengelola stack, dan menangani interupsi sistem.
\end{itemize}

\vspace{0.5cm}

% ============================================================
% 4. SUB-CPMK / INDIKATOR PENCAPAIAN
% ============================================================
\section*{4. Sub-CPMK / Indikator Pencapaian}

Penjabaran CPMK menjadi indikator yang lebih terukur dan dapat diuji:

\begin{itemize}[leftmargin=*]
  \item \textbf{Sub-CPMK 1.1:} Menjelaskan fungsi register AX, BX, CX, DX, SI, DI, BP, SP dalam prosesor 8086
  \item \textbf{Sub-CPMK 1.2:} Mengidentifikasi mode addressing (immediate, direct, indirect, indexed)
  \item \textbf{Sub-CPMK 1.3:} Mendemonstrasikan organisasi memori segmentasi (CS, DS, ES, SS)
  \item \textbf{Sub-CPMK 2.1:} Menulis program assembly sederhana dengan TASM syntax
  \item \textbf{Sub-CPMK 2.2:} Menggunakan direktif assembler (DB, DW, DD, EQU, ORG)
  \item \textbf{Sub-CPMK 3.1:} Mengimplementasikan operasi aritmatika (ADD, SUB, MUL, DIV)
  \item \textbf{Sub-CPMK 3.2:} Menerapkan operasi logika (AND, OR, XOR, NOT, SHIFT)
  \item \textbf{Sub-CPMK 4.1:} Mengembangkan prosedur dengan parameter passing
  \item \textbf{Sub-CPMK 4.2:} Menangani interupsi BIOS dan DOS (INT 10h, INT 21h)
\end{itemize}

\vspace{0.5cm}

% ============================================================
% 5. MATERI PEMBELAJARAN (BAHAN KAJIAN)
% ============================================================
\section*{5. Materi Pembelajaran (Bahan Kajian)}

Daftar topik materi yang relevan dengan Sub-CPMK dan CPMK:

\begin{enumerate}[leftmargin=*]
  \item \textbf{Minggu 1-2:} Pengenalan Arsitektur Komputer dan Bahasa Assembly
  \item \textbf{Minggu 3-4:} Struktur Prosesor Intel 8086 dan Register
  \item \textbf{Minggu 5-6:} Set Instruksi dan Mode Addressing
  \item \textbf{Minggu 7-8:} Operasi Aritmatika dan Logika
  \item \textbf{Minggu 9-10:} Struktur Kontrol dan Perulangan
  \item \textbf{Minggu 11-12:} Prosedur dan Manajemen Stack
  \item \textbf{Minggu 13-14:} Interupsi Sistem dan Layanan BIOS/DOS
  \item \textbf{Minggu 15-16:} Optimasi dan Debugging Program Assembly
\end{enumerate}

\textbf{Topik Spesifik per Minggu:}
\begin{itemize}[leftmargin=*]
  \item \textbf{Teori (30\%):} Konsep arsitektur, organisasi memori, set instruksi, mode addressing
  \item \textbf{Praktik (70\%):} Penggunaan GUI Turbo Assembler, debugging, pengembangan program
  \item \textbf{Aplikasi:} Program kalkulator, manipulasi string, grafik sederhana, kontrol hardware
\end{itemize}

\vspace{0.5cm}

% ============================================================
% 6. METODE PEMBELAJARAN
% ============================================================
\section*{6. Metode Pembelajaran}

Strategi atau pendekatan pembelajaran yang dipilih sesuai OBE yang menekankan aktivitas mahasiswa:

\begin{itemize}[leftmargin=*]
  \item \textbf{Ceramah Interaktif:} Penjelasan konsep arsitektur dan instruksi dengan demonstrasi live coding
  \item \textbf{Hands-on Lab:} Praktikum langsung menggunakan GUI Turbo Assembler (TASM) dengan bimbingan
  \item \textbf{Problem-Based Learning:} Mahasiswa menyelesaikan kasus pemrograman tingkat rendah
  \item \textbf{Pair Programming:} Kolaborasi berpasangan untuk debugging dan optimasi kode
  \item \textbf{Project-Based Learning:} Pengembangan aplikasi assembly mini sebagai proyek akhir
  \item \textbf{Flipped Classroom:} Mahasiswa mempelajari teori sebelum kelas, fokus praktik di kelas
  \item \textbf{Code Review Session:} Sesi review dan optimasi kode assembly secara bersama
\end{itemize}

\textbf{Lingkungan Pembelajaran:}
\begin{itemize}[leftmargin=*]
  \item GUI Turbo Assembler (TASM) dengan integrated debugger
  \item DOSBox untuk simulasi environment DOS
  \item Emulator 8086 untuk visualisasi eksekusi instruksi
  \item Documentation reference yang mudah diakses
\end{itemize}

\vspace{0.5cm}

% ============================================================
% 7. PENGALAMAN BELAJAR MAHASISWA
% ============================================================
\section*{7. Pengalaman Belajar Mahasiswa}

Deskripsi tugas, aktivitas, atau pengalaman belajar yang mendukung pencapaian Sub-CPMK:

\begin{itemize}[leftmargin=*]
  \item Menginstal dan mengkonfigurasi GUI Turbo Assembler (TASM) pada komputer
  \item Menulis program assembly sederhana untuk operasi aritmatika dasar
  \item Mengimplementasikan program kalkulator menggunakan instruksi 8086
  \item Mengembangkan program manipulasi string dengan prosedur dan parameter
  \item Membuat program grafik sederhana menggunakan interupsi BIOS
  \item Melakukan debugging program assembly dengan TASM debugger
  \item Mengoptimasi kode assembly untuk meningkatkan kinerja
  \item Berkolaborasi dalam tim untuk mengembangkan proyek aplikasi sistem
  \item Menganalisis kode assembly hasil kompilasi dari bahasa tingkat tinggi
  \item Membuat dokumentasi teknis untuk program assembly yang dikembangkan
\end{itemize}

\textbf{Proyek Akhir:}
\begin{itemize}[leftmargin=*]
  \item Pilihan 1: Sistem manajemen memori sederhana
  \item Pilihan 2: Program editor teks dengan fitur find/replace
  \item Pilihan 3: Game sederhana dengan grafik dan input keyboard
  \item Pilihan 4: Utility sistem untuk monitoring hardware
\end{itemize}

\vspace{0.5cm}

% ============================================================
% 8. KRITERIA, INDIKATOR, DAN BOBOT PENILAIAN
% ============================================================
\section*{8. Kriteria, Indikator, dan Bobot Penilaian}

Teknik/alat asesmen dipetakan ke Sub-CPMK/CPMK dengan bobot yang jelas:

\begin{longtable}{|>{\raggedright\arraybackslash}p{2.5cm}|>{\raggedright\arraybackslash}p{4cm}|>{\raggedright\arraybackslash}p{5.5cm}|c|}
\hline
\textbf{Komponen} & \textbf{Teknik Asesmen} & \textbf{Indikator/CPMK} & \textbf{Bobot (\%)} \\
\hline
\endfirsthead
\hline
\textbf{Komponen} & \textbf{Teknik Asesmen} & \textbf{Indikator/CPMK} & \textbf{Bobot (\%)} \\
\hline
\endhead
\hline
\endfoot

Tugas Individu & Assembly Programming & Sub-CPMK 1.1, 1.2, 2.1, 2.2 & 20 \\
\hline
Praktikum & Lab Exercise with TASM & Sub-CPMK 2.1, 2.2, 3.1, 3.2 & 30 \\
\hline
Kuis & Theory \& Practice & Sub-CPMK 1.1, 1.2, 1.3 & 10 \\
\hline
UTS & Written \& Coding Exam & CPMK-1, CPMK-2 & 20 \\
\hline
Proyek Akhir & Final Application & CPMK-2, CPMK-3, CPMK-4 & 20 \\
\hline
\textbf{Total} & & & \textbf{100} \\
\hline
\end{longtable}

\textbf{Kriteria Penilaian:}
\begin{itemize}[leftmargin=*]
  \item A (85-100): Menguasai semua CPMK dengan sangat baik, mampu mengembangkan program assembly kompleks
  \item B (70-84): Menguasai sebagian besar CPMK dengan baik, mampu membuat program assembly moderat
  \item C (60-69): Menguasai CPMK dasar dengan cukup, mampu membuat program assembly sederhana
  \item D (50-59): Menguasai sebagian kecil CPMK, program assembly terbatas
  \item E (<50): Belum menguasai CPMK yang ditetapkan
\end{itemize}

\textbf{Penekanan Penilaian:}
\begin{itemize}[leftmargin=*]
  \item \textbf{70\% Praktik:} Kemampuan menulis dan debugging kode assembly
  \item \textbf{30\% Teori:} Pemahaman konsep arsitektur dan instruksi
\end{itemize}

\vspace{0.5cm}

% ============================================================
% 9. EVALUASI DAN REFLEKSI PEMBELAJARAN (OPSIONAL)
% ============================================================
\section*{9. Evaluasi dan Refleksi Pembelajaran}

Penilaian sumatif/formatif untuk memantau ketercapaian outcome secara menyeluruh:

\begin{itemize}[leftmargin=*]
  \item \textbf{Evaluasi Formatif:} Kuis mingguan, latihan coding TASM, debugging session untuk feedback berkelanjutan
  \item \textbf{Evaluasi Sumatif:} UTS dan UAS untuk mengukur pencapaian CPMK secara komprehensif
  \item \textbf{Refleksi Mahasiswa:} Jurnal praktikum mingguan untuk refleksi terhadap pemahaman instruksi assembly
  \item \textbf{Code Review Session:} Sesi review kode assembly untuk identifikasi bug dan optimasi
  \item \textbf{Evaluasi Dosen:} Survey keefektifan GUI TASM dan materi praktikum
  \item \textbf{Continuous Improvement:} Analisis hasil proyek assembly untuk perbaikan kurikulum
\end{itemize}

\textbf{Monitoring Progress:}
\begin{itemize}[leftmargin=*]
  \item Weekly coding checkpoint dengan TASM
  \item Progress tracking untuk proyek akhir
  \item Peer assessment untuk praktikum
  \item Performance benchmarking untuk optimasi kode
\end{itemize}

\vspace{0.5cm}

% ============================================================
% 10. DAFTAR REFERENSI
% ============================================================
\section*{10. Daftar Referensi}

Sumber belajar utama yang digunakan dalam penyusunan materi dan asesmen:

\begin{enumerate}[leftmargin=*]
  \item Intel Corporation. (1990). \textit{Intel 8086 Family User's Manual}. Intel Corporation.
  
  \item Mazidi, M. A., \& Mazidi, J. (2018). \textit{The 80x86 Microprocessor and Assembly Language Programming} (2nd ed.). Pearson.
  
  \item Irvine, K. R. (2019). \textit{Assembly Language for x86 Processors} (8th ed.). Pearson.
  
  \item Abel, P. (2015). \textit{IBM PC Assembly Language and Programming} (5th ed.). Pearson.
  
  \item Borland International. (1992). \textit{Turbo Assembler User's Guide}. Borland International.
  
  \item Jones, W. (2020). \textit{GUI Turbo Assembler (TASM) Tutorial}. Retrieved from \url{https://github.com/ljnath/GUI-Turbo-Assembler}
  
  \item Brey, B. B. (2021). \textit{The Intel Microprocessors: Architecture, Programming, and Interfacing} (9th ed.). Pearson.
  
  \item Norton, P., \& Socha, J. (2018). \textit{Peter Norton's Assembly Language Book for the IBM PC}. Brady/Simon \& Schuster.
  
  \item ETSU. (2024). \textit{CSCI 2150 -- Turbo Assembler Laboratory}. Retrieved from \url{https://faculty.etsu.edu/TARNOFF/labs2150/tasm/tasm.htm}
  
  \item GitHub Repository. (2024). \textit{8086 Assembly Language Programs}. Retrieved from \url{https://github.com/Amey-Thakur/8086-ASSEMBLY-LANGUAGE-PROGRAMS}
\end{enumerate}

\textbf{Online Resources:}
\begin{itemize}[leftmargin=*]
  \item GUI Turbo Assembler Documentation: \url{https://sourceforge.net/projects/guitasm8086/}
  \item 8086 Assembly Tutorial: \url{https://yassinebridi.github.io/asm-docs/}
  \item DOSBox for TASM Environment: \url{https://www.dosbox.com/}
\end{itemize}

\vspace{1cm}

\begin{flushright}
\begin{tabular}{c}
Disusun oleh,\\[2cm]
\textbf{Dr. [Nama Dosen], M.Kom.}\\
NIP. [NIP Dosen]
\end{tabular}
\end{flushright}

\end{document}
