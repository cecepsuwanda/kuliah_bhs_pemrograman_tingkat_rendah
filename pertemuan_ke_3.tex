\chapter{Instalasi Turbo Assembler dan Lingkungan Pengembangan; Struktur Program COM/EXE; Direktif Dasar (ORG, END)}

\section{Tujuan Pembelajaran}
Setelah mengikuti pertemuan ini, mahasiswa diharapkan dapat:
\begin{itemize}
\item Menginstal dan mengkonfigurasi Turbo Assembler
\item Memahami lingkungan pengembangan Turbo Assembler
\item Mengetahui perbedaan struktur program COM dan EXE
\item Menggunakan direktif dasar ORG dan END
\item Mampu membuat program assembly sederhana
\end{itemize}

\section{Materi Pembelajaran}

\subsection{Instalasi Turbo Assembler}
\begin{itemize}
\item Persyaratan sistem
\item Proses instalasi
\item Konfigurasi lingkungan
\item Pengaturan path dan direktori
\end{itemize}

\subsection{Lingkungan Pengembangan}
\begin{itemize}
\item Editor Turbo Assembler
\item Kompiler (TASM)
\item Linker (TLINK)
\item Debugger (TD)
\item File bantuan dan dokumentasi
\end{itemize}

\subsection{Struktur Program COM}
\begin{itemize}
\item Karakteristik program COM
\item Format file COM
\item Batasan ukuran program
\item Penggunaan memori
\item Contoh struktur program COM
\end{itemize}

\subsection{Struktur Program EXE}
\begin{itemize}
\item Karakteristik program EXE
\item Format file EXE
\item Header file EXE
\item Segmentasi program
\item Contoh struktur program EXE
\end{itemize}

\subsection{Direktif Dasar}
\begin{itemize}
\item Direktif ORG (Origin)
\item Direktif END
\item Direktif lainnya (TITLE, PAGE, etc.)
\item Penggunaan direktif dalam program
\end{itemize}

\section{Praktikum}
\begin{enumerate}
\item Instalasi Turbo Assembler
\item Konfigurasi lingkungan pengembangan
\item Membuat program "Hello World" sederhana
\item Kompilasi dan eksekusi program
\item Penggunaan direktif ORG dan END
\end{enumerate}

\section{Contoh Kode}
\begin{verbatim}
; Program sederhana menggunakan direktif dasar
TITLE Program Sederhana
PAGE 60,132

.MODEL SMALL
.STACK 100h

.DATA
    pesan DB 'Hello World!$'

.CODE
    ORG 100h
    START:
        MOV AX, @DATA
        MOV DS, AX
        
        MOV DX, OFFSET pesan
        MOV AH, 09h
        INT 21h
        
        MOV AH, 4Ch
        INT 21h
    END START
\end{verbatim}

\section{Tugas}
\begin{itemize}
\item Instal Turbo Assembler di komputer
\item Buat program COM sederhana yang menampilkan nama Anda
\item Jelaskan perbedaan antara program COM dan EXE
\item Buat dokumentasi proses instalasi dan konfigurasi
\end{itemize}

\section{Referensi}
\begin{itemize}
\item Borland. \textit{Turbo Assembler (TASM) User's Guide}, Borland International.
\item Nopi, Arif. \textit{Tutorial Bahasa Assembly (x86)}, Jasakom, Edisi Online.
\end{itemize}

