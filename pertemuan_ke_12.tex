\chapter{Pemrograman Grafik Dasar: Menggambar Piksel dan Garis (Mode Grafik INT 10h)}

\section{Tujuan Pembelajaran}
Setelah mengikuti pertemuan ini, mahasiswa diharapkan dapat:
\begin{itemize}
\item Memahami konsep mode grafik
\item Menggunakan interupsi INT 10h untuk mode grafik
\item Menggambar piksel di layar
\item Menggambar garis menggunakan algoritma
\item Mampu membuat program grafik sederhana
\end{itemize}

\section{Materi Pembelajaran}

\subsection{Konsep Mode Grafik}
\begin{itemize}
\item Perbedaan mode teks dan grafik
\item Resolusi layar
\item Color palette
\item Video memory organization
\item Pixel addressing
\end{itemize}

\subsection{Interupsi INT 10h untuk Grafik}
\begin{itemize}
\item Fungsi 00h: Set Video Mode
\item Fungsi 0Ch: Write Pixel
\item Fungsi 0Dh: Read Pixel
\item Fungsi 0Fh: Get Video Mode
\item Mode grafik yang tersedia
\end{itemize}

\subsection{Menggambar Piksel}
\begin{itemize}
\item Koordinat piksel (X, Y)
\item Color value
\item Perhitungan alamat memori video
\item Optimasi akses memori video
\end{itemize}

\subsection{Algoritma Menggambar Garis}
\begin{itemize}
\item Algoritma DDA (Digital Differential Analyzer)
\item Algoritma Bresenham
\item Implementasi algoritma garis
\item Optimasi algoritma
\end{itemize}

\subsection{Koordinat dan Transformasi}
\begin{itemize}
\item Sistem koordinat layar
\item Transformasi koordinat
\item Clipping
\item Viewport
\end{itemize}

\section{Praktikum}
\begin{enumerate}
\item Program mengatur mode grafik
\item Program menggambar piksel tunggal
\item Program menggambar garis horizontal dan vertikal
\item Program menggambar garis diagonal
\item Program menggambar bentuk geometri sederhana
\end{enumerate}

\section{Contoh Kode}
\begin{verbatim}
; Program demonstrasi grafik dasar
TITLE Pemrograman Grafik
.MODEL SMALL
.STACK 100h

.DATA
    x1 DW 100
    y1 DW 100
    x2 DW 200
    y2 DW 150
    color DB 15  ; Putih

.CODE
START:
    MOV AX, @DATA
    MOV DS, AX
    
    ; Set mode grafik 320x200, 256 warna
    MOV AH, 00h
    MOV AL, 13h
    INT 10h
    
    ; Gambar piksel tunggal
    MOV AH, 0Ch
    MOV AL, color
    MOV BH, 00h
    MOV CX, 160  ; X coordinate
    MOV DX, 100  ; Y coordinate
    INT 10h
    
    ; Gambar garis horizontal
    MOV CX, 50   ; X start
    MOV DX, 50   ; Y coordinate
    MOV BX, 100  ; X end
    
gambar_horizontal:
    MOV AH, 0Ch
    MOV AL, color
    MOV BH, 00h
    INT 10h
    INC CX
    CMP CX, BX
    JLE gambar_horizontal
    
    ; Gambar garis vertikal
    MOV CX, 50   ; X coordinate
    MOV DX, 50   ; Y start
    MOV BX, 100  ; Y end
    
gambar_vertikal:
    MOV AH, 0Ch
    MOV AL, color
    MOV BH, 00h
    INT 10h
    INC DX
    CMP DX, BX
    JLE gambar_vertikal
    
    ; Gambar garis diagonal menggunakan DDA
    CALL gambar_garis_dda
    
    ; Tunggu input keyboard
    MOV AH, 00h
    INT 16h
    
    ; Kembali ke mode teks
    MOV AH, 00h
    MOV AL, 03h
    INT 10h
    
    MOV AH, 4Ch
    INT 21h

; Prosedur menggambar garis menggunakan DDA
gambar_garis_dda PROC
    PUSH AX
    PUSH BX
    PUSH CX
    PUSH DX
    
    MOV AX, x2
    SUB AX, x1
    MOV BX, y2
    SUB BX, y1
    
    ; Hitung jumlah steps
    CMP AX, BX
    JGE steps_x
    MOV CX, BX
    JMP hitung_delta
steps_x:
    MOV CX, AX
    
hitung_delta:
    ; Delta X dan Delta Y
    MOV DX, AX
    SAR DX, 8  ; Delta X = (x2-x1)/steps
    MOV AX, BX
    SAR AX, 8  ; Delta Y = (y2-y1)/steps
    
    ; Inisialisasi
    MOV BX, x1
    MOV DX, y1
    
gambar_loop:
    PUSH AX
    PUSH BX
    PUSH CX
    PUSH DX
    
    MOV AH, 0Ch
    MOV AL, color
    MOV BH, 00h
    MOV CX, BX  ; X
    MOV DX, DX  ; Y
    INT 10h
    
    POP DX
    POP CX
    POP BX
    POP AX
    
    ADD BX, DX  ; X += Delta X
    ADD DX, AX  ; Y += Delta Y
    
    LOOP gambar_loop
    
    POP DX
    POP CX
    POP BX
    POP AX
    RET
gambar_garis_dda ENDP

END START
\end{verbatim}

\section{Latihan}
\begin{enumerate}
\item Buat program yang menggambar kotak
\item Buat program yang menggambar segitiga
\item Buat program yang menggambar lingkaran
\item Buat program yang menggambar pola geometri
\end{enumerate}

\section{Tugas}
\begin{itemize}
\item Implementasikan algoritma Bresenham untuk menggambar garis
\item Buat program yang menggambar grafik fungsi matematika
\item Implementasikan program paint sederhana
\item Dokumentasikan perbedaan antara algoritma DDA dan Bresenham
\end{itemize}

\section{Referensi}
\begin{itemize}
\item Hyde, Randall. \textit{The Art of Assembly Language}, 2nd ed., No Starch Press, 2010.
\item Partoharsojo, Hartono. \textit{Tuntunan Praktis Pemrograman Assembly}, Penerbit Informatika.
\end{itemize}

