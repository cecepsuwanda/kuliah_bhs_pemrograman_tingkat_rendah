\chapter{Pengenalan Bahasa Rakitan dan Bahasa Tingkat Rendah; Sistem Bilangan (Biner, Heksadesimal)}

\section{Tujuan Pembelajaran}
Setelah mengikuti pertemuan ini, mahasiswa diharapkan dapat:
\begin{itemize}
\item Memahami konsep bahasa rakitan dan bahasa tingkat rendah
\item Mengetahui perbedaan antara bahasa tingkat tinggi dan bahasa tingkat rendah
\item Memahami sistem bilangan biner dan heksadesimal
\item Mampu melakukan konversi antar sistem bilangan
\end{itemize}

\section{Materi Pembelajaran}

\subsection{Pengenalan Bahasa Rakitan dan Bahasa Tingkat Rendah}
\begin{itemize}
\item Definisi bahasa rakitan (assembly language)
\item Karakteristik bahasa tingkat rendah
\item Perbandingan dengan bahasa tingkat tinggi
\item Keunggulan dan kelemahan bahasa rakitan
\item Aplikasi bahasa rakitan dalam pemrograman
\end{itemize}

\subsection{Sistem Bilangan}
\begin{itemize}
\item Sistem bilangan biner (basis 2)
\item Sistem bilangan heksadesimal (basis 16)
\item Konversi antar sistem bilangan
\item Operasi aritmatika dalam sistem bilangan biner
\item Representasi data dalam komputer
\end{itemize}

\section{Contoh Soal dan Latihan}
\begin{enumerate}
\item Konversikan bilangan desimal 255 ke biner dan heksadesimal
\item Konversikan bilangan biner 11010110 ke desimal dan heksadesimal
\item Konversikan bilangan heksadesimal A5F ke desimal dan biner
\item Lakukan operasi penjumlahan biner: 1011 + 1101
\end{enumerate}

\section{Tugas}
\begin{itemize}
\item Buat tabel konversi bilangan 0-15 dalam format desimal, biner, dan heksadesimal
\item Jelaskan mengapa sistem bilangan heksadesimal sering digunakan dalam pemrograman
\item Berikan contoh aplikasi bahasa rakitan dalam kehidupan sehari-hari
\end{itemize}

\section{Referensi}
\begin{itemize}
\item Hyde, Randall. \textit{The Art of Assembly Language}, 2nd ed., No Starch Press, 2010.
\item Susanto. \textit{Belajar Pemrograman Bahasa Assembly}, Elex Media Komputindo, 1995.
\end{itemize}

