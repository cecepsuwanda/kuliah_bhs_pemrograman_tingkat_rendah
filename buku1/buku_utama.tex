\documentclass[a4paper,12pt]{book}
\usepackage[utf8]{inputenc}
\usepackage[indonesian]{babel}
\usepackage{enumitem}
\usepackage{geometry}
\usepackage{fancyhdr}
\usepackage{titlesec}
\usepackage{tocloft}

% Pengaturan halaman
\geometry{margin=2cm}
\pagestyle{fancy}
\fancyhf{}
\fancyfoot[C]{\thepage}
\fancyhead[L]{Pemrograman Bahasa Tingkat Rendah}
\fancyhead[R]{}

% Pengaturan judul chapter
\titleformat{\chapter}[display]
{\normalfont\huge\bfseries}{\chaptertitlename\ \thechapter}{20pt}{\Huge}
\titlespacing*{\chapter}{0pt}{50pt}{40pt}

% Pengaturan daftar isi
\renewcommand{\cftchapleader}{\cftdotfill{\cftdotsep}}

\begin{document}

% Halaman judul
\begin{titlepage}
\begin{center}
\vspace*{2cm}
{\Huge\bfseries Pemrograman Bahasa Tingkat Rendah}\\[0.5cm]
{\Large\bfseries (Turbo Assembler)}\\[1cm]
{\large Mata Kuliah Teknik Informatika}\\[2cm]

\vfill

{\large Disusun oleh:}\\
{\large Cecep Suwanda, S.Si., M.Kom.}\\[1cm]

{\large Program Studi Teknik Informatika}\\
{\large Fakultas Teknologi Informasi}\\
{\large Universitas Bale Bandung}\\[1cm]

{\large \today}
\end{center}
\end{titlepage}


% Kata pengantar
\chapter*{Kata Pengantar}
\addcontentsline{toc}{chapter}{Kata Pengantar}

Buku ini disusun sebagai bahan ajar untuk mata kuliah Pemrograman Bahasa Tingkat Rendah dengan fokus pada Turbo Assembler. Buku ini mencakup 15 pertemuan yang dirancang untuk memberikan pemahaman komprehensif tentang pemrograman assembly pada mikroprosesor Intel 8086.

\vspace{1cm}

Penulis berharap buku ini dapat membantu mahasiswa dalam memahami konsep-konsep dasar pemrograman assembly dan mengembangkan kemampuan praktis dalam menggunakan Turbo Assembler.

% Halaman baru untuk daftar isi
\newpage

% Daftar isi
\tableofcontents
\newpage


% Import semua pertemuan
\chapter{Pengenalan Bahasa Rakitan dan Bahasa Tingkat Rendah; Sistem Bilangan (Biner, Heksadesimal)}

\section{Tujuan Pembelajaran}
Setelah mengikuti pertemuan ini, mahasiswa diharapkan dapat:
\begin{itemize}
\item Memahami konsep bahasa rakitan dan bahasa tingkat rendah
\item Mengetahui perbedaan antara bahasa tingkat tinggi dan bahasa tingkat rendah
\item Memahami sistem bilangan biner dan heksadesimal
\item Mampu melakukan konversi antar sistem bilangan
\end{itemize}

\section{Materi Pembelajaran}

\subsection{Pengenalan Bahasa Rakitan dan Bahasa Tingkat Rendah}
\begin{itemize}
\item Definisi bahasa rakitan (assembly language)
\item Karakteristik bahasa tingkat rendah
\item Perbandingan dengan bahasa tingkat tinggi
\item Keunggulan dan kelemahan bahasa rakitan
\item Aplikasi bahasa rakitan dalam pemrograman
\end{itemize}

\subsection{Sistem Bilangan}
\begin{itemize}
\item Sistem bilangan biner (basis 2)
\item Sistem bilangan heksadesimal (basis 16)
\item Konversi antar sistem bilangan
\item Operasi aritmatika dalam sistem bilangan biner
\item Representasi data dalam komputer
\end{itemize}

\section{Contoh Soal dan Latihan}
\begin{enumerate}
\item Konversikan bilangan desimal 255 ke biner dan heksadesimal
\item Konversikan bilangan biner 11010110 ke desimal dan heksadesimal
\item Konversikan bilangan heksadesimal A5F ke desimal dan biner
\item Lakukan operasi penjumlahan biner: 1011 + 1101
\end{enumerate}

\section{Tugas}
\begin{itemize}
\item Buat tabel konversi bilangan 0-15 dalam format desimal, biner, dan heksadesimal
\item Jelaskan mengapa sistem bilangan heksadesimal sering digunakan dalam pemrograman
\item Berikan contoh aplikasi bahasa rakitan dalam kehidupan sehari-hari
\end{itemize}

\section{Referensi}
\begin{itemize}
\item Hyde, Randall. \textit{The Art of Assembly Language}, 2nd ed., No Starch Press, 2010.
\item Susanto. \textit{Belajar Pemrograman Bahasa Assembly}, Elex Media Komputindo, 1995.
\end{itemize}


\chapter{Arsitektur Mikroprosesor Intel 8086: Register, Segmentasi Memori, Mode Pengalamatan; Struktur File ASM}

\section{Tujuan Pembelajaran}
Setelah mengikuti pertemuan ini, mahasiswa diharapkan dapat:
\begin{itemize}
\item Memahami arsitektur mikroprosesor Intel 8086
\item Mengetahui fungsi dan penggunaan register-register utama
\item Memahami konsep segmentasi memori
\item Mengenal berbagai mode pengalamatan
\item Memahami struktur file ASM
\end{itemize}

\section{Materi Pembelajaran}

\subsection{Arsitektur Mikroprosesor Intel 8086}
\begin{itemize}
\item Sejarah dan karakteristik Intel 8086
\item Arsitektur internal mikroprosesor
\item Bus sistem dan kontrol
\item Mode operasi (real mode)
\end{itemize}

\subsection{Register-Register Utama}
\begin{itemize}
\item Register umum (AX, BX, CX, DX)
\item Register indeks (SI, DI)
\item Register pointer (SP, BP)
\item Register segmen (CS, DS, SS, ES)
\item Register flag (FLAGS)
\item Register instruksi (IP)
\end{itemize}

\subsection{Segmentasi Memori}
\begin{itemize}
\item Konsep segmentasi memori
\item Register segmen dan offset
\item Perhitungan alamat fisik
\item Batasan memori dalam mode real
\end{itemize}

\subsection{Mode Pengalamatan}
\begin{itemize}
\item Pengalamatan register
\item Pengalamatan langsung
\item Pengalamatan memori langsung
\item Pengalamatan register tidak langsung
\item Pengalamatan berbasis indeks
\item Pengalamatan berbasis register
\end{itemize}

\subsection{Struktur File ASM}
\begin{itemize}
\item Format dasar file assembly
\item Direktif assembler
\item Komentar dalam kode
\item Organisasi kode program
\end{itemize}

\section{Contoh Soal dan Latihan}
\begin{enumerate}
\item Jelaskan fungsi register AX, BX, CX, dan DX
\item Hitung alamat fisik jika CS = 2000h dan IP = 0100h
\item Berikan contoh penggunaan mode pengalamatan register
\item Tuliskan struktur dasar file ASM
\end{enumerate}

\section{Tugas}
\begin{itemize}
\item Buat diagram arsitektur Intel 8086
\item Jelaskan perbedaan antara register segmen dan register umum
\item Berikan contoh penggunaan berbagai mode pengalamatan
\item Buat template struktur file ASM untuk program sederhana
\end{itemize}

\section{Referensi}
\begin{itemize}
\item Brey, Barry B. \textit{Mikroprosesor Intel 8086/8088 dsb.}, edisi terjemahan, Penerbit Informatika.
\item Intel Corporation. \textit{Intel 64 and IA-32 Architectures Software Developer's Manual}, Intel.
\end{itemize}


\chapter{Instalasi Turbo Assembler dan Lingkungan Pengembangan; Struktur Program COM/EXE; Direktif Dasar (ORG, END)}

\section{Tujuan Pembelajaran}
Setelah mengikuti pertemuan ini, mahasiswa diharapkan dapat:
\begin{itemize}
\item Menginstal dan mengkonfigurasi Turbo Assembler
\item Memahami lingkungan pengembangan Turbo Assembler
\item Mengetahui perbedaan struktur program COM dan EXE
\item Menggunakan direktif dasar ORG dan END
\item Mampu membuat program assembly sederhana
\end{itemize}

\section{Materi Pembelajaran}

\subsection{Instalasi Turbo Assembler}
\begin{itemize}
\item Persyaratan sistem
\item Proses instalasi
\item Konfigurasi lingkungan
\item Pengaturan path dan direktori
\end{itemize}

\subsection{Lingkungan Pengembangan}
\begin{itemize}
\item Editor Turbo Assembler
\item Kompiler (TASM)
\item Linker (TLINK)
\item Debugger (TD)
\item File bantuan dan dokumentasi
\end{itemize}

\subsection{Struktur Program COM}
\begin{itemize}
\item Karakteristik program COM
\item Format file COM
\item Batasan ukuran program
\item Penggunaan memori
\item Contoh struktur program COM
\end{itemize}

\subsection{Struktur Program EXE}
\begin{itemize}
\item Karakteristik program EXE
\item Format file EXE
\item Header file EXE
\item Segmentasi program
\item Contoh struktur program EXE
\end{itemize}

\subsection{Direktif Dasar}
\begin{itemize}
\item Direktif ORG (Origin)
\item Direktif END
\item Direktif lainnya (TITLE, PAGE, etc.)
\item Penggunaan direktif dalam program
\end{itemize}

\section{Praktikum}
\begin{enumerate}
\item Instalasi Turbo Assembler
\item Konfigurasi lingkungan pengembangan
\item Membuat program "Hello World" sederhana
\item Kompilasi dan eksekusi program
\item Penggunaan direktif ORG dan END
\end{enumerate}

\section{Contoh Kode}
\begin{verbatim}
; Program sederhana menggunakan direktif dasar
TITLE Program Sederhana
PAGE 60,132

.MODEL SMALL
.STACK 100h

.DATA
    pesan DB 'Hello World!$'

.CODE
    ORG 100h
    START:
        MOV AX, @DATA
        MOV DS, AX
        
        MOV DX, OFFSET pesan
        MOV AH, 09h
        INT 21h
        
        MOV AH, 4Ch
        INT 21h
    END START
\end{verbatim}

\section{Tugas}
\begin{itemize}
\item Instal Turbo Assembler di komputer
\item Buat program COM sederhana yang menampilkan nama Anda
\item Jelaskan perbedaan antara program COM dan EXE
\item Buat dokumentasi proses instalasi dan konfigurasi
\end{itemize}

\section{Referensi}
\begin{itemize}
\item Borland. \textit{Turbo Assembler (TASM) User's Guide}, Borland International.
\item Nopi, Arif. \textit{Tutorial Bahasa Assembly (x86)}, Jasakom, Edisi Online.
\end{itemize}


\chapter{Instruksi Dasar: Perpindahan Data (MOV), Operasi Aritmatika (ADD, SUB, MUL, DIV)}

\section{Tujuan Pembelajaran}
Setelah mengikuti pertemuan ini, mahasiswa diharapkan dapat:
\begin{itemize}
\item Menggunakan instruksi MOV untuk perpindahan data
\item Melakukan operasi aritmatika dengan ADD, SUB, MUL, DIV
\item Memahami pengaruh operasi terhadap flag register
\item Mampu menulis program dengan operasi aritmatika dasar
\end{itemize}

\section{Materi Pembelajaran}

\subsection{Instruksi Perpindahan Data (MOV)}
\begin{itemize}
\item Sintaks instruksi MOV
\item Aturan penggunaan MOV
\item Perpindahan data antar register
\item Perpindahan data dari/ke memori
\item Perpindahan data dengan konstanta
\item Contoh penggunaan MOV
\end{itemize}

\subsection{Operasi Aritmatika}
\begin{itemize}
\item Instruksi ADD (penjumlahan)
\item Instruksi SUB (pengurangan)
\item Instruksi MUL (perkalian)
\item Instruksi DIV (pembagian)
\item Pengaruh terhadap flag register
\item Penanganan overflow dan underflow
\end{itemize}

\subsection{Flag Register dan Operasi Aritmatika}
\begin{itemize}
\item Flag Carry (CF)
\item Flag Zero (ZF)
\item Flag Sign (SF)
\item Flag Overflow (OF)
\item Flag Parity (PF)
\item Flag Auxiliary Carry (AF)
\end{itemize}

\section{Praktikum}
\begin{enumerate}
\item Program perpindahan data antar register
\item Program penjumlahan dua bilangan
\item Program pengurangan dengan penanganan flag
\item Program perkalian dan pembagian
\item Program kalkulator sederhana
\end{enumerate}

\section{Contoh Kode}
\begin{verbatim}
; Program operasi aritmatika dasar
TITLE Operasi Aritmatika
.MODEL SMALL
.STACK 100h

.DATA
    bil1 DW 15
    bil2 DW 7
    hasil DW ?

.CODE
START:
    MOV AX, @DATA
    MOV DS, AX
    
    ; Penjumlahan
    MOV AX, bil1
    ADD AX, bil2
    MOV hasil, AX
    
    ; Pengurangan
    MOV AX, bil1
    SUB AX, bil2
    
    ; Perkalian
    MOV AX, bil1
    MUL bil2
    
    ; Pembagian
    MOV AX, bil1
    DIV bil2
    
    MOV AH, 4Ch
    INT 21h
END START
\end{verbatim}

\section{Latihan}
\begin{enumerate}
\item Buat program yang menghitung (A + B) - (C + D)
\item Buat program yang menghitung (X * Y) / Z
\item Jelaskan pengaruh operasi aritmatika terhadap flag register
\item Buat program yang menampilkan status flag setelah operasi
\end{enumerate}

\section{Tugas}
\begin{itemize}
\item Buat kalkulator sederhana dengan operasi +, -, *, /
\item Implementasikan penanganan error untuk pembagian dengan nol
\item Buat program yang menghitung rata-rata dari 5 bilangan
\item Dokumentasikan penggunaan flag register dalam operasi aritmatika
\end{itemize}

\section{Referensi}
\begin{itemize}
\item Hyde, Randall. \textit{The Art of Assembly Language}, 2nd ed., No Starch Press, 2010.
\item Susanto. \textit{Belajar Pemrograman Bahasa Assembly}, Elex Media Komputindo, 1995.
\end{itemize}


\chapter{Instruksi Logika (AND, OR, XOR, NOT); Output Teks ke Layar (INT 10h fungsi 02h)}

\section{Tujuan Pembelajaran}
Setelah mengikuti pertemuan ini, mahasiswa diharapkan dapat:
\begin{itemize}
\item Menggunakan instruksi logika AND, OR, XOR, NOT
\item Memahami operasi bitwise dan penggunaannya
\item Menggunakan interupsi INT 10h untuk output teks
\item Mampu menampilkan teks di layar dengan kontrol posisi
\end{itemize}

\section{Materi Pembelajaran}

\subsection{Instruksi Logika}
\begin{itemize}
\item Instruksi AND (operasi logika AND)
\item Instruksi OR (operasi logika OR)
\item Instruksi XOR (operasi logika XOR)
\item Instruksi NOT (operasi logika NOT)
\item Operasi bitwise dan aplikasinya
\item Pengaruh terhadap flag register
\end{itemize}

\subsection{Aplikasi Instruksi Logika}
\begin{itemize}
\item Manipulasi bit
\item Masking dan unmasking
\item Enkripsi sederhana
\item Operasi set dan clear bit
\item Perbandingan bit
\end{itemize}

\subsection{Interupsi BIOS INT 10h}
\begin{itemize}
\item Fungsi 02h: Set Cursor Position
\item Fungsi 09h: Write Character and Attribute
\item Fungsi 0Ah: Write Character Only
\item Fungsi 0Eh: Write Character in TTY Mode
\item Parameter dan penggunaan
\end{itemize}

\subsection{Output Teks ke Layar}
\begin{itemize}
\item Kontrol posisi kursor
\item Atribut karakter (warna, background)
\item Mode teks dan grafik
\item Penanganan karakter khusus
\end{itemize}

\section{Praktikum}
\begin{enumerate}
\item Program operasi logika dasar
\item Program manipulasi bit
\item Program enkripsi sederhana dengan XOR
\item Program output teks dengan kontrol posisi
\item Program menampilkan teks berwarna
\end{enumerate}

\section{Contoh Kode}
\begin{verbatim}
; Program operasi logika dan output teks
TITLE Operasi Logika dan Output
.MODEL SMALL
.STACK 100h

.DATA
    pesan DB 'Hello Assembly!$'
    nilai1 DW 0F0F0h
    nilai2 DW 0FF00h

.CODE
START:
    MOV AX, @DATA
    MOV DS, AX
    
    ; Operasi logika
    MOV AX, nilai1
    AND AX, nilai2    ; Masking
    OR AX, 000Fh      ; Set bit terakhir
    XOR AX, 0FFFFh    ; Inversi
    NOT AX            ; Komplemen
    
    ; Output teks ke layar
    MOV AH, 02h       ; Set cursor position
    MOV BH, 00h       ; Page 0
    MOV DH, 10        ; Row 10
    MOV DL, 20        ; Column 20
    INT 10h
    
    ; Tampilkan karakter
    MOV AH, 09h       ; Write character
    MOV AL, 'A'       ; Character 'A'
    MOV BH, 00h       ; Page 0
    MOV BL, 1Eh       ; Yellow on blue
    MOV CX, 1         ; Count
    INT 10h
    
    MOV AH, 4Ch
    INT 21h
END START
\end{verbatim}

\section{Latihan}
\begin{enumerate}
\item Buat program yang melakukan masking pada 8 bit terakhir
\item Buat program enkripsi sederhana menggunakan XOR
\item Buat program yang menampilkan teks di berbagai posisi layar
\item Buat program yang menampilkan teks dengan berbagai warna
\end{enumerate}

\section{Tugas}
\begin{itemize}
\item Implementasikan operasi logika untuk manipulasi data
\item Buat program yang menampilkan menu dengan kontrol posisi
\item Buat program enkripsi/dekripsi teks menggunakan XOR
\item Dokumentasikan penggunaan interupsi INT 10h
\end{itemize}

\section{Referensi}
\begin{itemize}
\item Hyde, Randall. \textit{The Art of Assembly Language}, 2nd ed., No Starch Press, 2010.
\item Partoharsojo, Hartono. \textit{Tuntunan Praktis Pemrograman Assembly}, Penerbit Informatika.
\end{itemize}


\chapter{Input dari Keyboard (INT 16h); Penanganan Keyboard Buffer}

\section{Tujuan Pembelajaran}
Setelah mengikuti pertemuan ini, mahasiswa diharapkan dapat:
\begin{itemize}
\item Menggunakan interupsi INT 16h untuk input keyboard
\item Memahami fungsi-fungsi INT 16h
\item Menangani keyboard buffer
\item Mampu membuat program interaktif dengan input keyboard
\end{itemize}

\section{Materi Pembelajaran}

\subsection{Interupsi BIOS INT 16h}
\begin{itemize}
\item Fungsi 00h: Read Key (Blocking)
\item Fungsi 01h: Check for Key (Non-blocking)
\item Fungsi 02h: Get Keyboard Flags
\item Parameter dan return values
\item Perbedaan blocking dan non-blocking
\end{itemize}

\subsection{Keyboard Buffer}
\begin{itemize}
\item Konsep keyboard buffer
\item Cara kerja buffer keyboard
\item Penanganan buffer penuh
\item Flushing keyboard buffer
\item Status keyboard buffer
\end{itemize}

\subsection{Input Karakter dan String}
\begin{itemize}
\item Input karakter tunggal
\item Input string dengan echo
\item Input string tanpa echo (password)
\item Validasi input
\item Penanganan karakter khusus
\end{itemize}

\subsection{Karakter Khusus}
\begin{itemize}
\item Scan code dan ASCII code
\item Function keys (F1-F12)
\item Arrow keys
\item Control keys (Ctrl, Alt, Shift)
\item Special keys (Enter, Escape, Backspace)
\end{itemize}

\section{Praktikum}
\begin{enumerate}
\item Program input karakter tunggal
\item Program input string dengan echo
\item Program input password (tanpa echo)
\item Program penanganan keyboard buffer
\item Program interaktif dengan menu
\end{enumerate}

\section{Contoh Kode}
\begin{verbatim}
; Program input keyboard dan penanganan buffer
TITLE Input Keyboard
.MODEL SMALL
.STACK 100h

.DATA
    prompt DB 'Masukkan nama: $'
    nama DB 50 DUP('$')
    pesan DB 'Halo, $'
    buffer DB 50 DUP(?)

.CODE
START:
    MOV AX, @DATA
    MOV DS, AX
    
    ; Tampilkan prompt
    MOV AH, 09h
    MOV DX, OFFSET prompt
    INT 21h
    
    ; Input string dengan echo
    MOV AH, 0Ah
    MOV DX, OFFSET buffer
    INT 21h
    
    ; Input karakter tunggal (blocking)
    MOV AH, 00h
    INT 16h
    ; AL = ASCII code, AH = Scan code
    
    ; Check for key (non-blocking)
    MOV AH, 01h
    INT 16h
    JZ no_key
    ; Key available, read it
    MOV AH, 00h
    INT 16h
    
no_key:
    ; Flush keyboard buffer
    MOV AH, 0Ch
    MOV AL, 00h
    INT 21h
    
    MOV AH, 4Ch
    INT 21h
END START
\end{verbatim}

\section{Latihan}
\begin{enumerate}
\item Buat program yang menunggu input dari keyboard
\item Buat program yang menampilkan scan code dari key yang ditekan
\item Buat program input password dengan masking karakter
\item Buat program yang menangani keyboard buffer penuh
\end{enumerate}

\section{Tugas}
\begin{itemize}
\item Buat program login dengan validasi username dan password
\item Implementasikan menu interaktif dengan navigasi menggunakan arrow keys
\item Buat program yang menangani input dengan timeout
\item Dokumentasikan penggunaan interupsi INT 16h
\end{itemize}

\section{Referensi}
\begin{itemize}
\item Hyde, Randall. \textit{The Art of Assembly Language}, 2nd ed., No Starch Press, 2010.
\item Susanto. \textit{Belajar Pemrograman Bahasa Assembly}, Elex Media Komputindo, 1995.
\end{itemize}


\chapter{Percabangan dan Loop: CMP, JE, JNE, JL, JG, LOOP; Interupsi Mouse (INT 33h)}

\section{Tujuan Pembelajaran}
Setelah mengikuti pertemuan ini, mahasiswa diharapkan dapat:
\begin{itemize}
\item Menggunakan instruksi CMP untuk perbandingan
\item Menggunakan instruksi percabangan (JE, JNE, JL, JG)
\item Menggunakan instruksi LOOP untuk perulangan
\item Memahami interupsi mouse INT 33h
\item Mampu membuat program dengan struktur kontrol
\end{itemize}

\section{Materi Pembelajaran}

\subsection{Instruksi Perbandingan (CMP)}
\begin{itemize}
\item Sintaks instruksi CMP
\item Cara kerja CMP
\item Pengaruh terhadap flag register
\item Perbandingan dengan operasi aritmatika
\end{itemize}

\subsection{Instruksi Percabangan}
\begin{itemize}
\item Instruksi JE (Jump if Equal)
\item Instruksi JNE (Jump if Not Equal)
\item Instruksi JL (Jump if Less)
\item Instruksi JG (Jump if Greater)
\item Instruksi lainnya (JLE, JGE, JB, JA, etc.)
\item Conditional jumps dan flag register
\end{itemize}

\subsection{Instruksi Perulangan (LOOP)}
\begin{itemize}
\item Instruksi LOOP
\item Instruksi LOOPZ/LOOPE
\item Instruksi LOOPNZ/LOOPNE
\item Penggunaan register CX
\item Kontrol perulangan
\end{itemize}

\subsection{Interupsi Mouse INT 33h}
\begin{itemize}
\item Fungsi 00h: Initialize Mouse
\item Fungsi 01h: Show Mouse Cursor
\item Fungsi 02h: Hide Mouse Cursor
\item Fungsi 03h: Get Mouse Position and Button Status
\item Fungsi 04h: Set Mouse Cursor Position
\item Fungsi 05h: Get Button Press Information
\end{itemize}

\section{Praktikum}
\begin{enumerate}
\item Program percabangan sederhana
\item Program perulangan dengan LOOP
\item Program kombinasi percabangan dan perulangan
\item Program interaksi dengan mouse
\item Program game sederhana dengan mouse
\end{enumerate}

\section{Contoh Kode}
\begin{verbatim}
; Program percabangan, loop, dan mouse
TITLE Percabangan dan Loop
.MODEL SMALL
.STACK 100h

.DATA
    pesan1 DB 'Bilangan sama$'
    pesan2 DB 'Bilangan berbeda$'
    pesan3 DB 'Loop selesai$'
    bil1 DW 10
    bil2 DW 10

.CODE
START:
    MOV AX, @DATA
    MOV DS, AX
    
    ; Percabangan
    MOV AX, bil1
    CMP AX, bil2
    JE sama
    ; Jika tidak sama
    MOV AH, 09h
    MOV DX, OFFSET pesan2
    INT 21h
    JMP lanjut
    
sama:
    MOV AH, 09h
    MOV DX, OFFSET pesan1
    INT 21h
    
lanjut:
    ; Perulangan
    MOV CX, 5
ulang:
    ; Kode yang diulang
    DEC CX
    JNZ ulang
    
    ; Inisialisasi mouse
    MOV AX, 00h
    INT 33h
    CMP AX, 0
    JE no_mouse
    
    ; Tampilkan cursor mouse
    MOV AX, 01h
    INT 33h
    
    ; Baca posisi mouse
    MOV AX, 03h
    INT 33h
    ; BX = button status, CX = X position, DX = Y position
    
no_mouse:
    MOV AH, 4Ch
    INT 21h
END START
\end{verbatim}

\section{Latihan}
\begin{enumerate}
\item Buat program yang membandingkan dua bilangan
\item Buat program yang menghitung faktorial menggunakan LOOP
\item Buat program yang menampilkan angka 1-10 menggunakan perulangan
\item Buat program yang merespon klik mouse
\end{enumerate}

\section{Tugas}
\begin{itemize}
\item Buat program kalkulator dengan menu percabangan
\item Implementasikan algoritma sorting sederhana dengan perulangan
\item Buat program game tebak angka dengan mouse
\item Dokumentasikan penggunaan instruksi percabangan dan perulangan
\end{itemize}

\section{Referensi}
\begin{itemize}
\item Hyde, Randall. \textit{The Art of Assembly Language}, 2nd ed., No Starch Press, 2010.
\item Partoharsojo, Hartono. \textit{Tuntunan Praktis Pemrograman Assembly}, Penerbit Informatika.
\end{itemize}


\chapter{Instruksi Stack: PUSH, POP, CALL, RET; Subrutin dan Parameter Sederhana}

\section{Tujuan Pembelajaran}
Setelah mengikuti pertemuan ini, mahasiswa diharapkan dapat:
\begin{itemize}
\item Memahami konsep stack dan operasinya
\item Menggunakan instruksi PUSH dan POP
\item Menggunakan instruksi CALL dan RET
\item Membuat subrutin dengan parameter sederhana
\item Mampu memodularisasi program
\end{itemize}

\section{Materi Pembelajaran}

\subsection{Konsep Stack}
\begin{itemize}
\item Definisi stack (tumpukan)
\item Prinsip LIFO (Last In, First Out)
\item Register stack pointer (SP)
\item Register stack segment (SS)
\item Operasi stack dalam assembly
\end{itemize}

\subsection{Instruksi Stack}
\begin{itemize}
\item Instruksi PUSH (memasukkan data ke stack)
\item Instruksi POP (mengambil data dari stack)
\item PUSH/POP register
\item PUSH/POP memori
\item PUSH/POP flag register
\item PUSH/POP immediate value
\end{itemize}

\subsection{Instruksi Subrutin}
\begin{itemize}
\item Instruksi CALL (memanggil subrutin)
\item Instruksi RET (kembali dari subrutin)
\item CALL near dan CALL far
\item RET dengan parameter
\item Penanganan return address
\end{itemize}

\subsection{Parameter dalam Subrutin}
\begin{itemize}
\item Parameter melalui register
\item Parameter melalui stack
\item Parameter melalui memori
\item Return value dari subrutin
\item Konvensi pemanggilan
\end{itemize}

\section{Praktikum}
\begin{enumerate}
\item Program demonstrasi operasi stack
\item Program subrutin sederhana
\item Program subrutin dengan parameter
\item Program subrutin dengan return value
\item Program modular dengan multiple subrutin
\end{enumerate}

\section{Contoh Kode}
\begin{verbatim}
; Program demonstrasi stack dan subrutin
TITLE Stack dan Subrutin
.MODEL SMALL
.STACK 100h

.DATA
    pesan1 DB 'Hello from main!$'
    pesan2 DB 'Hello from subrutin!$'
    nilai1 DW 10
    nilai2 DW 20
    hasil DW ?

.CODE
START:
    MOV AX, @DATA
    MOV DS, AX
    
    ; Demonstrasi stack
    MOV AX, 1234h
    PUSH AX
    MOV BX, 5678h
    PUSH BX
    
    ; Ambil dari stack (urutan terbalik)
    POP CX  ; CX = 5678h
    POP DX  ; DX = 1234h
    
    ; Panggil subrutin
    CALL tampilkan_pesan
    
    ; Subrutin dengan parameter
    PUSH nilai1
    PUSH nilai2
    CALL tambahkan
    ADD SP, 4  ; Bersihkan parameter dari stack
    
    MOV AH, 4Ch
    INT 21h

; Subrutin tanpa parameter
tampilkan_pesan PROC
    PUSH AX
    PUSH DX
    
    MOV AH, 09h
    MOV DX, OFFSET pesan2
    INT 21h
    
    POP DX
    POP AX
    RET
tampilkan_pesan ENDP

; Subrutin dengan parameter
tambahkan PROC
    PUSH BP
    MOV BP, SP
    
    PUSH AX
    PUSH BX
    
    MOV AX, [BP+6]  ; Parameter kedua
    MOV BX, [BP+4]  ; Parameter pertama
    ADD AX, BX
    MOV hasil, AX
    
    POP BX
    POP AX
    POP BP
    RET
tambahkan ENDP

END START
\end{verbatim}

\section{Latihan}
\begin{enumerate}
\item Buat program yang menggunakan stack untuk menyimpan dan mengambil data
\item Buat subrutin yang menghitung kuadrat dari sebuah bilangan
\item Buat subrutin yang menukar nilai dua variabel
\item Buat program yang menggunakan multiple subrutin
\end{enumerate}

\section{Tugas}
\begin{itemize}
\item Implementasikan subrutin untuk operasi aritmatika dasar
\item Buat program kalkulator modular menggunakan subrutin
\item Implementasikan subrutin untuk manipulasi string
\item Dokumentasikan penggunaan stack dan subrutin
\end{itemize}

\section{Referensi}
\begin{itemize}
\item Hyde, Randall. \textit{The Art of Assembly Language}, 2nd ed., No Starch Press, 2010.
\item Susanto. \textit{Belajar Pemrograman Bahasa Assembly}, Elex Media Komputindo, 1995.
\end{itemize}


\chapter{Array dan String: Penyimpanan Array, Instruksi String (MOVS, CMPS, SCAS, LODS, STOS)}

\section{Tujuan Pembelajaran}
Setelah mengikuti pertemuan ini, mahasiswa diharapkan dapat:
\begin{itemize}
\item Memahami konsep array dan string dalam assembly
\item Menyimpan dan mengakses data array
\item Menggunakan instruksi string MOVS, CMPS, SCAS, LODS, STOS
\item Mampu memanipulasi string dan array
\item Mengoptimalkan operasi string dengan prefix REP
\end{itemize}

\section{Materi Pembelajaran}

\subsection{Konsep Array dan String}
\begin{itemize}
\item Definisi array dan string
\item Penyimpanan array dalam memori
\item Pengalamatan elemen array
\item String sebagai array karakter
\item Null-terminated string
\end{itemize}

\subsection{Penyimpanan Array}
\begin{itemize}
\item Array satu dimensi
\item Array multi dimensi
\item Pengalamatan array dengan indeks
\item Perhitungan offset elemen array
\item Array dengan ukuran berbeda
\end{itemize}

\subsection{Instruksi String}
\begin{itemize}
\item Instruksi MOVS (Move String)
\item Instruksi CMPS (Compare String)
\item Instruksi SCAS (Scan String)
\item Instruksi LODS (Load String)
\item Instruksi STOS (Store String)
\item Penggunaan prefix REP
\end{itemize}

\subsection{Register untuk Operasi String}
\begin{itemize}
\item SI (Source Index)
\item DI (Destination Index)
\item CX (Counter)
\item AL/AX (Accumulator)
\item Direction Flag (DF)
\end{itemize}

\section{Praktikum}
\begin{enumerate}
\item Program demonstrasi operasi array
\item Program copy string menggunakan MOVS
\item Program perbandingan string menggunakan CMPS
\item Program pencarian karakter menggunakan SCAS
\item Program manipulasi string lengkap
\end{enumerate}

\section{Contoh Kode}
\begin{verbatim}
; Program demonstrasi array dan string
TITLE Array dan String
.MODEL SMALL
.STACK 100h

.DATA
    array DW 10, 20, 30, 40, 50
    string1 DB 'Hello World$'
    string2 DB 20 DUP('$')
    string3 DB 'Assembly$'
    string4 DB 'Assembly$'
    karakter DB 'l'
    panjang EQU $ - string1

.CODE
START:
    MOV AX, @DATA
    MOV DS, AX
    MOV ES, AX
    
    ; Akses elemen array
    MOV BX, 2        ; Indeks elemen ke-2
    MOV AX, array[BX] ; AX = 30
    
    ; Copy string menggunakan MOVS
    LEA SI, string1
    LEA DI, string2
    MOV CX, panjang
    CLD              ; Clear direction flag
    REP MOVSB
    
    ; Perbandingan string menggunakan CMPS
    LEA SI, string3
    LEA DI, string4
    MOV CX, 8
    CLD
    REPE CMPSB
    JE sama
    ; String berbeda
    JMP lanjut
    
sama:
    ; String sama
    
lanjut:
    ; Pencarian karakter menggunakan SCAS
    LEA DI, string1
    MOV AL, karakter
    MOV CX, panjang
    CLD
    REPNE SCASB
    JE ditemukan
    ; Karakter tidak ditemukan
    JMP selesai
    
ditemukan:
    ; Karakter ditemukan di posisi CX
    
selesai:
    ; Load string menggunakan LODS
    LEA SI, string1
    MOV CX, 5
    CLD
    LODSB  ; Load byte ke AL
    
    ; Store string menggunakan STOS
    LEA DI, string2
    MOV AL, 'X'
    MOV CX, 5
    CLD
    REP STOSB
    
    MOV AH, 4Ch
    INT 21h
END START
\end{verbatim}

\section{Latihan}
\begin{enumerate}
\item Buat program yang mengakses elemen array dengan indeks
\item Buat program yang menyalin string dari satu lokasi ke lokasi lain
\item Buat program yang membandingkan dua string
\item Buat program yang mencari karakter dalam string
\end{enumerate}

\section{Tugas}
\begin{itemize}
\item Implementasikan fungsi untuk menghitung panjang string
\item Buat program yang membalik string
\item Implementasikan fungsi untuk menggabungkan dua string
\item Buat program yang mengurutkan array bilangan
\end{itemize}

\section{Referensi}
\begin{itemize}
\item Hyde, Randall. \textit{The Art of Assembly Language}, 2nd ed., No Starch Press, 2010.
\item Partoharsojo, Hartono. \textit{Tuntunan Praktis Pemrograman Assembly}, Penerbit Informatika.
\end{itemize}


\chapter{Pemrograman Modular: PROC/ENDP, Penggunaan Makro Sederhana (MACRO)}

\section{Tujuan Pembelajaran}
Setelah mengikuti pertemuan ini, mahasiswa diharapkan dapat:
\begin{itemize}
\item Memahami konsep pemrograman modular
\item Menggunakan PROC/ENDP untuk membuat prosedur
\item Membuat dan menggunakan makro sederhana
\item Membedakan antara prosedur dan makro
\item Mampu merancang program modular
\end{itemize}

\section{Materi Pembelajaran}

\subsection{Konsep Pemrograman Modular}
\begin{itemize}
\item Definisi pemrograman modular
\item Keuntungan pemrograman modular
\item Prinsip modularitas
\item Reusability dan maintainability
\item Organisasi kode program
\end{itemize}

\subsection{Prosedur dengan PROC/ENDP}
\begin{itemize}
\item Sintaks PROC/ENDP
\item Deklarasi prosedur
\item Parameter prosedur
\item Local variables dalam prosedur
\item Return value dari prosedur
\item Konvensi pemanggilan prosedur
\end{itemize}

\subsection{Makro (MACRO)}
\begin{itemize}
\item Definisi makro
\item Sintaks MACRO/ENDM
\item Parameter makro
\item Makro dengan multiple parameters
\item Makro bersyarat
\item Perbedaan makro dan prosedur
\end{itemize}

\subsection{Organisasi Program Modular}
\begin{itemize}
\item Struktur file program
\item Include files
\item Library routines
\item Module dependencies
\item Code organization
\end{itemize}

\section{Praktikum}
\begin{enumerate}
\item Program dengan prosedur sederhana
\item Program dengan prosedur berparameter
\item Program dengan makro sederhana
\item Program dengan makro berparameter
\item Program modular lengkap
\end{enumerate}

\section{Contoh Kode}
\begin{verbatim}
; Program demonstrasi prosedur dan makro
TITLE Pemrograman Modular
.MODEL SMALL
.STACK 100h

.DATA
    pesan1 DB 'Hello from procedure!$'
    pesan2 DB 'Hello from macro!$'
    nilai1 DW 15
    nilai2 DW 25
    hasil DW ?

; Makro untuk menampilkan pesan
tampilkan_pesan MACRO pesan
    PUSH AX
    PUSH DX
    MOV AH, 09h
    MOV DX, OFFSET pesan
    INT 21h
    POP DX
    POP AX
ENDM

; Makro untuk pertukaran nilai
tukar_nilai MACRO var1, var2
    PUSH AX
    MOV AX, var1
    XCHG AX, var2
    MOV var1, AX
    POP AX
ENDM

.CODE
START:
    MOV AX, @DATA
    MOV DS, AX
    
    ; Panggil prosedur
    CALL tampilkan_hello
    
    ; Gunakan makro
    tampilkan_pesan pesan2
    
    ; Pertukaran nilai menggunakan makro
    tukar_nilai nilai1, nilai2
    
    ; Panggil prosedur dengan parameter
    PUSH nilai1
    PUSH nilai2
    CALL tambahkan
    ADD SP, 4
    
    MOV AH, 4Ch
    INT 21h

; Prosedur tanpa parameter
tampilkan_hello PROC
    PUSH AX
    PUSH DX
    
    MOV AH, 09h
    MOV DX, OFFSET pesan1
    INT 21h
    
    POP DX
    POP AX
    RET
tampilkan_hello ENDP

; Prosedur dengan parameter
tambahkan PROC
    PUSH BP
    MOV BP, SP
    
    PUSH AX
    PUSH BX
    
    MOV AX, [BP+6]  ; Parameter kedua
    MOV BX, [BP+4]  ; Parameter pertama
    ADD AX, BX
    MOV hasil, AX
    
    POP BX
    POP AX
    POP BP
    RET
tambahkan ENDP

END START
\end{verbatim}

\section{Latihan}
\begin{enumerate}
\item Buat prosedur untuk menghitung faktorial
\item Buat makro untuk menampilkan karakter
\item Buat prosedur untuk mengurutkan array
\item Buat makro untuk operasi aritmatika
\end{enumerate}

\section{Tugas}
\begin{itemize}
\item Implementasikan library prosedur untuk operasi string
\item Buat makro untuk debugging dan logging
\item Implementasikan program kalkulator modular
\item Dokumentasikan perbedaan antara prosedur dan makro
\end{itemize}

\section{Referensi}
\begin{itemize}
\item Hyde, Randall. \textit{The Art of Assembly Language}, 2nd ed., No Starch Press, 2010.
\item Borland. \textit{Turbo Assembler (TASM) User's Guide}, Borland International.
\end{itemize}


\chapter{Pemrograman Grafik Dasar: Menggambar Piksel dan Garis (Mode Grafik INT 10h)}

\section{Tujuan Pembelajaran}
Setelah mengikuti pertemuan ini, mahasiswa diharapkan dapat:
\begin{itemize}
\item Memahami konsep mode grafik
\item Menggunakan interupsi INT 10h untuk mode grafik
\item Menggambar piksel di layar
\item Menggambar garis menggunakan algoritma
\item Mampu membuat program grafik sederhana
\end{itemize}

\section{Materi Pembelajaran}

\subsection{Konsep Mode Grafik}
\begin{itemize}
\item Perbedaan mode teks dan grafik
\item Resolusi layar
\item Color palette
\item Video memory organization
\item Pixel addressing
\end{itemize}

\subsection{Interupsi INT 10h untuk Grafik}
\begin{itemize}
\item Fungsi 00h: Set Video Mode
\item Fungsi 0Ch: Write Pixel
\item Fungsi 0Dh: Read Pixel
\item Fungsi 0Fh: Get Video Mode
\item Mode grafik yang tersedia
\end{itemize}

\subsection{Menggambar Piksel}
\begin{itemize}
\item Koordinat piksel (X, Y)
\item Color value
\item Perhitungan alamat memori video
\item Optimasi akses memori video
\end{itemize}

\subsection{Algoritma Menggambar Garis}
\begin{itemize}
\item Algoritma DDA (Digital Differential Analyzer)
\item Algoritma Bresenham
\item Implementasi algoritma garis
\item Optimasi algoritma
\end{itemize}

\subsection{Koordinat dan Transformasi}
\begin{itemize}
\item Sistem koordinat layar
\item Transformasi koordinat
\item Clipping
\item Viewport
\end{itemize}

\section{Praktikum}
\begin{enumerate}
\item Program mengatur mode grafik
\item Program menggambar piksel tunggal
\item Program menggambar garis horizontal dan vertikal
\item Program menggambar garis diagonal
\item Program menggambar bentuk geometri sederhana
\end{enumerate}

\section{Contoh Kode}
\begin{verbatim}
; Program demonstrasi grafik dasar
TITLE Pemrograman Grafik
.MODEL SMALL
.STACK 100h

.DATA
    x1 DW 100
    y1 DW 100
    x2 DW 200
    y2 DW 150
    color DB 15  ; Putih

.CODE
START:
    MOV AX, @DATA
    MOV DS, AX
    
    ; Set mode grafik 320x200, 256 warna
    MOV AH, 00h
    MOV AL, 13h
    INT 10h
    
    ; Gambar piksel tunggal
    MOV AH, 0Ch
    MOV AL, color
    MOV BH, 00h
    MOV CX, 160  ; X coordinate
    MOV DX, 100  ; Y coordinate
    INT 10h
    
    ; Gambar garis horizontal
    MOV CX, 50   ; X start
    MOV DX, 50   ; Y coordinate
    MOV BX, 100  ; X end
    
gambar_horizontal:
    MOV AH, 0Ch
    MOV AL, color
    MOV BH, 00h
    INT 10h
    INC CX
    CMP CX, BX
    JLE gambar_horizontal
    
    ; Gambar garis vertikal
    MOV CX, 50   ; X coordinate
    MOV DX, 50   ; Y start
    MOV BX, 100  ; Y end
    
gambar_vertikal:
    MOV AH, 0Ch
    MOV AL, color
    MOV BH, 00h
    INT 10h
    INC DX
    CMP DX, BX
    JLE gambar_vertikal
    
    ; Gambar garis diagonal menggunakan DDA
    CALL gambar_garis_dda
    
    ; Tunggu input keyboard
    MOV AH, 00h
    INT 16h
    
    ; Kembali ke mode teks
    MOV AH, 00h
    MOV AL, 03h
    INT 10h
    
    MOV AH, 4Ch
    INT 21h

; Prosedur menggambar garis menggunakan DDA
gambar_garis_dda PROC
    PUSH AX
    PUSH BX
    PUSH CX
    PUSH DX
    
    MOV AX, x2
    SUB AX, x1
    MOV BX, y2
    SUB BX, y1
    
    ; Hitung jumlah steps
    CMP AX, BX
    JGE steps_x
    MOV CX, BX
    JMP hitung_delta
steps_x:
    MOV CX, AX
    
hitung_delta:
    ; Delta X dan Delta Y
    MOV DX, AX
    SAR DX, 8  ; Delta X = (x2-x1)/steps
    MOV AX, BX
    SAR AX, 8  ; Delta Y = (y2-y1)/steps
    
    ; Inisialisasi
    MOV BX, x1
    MOV DX, y1
    
gambar_loop:
    PUSH AX
    PUSH BX
    PUSH CX
    PUSH DX
    
    MOV AH, 0Ch
    MOV AL, color
    MOV BH, 00h
    MOV CX, BX  ; X
    MOV DX, DX  ; Y
    INT 10h
    
    POP DX
    POP CX
    POP BX
    POP AX
    
    ADD BX, DX  ; X += Delta X
    ADD DX, AX  ; Y += Delta Y
    
    LOOP gambar_loop
    
    POP DX
    POP CX
    POP BX
    POP AX
    RET
gambar_garis_dda ENDP

END START
\end{verbatim}

\section{Latihan}
\begin{enumerate}
\item Buat program yang menggambar kotak
\item Buat program yang menggambar segitiga
\item Buat program yang menggambar lingkaran
\item Buat program yang menggambar pola geometri
\end{enumerate}

\section{Tugas}
\begin{itemize}
\item Implementasikan algoritma Bresenham untuk menggambar garis
\item Buat program yang menggambar grafik fungsi matematika
\item Implementasikan program paint sederhana
\item Dokumentasikan perbedaan antara algoritma DDA dan Bresenham
\end{itemize}

\section{Referensi}
\begin{itemize}
\item Hyde, Randall. \textit{The Art of Assembly Language}, 2nd ed., No Starch Press, 2010.
\item Partoharsojo, Hartono. \textit{Tuntunan Praktis Pemrograman Assembly}, Penerbit Informatika.
\end{itemize}


\chapter{Pemrosesan File Dasar: Membuka, Menutup, Membaca, Menulis File (INT 21h fungsi 3Ch/3Fh/40h)}

\section{Tujuan Pembelajaran}
Setelah mengikuti pertemuan ini, mahasiswa diharapkan dapat:
\begin{itemize}
\item Memahami konsep file handling
\item Menggunakan interupsi INT 21h untuk operasi file
\item Membuka dan menutup file
\item Membaca dan menulis data ke file
\item Menangani error dalam operasi file
\end{itemize}

\section{Materi Pembelajaran}

\subsection{Konsep File Handling}
\begin{itemize}
\item Definisi file dan file system
\item File handle dan file descriptor
\item Mode akses file (read, write, append)
\item File attributes dan permissions
\item Error handling dalam operasi file
\end{itemize}

\subsection{Interupsi INT 21h untuk File}
\begin{itemize}
\item Fungsi 3Ch: Create File
\item Fungsi 3Dh: Open File
\item Fungsi 3Eh: Close File
\item Fungsi 3Fh: Read File
\item Fungsi 40h: Write File
\item Fungsi 41h: Delete File
\end{itemize}

\subsection{Operasi File Dasar}
\begin{itemize}
\item Membuat file baru
\item Membuka file yang sudah ada
\item Menutup file
\item Membaca data dari file
\item Menulis data ke file
\item Menghapus file
\end{itemize}

\subsection{File Handle dan Error Handling}
\begin{itemize}
\item File handle management
\item Error codes dan penanganannya
\item Validasi operasi file
\item Resource cleanup
\end{itemize}

\subsection{Buffer dan Data Transfer}
\begin{itemize}
\item File buffer
\item Block transfer
\item Sequential access
\item Data formatting
\end{itemize}

\section{Praktikum}
\begin{enumerate}
\item Program membuat file baru
\item Program menulis data ke file
\item Program membaca data dari file
\item Program menyalin file
\item Program dengan error handling
\end{enumerate}

\section{Contoh Kode}
\begin{verbatim}
; Program demonstrasi operasi file dasar
TITLE Pemrosesan File
.MODEL SMALL
.STACK 100h

.DATA
    nama_file DB 'test.txt', 0
    data_tulis DB 'Hello, World!', 13, 10
    data_baca DB 100 DUP(?)
    file_handle DW ?
    pesan_sukses DB 'File berhasil dibuat!$'
    pesan_error DB 'Error dalam operasi file!$'
    pesan_baca DB 'Data dari file: $'

.CODE
START:
    MOV AX, @DATA
    MOV DS, AX
    
    ; Buat file baru
    MOV AH, 3Ch
    MOV CX, 0        ; File attributes (normal)
    MOV DX, OFFSET nama_file
    INT 21h
    JC error_handler
    MOV file_handle, AX
    
    ; Tampilkan pesan sukses
    MOV AH, 09h
    MOV DX, OFFSET pesan_sukses
    INT 21h
    
    ; Tulis data ke file
    MOV AH, 40h
    MOV BX, file_handle
    MOV CX, 15       ; Jumlah byte yang ditulis
    MOV DX, OFFSET data_tulis
    INT 21h
    JC error_handler
    
    ; Tutup file
    MOV AH, 3Eh
    MOV BX, file_handle
    INT 21h
    JC error_handler
    
    ; Buka file untuk dibaca
    MOV AH, 3Dh
    MOV AL, 0        ; Mode read
    MOV DX, OFFSET nama_file
    INT 21h
    JC error_handler
    MOV file_handle, AX
    
    ; Baca data dari file
    MOV AH, 3Fh
    MOV BX, file_handle
    MOV CX, 100      ; Jumlah byte yang dibaca
    MOV DX, OFFSET data_baca
    INT 21h
    JC error_handler
    
    ; Tampilkan data yang dibaca
    MOV AH, 09h
    MOV DX, OFFSET pesan_baca
    INT 21h
    
    ; Tambahkan null terminator
    MOV BX, OFFSET data_baca
    ADD BX, AX
    MOV BYTE PTR [BX], '$'
    
    MOV AH, 09h
    MOV DX, OFFSET data_baca
    INT 21h
    
    ; Tutup file
    MOV AH, 3Eh
    MOV BX, file_handle
    INT 21h
    
    JMP selesai
    
error_handler:
    MOV AH, 09h
    MOV DX, OFFSET pesan_error
    INT 21h
    
selesai:
    MOV AH, 4Ch
    INT 21h
END START
\end{verbatim}

\section{Latihan}
\begin{enumerate}
\item Buat program yang menulis teks ke file
\item Buat program yang membaca dan menampilkan isi file
\item Buat program yang menyalin isi file ke file lain
\item Buat program yang menghitung jumlah karakter dalam file
\end{enumerate}

\section{Tugas}
\begin{itemize}
\item Implementasikan program text editor sederhana
\item Buat program yang menyimpan dan memuat data konfigurasi
\item Implementasikan program backup file
\item Dokumentasikan error codes dan penanganannya
\end{itemize}

\section{Referensi}
\begin{itemize}
\item Hyde, Randall. \textit{The Art of Assembly Language}, 2nd ed., No Starch Press, 2010.
\item Susanto. \textit{Belajar Pemrograman Bahasa Assembly}, Elex Media Komputindo, 1995.
\end{itemize}


\chapter{Pemrosesan File Lanjutan: Manipulasi Pointer (INT 21h fungsi 42h), Penyimpanan Data Terformat}

\section{Tujuan Pembelajaran}
Setelah mengikuti pertemuan ini, mahasiswa diharapkan dapat:
\begin{itemize}
\item Menggunakan fungsi 42h untuk manipulasi file pointer
\item Memahami konsep random access file
\item Menyimpan dan memuat data terformat
\item Mengimplementasikan database sederhana
\item Mampu membuat program file management
\end{itemize}

\section{Materi Pembelajaran}

\subsection{Manipulasi File Pointer}
\begin{itemize}
\item Konsep file pointer dan position
\item Fungsi 42h: LSEEK (Set File Pointer)
\item Mode seek (beginning, current, end)
\item Random access file
\item File positioning dan navigation
\end{itemize}

\subsection{Fungsi LSEEK (42h)}
\begin{itemize}
\item Sintaks fungsi 42h
\item Parameter AL (seek mode)
\item Parameter BX (file handle)
\item Parameter CX:DX (offset)
\item Return value dalam DX:AX
\end{itemize}

\subsection{Data Terformat}
\begin{itemize}
\item Konsep data terformat
\item Record-based data
\item Fixed-length records
\item Variable-length records
\item Data serialization
\end{itemize}

\subsection{Database Sederhana}
\begin{itemize}
\item Struktur record
\item Index file
\item CRUD operations (Create, Read, Update, Delete)
\item Data validation
\item File locking
\end{itemize}

\subsection{Optimasi File Access}
\begin{itemize}
\item Buffer management
\item Caching strategies
\item Batch operations
\item Error recovery
\end{itemize}

\section{Praktikum}
\begin{enumerate}
\item Program demonstrasi LSEEK
\item Program random access file
\item Program penyimpanan data terformat
\item Program database sederhana
\item Program file management
\end{enumerate}

\section{Contoh Kode}
\begin{verbatim}
; Program demonstrasi manipulasi file pointer dan data terformat
TITLE Pemrosesan File Lanjutan
.MODEL SMALL
.STACK 100h

.DATA
    nama_file DB 'data.dat', 0
    file_handle DW ?
    
    ; Struktur record
    record_size EQU 20
    nama DB 15 DUP(' ')
    umur DB ?
    gaji DW ?
    
    ; Data untuk disimpan
    data1 DB 'John Doe       ', 25, 1000
    data2 DB 'Jane Smith     ', 30, 1500
    data3 DB 'Bob Johnson    ', 35, 2000
    
    ; Buffer untuk membaca
    buffer DB record_size DUP(?)
    
    pesan_sukses DB 'Operasi berhasil!$'
    pesan_error DB 'Error dalam operasi!$'

.CODE
START:
    MOV AX, @DATA
    MOV DS, AX
    
    ; Buat file baru
    MOV AH, 3Ch
    MOV CX, 0
    MOV DX, OFFSET nama_file
    INT 21h
    JC error_handler
    MOV file_handle, AX
    
    ; Tulis record pertama
    MOV AH, 40h
    MOV BX, file_handle
    MOV CX, record_size
    MOV DX, OFFSET data1
    INT 21h
    JC error_handler
    
    ; Tulis record kedua
    MOV AH, 40h
    MOV BX, file_handle
    MOV CX, record_size
    MOV DX, OFFSET data2
    INT 21h
    JC error_handler
    
    ; Tulis record ketiga
    MOV AH, 40h
    MOV BX, file_handle
    MOV CX, record_size
    MOV DX, OFFSET data3
    INT 21h
    JC error_handler
    
    ; Baca record kedua (random access)
    MOV AH, 42h
    MOV AL, 0        ; Seek from beginning
    MOV BX, file_handle
    MOV CX, 0
    MOV DX, record_size  ; Offset ke record kedua
    INT 21h
    JC error_handler
    
    ; Baca record
    MOV AH, 3Fh
    MOV BX, file_handle
    MOV CX, record_size
    MOV DX, OFFSET buffer
    INT 21h
    JC error_handler
    
    ; Update record kedua
    MOV AH, 42h
    MOV AL, 1        ; Seek from current position
    MOV BX, file_handle
    MOV CX, 0
    MOV DX, -record_size  ; Kembali ke awal record
    INT 21h
    JC error_handler
    
    ; Modifikasi data
    MOV buffer[15], 31    ; Update umur
    MOV WORD PTR buffer[16], 1600  ; Update gaji
    
    ; Tulis record yang dimodifikasi
    MOV AH, 40h
    MOV BX, file_handle
    MOV CX, record_size
    MOV DX, OFFSET buffer
    INT 21h
    JC error_handler
    
    ; Baca semua record
    MOV AH, 42h
    MOV AL, 0        ; Seek to beginning
    MOV BX, file_handle
    MOV CX, 0
    MOV DX, 0
    INT 21h
    JC error_handler
    
    ; Baca dan tampilkan semua record
    MOV CX, 3        ; Jumlah record
baca_loop:
    PUSH CX
    
    MOV AH, 3Fh
    MOV BX, file_handle
    MOV CX, record_size
    MOV DX, OFFSET buffer
    INT 21h
    JC error_handler
    
    ; Tampilkan nama (15 karakter pertama)
    MOV buffer[15], '$'
    MOV AH, 09h
    MOV DX, OFFSET buffer
    INT 21h
    
    ; Tampilkan newline
    MOV AH, 02h
    MOV DL, 13
    INT 21h
    MOV DL, 10
    INT 21h
    
    POP CX
    LOOP baca_loop
    
    ; Tutup file
    MOV AH, 3Eh
    MOV BX, file_handle
    INT 21h
    
    JMP selesai
    
error_handler:
    MOV AH, 09h
    MOV DX, OFFSET pesan_error
    INT 21h
    
selesai:
    MOV AH, 4Ch
    INT 21h
END START
\end{verbatim}

\section{Latihan}
\begin{enumerate}
\item Buat program yang membaca record tertentu dari file
\item Buat program yang mengupdate record di posisi tertentu
\item Buat program yang menghapus record (mark as deleted)
\item Buat program yang mencari record berdasarkan kriteria
\end{enumerate}

\section{Tugas}
\begin{itemize}
\item Implementasikan program address book dengan file
\item Buat program inventory management sederhana
\item Implementasikan program student database
\item Dokumentasikan teknik optimasi file access
\end{itemize}

\section{Referensi}
\begin{itemize}
\item Hyde, Randall. \textit{The Art of Assembly Language}, 2nd ed., No Starch Press, 2010.
\item Partoharsojo, Hartono. \textit{Tuntunan Praktis Pemrograman Assembly}, Penerbit Informatika.
\end{itemize}


\chapter{Program Residen (TSR): Konsep Terminate-and-Stay-Resident; Contoh Implementasi Sederhana}

\section{Tujuan Pembelajaran}
Setelah mengikuti pertemuan ini, mahasiswa diharapkan dapat:
\begin{itemize}
\item Memahami konsep program TSR (Terminate-and-Stay-Resident)
\item Mengetahui cara kerja program residen
\item Mengimplementasikan TSR sederhana
\item Menggunakan interupsi untuk komunikasi TSR
\item Mampu membuat program background
\end{itemize}

\section{Materi Pembelajaran}

\subsection{Konsep Program TSR}
\begin{itemize}
\item Definisi TSR (Terminate-and-Stay-Resident)
\item Perbedaan program normal dan TSR
\item Keuntungan dan kelemahan TSR
\item Aplikasi program TSR
\item Memory management untuk TSR
\end{itemize}

\subsection{Arsitektur Program TSR}
\begin{itemize}
\item Initialization routine
\item Resident routine
\item Interrupt handler
\item Memory allocation
\item Program termination
\end{itemize}

\subsection{Interupsi untuk TSR}
\begin{itemize}
\item INT 21h fungsi 31h: Terminate and Stay Resident
\item INT 21h fungsi 25h: Set Interrupt Vector
\item INT 21h fungsi 35h: Get Interrupt Vector
\item Custom interrupt untuk komunikasi
\end{itemize}

\subsection{Implementasi TSR}
\begin{itemize}
\item Program structure
\item Memory management
\item Interrupt hooking
\item Communication mechanism
\item Error handling
\end{itemize}

\section{Praktikum}
\begin{enumerate}
\item Program TSR sederhana
\item Program TSR dengan interrupt handler
\item Program TSR dengan komunikasi
\item Program TSR dengan memory management
\item Program TSR dengan error handling
\end{enumerate}

\section{Contoh Kode}
\begin{verbatim}
; Program TSR sederhana
TITLE Program TSR
.MODEL SMALL
.STACK 100h

.DATA
    pesan_install DB 'TSR berhasil diinstall!$'
    pesan_uninstall DB 'TSR berhasil diuninstall!$'
    old_int21h DD ?
    tsr_installed DB 0

.CODE
START:
    MOV AX, @DATA
    MOV DS, AX
    
    ; Cek apakah TSR sudah diinstall
    MOV AH, 0FFh
    INT 21h
    CMP AL, 0AAh
    JE sudah_install
    
    ; Install TSR
    CALL install_tsr
    
    ; Tampilkan pesan
    MOV AH, 09h
    MOV DX, OFFSET pesan_install
    INT 21h
    
    ; Terminate and Stay Resident
    MOV AH, 31h
    MOV AL, 0
    MOV DX, 1000h  ; Ukuran program dalam paragraph
    INT 21h
    
sudah_install:
    ; Uninstall TSR
    CALL uninstall_tsr
    
    ; Tampilkan pesan
    MOV AH, 09h
    MOV DX, OFFSET pesan_uninstall
    INT 21h
    
    MOV AH, 4Ch
    INT 21h

; Prosedur install TSR
install_tsr PROC
    ; Simpan interrupt vector lama
    MOV AH, 35h
    MOV AL, 21h
    INT 21h
    MOV WORD PTR old_int21h, BX
    MOV WORD PTR old_int21h[2], ES
    
    ; Set interrupt vector baru
    MOV AH, 25h
    MOV AL, 21h
    MOV DX, OFFSET new_int21h
    INT 21h
    
    RET
install_tsr ENDP

; Prosedur uninstall TSR
uninstall_tsr PROC
    ; Restore interrupt vector lama
    MOV AH, 25h
    MOV AL, 21h
    MOV DX, WORD PTR old_int21h
    MOV DS, WORD PTR old_int21h[2]
    INT 21h
    
    RET
uninstall_tsr ENDP

; Interrupt handler baru
new_int21h PROC
    CMP AH, 0FFh
    JNE call_old_int
    
    ; Handler untuk komunikasi TSR
    MOV AL, 0AAh  ; Tanda TSR aktif
    IRET
    
call_old_int:
    ; Panggil interrupt handler lama
    JMP DWORD PTR old_int21h
    
new_int21h ENDP

END START
\end{verbatim}

\section{Latihan}
\begin{enumerate}
\item Buat TSR yang menampilkan waktu di layar
\item Buat TSR yang menangani hotkey
\item Buat TSR yang memantau aktivitas keyboard
\item Buat TSR yang melakukan backup otomatis
\end{enumerate}

\section{Tugas}
\begin{itemize}
\item Implementasikan TSR untuk system monitoring
\item Buat TSR untuk keyboard macro
\item Implementasikan TSR untuk file watcher
\item Dokumentasikan teknik memory management untuk TSR
\end{itemize}

\section{Referensi}
\begin{itemize}
\item Hyde, Randall. \textit{The Art of Assembly Language}, 2nd ed., No Starch Press, 2010.
\item Susanto. \textit{Belajar Pemrograman Bahasa Assembly}, Elex Media Komputindo, 1995.
\end{itemize}



% Daftar pustaka
\chapter*{Daftar Pustaka}
\addcontentsline{toc}{chapter}{Daftar Pustaka}

\begin{enumerate}
\item Hyde, Randall. \textit{The Art of Assembly Language}, 2nd ed., No Starch Press, 2010.
\item Susanto. \textit{Belajar Pemrograman Bahasa Assembly}, Elex Media Komputindo, 1995.
\item Partoharsojo, Hartono. \textit{Tuntunan Praktis Pemrograman Assembly}, Penerbit Informatika.
\item Brey, Barry B. \textit{Mikroprosesor Intel 8086/8088 dsb.}, edisi terjemahan, Penerbit Informatika.
\item Borland. \textit{Turbo Assembler (TASM) User's Guide}, Borland International.
\item Intel Corporation. \textit{Intel 64 and IA-32 Architectures Software Developer's Manual}, Intel.
\item Nopi, Arif. \textit{Tutorial Bahasa Assembly (x86)}, Jasakom, Edisi Online.
\end{enumerate}

\end{document}
