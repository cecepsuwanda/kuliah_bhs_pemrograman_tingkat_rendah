\chapter{Pemrosesan File Lanjutan: Manipulasi Pointer (INT 21h fungsi 42h), Penyimpanan Data Terformat}

\section{Tujuan Pembelajaran}
Setelah mengikuti pertemuan ini, mahasiswa diharapkan dapat:
\begin{itemize}
\item Menggunakan fungsi 42h untuk manipulasi file pointer
\item Memahami konsep random access file
\item Menyimpan dan memuat data terformat
\item Mengimplementasikan database sederhana
\item Mampu membuat program file management
\end{itemize}

\section{Materi Pembelajaran}

\subsection{Manipulasi File Pointer}
\begin{itemize}
\item Konsep file pointer dan position
\item Fungsi 42h: LSEEK (Set File Pointer)
\item Mode seek (beginning, current, end)
\item Random access file
\item File positioning dan navigation
\end{itemize}

\subsection{Fungsi LSEEK (42h)}
\begin{itemize}
\item Sintaks fungsi 42h
\item Parameter AL (seek mode)
\item Parameter BX (file handle)
\item Parameter CX:DX (offset)
\item Return value dalam DX:AX
\end{itemize}

\subsection{Data Terformat}
\begin{itemize}
\item Konsep data terformat
\item Record-based data
\item Fixed-length records
\item Variable-length records
\item Data serialization
\end{itemize}

\subsection{Database Sederhana}
\begin{itemize}
\item Struktur record
\item Index file
\item CRUD operations (Create, Read, Update, Delete)
\item Data validation
\item File locking
\end{itemize}

\subsection{Optimasi File Access}
\begin{itemize}
\item Buffer management
\item Caching strategies
\item Batch operations
\item Error recovery
\end{itemize}

\section{Praktikum}
\begin{enumerate}
\item Program demonstrasi LSEEK
\item Program random access file
\item Program penyimpanan data terformat
\item Program database sederhana
\item Program file management
\end{enumerate}

\section{Contoh Kode}
\begin{verbatim}
; Program demonstrasi manipulasi file pointer dan data terformat
TITLE Pemrosesan File Lanjutan
.MODEL SMALL
.STACK 100h

.DATA
    nama_file DB 'data.dat', 0
    file_handle DW ?
    
    ; Struktur record
    record_size EQU 20
    nama DB 15 DUP(' ')
    umur DB ?
    gaji DW ?
    
    ; Data untuk disimpan
    data1 DB 'John Doe       ', 25, 1000
    data2 DB 'Jane Smith     ', 30, 1500
    data3 DB 'Bob Johnson    ', 35, 2000
    
    ; Buffer untuk membaca
    buffer DB record_size DUP(?)
    
    pesan_sukses DB 'Operasi berhasil!$'
    pesan_error DB 'Error dalam operasi!$'

.CODE
START:
    MOV AX, @DATA
    MOV DS, AX
    
    ; Buat file baru
    MOV AH, 3Ch
    MOV CX, 0
    MOV DX, OFFSET nama_file
    INT 21h
    JC error_handler
    MOV file_handle, AX
    
    ; Tulis record pertama
    MOV AH, 40h
    MOV BX, file_handle
    MOV CX, record_size
    MOV DX, OFFSET data1
    INT 21h
    JC error_handler
    
    ; Tulis record kedua
    MOV AH, 40h
    MOV BX, file_handle
    MOV CX, record_size
    MOV DX, OFFSET data2
    INT 21h
    JC error_handler
    
    ; Tulis record ketiga
    MOV AH, 40h
    MOV BX, file_handle
    MOV CX, record_size
    MOV DX, OFFSET data3
    INT 21h
    JC error_handler
    
    ; Baca record kedua (random access)
    MOV AH, 42h
    MOV AL, 0        ; Seek from beginning
    MOV BX, file_handle
    MOV CX, 0
    MOV DX, record_size  ; Offset ke record kedua
    INT 21h
    JC error_handler
    
    ; Baca record
    MOV AH, 3Fh
    MOV BX, file_handle
    MOV CX, record_size
    MOV DX, OFFSET buffer
    INT 21h
    JC error_handler
    
    ; Update record kedua
    MOV AH, 42h
    MOV AL, 1        ; Seek from current position
    MOV BX, file_handle
    MOV CX, 0
    MOV DX, -record_size  ; Kembali ke awal record
    INT 21h
    JC error_handler
    
    ; Modifikasi data
    MOV buffer[15], 31    ; Update umur
    MOV WORD PTR buffer[16], 1600  ; Update gaji
    
    ; Tulis record yang dimodifikasi
    MOV AH, 40h
    MOV BX, file_handle
    MOV CX, record_size
    MOV DX, OFFSET buffer
    INT 21h
    JC error_handler
    
    ; Baca semua record
    MOV AH, 42h
    MOV AL, 0        ; Seek to beginning
    MOV BX, file_handle
    MOV CX, 0
    MOV DX, 0
    INT 21h
    JC error_handler
    
    ; Baca dan tampilkan semua record
    MOV CX, 3        ; Jumlah record
baca_loop:
    PUSH CX
    
    MOV AH, 3Fh
    MOV BX, file_handle
    MOV CX, record_size
    MOV DX, OFFSET buffer
    INT 21h
    JC error_handler
    
    ; Tampilkan nama (15 karakter pertama)
    MOV buffer[15], '$'
    MOV AH, 09h
    MOV DX, OFFSET buffer
    INT 21h
    
    ; Tampilkan newline
    MOV AH, 02h
    MOV DL, 13
    INT 21h
    MOV DL, 10
    INT 21h
    
    POP CX
    LOOP baca_loop
    
    ; Tutup file
    MOV AH, 3Eh
    MOV BX, file_handle
    INT 21h
    
    JMP selesai
    
error_handler:
    MOV AH, 09h
    MOV DX, OFFSET pesan_error
    INT 21h
    
selesai:
    MOV AH, 4Ch
    INT 21h
END START
\end{verbatim}

\section{Latihan}
\begin{enumerate}
\item Buat program yang membaca record tertentu dari file
\item Buat program yang mengupdate record di posisi tertentu
\item Buat program yang menghapus record (mark as deleted)
\item Buat program yang mencari record berdasarkan kriteria
\end{enumerate}

\section{Tugas}
\begin{itemize}
\item Implementasikan program address book dengan file
\item Buat program inventory management sederhana
\item Implementasikan program student database
\item Dokumentasikan teknik optimasi file access
\end{itemize}

\section{Referensi}
\begin{itemize}
\item Hyde, Randall. \textit{The Art of Assembly Language}, 2nd ed., No Starch Press, 2010.
\item Partoharsojo, Hartono. \textit{Tuntunan Praktis Pemrograman Assembly}, Penerbit Informatika.
\end{itemize}

