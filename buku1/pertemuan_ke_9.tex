\chapter{Instruksi Stack: PUSH, POP, CALL, RET; Subrutin dan Parameter Sederhana}

\section{Tujuan Pembelajaran}
Setelah mengikuti pertemuan ini, mahasiswa diharapkan dapat:
\begin{itemize}
\item Memahami konsep stack dan operasinya
\item Menggunakan instruksi PUSH dan POP
\item Menggunakan instruksi CALL dan RET
\item Membuat subrutin dengan parameter sederhana
\item Mampu memodularisasi program
\end{itemize}

\section{Materi Pembelajaran}

\subsection{Konsep Stack}
\begin{itemize}
\item Definisi stack (tumpukan)
\item Prinsip LIFO (Last In, First Out)
\item Register stack pointer (SP)
\item Register stack segment (SS)
\item Operasi stack dalam assembly
\end{itemize}

\subsection{Instruksi Stack}
\begin{itemize}
\item Instruksi PUSH (memasukkan data ke stack)
\item Instruksi POP (mengambil data dari stack)
\item PUSH/POP register
\item PUSH/POP memori
\item PUSH/POP flag register
\item PUSH/POP immediate value
\end{itemize}

\subsection{Instruksi Subrutin}
\begin{itemize}
\item Instruksi CALL (memanggil subrutin)
\item Instruksi RET (kembali dari subrutin)
\item CALL near dan CALL far
\item RET dengan parameter
\item Penanganan return address
\end{itemize}

\subsection{Parameter dalam Subrutin}
\begin{itemize}
\item Parameter melalui register
\item Parameter melalui stack
\item Parameter melalui memori
\item Return value dari subrutin
\item Konvensi pemanggilan
\end{itemize}

\section{Praktikum}
\begin{enumerate}
\item Program demonstrasi operasi stack
\item Program subrutin sederhana
\item Program subrutin dengan parameter
\item Program subrutin dengan return value
\item Program modular dengan multiple subrutin
\end{enumerate}

\section{Contoh Kode}
\begin{verbatim}
; Program demonstrasi stack dan subrutin
TITLE Stack dan Subrutin
.MODEL SMALL
.STACK 100h

.DATA
    pesan1 DB 'Hello from main!$'
    pesan2 DB 'Hello from subrutin!$'
    nilai1 DW 10
    nilai2 DW 20
    hasil DW ?

.CODE
START:
    MOV AX, @DATA
    MOV DS, AX
    
    ; Demonstrasi stack
    MOV AX, 1234h
    PUSH AX
    MOV BX, 5678h
    PUSH BX
    
    ; Ambil dari stack (urutan terbalik)
    POP CX  ; CX = 5678h
    POP DX  ; DX = 1234h
    
    ; Panggil subrutin
    CALL tampilkan_pesan
    
    ; Subrutin dengan parameter
    PUSH nilai1
    PUSH nilai2
    CALL tambahkan
    ADD SP, 4  ; Bersihkan parameter dari stack
    
    MOV AH, 4Ch
    INT 21h

; Subrutin tanpa parameter
tampilkan_pesan PROC
    PUSH AX
    PUSH DX
    
    MOV AH, 09h
    MOV DX, OFFSET pesan2
    INT 21h
    
    POP DX
    POP AX
    RET
tampilkan_pesan ENDP

; Subrutin dengan parameter
tambahkan PROC
    PUSH BP
    MOV BP, SP
    
    PUSH AX
    PUSH BX
    
    MOV AX, [BP+6]  ; Parameter kedua
    MOV BX, [BP+4]  ; Parameter pertama
    ADD AX, BX
    MOV hasil, AX
    
    POP BX
    POP AX
    POP BP
    RET
tambahkan ENDP

END START
\end{verbatim}

\section{Latihan}
\begin{enumerate}
\item Buat program yang menggunakan stack untuk menyimpan dan mengambil data
\item Buat subrutin yang menghitung kuadrat dari sebuah bilangan
\item Buat subrutin yang menukar nilai dua variabel
\item Buat program yang menggunakan multiple subrutin
\end{enumerate}

\section{Tugas}
\begin{itemize}
\item Implementasikan subrutin untuk operasi aritmatika dasar
\item Buat program kalkulator modular menggunakan subrutin
\item Implementasikan subrutin untuk manipulasi string
\item Dokumentasikan penggunaan stack dan subrutin
\end{itemize}

\section{Referensi}
\begin{itemize}
\item Hyde, Randall. \textit{The Art of Assembly Language}, 2nd ed., No Starch Press, 2010.
\item Susanto. \textit{Belajar Pemrograman Bahasa Assembly}, Elex Media Komputindo, 1995.
\end{itemize}

