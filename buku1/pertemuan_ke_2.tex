\chapter{Arsitektur Mikroprosesor Intel 8086: Register, Segmentasi Memori, Mode Pengalamatan; Struktur File ASM}

\section{Tujuan Pembelajaran}
Setelah mengikuti pertemuan ini, mahasiswa diharapkan dapat:
\begin{itemize}
\item Memahami arsitektur mikroprosesor Intel 8086
\item Mengetahui fungsi dan penggunaan register-register utama
\item Memahami konsep segmentasi memori
\item Mengenal berbagai mode pengalamatan
\item Memahami struktur file ASM
\end{itemize}

\section{Materi Pembelajaran}

\subsection{Arsitektur Mikroprosesor Intel 8086}
\begin{itemize}
\item Sejarah dan karakteristik Intel 8086
\item Arsitektur internal mikroprosesor
\item Bus sistem dan kontrol
\item Mode operasi (real mode)
\end{itemize}

\subsection{Register-Register Utama}
\begin{itemize}
\item Register umum (AX, BX, CX, DX)
\item Register indeks (SI, DI)
\item Register pointer (SP, BP)
\item Register segmen (CS, DS, SS, ES)
\item Register flag (FLAGS)
\item Register instruksi (IP)
\end{itemize}

\subsection{Segmentasi Memori}
\begin{itemize}
\item Konsep segmentasi memori
\item Register segmen dan offset
\item Perhitungan alamat fisik
\item Batasan memori dalam mode real
\end{itemize}

\subsection{Mode Pengalamatan}
\begin{itemize}
\item Pengalamatan register
\item Pengalamatan langsung
\item Pengalamatan memori langsung
\item Pengalamatan register tidak langsung
\item Pengalamatan berbasis indeks
\item Pengalamatan berbasis register
\end{itemize}

\subsection{Struktur File ASM}
\begin{itemize}
\item Format dasar file assembly
\item Direktif assembler
\item Komentar dalam kode
\item Organisasi kode program
\end{itemize}

\section{Contoh Soal dan Latihan}
\begin{enumerate}
\item Jelaskan fungsi register AX, BX, CX, dan DX
\item Hitung alamat fisik jika CS = 2000h dan IP = 0100h
\item Berikan contoh penggunaan mode pengalamatan register
\item Tuliskan struktur dasar file ASM
\end{enumerate}

\section{Tugas}
\begin{itemize}
\item Buat diagram arsitektur Intel 8086
\item Jelaskan perbedaan antara register segmen dan register umum
\item Berikan contoh penggunaan berbagai mode pengalamatan
\item Buat template struktur file ASM untuk program sederhana
\end{itemize}

\section{Referensi}
\begin{itemize}
\item Brey, Barry B. \textit{Mikroprosesor Intel 8086/8088 dsb.}, edisi terjemahan, Penerbit Informatika.
\item Intel Corporation. \textit{Intel 64 and IA-32 Architectures Software Developer's Manual}, Intel.
\end{itemize}

