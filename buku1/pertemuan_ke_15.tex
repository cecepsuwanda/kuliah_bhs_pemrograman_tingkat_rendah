\chapter{Program Residen (TSR): Konsep Terminate-and-Stay-Resident; Contoh Implementasi Sederhana}

\section{Tujuan Pembelajaran}
Setelah mengikuti pertemuan ini, mahasiswa diharapkan dapat:
\begin{itemize}
\item Memahami konsep program TSR (Terminate-and-Stay-Resident)
\item Mengetahui cara kerja program residen
\item Mengimplementasikan TSR sederhana
\item Menggunakan interupsi untuk komunikasi TSR
\item Mampu membuat program background
\end{itemize}

\section{Materi Pembelajaran}

\subsection{Konsep Program TSR}
\begin{itemize}
\item Definisi TSR (Terminate-and-Stay-Resident)
\item Perbedaan program normal dan TSR
\item Keuntungan dan kelemahan TSR
\item Aplikasi program TSR
\item Memory management untuk TSR
\end{itemize}

\subsection{Arsitektur Program TSR}
\begin{itemize}
\item Initialization routine
\item Resident routine
\item Interrupt handler
\item Memory allocation
\item Program termination
\end{itemize}

\subsection{Interupsi untuk TSR}
\begin{itemize}
\item INT 21h fungsi 31h: Terminate and Stay Resident
\item INT 21h fungsi 25h: Set Interrupt Vector
\item INT 21h fungsi 35h: Get Interrupt Vector
\item Custom interrupt untuk komunikasi
\end{itemize}

\subsection{Implementasi TSR}
\begin{itemize}
\item Program structure
\item Memory management
\item Interrupt hooking
\item Communication mechanism
\item Error handling
\end{itemize}

\section{Praktikum}
\begin{enumerate}
\item Program TSR sederhana
\item Program TSR dengan interrupt handler
\item Program TSR dengan komunikasi
\item Program TSR dengan memory management
\item Program TSR dengan error handling
\end{enumerate}

\section{Contoh Kode}
\begin{verbatim}
; Program TSR sederhana
TITLE Program TSR
.MODEL SMALL
.STACK 100h

.DATA
    pesan_install DB 'TSR berhasil diinstall!$'
    pesan_uninstall DB 'TSR berhasil diuninstall!$'
    old_int21h DD ?
    tsr_installed DB 0

.CODE
START:
    MOV AX, @DATA
    MOV DS, AX
    
    ; Cek apakah TSR sudah diinstall
    MOV AH, 0FFh
    INT 21h
    CMP AL, 0AAh
    JE sudah_install
    
    ; Install TSR
    CALL install_tsr
    
    ; Tampilkan pesan
    MOV AH, 09h
    MOV DX, OFFSET pesan_install
    INT 21h
    
    ; Terminate and Stay Resident
    MOV AH, 31h
    MOV AL, 0
    MOV DX, 1000h  ; Ukuran program dalam paragraph
    INT 21h
    
sudah_install:
    ; Uninstall TSR
    CALL uninstall_tsr
    
    ; Tampilkan pesan
    MOV AH, 09h
    MOV DX, OFFSET pesan_uninstall
    INT 21h
    
    MOV AH, 4Ch
    INT 21h

; Prosedur install TSR
install_tsr PROC
    ; Simpan interrupt vector lama
    MOV AH, 35h
    MOV AL, 21h
    INT 21h
    MOV WORD PTR old_int21h, BX
    MOV WORD PTR old_int21h[2], ES
    
    ; Set interrupt vector baru
    MOV AH, 25h
    MOV AL, 21h
    MOV DX, OFFSET new_int21h
    INT 21h
    
    RET
install_tsr ENDP

; Prosedur uninstall TSR
uninstall_tsr PROC
    ; Restore interrupt vector lama
    MOV AH, 25h
    MOV AL, 21h
    MOV DX, WORD PTR old_int21h
    MOV DS, WORD PTR old_int21h[2]
    INT 21h
    
    RET
uninstall_tsr ENDP

; Interrupt handler baru
new_int21h PROC
    CMP AH, 0FFh
    JNE call_old_int
    
    ; Handler untuk komunikasi TSR
    MOV AL, 0AAh  ; Tanda TSR aktif
    IRET
    
call_old_int:
    ; Panggil interrupt handler lama
    JMP DWORD PTR old_int21h
    
new_int21h ENDP

END START
\end{verbatim}

\section{Latihan}
\begin{enumerate}
\item Buat TSR yang menampilkan waktu di layar
\item Buat TSR yang menangani hotkey
\item Buat TSR yang memantau aktivitas keyboard
\item Buat TSR yang melakukan backup otomatis
\end{enumerate}

\section{Tugas}
\begin{itemize}
\item Implementasikan TSR untuk system monitoring
\item Buat TSR untuk keyboard macro
\item Implementasikan TSR untuk file watcher
\item Dokumentasikan teknik memory management untuk TSR
\end{itemize}

\section{Referensi}
\begin{itemize}
\item Hyde, Randall. \textit{The Art of Assembly Language}, 2nd ed., No Starch Press, 2010.
\item Susanto. \textit{Belajar Pemrograman Bahasa Assembly}, Elex Media Komputindo, 1995.
\end{itemize}

