\chapter{Array dan String: Penyimpanan Array, Instruksi String (MOVS, CMPS, SCAS, LODS, STOS)}

\section{Tujuan Pembelajaran}
Setelah mengikuti pertemuan ini, mahasiswa diharapkan dapat:
\begin{itemize}
\item Memahami konsep array dan string dalam assembly
\item Menyimpan dan mengakses data array
\item Menggunakan instruksi string MOVS, CMPS, SCAS, LODS, STOS
\item Mampu memanipulasi string dan array
\item Mengoptimalkan operasi string dengan prefix REP
\end{itemize}

\section{Materi Pembelajaran}

\subsection{Konsep Array dan String}
\begin{itemize}
\item Definisi array dan string
\item Penyimpanan array dalam memori
\item Pengalamatan elemen array
\item String sebagai array karakter
\item Null-terminated string
\end{itemize}

\subsection{Penyimpanan Array}
\begin{itemize}
\item Array satu dimensi
\item Array multi dimensi
\item Pengalamatan array dengan indeks
\item Perhitungan offset elemen array
\item Array dengan ukuran berbeda
\end{itemize}

\subsection{Instruksi String}
\begin{itemize}
\item Instruksi MOVS (Move String)
\item Instruksi CMPS (Compare String)
\item Instruksi SCAS (Scan String)
\item Instruksi LODS (Load String)
\item Instruksi STOS (Store String)
\item Penggunaan prefix REP
\end{itemize}

\subsection{Register untuk Operasi String}
\begin{itemize}
\item SI (Source Index)
\item DI (Destination Index)
\item CX (Counter)
\item AL/AX (Accumulator)
\item Direction Flag (DF)
\end{itemize}

\section{Praktikum}
\begin{enumerate}
\item Program demonstrasi operasi array
\item Program copy string menggunakan MOVS
\item Program perbandingan string menggunakan CMPS
\item Program pencarian karakter menggunakan SCAS
\item Program manipulasi string lengkap
\end{enumerate}

\section{Contoh Kode}
\begin{verbatim}
; Program demonstrasi array dan string
TITLE Array dan String
.MODEL SMALL
.STACK 100h

.DATA
    array DW 10, 20, 30, 40, 50
    string1 DB 'Hello World$'
    string2 DB 20 DUP('$')
    string3 DB 'Assembly$'
    string4 DB 'Assembly$'
    karakter DB 'l'
    panjang EQU $ - string1

.CODE
START:
    MOV AX, @DATA
    MOV DS, AX
    MOV ES, AX
    
    ; Akses elemen array
    MOV BX, 2        ; Indeks elemen ke-2
    MOV AX, array[BX] ; AX = 30
    
    ; Copy string menggunakan MOVS
    LEA SI, string1
    LEA DI, string2
    MOV CX, panjang
    CLD              ; Clear direction flag
    REP MOVSB
    
    ; Perbandingan string menggunakan CMPS
    LEA SI, string3
    LEA DI, string4
    MOV CX, 8
    CLD
    REPE CMPSB
    JE sama
    ; String berbeda
    JMP lanjut
    
sama:
    ; String sama
    
lanjut:
    ; Pencarian karakter menggunakan SCAS
    LEA DI, string1
    MOV AL, karakter
    MOV CX, panjang
    CLD
    REPNE SCASB
    JE ditemukan
    ; Karakter tidak ditemukan
    JMP selesai
    
ditemukan:
    ; Karakter ditemukan di posisi CX
    
selesai:
    ; Load string menggunakan LODS
    LEA SI, string1
    MOV CX, 5
    CLD
    LODSB  ; Load byte ke AL
    
    ; Store string menggunakan STOS
    LEA DI, string2
    MOV AL, 'X'
    MOV CX, 5
    CLD
    REP STOSB
    
    MOV AH, 4Ch
    INT 21h
END START
\end{verbatim}

\section{Latihan}
\begin{enumerate}
\item Buat program yang mengakses elemen array dengan indeks
\item Buat program yang menyalin string dari satu lokasi ke lokasi lain
\item Buat program yang membandingkan dua string
\item Buat program yang mencari karakter dalam string
\end{enumerate}

\section{Tugas}
\begin{itemize}
\item Implementasikan fungsi untuk menghitung panjang string
\item Buat program yang membalik string
\item Implementasikan fungsi untuk menggabungkan dua string
\item Buat program yang mengurutkan array bilangan
\end{itemize}

\section{Referensi}
\begin{itemize}
\item Hyde, Randall. \textit{The Art of Assembly Language}, 2nd ed., No Starch Press, 2010.
\item Partoharsojo, Hartono. \textit{Tuntunan Praktis Pemrograman Assembly}, Penerbit Informatika.
\end{itemize}

