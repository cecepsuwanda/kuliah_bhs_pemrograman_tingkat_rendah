\chapter{Pemrosesan File Dasar: Membuka, Menutup, Membaca, Menulis File (INT 21h fungsi 3Ch/3Fh/40h)}

\section{Tujuan Pembelajaran}
Setelah mengikuti pertemuan ini, mahasiswa diharapkan dapat:
\begin{itemize}
\item Memahami konsep file handling
\item Menggunakan interupsi INT 21h untuk operasi file
\item Membuka dan menutup file
\item Membaca dan menulis data ke file
\item Menangani error dalam operasi file
\end{itemize}

\section{Materi Pembelajaran}

\subsection{Konsep File Handling}
\begin{itemize}
\item Definisi file dan file system
\item File handle dan file descriptor
\item Mode akses file (read, write, append)
\item File attributes dan permissions
\item Error handling dalam operasi file
\end{itemize}

\subsection{Interupsi INT 21h untuk File}
\begin{itemize}
\item Fungsi 3Ch: Create File
\item Fungsi 3Dh: Open File
\item Fungsi 3Eh: Close File
\item Fungsi 3Fh: Read File
\item Fungsi 40h: Write File
\item Fungsi 41h: Delete File
\end{itemize}

\subsection{Operasi File Dasar}
\begin{itemize}
\item Membuat file baru
\item Membuka file yang sudah ada
\item Menutup file
\item Membaca data dari file
\item Menulis data ke file
\item Menghapus file
\end{itemize}

\subsection{File Handle dan Error Handling}
\begin{itemize}
\item File handle management
\item Error codes dan penanganannya
\item Validasi operasi file
\item Resource cleanup
\end{itemize}

\subsection{Buffer dan Data Transfer}
\begin{itemize}
\item File buffer
\item Block transfer
\item Sequential access
\item Data formatting
\end{itemize}

\section{Praktikum}
\begin{enumerate}
\item Program membuat file baru
\item Program menulis data ke file
\item Program membaca data dari file
\item Program menyalin file
\item Program dengan error handling
\end{enumerate}

\section{Contoh Kode}
\begin{verbatim}
; Program demonstrasi operasi file dasar
TITLE Pemrosesan File
.MODEL SMALL
.STACK 100h

.DATA
    nama_file DB 'test.txt', 0
    data_tulis DB 'Hello, World!', 13, 10
    data_baca DB 100 DUP(?)
    file_handle DW ?
    pesan_sukses DB 'File berhasil dibuat!$'
    pesan_error DB 'Error dalam operasi file!$'
    pesan_baca DB 'Data dari file: $'

.CODE
START:
    MOV AX, @DATA
    MOV DS, AX
    
    ; Buat file baru
    MOV AH, 3Ch
    MOV CX, 0        ; File attributes (normal)
    MOV DX, OFFSET nama_file
    INT 21h
    JC error_handler
    MOV file_handle, AX
    
    ; Tampilkan pesan sukses
    MOV AH, 09h
    MOV DX, OFFSET pesan_sukses
    INT 21h
    
    ; Tulis data ke file
    MOV AH, 40h
    MOV BX, file_handle
    MOV CX, 15       ; Jumlah byte yang ditulis
    MOV DX, OFFSET data_tulis
    INT 21h
    JC error_handler
    
    ; Tutup file
    MOV AH, 3Eh
    MOV BX, file_handle
    INT 21h
    JC error_handler
    
    ; Buka file untuk dibaca
    MOV AH, 3Dh
    MOV AL, 0        ; Mode read
    MOV DX, OFFSET nama_file
    INT 21h
    JC error_handler
    MOV file_handle, AX
    
    ; Baca data dari file
    MOV AH, 3Fh
    MOV BX, file_handle
    MOV CX, 100      ; Jumlah byte yang dibaca
    MOV DX, OFFSET data_baca
    INT 21h
    JC error_handler
    
    ; Tampilkan data yang dibaca
    MOV AH, 09h
    MOV DX, OFFSET pesan_baca
    INT 21h
    
    ; Tambahkan null terminator
    MOV BX, OFFSET data_baca
    ADD BX, AX
    MOV BYTE PTR [BX], '$'
    
    MOV AH, 09h
    MOV DX, OFFSET data_baca
    INT 21h
    
    ; Tutup file
    MOV AH, 3Eh
    MOV BX, file_handle
    INT 21h
    
    JMP selesai
    
error_handler:
    MOV AH, 09h
    MOV DX, OFFSET pesan_error
    INT 21h
    
selesai:
    MOV AH, 4Ch
    INT 21h
END START
\end{verbatim}

\section{Latihan}
\begin{enumerate}
\item Buat program yang menulis teks ke file
\item Buat program yang membaca dan menampilkan isi file
\item Buat program yang menyalin isi file ke file lain
\item Buat program yang menghitung jumlah karakter dalam file
\end{enumerate}

\section{Tugas}
\begin{itemize}
\item Implementasikan program text editor sederhana
\item Buat program yang menyimpan dan memuat data konfigurasi
\item Implementasikan program backup file
\item Dokumentasikan error codes dan penanganannya
\end{itemize}

\section{Referensi}
\begin{itemize}
\item Hyde, Randall. \textit{The Art of Assembly Language}, 2nd ed., No Starch Press, 2010.
\item Susanto. \textit{Belajar Pemrograman Bahasa Assembly}, Elex Media Komputindo, 1995.
\end{itemize}

