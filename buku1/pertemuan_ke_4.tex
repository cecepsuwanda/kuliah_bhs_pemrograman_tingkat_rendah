\chapter{Instruksi Dasar: Perpindahan Data (MOV), Operasi Aritmatika (ADD, SUB, MUL, DIV)}

\section{Tujuan Pembelajaran}
Setelah mengikuti pertemuan ini, mahasiswa diharapkan dapat:
\begin{itemize}
\item Menggunakan instruksi MOV untuk perpindahan data
\item Melakukan operasi aritmatika dengan ADD, SUB, MUL, DIV
\item Memahami pengaruh operasi terhadap flag register
\item Mampu menulis program dengan operasi aritmatika dasar
\end{itemize}

\section{Materi Pembelajaran}

\subsection{Instruksi Perpindahan Data (MOV)}
\begin{itemize}
\item Sintaks instruksi MOV
\item Aturan penggunaan MOV
\item Perpindahan data antar register
\item Perpindahan data dari/ke memori
\item Perpindahan data dengan konstanta
\item Contoh penggunaan MOV
\end{itemize}

\subsection{Operasi Aritmatika}
\begin{itemize}
\item Instruksi ADD (penjumlahan)
\item Instruksi SUB (pengurangan)
\item Instruksi MUL (perkalian)
\item Instruksi DIV (pembagian)
\item Pengaruh terhadap flag register
\item Penanganan overflow dan underflow
\end{itemize}

\subsection{Flag Register dan Operasi Aritmatika}
\begin{itemize}
\item Flag Carry (CF)
\item Flag Zero (ZF)
\item Flag Sign (SF)
\item Flag Overflow (OF)
\item Flag Parity (PF)
\item Flag Auxiliary Carry (AF)
\end{itemize}

\section{Praktikum}
\begin{enumerate}
\item Program perpindahan data antar register
\item Program penjumlahan dua bilangan
\item Program pengurangan dengan penanganan flag
\item Program perkalian dan pembagian
\item Program kalkulator sederhana
\end{enumerate}

\section{Contoh Kode}
\begin{verbatim}
; Program operasi aritmatika dasar
TITLE Operasi Aritmatika
.MODEL SMALL
.STACK 100h

.DATA
    bil1 DW 15
    bil2 DW 7
    hasil DW ?

.CODE
START:
    MOV AX, @DATA
    MOV DS, AX
    
    ; Penjumlahan
    MOV AX, bil1
    ADD AX, bil2
    MOV hasil, AX
    
    ; Pengurangan
    MOV AX, bil1
    SUB AX, bil2
    
    ; Perkalian
    MOV AX, bil1
    MUL bil2
    
    ; Pembagian
    MOV AX, bil1
    DIV bil2
    
    MOV AH, 4Ch
    INT 21h
END START
\end{verbatim}

\section{Latihan}
\begin{enumerate}
\item Buat program yang menghitung (A + B) - (C + D)
\item Buat program yang menghitung (X * Y) / Z
\item Jelaskan pengaruh operasi aritmatika terhadap flag register
\item Buat program yang menampilkan status flag setelah operasi
\end{enumerate}

\section{Tugas}
\begin{itemize}
\item Buat kalkulator sederhana dengan operasi +, -, *, /
\item Implementasikan penanganan error untuk pembagian dengan nol
\item Buat program yang menghitung rata-rata dari 5 bilangan
\item Dokumentasikan penggunaan flag register dalam operasi aritmatika
\end{itemize}

\section{Referensi}
\begin{itemize}
\item Hyde, Randall. \textit{The Art of Assembly Language}, 2nd ed., No Starch Press, 2010.
\item Susanto. \textit{Belajar Pemrograman Bahasa Assembly}, Elex Media Komputindo, 1995.
\end{itemize}

