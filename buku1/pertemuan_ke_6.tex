\chapter{Input dari Keyboard (INT 16h); Penanganan Keyboard Buffer}

\section{Tujuan Pembelajaran}
Setelah mengikuti pertemuan ini, mahasiswa diharapkan dapat:
\begin{itemize}
\item Menggunakan interupsi INT 16h untuk input keyboard
\item Memahami fungsi-fungsi INT 16h
\item Menangani keyboard buffer
\item Mampu membuat program interaktif dengan input keyboard
\end{itemize}

\section{Materi Pembelajaran}

\subsection{Interupsi BIOS INT 16h}
\begin{itemize}
\item Fungsi 00h: Read Key (Blocking)
\item Fungsi 01h: Check for Key (Non-blocking)
\item Fungsi 02h: Get Keyboard Flags
\item Parameter dan return values
\item Perbedaan blocking dan non-blocking
\end{itemize}

\subsection{Keyboard Buffer}
\begin{itemize}
\item Konsep keyboard buffer
\item Cara kerja buffer keyboard
\item Penanganan buffer penuh
\item Flushing keyboard buffer
\item Status keyboard buffer
\end{itemize}

\subsection{Input Karakter dan String}
\begin{itemize}
\item Input karakter tunggal
\item Input string dengan echo
\item Input string tanpa echo (password)
\item Validasi input
\item Penanganan karakter khusus
\end{itemize}

\subsection{Karakter Khusus}
\begin{itemize}
\item Scan code dan ASCII code
\item Function keys (F1-F12)
\item Arrow keys
\item Control keys (Ctrl, Alt, Shift)
\item Special keys (Enter, Escape, Backspace)
\end{itemize}

\section{Praktikum}
\begin{enumerate}
\item Program input karakter tunggal
\item Program input string dengan echo
\item Program input password (tanpa echo)
\item Program penanganan keyboard buffer
\item Program interaktif dengan menu
\end{enumerate}

\section{Contoh Kode}
\begin{verbatim}
; Program input keyboard dan penanganan buffer
TITLE Input Keyboard
.MODEL SMALL
.STACK 100h

.DATA
    prompt DB 'Masukkan nama: $'
    nama DB 50 DUP('$')
    pesan DB 'Halo, $'
    buffer DB 50 DUP(?)

.CODE
START:
    MOV AX, @DATA
    MOV DS, AX
    
    ; Tampilkan prompt
    MOV AH, 09h
    MOV DX, OFFSET prompt
    INT 21h
    
    ; Input string dengan echo
    MOV AH, 0Ah
    MOV DX, OFFSET buffer
    INT 21h
    
    ; Input karakter tunggal (blocking)
    MOV AH, 00h
    INT 16h
    ; AL = ASCII code, AH = Scan code
    
    ; Check for key (non-blocking)
    MOV AH, 01h
    INT 16h
    JZ no_key
    ; Key available, read it
    MOV AH, 00h
    INT 16h
    
no_key:
    ; Flush keyboard buffer
    MOV AH, 0Ch
    MOV AL, 00h
    INT 21h
    
    MOV AH, 4Ch
    INT 21h
END START
\end{verbatim}

\section{Latihan}
\begin{enumerate}
\item Buat program yang menunggu input dari keyboard
\item Buat program yang menampilkan scan code dari key yang ditekan
\item Buat program input password dengan masking karakter
\item Buat program yang menangani keyboard buffer penuh
\end{enumerate}

\section{Tugas}
\begin{itemize}
\item Buat program login dengan validasi username dan password
\item Implementasikan menu interaktif dengan navigasi menggunakan arrow keys
\item Buat program yang menangani input dengan timeout
\item Dokumentasikan penggunaan interupsi INT 16h
\end{itemize}

\section{Referensi}
\begin{itemize}
\item Hyde, Randall. \textit{The Art of Assembly Language}, 2nd ed., No Starch Press, 2010.
\item Susanto. \textit{Belajar Pemrograman Bahasa Assembly}, Elex Media Komputindo, 1995.
\end{itemize}

