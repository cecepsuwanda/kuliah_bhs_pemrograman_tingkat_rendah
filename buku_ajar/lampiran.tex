\chapter*{Lampiran}
\addcontentsline{toc}{chapter}{Lampiran}

% ============================================================
% Lampiran A: Rubrik Penilaian
% ============================================================
\section*{Lampiran A: Rubrik Penilaian Tugas Praktik}

\begin{table}[htbp]
\centering
\begin{tabular}{|>{\raggedright\arraybackslash}p{3cm}|>{\raggedright\arraybackslash}p{3.5cm}|>{\raggedright\arraybackslash}p{3.5cm}|>{\raggedright\arraybackslash}p{3.5cm}|}
\hline
\textbf{Kriteria} & \textbf{Sangat Baik} & \textbf{Baik} & \textbf{Perlu Perbaikan} \\
\hline
Correctness & Program berfungsi sempurna sesuai spesifikasi & Berfungsi dengan minor bug & Banyak bug atau tidak berjalan \\
\hline
Modularity & Prosedur terstruktur, code reuse optimal & Struktur prosedur cukup baik & Struktur kurang modular \\
\hline
Documentation & Komentar lengkap, struktur jelas & Dokumentasi adequate & Minimal atau tidak ada dokumentasi \\
\hline
Optimization & Penggunaan register efisien, strength reduction & Optimasi diterapkan & Tanpa optimasi \\
\hline
Testing & Pengujian lengkap dengan TASM debugger & Pengujian sebagian & Tidak ada pengujian \\
\hline
\end{tabular}
\caption{Rubrik Penilaian Tugas Praktik}
\end{table}

\textbf{Pemetaan Kriteria Rubrik ke CPMK:}

\begin{table}[htbp]
\centering
\begin{tabular}{|>{\raggedright\arraybackslash}p{3cm}|>{\raggedright\arraybackslash}p{4.5cm}|}
\hline
\textbf{Kriteria Rubrik} & \textbf{CPMK Terkait} \\
\hline
Correctness & CPMK-3: Implementasi program sesuai spesifikasi dan standar Intel 8086 \\
\hline
Modularity & CPMK-2: Merancang program assembly dengan struktur modular \\
\hline
Documentation & CPMK-2, CPMK-4: Dokumentasi mendukung desain dan evaluasi kode \\
\hline
Optimization & CPMK-4: Menganalisis dan mengoptimasi kode assembly \\
\hline
Testing & CPMK-3, CPMK-4: Pengujian memvalidasi implementasi dan kualitas \\
\hline
\end{tabular}
\caption{Pemetaan Kriteria Rubrik ke CPMK}
\end{table}

% ============================================================
% Lampiran B: Contoh Tugas, Portofolio, dan Rubrik Refleksi
% ============================================================
\section*{Lampiran B: Contoh Tugas, Portofolio, dan Rubrik Refleksi}

\subsection*{B.1 Contoh Tugas Praktik Assembly}

\textbf{Tugas}: Buat program assembly yang menerima input dua bilangan (1 digit masing-masing), menjumlahkannya, dan menampilkan hasil ke layar. Gunakan INT 21h untuk input/output.

\textbf{Spesifikasi:}
\begin{itemize}
  \item Gunakan interrupt DOS (INT 21h) dengan fungsi 01h (input) dan 02h/09h (output)
  \item Konversi ASCII ke nilai numerik sebelum operasi
  \item Program berformat .COM (ORG 100h)
  \item Cantumkan komentar pada setiap bagian kode
\end{itemize}

\textbf{Indikator Penilaian}: Correctness, Modularity, Documentation (lihat Lampiran A).

\subsection*{B.2 Struktur Portofolio Mahasiswa}

Portofolio mahasiswa untuk mata kuliah ini memuat kumpulan dokumen berikut:

\begin{enumerate}
  \item \textbf{Biodata dan Identitas}: Nama, NIM, kelas, foto
  \item \textbf{Daftar Tugas}: Kumpulan laporan tugas praktik (Bab 2--13) dengan cuplikan kode dan hasil eksekusi
  \item \textbf{Proyek Akhir}: Laporan proyek assembly yang mengintegrasikan konsep (direktif, prosedur, interupsi)
  \item \textbf{Refleksi Pembelajaran}: Jurnal refleksi per bab atau per milestone: apa yang dipelajari, tantangan, dan rencana perbaikan
  \item \textbf{Checklist Kompetensi}: Salinan checklist yang telah diisi untuk self-assessment
\end{enumerate}

\subsection*{B.3 Template Laporan Tugas}

\begin{enumerate}
  \item Judul dan Identitas Mahasiswa
  \item Deskripsi Masalah
  \item Struktur Program dan Flowchart/Algoritma
  \item Implementasi (cuplikan kode assembly penting)
  \item Hasil Pengujian dengan TASM
  \item Refleksi dan Kesimpulan
\end{enumerate}

\subsection*{B.4 Rubrik Penilaian Refleksi}

\begin{table}[htbp]
\centering
\begin{tabular}{|>{\raggedright\arraybackslash}p{2.8cm}|>{\raggedright\arraybackslash}p{3.2cm}|>{\raggedright\arraybackslash}p{3.2cm}|>{\raggedright\arraybackslash}p{3.2cm}|}
\hline
\textbf{Kriteria} & \textbf{Sangat Baik} & \textbf{Baik} & \textbf{Perlu Perbaikan} \\
\hline
Kedalaman Analisis & Merefleksi konsep, kesalahan, dan insight dengan rinci & Refleksi memadai dengan contoh & Hanya ringkasan dangkal \\
\hline
Rencana Tindak Lanjut & Action plan jelas dan terukur & Ada rencana perbaikan & Tidak ada rencana konkret \\
\hline
Keterkaitan CPMK & Menghubungkan refleksi ke Sub-CPMK yang relevan & Sebagian terhubung ke kompetensi & Tidak mengacu kompetensi \\
\hline
Kejujuran dan Kritik Diri & Mengakui kelemahan dan strategi perbaikan & Cukup objektif & Mengabaikan area improvement \\
\hline
\end{tabular}
\caption{Rubrik Penilaian Refleksi Pembelajaran}
\end{table}

% ============================================================
% Lampiran C: Glosarium
% ============================================================
\section*{Lampiran C: Glosarium Istilah Assembly}
\begin{itemize}
  \item \textbf{Register}: Penyimpan sementara dalam CPU untuk operasi data
  \item \textbf{Segment}: Blok memori 64KB dalam model segmentasi Intel 8086
  \item \textbf{Mnemonic}: Kode instruksi yang mudah dibaca (mis. MOV, ADD)
  \item \textbf{Directive}: Instruksi untuk assembler (mis. DB, DW, SEGMENT)
  \item \textbf{Stack Frame}: Area stack yang digunakan prosedur untuk parameter dan variabel lokal
  \item \textbf{Interrupt}: Sinyal yang menghentikan eksekusi untuk layanan sistem
  \item \textbf{Addressing Mode}: Cara menentukan operan dalam instruksi
  \item \textbf{Label}: Penanda alamat dalam kode untuk jump dan referensi
\end{itemize}

% ============================================================
% Lampiran D: Referensi Tambahan
% ============================================================
\section*{Lampiran D: Referensi Tambahan}
\begin{itemize}
  \item Intel 8086 Family User's Manual: \url{https://edge.edx.org/c4x/BITSPilani/EEE231/asset/8086_family_Users_Manual_1_.pdf}
  \item TASM (Turbo Assembler) Documentation
  \item Assembly Language Tutorial: \url{https://yassinebridi.github.io/asm-docs/}
  \item x86 Assembly Guide: \url{https://www.cs.virginia.edu/~evans/cs216/guides/x86.html}
\end{itemize}
