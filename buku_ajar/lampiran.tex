\chapter*{Lampiran}
\addcontentsline{toc}{chapter}{Lampiran}

% ============================================================
% Lampiran A: Rubrik Penilaian
% ============================================================
\section*{Lampiran A: Rubrik Penilaian Tugas Praktik}

\begin{table}[htbp]
\centering
\small
\begin{tabular}{|>{\raggedright\arraybackslash}p{3cm}|>{\raggedright\arraybackslash}p{3.5cm}|>{\raggedright\arraybackslash}p{3.5cm}|>{\raggedright\arraybackslash}p{3.5cm}|}
\hline
\textbf{Kriteria} & \textbf{Sangat Baik} & \textbf{Baik} & \textbf{Perlu Perbaikan} \\
\hline
Correctness & Program berfungsi sempurna sesuai spesifikasi & Berfungsi dengan minor bug & Banyak bug atau tidak berjalan \\
\hline
Modularity & Prosedur terstruktur, code reuse optimal & Struktur prosedur cukup baik & Struktur kurang modular \\
\hline
Documentation & Komentar lengkap, struktur jelas & Dokumentasi adequate & Minimal atau tidak ada dokumentasi \\
\hline
Optimization & Penggunaan register efisien, strength reduction & Optimasi diterapkan & Tanpa optimasi \\
\hline
Testing & Pengujian lengkap dengan TASM debugger & Pengujian sebagian & Tidak ada pengujian \\
\hline
\end{tabular}
\caption{Rubrik Penilaian Tugas Praktik}
\end{table}

% ============================================================
% Lampiran B: Contoh Template Laporan
% ============================================================
\section*{Lampiran B: Contoh Template Laporan Tugas}
\begin{enumerate}
  \item Judul dan Identitas Mahasiswa
  \item Deskripsi Masalah
  \item Struktur Program dan Flowchart/Algoritma
  \item Implementasi (cuplikan kode assembly penting)
  \item Hasil Pengujian dengan TASM
  \item Refleksi dan Kesimpulan
\end{enumerate}

% ============================================================
% Lampiran C: Glosarium
% ============================================================
\section*{Lampiran C: Glosarium Istilah Assembly}
\begin{itemize}
  \item \textbf{Register}: Penyimpan sementara dalam CPU untuk operasi data
  \item \textbf{Segment}: Blok memori 64KB dalam model segmentasi Intel 8086
  \item \textbf{Mnemonic}: Kode instruksi yang mudah dibaca (mis. MOV, ADD)
  \item \textbf{Directive}: Instruksi untuk assembler (mis. DB, DW, SEGMENT)
  \item \textbf{Stack Frame}: Area stack yang digunakan prosedur untuk parameter dan variabel lokal
  \item \textbf{Interrupt}: Sinyal yang menghentikan eksekusi untuk layanan sistem
  \item \textbf{Addressing Mode}: Cara menentukan operan dalam instruksi
  \item \textbf{Label}: Penanda alamat dalam kode untuk jump dan referensi
\end{itemize}

% ============================================================
% Lampiran D: Referensi Tambahan
% ============================================================
\section*{Lampiran D: Referensi Tambahan}
\begin{itemize}
  \item Intel 8086 Family User's Manual: \url{https://edge.edx.org/c4x/BITSPilani/EEE231/asset/8086_family_Users_Manual_1_.pdf}
  \item TASM (Turbo Assembler) Documentation
  \item Assembly Language Tutorial: \url{https://yassinebridi.github.io/asm-docs/}
  \item x86 Assembly Guide: \url{https://www.cs.virginia.edu/~evans/cs216/guides/x86.html}
\end{itemize}
