\frontmatter

% ============================================================
% Halaman Sampul
% ============================================================
\begin{titlepage}
  \centering
  \vspace*{1cm}
  {\LARGE\bfseries Pemrograman Assembly Intel 8086\par}
  \vspace{0.5cm}
  {\large Buku Ajar Berbasis Outcome-Based Education (OBE)\par}
  \vspace{1cm}
  {\large Mata Kuliah: Pemrograman Bahasa Tingkat Rendah\par}
  {\large Kode: TIF-301\par}
  \vspace{2cm}
  {\large Program Studi Teknik Informatika\par}
  {\large Fakultas Teknik\par}
  {\large Universitas Lorem Ipsum\par}
  \vfill
  {\large 2026\par}
\end{titlepage}

\cleardoublepage

% ============================================================
% Kata Pengantar
% ============================================================
\chapter*{Kata Pengantar}
\addcontentsline{toc}{chapter}{Kata Pengantar}

Puji syukur kehadirat Tuhan Yang Maha Esa atas terselesaikannya buku ajar \textit{Pemrograman Assembly Intel 8086} ini. Buku ini disusun dengan pendekatan \textbf{Outcome-Based Education (OBE)}, yang berfokus pada pencapaian kompetensi terukur sesuai dengan Capaian Pembelajaran Lulusan (CPL) dan Capaian Pembelajaran Mata Kuliah (CPMK).

Pemrograman Assembly Intel 8086 merupakan fondasi fundamental untuk memahami arsitektur komputer dan sistem operasi. Paradigma ini tidak hanya mengajarkan cara menulis kode tingkat rendah, tetapi juga cara berpikir dalam merancang solusi perangkat lunak yang efisien, teroptimasi, dan dekat dengan hardware.

Buku ini dirancang untuk mendukung pembelajaran mahasiswa semester 3 dengan pendekatan student-centered learning. Setiap bab dilengkapi dengan:
\begin{itemize}
  \item Sub-CPMK yang jelas dan terukur
  \item Materi pokok dengan contoh kode assembly yang lengkap
  \item Aktivitas pembelajaran yang mendorong eksplorasi mandiri
  \item Latihan dan refleksi untuk penguatan pemahaman
  \item Asesmen untuk mengukur pencapaian kompetensi
  \item Checklist kompetensi untuk self-assessment
\end{itemize}

Kami berharap buku ini dapat menjadi panduan yang efektif dalam perjalanan pembelajaran Anda menguasai Pemrograman Assembly Intel 8086 dengan Turbo Assembler (TASM).

\vspace{1cm}
\begin{flushright}
Penyusun\\
Dr. Lorem Ipsum, M.Kom.
\end{flushright}

\cleardoublepage

% ============================================================
% Cara Menggunakan Buku Ini
% ============================================================
\chapter*{Cara Menggunakan Buku Ini}
\addcontentsline{toc}{chapter}{Cara Menggunakan Buku Ini}

Buku ajar ini dirancang dengan pendekatan OBE untuk memaksimalkan pencapaian pembelajaran Anda. Berikut panduan penggunaan buku ini:

\section*{Struktur Buku}

\textbf{Bab I: Pendahuluan dan Orientasi}\\
Memperkenalkan tujuan buku, keterkaitan dengan RPS, dan konteks kurikulum OBE.

\textbf{Bab II: Landasan Teori}\\
Menyajikan fondasi teoretis Assembly yang menjadi basis pembelajaran seluruh bab berikutnya.

\textbf{Bab III-XIII: Unit Materi Inti}\\
Setiap bab mencakup satu topik utama Assembly Intel 8086 dengan struktur lengkap: Sub-CPMK, materi, aktivitas, latihan, asesmen, dan checklist.

\textbf{Bab XIV: Evaluasi dan Integrasi}\\
Berisi asesmen komprehensif dan panduan refleksi untuk mengukur pencapaian kompetensi secara menyeluruh.

\textbf{Lampiran}\\
Menyediakan rubrik penilaian, glosarium istilah Assembly, dan referensi tambahan.

\section*{Komponen dalam Setiap Bab}

\begin{enumerate}
  \item \textbf{Sub-CPMK}: Baca dengan seksama untuk memahami kompetensi yang harus dicapai
  \item \textbf{Materi Pokok}: Pelajari dengan cermat, jalankan semua contoh kode
  \item \textbf{Aktivitas Pembelajaran}: Lakukan secara mandiri atau berkelompok
  \item \textbf{Latihan}: Kerjakan untuk menguji pemahaman Anda
  \item \textbf{Asesmen}: Gunakan untuk mengukur pencapaian Sub-CPMK
  \item \textbf{Checklist}: Centang setelah yakin menguasai setiap indikator
\end{enumerate}

\section*{Tips Belajar Efektif}

\begin{itemize}
  \item Jangan hanya membaca, praktikkan semua contoh kode
  \item Gunakan TASM, DOSBox, atau emulator 8086 untuk eksperimen
  \item Kerjakan latihan sebelum melihat solusi
  \item Diskusikan konsep yang sulit dengan teman atau dosen
  \item Manfaatkan checklist untuk self-assessment berkala
  \item Kerjakan proyek mini untuk mengintegrasikan konsep yang dipelajari
\end{itemize}

\cleardoublepage

% ============================================================
% Identitas Mata Kuliah
% ============================================================
\chapter*{Identitas Mata Kuliah}
\addcontentsline{toc}{chapter}{Identitas Mata Kuliah}

\begin{tabular}{ll}
  Nama Program Studi & : Teknik Informatika \\
  Nama Mata Kuliah & : Pemrograman Bahasa Tingkat Rendah \\
  Kode Mata Kuliah & : TIF-201 \\
  Semester & : 2 (Dua) \\
  SKS / Bobot Kredit & : 3 SKS (1 Teori, 2 Praktikum) \\
  Dosen Pengampu & : Dr. Lorem Ipsum, M.Kom. \\
  Tanggal Penyusunan & : 31 Januari 2026 \\
\end{tabular}

\vspace{1cm}

\section*{Capaian Pembelajaran Lulusan (CPL)}

CPL yang dibebankan pada mata kuliah ini mencakup kompetensi lulusan dalam aspek pengetahuan, keterampilan, dan sikap:

\begin{enumerate}
  \item \textbf{CPL-1 (Pengetahuan):} Menguasai konsep teoretis bidang pengetahuan tertentu secara umum dan konsep teoretis bagian khusus dalam bidang pengetahuan tersebut secara mendalam, serta mampu memformulasikan penyelesaian masalah prosedural.
  
  \item \textbf{CPL-2 (Keterampilan Umum):} Mampu menerapkan pemikiran logis, kritis, sistematis, dan inovatif dalam konteks pengembangan atau implementasi ilmu pengetahuan dan teknologi yang memperhatikan dan menerapkan nilai humaniora.
  
  \item \textbf{CPL-3 (Keterampilan Khusus):} Mampu merancang, mengimplementasikan, dan mengevaluasi program assembly dengan mempertimbangkan efisiensi dan standar kualitas perangkat lunak tingkat rendah.
  
  \item \textbf{CPL-4 (Sikap):} Menunjukkan sikap bertanggung jawab atas pekerjaan di bidang keahliannya secara mandiri dan mampu bekerja sama dalam tim.
\end{enumerate}

\section*{Capaian Pembelajaran Mata Kuliah (CPMK)}

Kemampuan atau kompetensi spesifik yang diharapkan mahasiswa kuasai setelah menyelesaikan mata kuliah:

\begin{enumerate}
  \item \textbf{CPMK-1:} Mahasiswa mampu memahami dan menjelaskan konsep register, segmentasi memori, dan instruksi dasar Intel 8086.
  
  \item \textbf{CPMK-2:} Mahasiswa mampu merancang program assembly dengan struktur modular menggunakan prosedur dan segmentasi.
  
  \item \textbf{CPMK-3:} Mahasiswa mampu mengimplementasikan program assembly dengan TASM sesuai spesifikasi dan standar Intel 8086.
  
  \item \textbf{CPMK-4:} Mahasiswa mampu menganalisis dan mengoptimasi kode assembly berdasarkan teknik optimasi dan best practices.
\end{enumerate}

\section*{Matriks Keterkaitan CPL dan CPMK}

\begin{table}[htbp]
\centering
\begin{tabular}{|c|c|c|p{5cm}|}
  \hline
  \textbf{CPL} & \textbf{CPMK} & \textbf{Kontribusi} & \textbf{Keterangan} \\
  \hline
  CPL-1 & CPMK-1 & Tinggi & Penguasaan konsep teoretis Assembly \\
  \hline
  CPL-1 & CPMK-2 & Tinggi & Kemampuan merancang program modular \\
  \hline
  CPL-2 & CPMK-3 & Tinggi & Penerapan pemikiran sistematis dalam implementasi \\
  \hline
  CPL-2 & CPMK-4 & Sedang & Pemikiran kritis dalam evaluasi kode \\
  \hline
  CPL-3 & CPMK-2 & Tinggi & Perancangan program assembly \\
  \hline
  CPL-3 & CPMK-3 & Tinggi & Implementasi dengan TASM \\
  \hline
  CPL-3 & CPMK-4 & Tinggi & Optimasi dan evaluasi kode assembly \\
  \hline
  CPL-4 & CPMK-3 & Sedang & Tanggung jawab dalam implementasi \\
  \hline
  CPL-4 & CPMK-4 & Sedang & Kerja sama dalam code review \\
  \hline
\end{tabular}
\caption{Matriks Keterkaitan CPL dan CPMK}
\end{table}

\cleardoublepage

% ============================================================
% Daftar Isi
% ============================================================
\phantomsection
\addcontentsline{toc}{chapter}{Daftar Isi}
\tableofcontents
\cleardoublepage

\clearpage
\phantomsection
\addcontentsline{toc}{chapter}{Daftar Gambar}
\listoffigures
\cleardoublepage

\phantomsection
\addcontentsline{toc}{chapter}{Daftar Tabel}
\listoftables
\cleardoublepage

\phantomsection
\addcontentsline{toc}{chapter}{Daftar Kode Program}
\lstlistoflistings
\cleardoublepage

