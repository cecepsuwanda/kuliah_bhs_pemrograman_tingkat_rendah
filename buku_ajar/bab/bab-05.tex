\documentclass[../main.tex]{subfiles}
\ifSubfilesClassLoaded{\setcounter{chapter}{4}}{}
\begin{document}

\chapter{Operasi Aritmatika dan Logika}

\begin{subcpmk}
  \item Sub-CPMK 3.1: Mengimplementasikan operasi aritmatika (ADD, SUB, MUL, DIV)
  \item Sub-CPMK 3.2: Menerapkan operasi logika (AND, OR, XOR, NOT, SHIFT)
  \item Sub-CPMK 2.1: Menulis program assembly sederhana dengan TASM syntax
\end{subcpmk}

% ============================================================
% MATERI POKOK
% ============================================================
% ============================================================
% MATERI POKOK
% ============================================================
\section{Operasi Aritmatika}

\subsection{Penjumlahan dan Pengurangan}
Instruksi ADD dan SUB adalah operasi aritmatika dasar dalam assembly 8086.

\begin{verbatim}
; Addition examples
ADD AX, BX      ; AX = AX + BX, flags terpengaruh
ADD CX, 10       ; CX = CX + 10 (immediate)
ADD [SI], AL     ; memori[SI] = memori[SI] + AL

; Subtraction examples  
SUB DX, CX      ; DX = DX - CX, flags terpengaruh
SUB AX, 5       ; AX = AX - 5 (immediate)
SUB [DI], BL     ; memori[DI] = memori[DI] - BL
\end{verbatim}

\textbf{Flags yang terpengaruh:}
\begin{itemize}
  \item \textbf{ZF (Zero Flag)}: Set jika hasil = 0
  \item \textbf{SF (Sign Flag)}: Set jika hasil negatif (MSB = 1)
  \item \textbf{CF (Carry Flag)}: Set jika ada borrow (SUB) atau carry (ADD)
  \item \textbf{OF (Overflow Flag)}: Set jika overflow signed
  \item \textbf{AF (Auxiliary Flag)}: Set jika carry dari bit 3 ke bit 4
\end{itemize}

\subsection{Perkalian dan Pembagian}
Instruksi MUL dan DIV untuk operasi perkalian dan pembagian.

\begin{verbatim}
; Unsigned multiplication
MUL BL          ; AX = AL * BL (8-bit x 8-bit = 16-bit)
MUL CX          ; DX:AX = AX * CX (16-bit x 16-bit = 32-bit)

; Signed multiplication  
IMUL BX         ; DX:AX = AX * BX (signed)

; Unsigned division
DIV BL          ; AL = AX / BL, AH = AX % BL
DIV CX          ; AX = DX:AX / CX, DX = DX:AX % CX

; Signed division
IDIV BX         ; AX = DX:AX / BX (signed), DX = DX:AX % BX (signed)
\end{verbatim}

\textbf{Catatan penting:}
\begin{itemize}
  \item MUL selalu menggunakan A register (AL/AX) sebagai operand implisit
  \item Hasil 8-bit: AL, 16-bit: AX, 32-bit: DX:AX
  \item DIV error (divide by zero) trigger interrupt 0
  \item Overflow pada MUL/DIV set CF dan OF
\end{itemize}

\subsection{Increment dan Decrement}
Instruksi INC dan DEC untuk operasi +1 dan -1.

\begin{verbatim}
; Increment
INC AX          ; AX = AX + 1
INC [SI]        ; memori[SI] = memori[SI] + 1
INC BL          ; BL = BL + 1

; Decrement  
DEC CX          ; CX = CX - 1
DEC [DI]        ; memori[DI] = memori[DI] - 1
DEC BH          ; BH = BH - 1
\end{verbatim}

\textbf{Pengaruh flags:}
\begin{itemize}
  \item INC/DEC mempengaruhi OF, SF, ZF, AF
  \item Tidak mempengaruhi CF (kecuali DEC word menjadi 0)
\end{itemize}

\subsection{Negasi dan Komplemen}
Instruksi NEG dan NOT untuk operasi negasi.

\begin{verbatim}
; Two's complement (negasi)
NEG AX          ; AX = -AX (two's complement)
NEG BL          ; BL = -BL

; One's complement (komplemen bitwise)
NOT AX          ; AX = ~AX (bitwise NOT)
NOT BL          ; BL = ~BL
\end{verbatim}

\textbf{Perbedaan NEG vs NOT:}
\begin{itemize}
  \item NEG: arithmetic negation (-n)
  \item NOT: bitwise complement (~n)
  \item NEG = NOT + 1
\end{itemize}

\section{Operasi Logika}

\subsection{AND Operation}
Bitwise AND untuk masking dan testing bits.

\begin{verbatim}
; Basic AND
AND AX, BX      ; AX = AX & BX
AND AL, 0Fh     ; Mask lower nibble (keep 4 LSB)
AND BL, 01h     ; Test bit 0 (result 0 atau 1)

; Memory operations
AND [SI], 80h    ; memori[SI] = memori[SI] & 80h
AND WORD PTR [DI], 0FF00h ; Mask upper byte
\end{verbatim}

\textbf{Aplikasi AND:}
\begin{itemize}
  \item Clear bits: AND dengan 0
  \item Test bits: AND dengan mask, cek ZF
  \item Extract bits: AND dengan mask spesifik
\end{itemize}

\subsection{OR Operation}
Bitwise OR untuk setting bits.

\begin{verbatim}
; Basic OR
OR AX, BX       ; AX = AX | BX
OR AL, 80h      ; Set bit 7
OR BL, 01h      ; Set bit 0

; Memory operations
OR [SI], 0Fh    ; Set lower nibble
OR WORD PTR [DI], 8000h ; Set bit 15
\end{verbatim}

\textbf{Aplikasi OR:}
\begin{itemize}
  \item Set bits ke 1
  \item Combine flag bits
  \item Create bit patterns
\end{itemize}

\subsection{XOR Operation}
Bitwise XOR untuk toggling dan enkripsi sederhana.

\begin{verbatim}
; Basic XOR
XOR AX, BX      ; AX = AX ^ BX
XOR AL, 01h     ; Toggle bit 0
XOR BL, 0FFh    ; Toggle semua bits

; Clear register (XOR dengan diri sendiri)
XOR AX, AX       ; AX = 0
XOR CX, CX       ; CX = 0

; Memory operations
XOR [SI], 80h    ; Toggle bit 7 di memori
\end{verbatim}

\textbf{Aplikasi XOR:}
\begin{itemize}
  \item Toggle bits
  \item Simple encryption/decryption
  \item Clear registers
  \item Parity checking
\end{itemize}

\subsection{TEST Instruction}
TEST seperti AND tetapi tidak mengubah hasil, hanya mengubah flags.

\begin{verbatim}
; Test individual bits
TEST AL, 01h     ; Test bit 0, set ZF jika bit 0
TEST AH, 80h     ; Test bit 7

; Test multiple bits
TEST AX, 0F0Fh   ; Test lower byte, set ZF jika 0
TEST BX, 8000h   ; Test bit 15

; Test untuk jump condition
TEST CX, CX       ; Test apakah CX = 0, set ZF
JZ zero           ; Jump jika CX = 0
\end{verbatim}

\textbf{Keuntungan TEST:}
\begin{itemize}
  \item Tidak mengubah data asli
  \item Efisien untuk conditional testing
  \item Sering digunakan sebelum conditional jump
\end{itemize}

\section{Operasi Shift dan Rotate}

\subsection{Logical Shift}
SHL dan SHR untuk pergeseran bit dengan mempertahankan sign.

\begin{verbatim}
; Shift Left (SHL)
SHL AX, 1       ; AX = AX << 1, CF = bit yang keluar
SHL BX, CL       ; BX = BX << CL, CF = bit yang keluar

; Shift Right (SHR) 
SHR AX, 1       ; AX = AX >> 1 (unsigned), CF = bit yang keluar
SHR BX, CL       ; BX = BX >> CL (unsigned)
\end{verbatim}

\textbf{Pengaruh flags:}
\begin{itemize}
  \item CF berisi bit yang digeser keluar
  \item OF tergantung pada shift left (SHL)
  \item ZF set jika hasil = 0
\end{itemize}

\subsection{Arithmetic Shift}
SAL dan SAR untuk pergeseran dengan mempertahankan sign.

\begin{verbatim}
; Shift Arithmetic Left (SAL) - sama dengan SHL
SAL AX, 1       ; AX = AX << 1 (signed)

; Shift Arithmetic Right (SAR)
SAR AX, 1       ; AX = AX >> 1 (signed), sign bit preserved
SAR BX, CL       ; BX = BX >> CL (signed)
\end{verbatim}

\textbf{Perbedaan SHR vs SAR:}
\begin{itemize}
  \item SHR: mengisi MSB dengan 0 (unsigned shift)
  \item SAR: mempertahankan sign bit (signed shift)
  \item SAR mempertahankan tanda bilangan negatif
\end{itemize}

\subsection{Rotate Operations}
RCL, RCR, ROL, ROR untuk rotasi bit.

\begin{verbatim}
; Rotate Left (ROL)
ROL AL, 1       ; Rotate left 1 bit, CF = bit yang keluar
ROL AX, CL       ; Rotate left CL bits

; Rotate Right (ROR)  
ROR AL, 1       ; Rotate right 1 bit
ROR AX, CL       ; Rotate right CL bits

; Rotate through Carry (RCL/RCR)
RCL AL, 1       ; Rotate left through carry
RCR AL, 1       ; Rotate right through carry
\end{verbatim}

\textbf{Aplikasi Rotate:}
\begin{itemize}
  \item Circular buffer
  \item Cryptography
  \item Bit manipulation
  \item Graphics operations
\end{itemize}

% ============================================================
% AKTIVITAS PEMBELAJARAN
% ============================================================

\begin{aktivitas}
  \item \textbf{Arithmetic Calculator}: Implementasikan kalkulator dengan operasi ADD, SUB, MUL, DIV.
  
  \item \textbf{Logic Gates}: Simulasi logic gates (AND, OR, XOR, NOT) menggunakan instruksi assembly.
  
  \item \textbf{Bit Manipulation}: Buat program untuk mengekstrak, mengatur, dan menguji bit individual.
  
  \item \textbf{Shift Operations}: Implementasikan program enkripsi sederhana dengan shift dan rotate.
  
  \item \textbf{Flag Analysis}: Gunakan TASM debugger untuk mengamati perubahan flag register.
  
  \item \textbf{Performance Testing}: Bandingkan kecepatan berbagai operasi aritmatika dan logika.
\end{aktivitas}

% ============================================================
% LATIHAN DAN REFLEKSI
% ============================================================

\begin{latihan}
  \item Buat program untuk menghitung faktorial menggunakan MUL dan loop.
  
  \item Implementasikan program untuk mengkonversi bilangan desimal ke biner dengan shift operations.
  
  \item Buat program yang mengecek apakah suatu bilangan adalah pangkat dari 2 menggunakan TEST instruksi.
  
  \item Implementasikan program enkripsi Caesar cipher dengan XOR operations.
  
  \item Buat program untuk menghitung jumlah bit yang set (population count) dalam sebuah register.
  
  \item Implementasikan swap dua variabel tanpa menggunakan temporary variable (hanya XOR).
  
  \item Buat program untuk membagi bilangan besar dengan DIV dan handle overflow.
  
  \item \textbf{Refleksi}: Operasi logika mana yang paling sulit dipahami? Bagaimana Anda mengatasi kesulitan tersebut?
\end{latihan}

% ============================================================
% ASESMEN
% ============================================================

\begin{asesmen}
\textbf{Instrumen Penilaian untuk Sub-CPMK 3.1, 3.2}

\textbf{A. Pilihan Ganda}

\begin{enumerate}
  \item Instruksi untuk perkalian unsigned 8-bit adalah:
  \begin{enumerate}
    \item MUL
    \item IMUL
    \item DIV
    \item IDIV
  \end{enumerate}
  
  \item Flag yang tidak terpengaruh oleh instruksi AND adalah:
  \begin{enumerate}
    \item CF (Carry Flag)
    \item OF (Overflow Flag)
    \item ZF (Zero Flag)
    \item SF (Sign Flag)
  \end{enumerate}
  
  \item Instruksi untuk shift right yang mempertahankan sign bit adalah:
  \begin{enumerate}
    \item SHR
    \item SHL
    \item SAR
    \item SAL
  \end{enumerate}
  
  \item Hasil dari XOR AX, AX adalah:
  \begin{enumerate}
    \item AX
    \item 0
    \item 0FFFFh
    \item Tidak terdefinisi
  \end{enumerate}
\end{enumerate}

\textbf{B. Essay}

\begin{enumerate}
  \item Jelaskan perbedaan antara instruksi TEST dan CMP! Kapan sebaiknya menggunakan masing-masing?
  
  \item Mengapa instruksi MUL memiliki format yang berbeda dengan instruksi ADD dalam hal operand?
\end{enumerate}

\textbf{C. Practical Challenge}

\begin{enumerate}
  \item Buat program kalkulator scientific:
  \begin{itemize}
    \item Support operasi dasar (+, -, *, /)
    \item Implementasikan fungsi trigonometri dengan lookup table
    \item Gunakan operasi logika untuk validasi input
    \item Handle overflow dan underflow
    \item Tampilkan hasil dengan format yang baik
  \end{itemize}
\end{enumerate}

\textbf{Rubrik Penilaian}: Lihat Lampiran A
\end{asesmen}

% ============================================================
% CHECKLIST KOMPETENSI
% ============================================================

\begin{checklist}
  \item Saya dapat mengimplementasikan operasi aritmatika (ADD, SUB, MUL, DIV)
  \item Saya dapat menerapkan operasi logika (AND, OR, XOR, NOT, TEST)
  \item Saya dapat menggunakan shift dan rotate operations
  \item Saya memahami pengaruh operasi terhadap flag register
  \item Saya dapat menangani overflow dan underflow
  \item Saya dapat mengoptimasi operasi bit manipulation
  \item Saya dapat menggunakan TASM debugger untuk tracing operasi
  \item Saya dapat menerapkan konsep two's complement
\end{checklist}

% ============================================================
% RANGKUMAN
% ============================================================

\begin{rangkuman}
Bab ini membahas operasi aritmatika dan logika dalam assembly 8086, termasuk instruksi dasar, flag manipulation, dan teknik bit manipulation.

\textbf{Poin Kunci:}
\begin{itemize}
  \item Operasi aritmatika meliputi ADD, SUB, MUL, DIV, INC, DEC, NEG
  \item Operasi logika meliputi AND, OR, XOR, NOT, TEST
  \item Shift operations: SHL, SHR, SAL, SAR untuk pergeseran bit
  \item Rotate operations: ROL, ROR, RCL, RCR untuk rotasi bit
  \item Flag register penting untuk conditional operations dan error detection
  \item Bit manipulation essential untuk encryption, compression, dan graphics
  \item Overflow handling critical untuk numerical computations
  \item TASM debugger membantu analisis operasi langkah demi langkah
\end{itemize}

\textbf{Kata Kunci}: \asm{Aritmatika}, \asm{Logika}, \asm{ADD}, \asm{SUB}, \asm{MUL}, \asm{DIV}, \asm{AND}, \asm{OR}, \asm{XOR}, \asm{NOT}, \asm{TEST}, \asm{SHL}, \asm{SHR}, \asm{SAL}, \asm{SAR}, \asm{ROL}, \asm{ROR}, \asm{Flag}, \asm{TASM}
\end{rangkuman}


% ============================================================
% AKTIVITAS PEMBELAJARAN
% ============================================================

\begin{aktivitas}
  \item \textbf{Arithmetic Calculator}: Buat kalkulator yang implementasikan semua operasi aritmatika dengan input dari user.
  
  \item \textbf{Flag Analysis}: Gunakan TASM debugger untuk mengamati perubahan flag setiap operasi aritmatika dan logika.
  
  \item \textbf{Bit Manipulation}: Implementasikan program untuk menghitung jumlah bit set (population count).
  
  \item \textbf{Shift Operations}: Buat program untuk konversi bilangan desimal ke biner menggunakan shift operations.
  
  \item \textbf{Logical Expression}: Implementasikan ekspresi logika kompleks dengan kombinasi AND, OR, XOR.
  
  \item \textbf{Performance Comparison}: Bandingkan kecepatan MUL vs shift untuk operasi perkalian dengan pangkat 2.
\end{aktivitas}

% ============================================================
% LATIHAN DAN REFLEKSI
% ============================================================

\begin{latihan}
  \item Jelaskan perbedaan antara MUL dan IMU! Kapan sebaiknya menggunakan masing-masing?
  
  \item Analisis flag yang berubah setelah operasi ADD AX, 7FFFh dengan AX = 0001h. Jelaskan mengapa overflow terjadi.
  
  \item Buat program yang:
  \begin{itemize}
    \item Menghitung kuadrat bilangan menggunakan MUL
    \item Menguji hasil positif/negatif dengan flag register
    \item Menampilkan hasil yang sesuai
  \end{itemize}
  
  \item Implementasikan fungsi untuk menghitung absolute value menggunakan operasi logika.
  
  \item Buat program untuk menghitung gcd dua bilangan menggunakan DIV dan SUB.
  
  \item Gunakan operasi shift untuk implementasi perkalian dan pembagian dengan pangkat 2.
  
  \item \textbf{Refleksi}: Operasi mana yang paling sulit dipahami (aritmatika vs logika)? Bagaimana Anda mengatasi kesulitan tersebut?
\end{latihan}

% ============================================================
% ASESMEN
% ============================================================

\begin{asesmen}
\textbf{Instrumen Penilaian untuk Sub-CPMK 3.1, 3.2, 2.1}

\textbf{A. Pilihan Ganda}

\begin{enumerate}
  \item Setelah operasi MUL dengan operand 8-bit, hasil disimpan di:
  \begin{enumerate}
    \item AX
    \item AL
    \item DX:AX
    \item AH
  \end{enumerate}
  
  \item Instruksi yang tidak mempengaruhi carry flag adalah:
  \begin{enumerate}
    \item ADD
    \item SUB
    \item AND
    \item INC
  \end{enumerate}
  
  \item Untuk membagi DX:AX dengan operand 16-bit, gunakan:
  \begin{enumerate}
    \item DIV
    \item IDIV
    \item MUL
    \item IMUL
  \end{enumerate}
  
  \item Shift yang preserve sign bit adalah:
  \begin{enumerate}
    \item SHL
    \item SHR
    \item SAL
    \item SAR
  \end{enumerate}
\end{enumerate}

\textbf{B. Essay}

\begin{enumerate}
  \item Jelaskan perbedaan antara logical shift dan arithmetic shift! Berikan contoh kasus penggunaan.
  
  \item Mengapa instruksi TEST digunakan instead of AND untuk testing bit?
\end{enumerate}

\textbf{C. Practical Challenge}

\begin{enumerate}
  \item Buat program scientific calculator:
  \begin{itemize}
    \item Implementasikan semua operasi aritmatika dasar
    \item Tambahkan operasi logika bitwise
    \item Implementasikan fungsi power of 2 dengan shift
    \item Validasi input dan handle overflow
    \item Tampilkan hasil dalam format desimal dan heksadesimal
  \end{itemize}
\end{enumerate}

\textbf{Rubrik Penilaian}: Lihat Lampiran A
\end{asesmen}

% ============================================================
% CHECKLIST KOMPETENSI
% ============================================================

\begin{checklist}
  \item Saya dapat mengimplementasikan operasi aritmatika (ADD, SUB, MUL, DIV)
  \item Saya dapat menganalisis flag register setelah operasi aritmatika
  \item Saya dapat menerapkan operasi logika (AND, OR, XOR, NOT)
  \item Saya dapat menggunakan operasi shift dan rotate
  \item Saya dapat menangani overflow dan underflow
  \item Saya dapat mengoptimasi operasi dengan shift operations
  \item Saya dapat menulis program assembly dengan operasi kombinasi
  \item Saya dapat menggunakan TASM debugger untuk analisis operasi
\end{checklist}

% ============================================================
% RANGKUMAN
% ============================================================

\begin{rangkuman}
Bab ini membahas operasi aritmatika dan logika dalam assembly language, termasuk implementasi, flag handling, dan optimasi.

\textbf{Poin Kunci:}
\begin{itemize}
  \item Operasi aritmatika meliputi ADD, SUB, MUL, DIV dengan aturan register spesifik
  \item Flag register penting untuk deteksi overflow, carry, zero, dan sign
  \item Operasi logika bitwise essential untuk bit manipulation
  \item Shift operations lebih efisien untuk perkalian/pembagian pangkat 2
  \item Pemahaman operand size dan register usage kunci untuk operasi yang benar
  \item TASM debugger membantu analisis operasi langkah per langkah
  \item Optimasi operasi meningkatkan kinerja program assembly
\end{itemize}

\textbf{Kata Kunci}: \textbf{\texttt{Aritmatika}}, \textbf{\texttt{Logika}}, \textbf{\texttt{ADD}}, \textbf{\texttt{SUB}}, \textbf{\texttt{MUL}}, \textbf{\texttt{DIV}}, \textbf{\texttt{AND}}, \textbf{\texttt{OR}}, \textbf{\texttt{XOR}}, \textbf{\texttt{SHIFT}}, \textbf{\texttt{Flag Register}}, \textbf{\texttt{TASM}}
\end{rangkuman}

\ifSubfilesClassLoaded{
  \renewcommand{\bibname}{Daftar Pustaka}
  \bibliographystyle{plain}
  \bibliography{../references}
}{}
\end{document}
