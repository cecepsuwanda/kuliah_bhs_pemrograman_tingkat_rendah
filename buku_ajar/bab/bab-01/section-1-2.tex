\section{Tujuan Pembelajaran dan Kompetensi}

Buku ajar ini dirancang untuk mencapai tujuan pembelajaran spesifik dalam pemrograman assembly Intel 8086, dengan fokus pada pengembangan keterampilan praktis dan pemahaman konseptual yang mendalam \cite{ref8}. Tujuan pembelajaran ini diwujudkan melalui:

\subsection{Keterkaitan Buku Ajar dengan RPS Berbasis OBE}

Buku ajar ini disusun selaras dengan Rencana Pembelajaran Semester (RPS) berbasis Outcome-Based Education (OBE). Keterkaitan utama:

\begin{itemize}
  \item \textbf{RPS sebagai Acuan}: RPS menentukan CPL, CPMK, Sub-CPMK, dan strategi penilaian. Setiap bab dalam buku ini memetakan ke Sub-CPMK tertentu yang tercantum dalam RPS.
  \item \textbf{Matriks Keterkaitan}: Matriks CPL--CPMK (lihat halaman Identitas Mata Kuliah) menjelaskan kontribusi mata kuliah ini terhadap capaian program studi. Buku ini mendukung CPL-1, CPL-2, CPL-3, dan CPL-4 melalui CPMK-1 hingga CPMK-4.
  \item \textbf{Alur Pembelajaran}: Urutan bab mengikuti struktur RPS---dari pendahuluan, landasan teori, unit materi per minggu, hingga evaluasi akhir.
  \item \textbf{Asesmen Terpadu}: Instrumen asesmen di setiap bab dirancang mengikuti indikator penilaian RPS, sehingga mahasiswa dapat memenuhi kriteria kelulusan yang ditetapkan.
\end{itemize}

\subsection{Ringkasan Konteks Kurikulum OBE}

Outcome-Based Education menekankan \textit{apa yang mahasiswa capai} setelah pembelajaran, bukan sekadar materi yang diajarkan. Konteks kurikulum OBE untuk mata kuliah ini:

\begin{itemize}
  \item \textbf{CPL (Capaian Pembelajaran Lulusan)}: Kompetensi yang harus dimiliki lulusan program studi---pengetahuan, keterampilan umum, keterampilan khusus, dan sikap. Mata kuliah Pemrograman Assembly berkontribusi pada keempat aspek tersebut.
  \item \textbf{CPMK (Capaian Pembelajaran Mata Kuliah)}: Empat kompetensi utama---memahami konsep register dan segmentasi (CPMK-1), merancang program modular (CPMK-2), mengimplementasikan dengan TASM (CPMK-3), serta menganalisis dan mengoptimasi kode (CPMK-4).
  \item \textbf{Sub-CPMK}: Turunan CPMK yang lebih spesifik, diukur per bab melalui aktivitas, latihan, dan asesmen.
  \item \textbf{Peran Mata Kuliah}: Sebagai fondasi pemrograman tingkat rendah, mata kuliah ini mendukung pemahaman arsitektur komputer, sistem operasi, dan optimasi perangkat lunak---kompetensi yang diperlukan dalam ranah embedded systems, driver development, dan reverse engineering.
\end{itemize}

\subsection{Kompetensi Pembelajaran}

Setiap bab dalam buku ini dikembangkan untuk mencapai kompetensi spesifik:
\begin{itemize}
  \item \textbf{Pemahaman Konseptual}: Master konsep processor, instruksi, dan addressing modes
  \item \textbf{Keterampilan Praktis}: Menulis dan debug program assembly dengan TASM
  \item \textbf{Optimasi Kode}: Menerapkan teknik optimasi untuk performa maksimal
  \item \textbf{Problem Solving}: Menyelesaikan masalah dengan pendekatan assembly
  \item \textbf{Debugging}: Mengidentifikasi dan memperbaiki bug dalam kode assembly
\end{itemize}

\subsection{Metodologi Pembelajaran}

Buku ini mengadopsi metodologi pembelajaran yang efektif untuk assembly programming:
\begin{itemize}
  \item \textbf{Hands-On Practice}: 70\% praktikum dengan TASM dan debugger
  \item \textbf{Progressive Complexity}: Dari konsep dasar hingga aplikasi kompleks
  \item \textbf{Real-World Examples}: Studi kasus aplikasi assembly industri
  \item \textbf{Performance Analysis}: Pengukuran dan optimasi performa kode
  \item \textbf{Interactive Learning}: Demonstrasi langsung dengan TASM
\end{itemize}

\subsection{Struktur Pembelajaran}

Struktur pembelajaran dirancang untuk kesuksesan mahasiswa:
\begin{itemize}
  \item \textbf{Teori (30\%)}: Konsep fundamental processor dan assembly
  \item \textbf{Praktik (50\%)}: Hands-on programming dengan TASM
  \item \textbf{Proyek (20\%)}: Aplikasi nyata untuk integrasi konsep
\end{itemize}
