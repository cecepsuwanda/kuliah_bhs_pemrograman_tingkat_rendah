\section{Tujuan Pembelajaran dan Kompetensi}

Buku ajar ini dirancang untuk mencapai tujuan pembelajaran spesifik dalam pemrograman assembly Intel 8086, dengan fokus pada pengembangan keterampilan praktis dan pemahaman konseptual yang mendalam \cite{ref8}. Tujuan pembelajaran ini diwujudkan melalui:

\subsection{Kompetensi Pembelajaran}

Setiap bab dalam buku ini dikembangkan untuk mencapai kompetensi spesifik:
\begin{itemize}
  \item \textbf{Pemahaman Konseptual}: Master konsep processor, instruksi, dan addressing modes
  \item \textbf{Keterampilan Praktis}: Menulis dan debug program assembly dengan TASM
  \item \textbf{Optimasi Kode}: Menerapkan teknik optimasi untuk performa maksimal
  \item \textbf{Problem Solving}: Menyelesaikan masalah dengan pendekatan assembly
  \item \textbf{Debugging}: Mengidentifikasi dan memperbaiki bug dalam kode assembly
\end{itemize}

\subsection{Metodologi Pembelajaran}

Buku ini mengadopsi metodologi pembelajaran yang efektif untuk assembly programming:
\begin{itemize}
  \item \textbf{Hands-On Practice}: 70\% praktikum dengan TASM dan debugger
  \item \textbf{Progressive Complexity}: Dari konsep dasar hingga aplikasi kompleks
  \item \textbf{Real-World Examples}: Studi kasus aplikasi assembly industri
  \item \textbf{Performance Analysis}: Pengukuran dan optimasi performa kode
  \item \textbf{Interactive Learning}: Demonstrasi langsung dengan TASM
\end{itemize}

\subsection{Struktur Pembelajaran}

Struktur pembelajaran dirancang untuk kesuksesan mahasiswa:
\begin{itemize}
  \item \textbf{Teori (30\%)}: Konsep fundamental processor dan assembly
  \item \textbf{Praktik (50\%)}: Hands-on programming dengan TASM
  \item \textbf{Proyek (20\%)}: Aplikasi nyata untuk integrasi konsep
\end{itemize}
