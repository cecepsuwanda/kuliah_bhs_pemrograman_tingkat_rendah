\section{Pengenalan Bahasa Rakitan dan Bahasa Tingkat Rendah}

\subsection{Latar Belakang dan Konteks Historis}

Bahasa rakitan (assembly language) merupakan salah satu bahasa pemrograman tertua yang masih digunakan hingga saat ini \cite{wiki_assembly_language}. Konsep assembly language pertama kali dikembangkan pada tahun 1940-an sebagai respons terhadap kompleksitas pemrograman dalam bahasa mesin murni. Sebelum adanya assembly language, programmer harus menulis program menggunakan kode biner (0 dan 1) yang sangat sulit dibaca dan rentan terhadap kesalahan.

Perkembangan assembly language dimulai dengan munculnya komputer-komputer pertama seperti ENIAC dan UNIVAC \cite{wiki_8086}. Pada masa itu, programmer menggunakan machine code yang terdiri dari angka-angka biner yang mewakili instruksi dasar komputer. Assembly language kemudian dikembangkan sebagai solusi dengan menyediakan mnemonik (singkatan yang mudah diingat) untuk menggantikan kode biner tersebut \cite{hyde2010art}.

\subsection{Definisi dan Karakteristik Bahasa Rakitan}

Assembly language adalah bahasa pemrograman berlevel rendah yang menyediakan antarmuka langsung terhadap instruksi mesin (machine instructions) dari sebuah \textit{Instruction Set Architecture} (ISA) \cite{computer_organization_design}. Assembly language merupakan representasi mnemonik dari instruksi mesin. Setiap mnemonik (misal, \texttt{MOV}, \texttt{ADD}, \texttt{JMP}) biasanya berkorelasi dekat dengan \textit{opcode} biner yang dieksekusi CPU. Kode assembly dirakit (\textit{assembled}) oleh \textit{assembler} (misal: TASM, MASM, NASM) menjadi \textit{object code} atau berkas executable \cite{tutorials_point_assembly}.

Dalam konteks pembelajaran ini, kita akan fokus pada keluarga x86, khususnya Intel 8086 dan penerusnya \cite{intel2019manual32}.

\subsection{Perbandingan dengan Bahasa Tingkat Tinggi}

\begin{table}[H]
\centering
\caption{Perbandingan Assembly dan Bahasa Tingkat Tinggi}
\setlength{\tabcolsep}{3pt}
\resizebox{\textwidth}{!}{%
\scriptsize
\begin{tabular}{|p{1.8cm}|p{3.2cm}|p{3.2cm}|}
\hline
\textbf{Aspek} & \textbf{Assembly} & \textbf{Bahasa Tingkat Tinggi} \\
\hline
Abstraksi & Rendah; dekat hardware & Tinggi; jauh dari hardware \\
Portabilitas & Rendah (spesifik ISA) & Lebih tinggi (kompiler/VM) \\
Produktivitas & Rendah & Tinggi \\
Kinerja puncak & Sangat tinggi (diopt.) & Baik (optimasi auto) \\
Pemeliharaan & Sulit & Lebih mudah \\
Manajemen Memori & Manual & Otomatis \\
Pustaka & Minim & Kaya \\
\hline
\end{tabular}
}
\end{table}

\subsection{Keunggulan dan Kelemahan}

\textbf{Keunggulan:}
\begin{itemize}
  \item Kontrol penuh terhadap perangkat keras—akses langsung ke register, flag, dan layout memori
  \item Optimasi mikroskopik untuk kinerja dan ukuran kode
  \item Pemahaman mendalam atas runtime dan calling convention
  \item Kinerja dan jejak memori optimal untuk aplikasi embedded dan komponen runtime-critical
\end{itemize}

\textbf{Kelemahan:}
\begin{itemize}
  \item Pengembangan lambat dan rentan kesalahan
  \item Sulit dipelihara dan kurang portabel (spesifik arsitektur x86/ARM)
  \item Minim dukungan pustaka
  \item Kurangnya abstraksi tingkat tinggi: tidak ada struktur kontrol kompleks, garbage collector, atau tipe data kompleks secara bawaan
\end{itemize}

\subsection{Aplikasi Bahasa Rakitan}

Assembly language masih digunakan dalam berbagai bidang khusus \cite{wiki_assembly_language}:

\begin{itemize}
  \item \textbf{Embedded systems}: Firmware mikrokontroler, device driver, interrupt service routine
  \item \textbf{Bootloader dan BIOS/UEFI}: Kode awal yang dijalankan sebelum sistem operasi
  \item \textbf{Driver dan kernel}: Bagian kernel yang sensitif terhadap kinerja
  \item \textbf{Keamanan siber}: Analisis malware, exploit development, reverse engineering
  \item \textbf{Optimasi hotspot}: Bagian kecil dari aplikasi yang memerlukan latensi minimal
  \item \textbf{Pengembangan compiler}: Komponen runtime yang sensitif terhadap kinerja
\end{itemize}

\subsection{Contoh Program Assembly Sederhana}

Berikut contoh program assembly 8086 untuk mencetak karakter \cite{susanto1995belajar}:

\begin{lstlisting}[style=assemblystyle, caption={Program Assembly Sederhana - Output Karakter (A0.asm)}]
;   PROGRAM : A0.ASM
;   FUNGSI  : MENCETAK KARAKTER 'A' DENGAN INT 21h
.MODEL SMALL
.CODE
ORG 100h
Proses:
    MOV  AH, 02h   ; Nilai servis untuk mencetak karakter
    MOV  DL, 'A'   ; DL = Karakter ASCII yang akan dicetak
    INT  21h       ; Cetak karakter
END Proses
\end{lstlisting}

Program ini menunjukkan:
\begin{itemize}
  \item Penggunaan direktif \asm{.MODEL SMALL} untuk model memori kecil
  \item Penggunaan direktif \asm{ORG 100h} untuk menentukan starting address
  \item Instruksi \asm{MOV} untuk memindahkan data ke register
  \item Penggunaan interrupt \asm{INT 21h} dengan servis \asm{AH=02h} untuk output karakter
\end{itemize}
