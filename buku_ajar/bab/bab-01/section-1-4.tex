\section{Arsitektur Intel 8086}

\subsection{Overview Processor Intel 8086}

Intel 8086 adalah 16-bit processor yang menjadi standar industri untuk komputer personal. Processor ini diperkenalkan pada tahun 1978 dan menjadi fondasi untuk arsitektur x86 modern.

\textbf{Fitur Utama Intel 8086:}
\begin{itemize}
  \item \textbf{16-bit Architecture}: 16-bit data bus dan 20-bit address bus
  \item \textbf{Register Set}: 8 general-purpose registers 16-bit
  \item \textbf{Instruction Set}: Komprehensif untuk operasi data dan kontrol
  \item \textbf{Addressing Modes}: Fleksibel untuk akses memori
  \item \textbf{Segmentation}: Memori management dengan segment registers
\end{itemize}

\subsection{Register Architecture}

Intel 8086 memiliki 8 general-purpose registers:
\begin{verbatim}
General Purpose Registers (16-bit):
AX  - Accumulator (arithmetic, I/O)
BX  - Base Register (memory addressing)
CX  - Count Register (loops, strings)
DX  - Data Register (arithmetic, I/O)

Special Purpose Registers:
SP  - Stack Pointer
BP  - Base Pointer
SI  - Source Index
DI  - Destination Index

Segment Registers:
CS  - Code Segment
DS  - Data Segment
SS  - Stack Segment
ES  - Extra Segment
\end{verbatim}

\subsection{Memory Organization}

Intel 8086 menggunakan segmented memory architecture:
\begin{itemize}
  \item \textbf{Physical Address}: 20-bit (1MB addressable memory)
  \item \textbf{Logical Address}: Segment:Offset (16:16 bit)
  \item \textbf{Segmentation}: 4 segment registers untuk 64KB segments
  \item \textbf{Address Calculation}: Physical = Segment × 16 + Offset
\end{itemize}

\begin{verbatim}
; Address calculation example
MOV AX, 1234h    ; Load value into AX
MOV DS, AX       ; Set DS = 1234h
MOV SI, 56h      ; Set SI = 56h
; Physical address = 1234h × 16 + 56h = 12340h + 56h = 12396h
\end{verbatim}

\subsection{Flag Register}

Flag register menyimpan status operasi:
\begin{verbatim}
Flag Register (16-bit):
CF - Carry Flag (carry/borrow)
PF - Parity Flag (even parity)
AF - Auxiliary Flag (BCD adjust)
ZF - Zero Flag (result = 0)
SF - Sign Flag (negative result)
TF - Trap Flag (debug)
IF - Interrupt Flag (interrupt enable)
DF - Direction Flag (string direction)
OF - Overflow Flag (signed overflow)
\end{verbatim}
