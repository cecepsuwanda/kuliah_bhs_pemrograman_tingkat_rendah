\documentclass[../main.tex]{subfiles}
\ifSubfilesClassLoaded{\setcounter{chapter}{0}}{}
\begin{document}

\chapter{Pendahuluan Pemrograman Assembly Intel 8086}

\section{Tujuan Buku Ajar}

Buku ajar ini dirancang sebagai panduan komprehensif untuk menguasai Pemrograman Assembly Intel 8086 menggunakan Turbo Assembler (TASM) \cite{ref2}. Fokus utama buku ini adalah pada pemahaman arsitektur komputer tingkat rendah dan implementasi program assembly yang efisien. Tujuan spesifik buku ini adalah:

\begin{enumerate}
  \item Memberikan pemahaman mendalam tentang arsitektur processor Intel 8086
  \item Mengembangkan kemampuan menulis program assembly dengan TASM
  \item Membangun keterampilan optimasi dan debugging program assembly
  \item Menumbuhkan kemampuan analisis dan evaluasi performa kode
  \item Memfasilitasi pencapaian CPL dan CPMK yang telah ditetapkan
\end{enumerate}

Setelah mempelajari buku ini secara menyeluruh, mahasiswa diharapkan mampu:
\begin{itemize}
  \item Menjelaskan konsep fundamental processor Intel 8086 (registers, addressing modes, instruction set)
  \item Merancang program assembly dengan struktur yang efisien
  \item Mengimplementasikan aplikasi assembly dengan Turbo Assembler
  \item Mengidentifikasi dan menerapkan teknik optimasi assembly
  \item Menulis kode yang efisien, terstruktur, dan mudah di-debug
\end{itemize}

\section{Tujuan Pembelajaran dan Kompetensi}

Buku ajar ini dirancang untuk mencapai tujuan pembelajaran spesifik dalam pemrograman assembly Intel 8086, dengan fokus pada pengembangan keterampilan praktis dan pemahaman konseptual yang mendalam \cite{ref8}. Tujuan pembelajaran ini diwujudkan melalui:

\subsection{Keterkaitan Buku Ajar dengan RPS Berbasis OBE}

Buku ajar ini disusun selaras dengan Rencana Pembelajaran Semester (RPS) berbasis Outcome-Based Education (OBE). Keterkaitan utama:

\begin{itemize}
  \item \textbf{RPS sebagai Acuan}: RPS menentukan CPL, CPMK, Sub-CPMK, dan strategi penilaian. Setiap bab dalam buku ini memetakan ke Sub-CPMK tertentu yang tercantum dalam RPS.
  \item \textbf{Matriks Keterkaitan}: Matriks CPL--CPMK (lihat halaman Identitas Mata Kuliah) menjelaskan kontribusi mata kuliah ini terhadap capaian program studi. Buku ini mendukung CPL-1, CPL-2, CPL-3, dan CPL-4 melalui CPMK-1 hingga CPMK-4.
  \item \textbf{Alur Pembelajaran}: Urutan bab mengikuti struktur RPS---dari pendahuluan, landasan teori, unit materi per minggu, hingga evaluasi akhir.
  \item \textbf{Asesmen Terpadu}: Instrumen asesmen di setiap bab dirancang mengikuti indikator penilaian RPS, sehingga mahasiswa dapat memenuhi kriteria kelulusan yang ditetapkan.
\end{itemize}

\subsection{Ringkasan Konteks Kurikulum OBE}

Outcome-Based Education menekankan \textit{apa yang mahasiswa capai} setelah pembelajaran, bukan sekadar materi yang diajarkan. Konteks kurikulum OBE untuk mata kuliah ini:

\begin{itemize}
  \item \textbf{CPL (Capaian Pembelajaran Lulusan)}: Kompetensi yang harus dimiliki lulusan program studi---pengetahuan, keterampilan umum, keterampilan khusus, dan sikap. Mata kuliah Pemrograman Assembly berkontribusi pada keempat aspek tersebut.
  \item \textbf{CPMK (Capaian Pembelajaran Mata Kuliah)}: Empat kompetensi utama---memahami konsep register dan segmentasi (CPMK-1), merancang program modular (CPMK-2), mengimplementasikan dengan TASM (CPMK-3), serta menganalisis dan mengoptimasi kode (CPMK-4).
  \item \textbf{Sub-CPMK}: Turunan CPMK yang lebih spesifik, diukur per bab melalui aktivitas, latihan, dan asesmen.
  \item \textbf{Peran Mata Kuliah}: Sebagai fondasi pemrograman tingkat rendah, mata kuliah ini mendukung pemahaman arsitektur komputer, sistem operasi, dan optimasi perangkat lunak---kompetensi yang diperlukan dalam ranah embedded systems, driver development, dan reverse engineering.
\end{itemize}

\subsection{Kompetensi Pembelajaran}

Setiap bab dalam buku ini dikembangkan untuk mencapai kompetensi spesifik:
\begin{itemize}
  \item \textbf{Pemahaman Konseptual}: Master konsep processor, instruksi, dan addressing modes
  \item \textbf{Keterampilan Praktis}: Menulis dan debug program assembly dengan TASM
  \item \textbf{Optimasi Kode}: Menerapkan teknik optimasi untuk performa maksimal
  \item \textbf{Problem Solving}: Menyelesaikan masalah dengan pendekatan assembly
  \item \textbf{Debugging}: Mengidentifikasi dan memperbaiki bug dalam kode assembly
\end{itemize}

\subsection{Metodologi Pembelajaran}

Buku ini mengadopsi metodologi pembelajaran yang efektif untuk assembly programming:
\begin{itemize}
  \item \textbf{Hands-On Practice}: 70\% praktikum dengan TASM dan debugger
  \item \textbf{Progressive Complexity}: Dari konsep dasar hingga aplikasi kompleks
  \item \textbf{Real-World Examples}: Studi kasus aplikasi assembly industri
  \item \textbf{Performance Analysis}: Pengukuran dan optimasi performa kode
  \item \textbf{Interactive Learning}: Demonstrasi langsung dengan TASM
\end{itemize}

\subsection{Struktur Pembelajaran}

Struktur pembelajaran dirancang untuk kesuksesan mahasiswa:
\begin{itemize}
  \item \textbf{Teori (30\%)}: Konsep fundamental processor dan assembly
  \item \textbf{Praktik (50\%)}: Hands-on programming dengan TASM
  \item \textbf{Proyek (20\%)}: Aplikasi nyata untuk integrasi konsep
\end{itemize}

\section{Petunjuk Penggunaan Buku Ajar}

Untuk gambaran lengkap struktur buku, komponen setiap bab, dan tips belajar efektif, pembaca dapat merujuk ke bab \textit{Cara Menggunakan Buku Ini} pada halaman awal (frontmatter). Berikut petunjuk spesifik untuk mahasiswa dan dosen:

\subsection{Untuk Mahasiswa}

\textbf{Sebelum Perkuliahan:}
\begin{enumerate}
  \item Siapkan lingkungan TASM (Turbo Assembler dan Debugger)
  \item Pelajari Sub-CPMK di awal bab untuk memahami target pembelajaran
  \item Baca materi pokok dengan seksama, fokus pada konsep processor
  \item Siapkan pertanyaan tentang instruksi dan addressing modes
\end{enumerate}

\textbf{Selama Perkuliahan:}
\begin{enumerate}
  \item Praktikkan semua contoh kode dengan TASM
  \item Gunakan debugger untuk menganalisis eksekusi program
  \item Diskusikan teknik optimasi dengan dosen dan teman
  \item Aktif dalam debugging dan troubleshooting
  \item Dokumentasikan hasil praktikum dan analisis
\end{enumerate}

\textbf{Setelah Perkuliahan:}
\begin{enumerate}
  \item Kerjakan latihan dan refleksi programming
  \item Lakukan asesmen mandiri dengan checklist kompetensi
  \item Optimasi kode yang telah dibuat untuk performa lebih baik
  \item Kerjakan proyek assembly untuk aplikasi nyata
  \item Eksplorasi topik advanced dan research
\end{enumerate}

\subsection{Untuk Dosen}

Buku ini dapat digunakan sebagai:
\begin{itemize}
  \item Bahan ajar utama untuk perkuliahan assembly programming
  \item Sumber praktikum TASM dan debugging exercises
  \item Referensi untuk menyusun soal ujian praktik
  \item Panduan untuk merancang aktivitas hands-on
  \item Alat untuk mengukur pencapaian Sub-CPMK mahasiswa
  \item Framework untuk pengembangan kurikulum assembly
\end{itemize}

\subsection{Tips Belajar Assembly Programming}

\begin{itemize}
  \item \textbf{Praktik Rutin}: Luangkan waktu setiap hari untuk coding assembly
  \item \textbf{Debug Actively}: Gunakan TASM debugger untuk memahami eksekusi
  \item \textbf{Document Progress}: Catat pembelajaran dan breakthrough moments
  \item \textbf{Join Community}: Berpartisipasi dalam forum assembly programming
  \item \textbf{Real Projects}: Terapkan konsep pada proyek nyata
\end{itemize}

\section{Pengenalan Bahasa Rakitan dan Bahasa Tingkat Rendah}

\subsection{Latar Belakang dan Konteks Historis}

Bahasa rakitan (assembly language) merupakan salah satu bahasa pemrograman tertua yang masih digunakan hingga saat ini \cite{wiki_assembly_language}. Konsep assembly language pertama kali dikembangkan pada tahun 1940-an sebagai respons terhadap kompleksitas pemrograman dalam bahasa mesin murni. Sebelum adanya assembly language, programmer harus menulis program menggunakan kode biner (0 dan 1) yang sangat sulit dibaca dan rentan terhadap kesalahan.

Perkembangan assembly language dimulai dengan munculnya komputer-komputer pertama seperti ENIAC dan UNIVAC \cite{wiki_8086}. Pada masa itu, programmer menggunakan machine code yang terdiri dari angka-angka biner yang mewakili instruksi dasar komputer. Assembly language kemudian dikembangkan sebagai solusi dengan menyediakan mnemonik (singkatan yang mudah diingat) untuk menggantikan kode biner tersebut \cite{hyde2010art}.

\subsection{Definisi dan Karakteristik Bahasa Rakitan}

Assembly language adalah bahasa pemrograman berlevel rendah yang menyediakan antarmuka langsung terhadap instruksi mesin (machine instructions) dari sebuah \textit{Instruction Set Architecture} (ISA) \cite{computer_organization_design}. Assembly language merupakan representasi mnemonik dari instruksi mesin. Setiap mnemonik (misal, \texttt{MOV}, \texttt{ADD}, \texttt{JMP}) biasanya berkorelasi dekat dengan \textit{opcode} biner yang dieksekusi CPU. Kode assembly dirakit (\textit{assembled}) oleh \textit{assembler} (misal: TASM, MASM, NASM) menjadi \textit{object code} atau berkas executable \cite{tutorials_point_assembly}.

Dalam konteks pembelajaran ini, kita akan fokus pada keluarga x86, khususnya Intel 8086 dan penerusnya \cite{intel2019manual32}.

\subsection{Perbandingan dengan Bahasa Tingkat Tinggi}

\begin{table}[H]
\centering
\caption{Perbandingan Assembly dan Bahasa Tingkat Tinggi}
\setlength{\tabcolsep}{3pt}
\resizebox{\textwidth}{!}{%
\scriptsize
\begin{tabular}{|p{1.8cm}|p{3.2cm}|p{3.2cm}|}
\hline
\textbf{Aspek} & \textbf{Assembly} & \textbf{Bahasa Tingkat Tinggi} \\
\hline
Abstraksi & Rendah; dekat hardware & Tinggi; jauh dari hardware \\
Portabilitas & Rendah (spesifik ISA) & Lebih tinggi (kompiler/VM) \\
Produktivitas & Rendah & Tinggi \\
Kinerja puncak & Sangat tinggi (diopt.) & Baik (optimasi auto) \\
Pemeliharaan & Sulit & Lebih mudah \\
Manajemen Memori & Manual & Otomatis \\
Pustaka & Minim & Kaya \\
\hline
\end{tabular}
}
\end{table}

\subsection{Keunggulan dan Kelemahan}

\textbf{Keunggulan:}
\begin{itemize}
  \item Kontrol penuh terhadap perangkat keras—akses langsung ke register, flag, dan layout memori
  \item Optimasi mikroskopik untuk kinerja dan ukuran kode
  \item Pemahaman mendalam atas runtime dan calling convention
  \item Kinerja dan jejak memori optimal untuk aplikasi embedded dan komponen runtime-critical
\end{itemize}

\textbf{Kelemahan:}
\begin{itemize}
  \item Pengembangan lambat dan rentan kesalahan
  \item Sulit dipelihara dan kurang portabel (spesifik arsitektur x86/ARM)
  \item Minim dukungan pustaka
  \item Kurangnya abstraksi tingkat tinggi: tidak ada struktur kontrol kompleks, garbage collector, atau tipe data kompleks secara bawaan
\end{itemize}

\subsection{Aplikasi Bahasa Rakitan}

Assembly language masih digunakan dalam berbagai bidang khusus \cite{wiki_assembly_language}:

\begin{itemize}
  \item \textbf{Embedded systems}: Firmware mikrokontroler, device driver, interrupt service routine
  \item \textbf{Bootloader dan BIOS/UEFI}: Kode awal yang dijalankan sebelum sistem operasi
  \item \textbf{Driver dan kernel}: Bagian kernel yang sensitif terhadap kinerja
  \item \textbf{Keamanan siber}: Analisis malware, exploit development, reverse engineering
  \item \textbf{Optimasi hotspot}: Bagian kecil dari aplikasi yang memerlukan latensi minimal
  \item \textbf{Pengembangan compiler}: Komponen runtime yang sensitif terhadap kinerja
\end{itemize}

\subsection{Contoh Program Assembly Sederhana}

Berikut contoh program assembly 8086 untuk mencetak karakter \cite{susanto1995belajar}:

\begin{lstlisting}[style=assemblystyle, caption={Program Assembly Sederhana - Output Karakter (A0.asm)}]
;   PROGRAM : A0.ASM
;   FUNGSI  : MENCETAK KARAKTER 'A' DENGAN INT 21h
.MODEL SMALL
.CODE
ORG 100h
Proses:
    MOV  AH, 02h   ; Nilai servis untuk mencetak karakter
    MOV  DL, 'A'   ; DL = Karakter ASCII yang akan dicetak
    INT  21h       ; Cetak karakter
END Proses
\end{lstlisting}

Program ini menunjukkan:
\begin{itemize}
  \item Penggunaan direktif \asm{.MODEL SMALL} untuk model memori kecil
  \item Penggunaan direktif \asm{ORG 100h} untuk menentukan starting address
  \item Instruksi \asm{MOV} untuk memindahkan data ke register
  \item Penggunaan interrupt \asm{INT 21h} dengan servis \asm{AH=02h} untuk output karakter
\end{itemize}

\section{Karir dan Aplikasi Assembly Programming}

Pemrograman assembly Intel 8086 membuka berbagai peluang karir di industri teknologi:

\begin{enumerate}
  \item \textbf{Systems Programming}: Pengembangan sistem operasi dan driver
  \item \textbf{Embedded Systems}: Programming untuk perangkat embedded
  \item \textbf{Firmware Development}: Software untuk hardware-software interface
  \item \textbf{Security Research}: Reverse engineering dan vulnerability analysis
  \item \textbf{Performance Engineering}: Optimasi sistem dan aplikasi
\end{enumerate}

\textbf{Industri Applications:}
\begin{itemize}
  \item \textbf{Automotive}: ECU programming, control systems
  \item \textbf{Aerospace}: Flight control systems, avionics
  \item \textbf{Medical Devices}: Equipment programming, diagnostics
  \item \textbf{IoT}: Device firmware, sensor programming
  \item \textbf{Gaming}: Engine development, optimization
  \item \textbf{Networking}: Protocol implementation, network drivers
\end{itemize}

\textbf{Alur Pembelajaran:}
\begin{itemize}
  \item Bab II-IV: Fundamental processor dan assembly concepts
  \item Bab V-VII: Programming constructs dan prosedur
  \item Bab VIII-X: System programming dan interrupts
  \item Bab XI-XII: Optimasi, debugging, dan project development
  \item Bab XIII-XIV: Advanced topics dan evaluasi komprehensif
\end{itemize}

\textbf{Skills Development:}
\begin{itemize}
  \item \textbf{Technical Skills}: Assembly programming, debugging, optimasi
  \item \textbf{Analytical Skills}: Problem solving, performance analysis
  \item \textbf{Soft Skills}: Documentation, collaboration, continuous learning
  \item \textbf{Industry Skills}: Code review, testing, maintenance
\end{itemize}


% ============================================================
% Rangkuman Bab
% ============================================================
\begin{rangkuman}
Bab ini memperkenalkan tujuan buku ajar, keterkaitan dengan RPS berbasis OBE, petunjuk penggunaan, dan konteks kurikulum OBE untuk mata kuliah Pemrograman Bahasa Tingkat Rendah. Pemahaman yang baik tentang struktur dan pendekatan buku ini akan membantu Anda memaksimalkan pembelajaran assembly language Intel 8086 dengan Turbo Assembler.

\textbf{Poin Kunci:}
\begin{itemize}
  \item Buku ini dirancang dengan pendekatan OBE yang fokus pada pencapaian kompetensi terukur dalam pemrograman tingkat rendah
  \item Setiap bab dipetakan ke Sub-CPMK yang berkontribusi pada CPL pemrograman sistem
  \item Gunakan komponen OBE (Sub-CPMK, aktivitas, latihan, asesmen, checklist) secara optimal
  \item Pembelajaran assembly tersusun sistematis dari arsitektur komputer hingga aplikasi praktik
  \item Fokus praktik 70\% dengan GUI Turbo Assembler (TASM) untuk pengalaman hands-on
\end{itemize}
\end{rangkuman}

\ifSubfilesClassLoaded{
  \renewcommand{\bibname}{Daftar Pustaka}
  \bibliographystyle{plain}
  \bibliography{../references}
}{}
\end{document}
