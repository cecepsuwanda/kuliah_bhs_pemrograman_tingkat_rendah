% ============================================================
% MATERI POKOK
% ============================================================
\section{Integrasi Konsep Assembly}

\subsection{Konsep Integrasi}
Integrasi adalah proses menggabungkan semua konsep assembly programming yang telah dipelajari menjadi aplikasi yang komprehensif.

\textbf{Komponen integrasi:}
\begin{itemize}
  \item \textbf{Processor Architecture}: Registers, addressing modes, segmentation
  \item \textbf{Instruction Set}: Data transfer, arithmetic, logic, control flow
  \item \textbf{Programming Constructs}: Procedures, loops, conditional statements
  \item \textbf{System Programming}: Interrupts, BIOS/DOS services
  \item \textbf{Optimization}: Performance tuning, debugging techniques
  \item \textbf{Project Development}: Modular design, library development
\end{itemize}

\subsection{Integrated Application Example}
Contoh aplikasi yang menggabungkan semua konsep assembly.

\begin{verbatim}
; Integrated Application: Student Management System
; ==================================================
; Features:
; 1. Student data management (arrays, file I/O)
; 2. Grade calculation (arithmetic operations)
; 3. Report generation (string operations, file output)
; 4. User interface (BIOS/DOS interrupts)
; 5. Data validation (conditional logic)
; 6. Performance optimization (register usage, loops)

; Main program structure
MAIN PROC
    ; Initialize system
    CALL initialize_system
    
    ; Load student data
    CALL load_student_data
    
    ; Display menu
    CALL display_menu
    
    ; Process user input
    CALL process_menu_selection
    
    ; Generate reports
    CALL generate_reports
    
    ; Cleanup and exit
    CALL cleanup_and_exit
MAIN ENDP

; Supporting procedures
initialize_system PROC
    ; Set up video mode
    MOV AH, 00h
    MOV AL, 03h
    INT 10h
    
    ; Initialize data structures
    MOV SI, student_array
    MOV CX, MAX_STUDENTS
    CALL initialize_student_array
    RET
initialize_system ENDP

load_student_data PROC
    ; Open data file
    MOV AH, 3Dh
    MOV AL, 00h
    MOV DX, student_data_file
    INT 21h
    JC file_error
    
    MOV file_handle, AX
    
    ; Read student records
    MOV CX, MAX_STUDENTS
    MOV SI, student_array
read_loop:
    CALL read_student_record
    ADD SI, STUDENT_SIZE
    LOOP read_loop
    
    ; Close file
    MOV AH, 3Eh
    MOV BX, file_handle
    INT 21h
    RET
load_student_data ENDP
\end{verbatim}

\section{Refleksi Pembelajaran}

\subsection{Perjalanan Pembelajaran Assembly}
Refleksi tentang proses pembelajaran assembly programming dari dasar hingga advanced.

\textbf{Tahapan pembelajaran:}
\begin{enumerate}
  \item \textbf{Foundation (Minggu 1-2)}: Konsep dasar processor dan assembly
  \item \textbf{Core Skills (Minggu 3-6)}: Instruksi, addressing, operasi dasar
  \item \textbf{Advanced Concepts (Minggu 7-10)}: Prosedur, interrupts, optimasi
  \item \textbf{Application (Minggu 11-12)}: Project development, best practices
  \item \textbf{Mastery (Minggu 13-14)}: Integration, assessment, reflection
\end{enumerate}

\subsection{Pencapaian Kompetensi}
Evaluasi pencapaian Sub-CPMK selama pembelajaran.

\begin{table}[htbp]
\centering
\begin{tabular}{|p{3cm}|p{2cm}|p{3cm}|p{3cm}|}
\hline
\textbf{Sub-CPMK} & \textbf{Target} & \textbf{Pencapaian} & \textbf{Action} \\
\hline
1.1: Processor Architecture & 85\% & 90\% & Maintain \\
\hline
1.2: Addressing Modes & 80\% & 85\% & Practice \\
\hline
2.1: TASM Syntax & 90\% & 95\% & Maintain \\
\hline
2.2: Assembler Directives & 75\% & 80\% & Review \\
\hline
3.1: Arithmetic Operations & 85\% & 90\% & Maintain \\
\hline
3.2: Logical Operations & 80\% & 85\% & Practice \\
\hline
4.1: Control Structures & 85\% & 90\% & Maintain \\
\hline
4.2: Procedures & 80\% & 85\% & Practice \\
\hline
\end{tabular}
\caption{Pencapaian Sub-CPMK}
\end{table}

\section{Best Practices Summary}

\subsection{Programming Best Practices}
Ringkasan praktik terbaik dalam assembly programming.

\textbf{Code Quality:}
\begin{itemize}
  \item Gunakan meaningful variable dan procedure names
  \item Dokumentasikan kode dengan komentar yang jelas
  \item Implementasikan error handling yang komprehensif
  \item Gunakan modular design untuk maintainability
  \item Optimasi kode untuk performa yang baik
\end{itemize}

\textbf{Development Process:}
\begin{itemize}
  \item Plan sebelum coding (requirements, design)
  \item Test secara bertahap (unit, integration, system)
  \item Debug secara sistematis dengan TASM debugger
  \item Review kode untuk best practices
  \item Document untuk maintainability
\end{itemize}

\subsection{Common Pitfalls dan Solutions}
Masalah umum dan solusinya dalam assembly programming.

\begin{table}[htbp]
\centering
\begin{tabular}{|p{4cm}|p{5cm}|p{4cm}|}
\hline
\textbf{Common Pitfall} & \textbf{Description} & \textbf{Solution} \\
\hline
Register Corruption & Mengubah register tanpa menyimpan nilai asli & Gunakan PUSH/POP untuk preservation \\
\hline
Stack Overflow & Rekursi tanpa base case atau terlalu dalam & Tambah base case, limit depth \\
\hline
Off-by-One Error & Loop count atau array index salah & Validasi boundary conditions \\
\hline
Memory Leaks & Tidak cleanup allocated resources & Implement proper cleanup \\
\hline
Poor Performance & Inefficient algorithms atau data structures & Use profiling, optimize \\
\hline
Hard to Debug & Kode tidak terorganisir dengan baik & Modular design, good documentation \\
\hline
\end{tabular}
\caption{Common Pitfalls dan Solutions}
\end{table}

\section{Career Pathways}

\subsection{Career Opportunities}
Peluang karir untuk mahasiswa dengan keahlian assembly programming.

\textbf{Technical Roles:}
\begin{itemize}
  \item \textbf{Systems Programmer}: Low-level system development
  \item \textbf{Embedded Systems Engineer}: Device programming
  \item \textbf{Firmware Developer}: Hardware-software interface
  \item \textbf{Security Researcher}: Reverse engineering, vulnerability analysis
  \item \textbf{Performance Engineer}: System optimization
\end{itemize}

\textbf{Industry Applications:}
\begin{itemize}
  \item \textbf{Automotive}: ECU programming, control systems
  \item \textbf{Aerospace}: Flight control systems, avionics
  \item \textbf{Medical Devices}: Equipment programming, diagnostics
  \item \textbf{IoT}: Device firmware, sensor programming
  \item \textbf{Gaming}: Engine development, optimization
\end{itemize}

\subsection{Continuous Learning}
Strategi untuk pembelajaran berkelanjutan.

\textbf{Advanced Topics:}
\begin{itemize}
  \item Modern processor architectures (x86-64, ARM)
  \item Operating system internals
  \item Compiler design and construction
  \item Computer architecture and organization
  \item Parallel and distributed computing
\end{itemize}

\textbf{Learning Resources:}
\begin{itemize}
  \item Intel Developer Manuals
  \item AMD Processor Programming Guides
  \item Open source assembly projects
  \item Online communities and forums
  \item Technical conferences and workshops
\end{itemize}
