% ============================================================
% AKTIVITAS PEMBELAJARAN
% ============================================================

\begin{aktivitas}
  \item \textbf{Integration Project}: Buat aplikasi lengkap yang menggabungkan semua konsep assembly.
  
  \item \textbf{Self-Reflection}: Lakukan refleksi pembelajaran dan identifikasi area improvement.
  
  \item \textbf{Best Practices Review}: Review kode assembly yang telah dibuat untuk best practices.
  
  \item \textbf{Career Planning}: Rancang career path berdasarkan keahlian assembly programming.
  
  \item \textbf{Portfolio Completion}: Finalisasi portfolio dengan semua project terbaik.
  
  \item \textbf{Knowledge Sharing}: Bagikan pengalaman dan tips dengan mahasiswa lain.
\end{aktivitas}

% ============================================================
% LATIHAN DAN REFLEKSI
% ============================================================

\begin{latihan}
  \item Buat aplikasi database management system yang mengintegrasikan semua konsep assembly.
  
  \item Lakukan self-assessment untuk semua Sub-CPMK dan buat action plan untuk improvement.
  
  \item Identifikasi dan dokumentasikan 5 best practices yang telah Anda pelajari.
  
  \item Rancang career development plan untuk 5 tahun ke depan.
  
  \item Buat comprehensive portfolio dengan semua project dan pencapaian.
  
  \item Tulis reflection essay tentang perjalanan pembelajaran assembly programming.
  
  \item Buat knowledge sharing presentation untuk mahasiswa angkatan berikutnya.
  
  \item \textbf{Refleksi}: Konsep assembly mana yang paling bermanfaat untuk karir Anda? Bagaimana Anda mengaplikasikannya?
\end{latihan}

% ============================================================
% ASESMEN
% ============================================================

\begin{asesmen}
\textbf{Instrumen Penilaian untuk Sub-CPMK 10.1, 10.2}

\textbf{A. Pilihan Ganda}

\begin{enumerate}
  \item Tujuan utama integrasi konsep assembly adalah:
  \begin{enumerate}
    \item Membuat program yang kompleks
    \item Menggabungkan semua konsep yang dipelajari
    \item Meningkatkan performa program
    \item Menunjukkan kemampuan individual
  \end{enumerate}
  
  \item Komponen penting dalam refleksi pembelajaran adalah:
  \begin{enumerate}
    \item Identifikasi kekuatan dan kelemahan
    \item Membuat action plan untuk improvement
    \item Mengevaluasi pencapaian kompetensi
    \item Semua jawaban benar
  \end{enumerate}
  
  \item Best practice yang paling penting dalam assembly programming adalah:
  \begin{enumerate}
    \item Code documentation
    \item Performance optimization
    \item Error handling
    \item Modular design
  \end{enumerate}
  
  \item Career pathway yang paling relevan untuk assembly programming adalah:
  \begin{enumerate}
    \item Web development
    \item Mobile development
    \item Systems programming
    \item Database administration
  \end{enumerate}
\end{enumerate}

\textbf{B. Essay}

\begin{enumerate}
  \item Jelaskan bagaimana integrasi konsep assembly penting untuk pengembangan karir!
  
  \item Mengapa refleksi pembelajaran penting untuk continuous improvement?
\end{enumerate}

\textbf{C. Practical Challenge}

\begin{enumerate}
  \item Buat capstone project:
  \begin{itemize}
    \item Integrated application yang menggabungkan semua konsep assembly
    \item Comprehensive documentation dan user manual
    \item Performance analysis dan optimization report
    \item Testing framework dengan unit tests
    \item Deployment guide dan installation instructions
    \item Maintenance manual dan troubleshooting guide
    \item Knowledge transfer documentation
    \item Presentation materials untuk demo
    \item Code review checklist dan quality assurance
    \item Future enhancement recommendations
  \end{itemize}
\end{enumerate}

\textbf{Rubrik Penilaian}: Lihat Lampiran A
\end{asesmen}

% ============================================================
% CHECKLIST KOMPETENSI
% ============================================================

\begin{checklist}
  \item Saya dapat mengintegrasikan semua konsep assembly programming
  \item Saya dapat melakukan refleksi pembelajaran yang mendalam
  \item Saya dapat mengidentifikasi best practices dalam assembly programming
  \item Saya dapat merancang career development yang realistis
  \item Saya dapat membuat portfolio yang komprehensif
  \item Saya dapat berbagi pengetahuan dengan orang lain
  \item Saya dapat merencanakan continuous learning
\end{checklist}

% ============================================================
% RANGKUMAN
% ============================================================

\begin{rangkuman}
Bab ini membahas integrasi dan refleksi dalam assembly programming, termasuk penggabungan konsep, evaluasi pembelajaran, dan persiapan karir.

\textbf{Poin Kunci:}
\begin{itemize}
  \item Integrasi menggabungkan semua konsep assembly menjadi aplikasi komprehensif
  \item Refleksi penting untuk continuous improvement
  \item Best practices memastikan kode berkualitas dan maintainable
  \item Career pathways tersedia untuk berbagai industri
  \item Continuous learning essential untuk tetap relevan
  \item Portfolio mendokumentasikan pencapaian kompetensi
  \item Knowledge sharing memperluas dampak pembelajaran
  \item Action plan memastikan development berkelanjutan
\end{itemize}

\textbf{Kata Kunci}: \asm{Integrasi}, \asm{Refleksi}, \asm{Best Practices}, \asm{Career Path}, \asm{Continuous Learning}, \asm{Portfolio}, \asm{Knowledge Sharing}, \asm{Action Plan}
\end{rangkuman}
