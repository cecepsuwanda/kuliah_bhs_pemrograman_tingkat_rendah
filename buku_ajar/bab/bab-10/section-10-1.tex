% ============================================================
% MATERI POKOK
% ============================================================
\section{Project Development Lifecycle}

\subsection{Konsep Project Development}
Project development dalam assembly programming memerlukan pendekatan sistematis untuk menghasilkan program yang efisien dan maintainable.

\textbf{Tahapan project development:}
\begin{itemize}
  \item \textbf{Requirements Analysis}: Memahami kebutuhan dan constraints
  \item \textbf{Design}: Merancang arsitektur dan algoritma
  \item \textbf{Implementation}: Menulis kode assembly
  \item \textbf{Testing}: Debugging dan validasi
  \item \textbf{Optimization}: Meningkatkan performa
  \item \textbf{Documentation}: Menulis dokumentasi
\end{itemize}

\subsection{Project Planning}
Perencanaan yang baik adalah kunci keberhasilan project assembly.

\begin{verbatim}
; Project structure template
; -----------------------
; main.asm          - Entry point
; constants.inc     - Constants dan definitions
; procedures.inc    - Procedure declarations
; data.inc          - Data definitions
; utils.asm         - Utility procedures
; io.asm            - Input/output procedures
; math.asm          - Mathematical operations
\end{verbatim}

\textbf{Komponen project planning:}
\begin{itemize}
  \item Definisi scope dan requirements
  \item Pemilihan algoritma yang tepat
  \item Estimasi waktu dan resources
  \item Risk assessment
  \item Milestone definition
\end{itemize}

\section{Modular Programming}

\subsection{Modular Design Principles}
Membagi program menjadi modul-modul yang terpisah untuk maintainability.

\begin{verbatim}
; Module structure example
; math.asm - Mathematical operations module

PUBLIC add_numbers, subtract_numbers, multiply_numbers
EXTERN print_result

; Code segment
CODE SEGMENT
    ASSUME CS:CODE, DS:DATA

add_numbers PROC
    ; Implementation
    RET
add_numbers ENDP

subtract_numbers PROC
    ; Implementation
    RET
subtract_numbers ENDP

multiply_numbers PROC
    ; Implementation
    RET
multiply_numbers ENDP

CODE ENDS
END
\end{verbatim}

\textbf{Keuntungan modular programming:}
\begin{itemize}
  \item Code reuse dan maintainability
  \item Parallel development
  \item Easier testing dan debugging
  \item Better organization
\end{itemize}

\subsection{Interface Design}
Mendefinisikan interface yang jelas antar modul.

\begin{verbatim}
; Interface definitions
; io_interface.inc

; Input procedures
PUBLIC read_char, read_string, read_number
read_char PROC NEAR
    ; Returns character in AL
    RET
read_char ENDP

read_string PROC NEAR
    ; DS:DX = buffer address
    RET
read_string ENDP

; Output procedures
PUBLIC print_char, print_string, print_number
print_char PROC NEAR
    ; AL = character to print
    RET
print_char ENDP
\end{verbatim}

\section{Library Development}

\subsection{Creating Reusable Libraries}
Mengembangkan library untuk fungsi-fungsi yang sering digunakan.

\begin{verbatim}
; Standard library structure
; stdlib.asm

; String operations
PUBLIC strlen, strcpy, strcmp, strcat
strlen PROC NEAR
    PUSH SI
    MOV SI, DX        ; DX = string address
    MOV CX, 0
count_loop:
    CMP BYTE PTR [SI], 0
    JE end_count
    INC CX
    INC SI
    JMP count_loop
end_count:
    MOV AX, CX        ; Return length in AX
    POP SI
    RET
strlen ENDP

strcpy PROC NEAR
    ; DS:SI = source, ES:DI = destination
    PUSH SI
    PUSH DI
copy_loop:
    MOV AL, [SI]
    MOV [DI], AL
    CMP AL, 0
    JE end_copy
    INC SI
    INC DI
    JMP copy_loop
end_copy:
    POP DI
    POP SI
    RET
strcpy ENDP
\end{verbatim}

\subsection{Library Organization}
Mengorganisir library untuk kemudahan penggunaan.

\textbf{Kategori library:}
\begin{itemize}
  \item \textbf{String Library}: strlen, strcpy, strcmp, strcat
  \item \textbf{Math Library}: add, subtract, multiply, divide, power
  \item \textbf{I/O Library}: \texttt{read\_char}, \texttt{print\_string}, file operations
  \item \textbf{Utility Library}: memory operations, conversions
  \item \textbf{Graphics Library}: pixel operations, line drawing
\end{itemize}

\section{Best Practices}

\subsection{Code Organization}
Praktik terbaik untuk organisasi kode assembly.

\begin{verbatim}
; Standard file organization
; ------------------------
; 1. Header comments
; 2. Include files
; 3. Constants and equates
; 4. Data segment
; 5. Code segment
; 6. Procedures
; 7. Entry point

; Example structure
; main.asm

; ============================================================
; PROGRAM: Calculator Application
; AUTHOR: Your Name
; DATE: Current Date
; DESCRIPTION: Simple calculator with basic operations
; ============================================================

INCLUDE constants.inc
INCLUDE procedures.inc

; Data segment
DATA SEGMENT
    prompt_msg DB 'Enter expression: $'
    result_msg DB 'Result: $'
    buffer DB 80 DUP(?)
DATA ENDS

; Code segment
CODE SEGMENT
    ASSUME CS:CODE, DS:DATA

start:
    ; Initialize DS
    MOV AX, DATA
    MOV DS, AX
    
    ; Main program logic
    CALL main_loop
    
    ; Exit program
    MOV AH, 4Ch
    INT 21h

main_loop PROC
    ; Implementation
    RET
main_loop ENDP

CODE ENDS
END start
\end{verbatim}

\subsection{Naming Conventions}
Konvensi penamaan untuk konsistensi kode.

\textbf{Recommended conventions:}
\begin{itemize}
  \item \textbf{Procedures}: camelCase (calculateSum, printResult)
  \item \textbf{Variables}: snake\_case (user\_input, buffer\_ptr)
  \item \textbf{Constants}: UPPER\_CASE (MAX\_SIZE, ERROR\_CODE)
  \item \textbf{Labels}: descriptive\_name (loop\_start, error\_handler)
\end{itemize}

\subsection{Documentation Standards}
Standar dokumentasi untuk kode assembly.

\begin{verbatim}
; Procedure documentation template
; -------------------------------------------------
; Procedure: calculateSum
; Purpose: Calculate sum of two numbers
; Input: AX = first number, BX = second number
; Output: AX = sum
; Modifies: AX, flags
; Calls: None
; -------------------------------------------------
calculateSum PROC
    ADD AX, BX       ; Calculate sum
    RET
calculateSum ENDP
\end{verbatim}

\section{Version Control}

\subsection{Source Code Management}
Mengelola versi kode assembly dengan version control.

\textbf{Best practices:}
\begin{itemize}
  \item Gunakan meaningful commit messages
  \item Commit frequently dengan logical units
  \item Gunakan branching untuk features
  \item Review code sebelum merge
  \item Maintain changelog
\end{itemize}

\subsection{Backup and Recovery}
Strategi backup dan recovery untuk project assembly.

\begin{verbatim}
; Backup strategy
; -------------
; 1. Daily automatic backup
; 2. Weekly full backup
; 3. Version control repository
; 4. Offsite backup
; 5. Documentation backup
\end{verbatim}
