% ============================================================
% AKTIVITAS PEMBELAJARAN
% ============================================================

\begin{aktivitas}
  \item \textbf{Project Planning}: Buat project plan untuk aplikasi calculator dengan modular design.
  
  \item \textbf{Library Development}: Kembangkan string library dengan fungsi strlen, strcpy, strcmp.
  
  \item \textbf{Modular Design}: Implementasikan text editor dengan modular architecture.
  
  \item \textbf{Code Organization}: Reorganize existing code mengikuti best practices.
  
  \item \textbf{Documentation}: Tulis dokumentasi lengkap untuk assembly project.
  
  \item \textbf{Version Control}: Setup version control untuk assembly project development.
\end{aktivitas}

% ============================================================
% LATIHAN DAN REFLEKSI
% ============================================================

\begin{latihan}
  \item Buat project plan untuk game sederhana dengan modular architecture.
  
  \item Kembangkan math library dengan operasi dasar dan fungsi trigonometri.
  
  \item Implementasikan file I/O library dengan error handling yang baik.
  
  \item Buat documentation template untuk assembly project.
  
  \item Design modular architecture untuk database management system.
  
  \item Implementasikan graphics library untuk drawing operations.
  
  \item Buat build system untuk multi-module assembly project.
  
  \item \textbf{Refleksi}: Aspek mana dari project development yang paling menantang? Bagaimana Anda mengatasi kesulitan tersebut?
\end{latihan}

% ============================================================
% ASESMEN
% ============================================================

\begin{asesmen}
\textbf{Instrumen Penilaian untuk Sub-CPMK 8.1, 8.2}

\textbf{A. Pilihan Ganda}

\begin{enumerate}
  \item Tujuan utama modular programming adalah:
  \begin{enumerate}
    \item Meningkatkan ukuran kode
    \item Memudahkan maintainability dan reuse
    \item Mengurangi performa
    \item Memperumit testing
  \end{enumerate}
  
  \item Directive untuk membuat procedure visible ke modul lain adalah:
  \begin{enumerate}
    \item EXTERN
    \item PUBLIC
    \item GLOBAL
    \item EXPORT
  \end{enumerate}
  
  \item Bagian pertama yang harus ada dalam assembly file adalah:
  \begin{enumerate}
    \item Data segment
    \item Code segment
    \item Header comments
    \item Include files
  \end{enumerate}
  
  \item Naming convention yang direkomendasikan untuk procedures adalah:
  \begin{enumerate}
    \item UPPERCASE
    \item lowercase
    \item camelCase
    \item snake\_case
  \end{enumerate}
  
  \item Urutan tahapan project development lifecycle yang benar adalah:
  \begin{enumerate}
    \item Requirements Analysis $\rightarrow$ Design $\rightarrow$ Implementation $\rightarrow$ Testing $\rightarrow$ Optimization $\rightarrow$ Documentation
    \item Implementation $\rightarrow$ Testing $\rightarrow$ Design
    \item Design $\rightarrow$ Implementation $\rightarrow$ Requirements
    \item Documentation $\rightarrow$ Implementation $\rightarrow$ Testing
  \end{enumerate}
  
  \item Pada tahap Requirements Analysis untuk project assembly, yang dianalisis antara lain:
  \begin{enumerate}
    \item Kebutuhan fungsional, constraints (ukuran .COM 64KB, interupsi DOS/BIOS)
    \item Hanya nama file
    \item Hanya warna tampilan
    \item Tidak ada yang spesifik
  \end{enumerate}
\end{enumerate}

\textbf{B. Essay}

\begin{enumerate}
  \item Jelaskan keuntungan modular programming dalam assembly development!
  
  \item Mengapa documentation penting dalam assembly programming?
  
  \item Jelaskan tahapan Design untuk project assembly! Apa yang perlu dirancang dalam hal register usage, model memori, dan struktur prosedur?
  
  \item Mengapa Testing dengan Turbo Debugger penting dalam project development assembly?
\end{enumerate}

\textbf{C. Practical Challenge}

\begin{enumerate}
  \item Buat comprehensive project:
  \begin{itemize}
    \item Text editor dengan modular architecture
    \item String library (strlen, strcpy, strcmp, strcat)
    \item File I/O operations
    \item Search dan replace functionality
    \item Multiple file handling
    \item Configuration management
    \item Complete documentation
    \item Build system dengan makefile
    \item Version control integration
    \item Testing framework
  \end{itemize}
  
  \item Untuk proyek yang dikerjakan, buat penjelasan singkat (1–2 kalimat) per tahap project development lifecycle: Requirements, Design, Implementation, Testing, Optimization, Documentation.
\end{enumerate}

\textbf{Rubrik Penilaian}: Lihat Lampiran A
\end{asesmen}

% ============================================================
% CHECKLIST KOMPETENSI
% ============================================================

\begin{checklist}
  \item Saya dapat merancang project assembly dengan struktur yang baik
  \item Saya dapat mengimplementasikan modular programming
  \item Saya dapat mengembangkan reusable libraries
  \item Saya dapat mengorganisir kode assembly dengan best practices
  \item Saya dapat menulis dokumentasi yang komprehensif
  \item Saya dapat menggunakan version control untuk assembly project
  \item Saya dapat membuat build system untuk multi-module project
  \item Saya dapat menerapkan quality assurance dalam assembly development
\end{checklist}

% ============================================================
% RANGKUMAN
% ============================================================

\begin{rangkuman}
Bab ini membahas project development dalam assembly programming, termasuk modular programming, library development, dan best practices.

\textbf{Poin Kunci:}
\begin{itemize}
  \item Project development lifecycle memerlukan perencanaan sistematis
  \item Modular programming meningkatkan maintainability dan code reuse
  \item Library development menyediakan fungsi-fungsi yang reusable
  \item Best practices memastikan kode yang organized dan maintainable
  \item Documentation penting untuk long-term maintenance
  \item Version control memungkinkan collaboration dan tracking
  \item Code organization mengikuti standar industri
  \item Quality assurance memastikan deliverables yang berkualitas
\end{itemize}

\textbf{Kata Kunci}: \asm{Project Development}, \asm{Modular Programming}, \asm{Library}, \asm{Documentation}, \asm{Version Control}, \asm{Best Practices}, \asm{Code Organization}, \asm{Build System}
\end{rangkuman}
