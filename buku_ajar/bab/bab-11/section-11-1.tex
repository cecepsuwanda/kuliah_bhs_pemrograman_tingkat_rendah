% ============================================================
% MATERI POKOK
% ============================================================
\section{Jenis-jenis Asesmen}

\subsection{Konsep Asesmen dalam Assembly Programming}
Asesmen adalah proses pengukuran pencapaian kompetensi mahasiswa dalam pemrograman assembly.

\textbf{Tujuan asesmen:}
\begin{itemize}
  \item Mengukur pemahaman konsep assembly
  \item Mengevaluasi kemampuan praktis programming
  \item Memberikan feedback untuk improvement
  \item Menentukan pencapaian Sub-CPMK
\end{itemize}

\subsection{Formative Assessment}
Asesmen formatif untuk monitoring progress pembelajaran.

\begin{verbatim}
; Contoh formatif assessment - debugging exercise
; Student harus menemukan dan memperbaiki bug

; Bug code (diberikan ke mahasiswa)
buggy_code PROC
    MOV AX, [array]  ; Bug: tidak mengecek array bounds
    ADD AX, [array+2]
    ADD AX, [array+4]
    ; ... potensial array overflow
    RET
buggy_code ENDP

; Expected fix (mahasiswa harus menemukan)
fixed_code PROC
    MOV CX, array_size
    MOV SI, 0
    MOV AX, 0
sum_loop:
    CMP SI, CX
    JGE end_sum
    ADD AX, [array+SI]
    ADD SI, 2
    JMP sum_loop
end_sum:
    RET
fixed_code ENDP
\end{verbatim}

\textbf{Formatif assessment types:}
\begin{itemize}
  \item \textbf{Code Review}: Review kode assembly untuk best practices
  \item \textbf{Debugging Exercises}: Temukan dan perbaiki bug
  \item \textbf{Pair Programming}: Kolaborasi debugging
  \item \textbf{Quick Quiz}: Pengetahuan konseptual singkat
\end{itemize}

\subsection{Summative Assessment}
Asesmen sumatif untuk evaluasi akhir kompetensi.

\begin{verbatim}
; Contoh summative assessment - comprehensive project
; Mahasiswa harus implementasikan program lengkap

; Project requirements:
; 1. Text editor dengan fitur:
;    - File operations (open, save, close)
;    - Text editing (insert, delete, replace)
;    - Search functionality
;    - Menu system
; 2. Modular design dengan library
; 3. Documentation lengkap
; 4. Error handling
; 5. Performance optimization
\end{verbatim}

\textbf{Summative assessment components:}
\begin{itemize}
  \item \textbf{Practical Exam}: Implementasi program assembly
  \item \textbf{Written Exam}: Konseptual understanding
  \item \textbf{Project}: Development project lengkap
  \item \textbf{Presentation}: Demo dan explanation
\end{itemize}

\section{Rubrik Penilaian}

\subsection{Rubrik untuk Kode Assembly}
Standar penilaian untuk kualitas kode assembly.

\begin{table}[htbp]
\centering
\begin{tabular}{|p{2cm}|p{3cm}|p{3cm}|p{3cm}|p{2cm}|}
\hline
\textbf{Kriteria} & \textbf{Excellent (4)} & \textbf{Good (3)} & \textbf{Fair (2)} & \textbf{Poor (1)} \\
\hline
Correctness & No bugs, perfect functionality & Minor bugs, good functionality & Some bugs, basic functionality & Major bugs, poor functionality \\
\hline
Efficiency & Optimized, minimal cycles & Good optimization & Basic optimization & No optimization \\
\hline
Readability & Well documented, clear structure & Good documentation & Basic documentation & Poor/no documentation \\
\hline
Modularity & Excellent modular design & Good modular design & Basic modularization & No modularity \\
\hline
Error Handling & Comprehensive error handling & Good error handling & Basic error handling & No error handling \\
\hline
\end{tabular}
\caption{Rubrik Penilaian Kode Assembly}
\end{table}

\subsection{Rubrik untuk Problem Solving}
Standar penilaian untuk kemampuan problem solving.

\begin{table}[htbp]
\centering
\begin{tabular}{|p{2cm}|p{3cm}|p{3cm}|p{3cm}|p{2cm}|}
\hline
\textbf{Kriteria} & \textbf{Excellent (4)} & \textbf{Good (3)} & \textbf{Fair (2)} & \textbf{Poor (1)} \\
\hline
Analysis & Deep understanding of problem & Good understanding & Basic understanding & Poor understanding \\
\hline
Algorithm Design & Optimal algorithm design & Good algorithm design & Basic algorithm design & Poor algorithm design \\
\hline
Implementation & Flawless implementation & Good implementation & Basic implementation & Poor implementation \\
\hline
Optimization & Excellent optimization & Good optimization & Basic optimization & No optimization \\
\hline
Testing & Comprehensive testing & Good testing & Basic testing & No testing \\
\hline
\end{tabular}
\caption{Rubrik Penilaian Problem Solving}
\end{table}

\section{Instrumen Asesmen}

\subsection{Practical Exam Template}
Template untuk ujian praktik assembly programming.

\begin{verbatim}
; Practical Exam Template
; ====================
; Duration: 120 minutes
; Total Points: 100

; Problem 1: String Manipulation (25 points)
; Implementasikan fungsi string reverse dengan optimasi

; Problem 2: Array Processing (25 points)
; Buat program untuk sorting array dengan bubble sort

; Problem 3: File Operations (25 points)
; Implementasikan file reader dengan error handling

; Problem 4: System Programming (25 points)
; Buat program yang menampilkan informasi sistem

; Evaluation Criteria:
; - Correctness: 40%
; - Efficiency: 20%
; - Documentation: 20%
; - Error Handling: 20%
\end{verbatim}

\subsection{Written Exam Template}
Template untuk ujian tertulis konseptual.

\textbf{Sample Questions:}

\begin{enumerate}
  \item Jelaskan perbedaan antara CALL dan RET instruksi! (10 points)
  
  \item Implementasikan algoritma binary search dalam assembly! (20 points)
  
  \item Analisis kode berikut dan identifikasi optimasi opportunities: (15 points)
  \begin{verbatim}
  MOV CX, 100
  MOV SI, 0
  loop_start:
      MOV AL, [array+SI]
      CMP AL, 0
      JE found
      INC SI
      LOOP loop_start
  \end{verbatim}
  
  \item Jelaskan konsep stack frame dan manajemen register dalam prosedur! (15 points)
  
  \item Design modular architecture untuk calculator application! (20 points)
  
  \item Jelaskan optimasi techniques yang dapat diterapkan pada kode assembly! (20 points)
\end{enumerate}

\section{Self-Assessment Tools}

\subsection{Checklist Kompetensi}
Checklist untuk self-assessment mahasiswa.

\textbf{Technical Skills Checklist:}
\begin{itemize}
  \item[$\square$] Saya dapat menulis program assembly dasar
  \item[$\square$] Saya dapat menggunakan instruksi aritmatika dan logika
  \item[$\square$] Saya dapat mengimplementasikan struktur kontrol
  \item[$\square$] Saya dapat membuat prosedur dengan parameter
  \item[$\square$] Saya dapat menggunakan interupsi BIOS/DOS
  \item[$\square$] Saya dapat melakukan debugging dengan TASM
  \item[$\square$] Saya dapat mengoptimasi kode assembly
  \item[$\square$] Saya dapat mengembangkan modular program
\end{itemize}

\textbf{Problem Solving Checklist:}
\begin{itemize}
  \item[$\square$] Saya dapat menganalisis problem requirements
  \item[$\square$] Saya dapat merancang algoritma yang efisien
  \item[$\square$] Saya dapat mengidentifikasi dan memperbaiki bug
  \item[$\square$] Saya dapat mengoptimasi performa kode
  \item[$\square$] Saya dapat mendokumentasikan solusi
  \item[$\square$] Saya dapat menguji dan validasi solusi
\end{itemize}

\subsection{Portfolio Assessment}
Portfolio untuk dokumentasi pencapaian kompetensi.

\textbf{Portfolio Components:}
\begin{itemize}
  \item \textbf{Code Samples}: 5 best assembly programs
  \item \textbf{Projects}: 2-3 complete projects
  \item \textbf{Debugging Logs}: Bug fixing examples
  \item \textbf{Optimization Reports}: Before/after comparisons
  \item \textbf{Documentation}: Complete documentation
  \item \textbf{Reflection}: Learning journey reflection
\end{itemize}

\begin{verbatim}
; Portfolio Entry Template
; =====================
; Project: Calculator Application
; Date: [Current Date]
; Skills Demonstrated:
; - Modular programming
; - File I/O operations
; - User interface design
; - Error handling
; - Performance optimization
; 
; Lessons Learned:
; - Importance of modular design
; - Debugging techniques
; - Optimization strategies
; 
; Future Improvements:
; - Add advanced functions
; - Improve user interface
; - Add configuration options
\end{verbatim}
