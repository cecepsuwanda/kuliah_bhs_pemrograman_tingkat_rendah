% ============================================================
% AKTIVITAS PEMBELAJARAN
% ============================================================

\begin{aktivitas}
  \item \textbf{Assessment Design}: Buat rubrik penilaian untuk assembly programming project.
  
  \item \textbf{Peer Review}: Implementasikan sistem peer review untuk kode assembly.
  
  \item \textbf{Self-Assessment}: Gunakan checklist kompetensi untuk self-evaluation.
  
  \item \textbf{Portfolio Development}: Buat portfolio untuk dokumentasi pencapaian.
  
  \item \textbf{Exam Creation}: Rancang ujian praktik dan tertulis untuk assembly programming.
  
  \item \textbf{Feedback System}: Implementasikan sistem feedback yang konstruktif.
\end{aktivitas}

% ============================================================
% LATIHAN DAN REFLEKSI
% ============================================================

\begin{latihan}
  \item Buat rubrik penilaian lengkap untuk text editor assembly project.
  
  \item Rancang ujian praktik 120 menit untuk mengevaluasi kompetensi assembly.
  
  \item Implementasikan sistem peer review dengan template yang standar.
  
  \item Buat checklist self-assessment untuk Sub-CPMK 1.1-1.3.
  
  \item Design portfolio assessment untuk semester project.
  
  \item Buat template feedback untuk memberikan komentar konstruktif.
  
  \item Implementasikan automated testing framework untuk assembly programs.
  
  \item \textbf{Refleksi}: Aspek mana dari assessment yang paling sulit diimplementasikan? Bagaimana Anda mengatasi kesulitan tersebut?
\end{latihan}

% ============================================================
% ASESMEN
% ============================================================

\begin{asesmen}
\textbf{Instrumen Penilaian untuk Sub-CPMK 9.1, 9.2}

\textbf{A. Pilihan Ganda}

\begin{enumerate}
  \item Tujuan utama formative assessment adalah:
  \begin{enumerate}
    \item Evaluasi akhir kompetensi
    \item Monitoring progress pembelajaran
    \item Memberikan nilai akhir
    \item Seleksi mahasiswa terbaik
  \end{enumerate}
  
  \item Komponen penting dalam rubrik penilaian kode assembly adalah:
  \begin{enumerate}
    \item Correctness dan efficiency
    \item Documentation dan modularity
    \item Error handling dan testing
    \item Semua jawaban benar
  \end{enumerate}
  
  \item Self-assessment tool yang paling efektif adalah:
  \begin{enumerate}
    \item Multiple choice quiz
    \item Checklist kompetensi
    \item Essay questions
    \item Practical exam
  \end{enumerate}
  
  \item Portfolio assessment mengevaluasi:
  \begin{enumerate}
    \item Pencapaian kompetensi overtime
    \item Pengetahuan teoretis saja
    \item Kemampuan memorisasi
    \item Kehadiran di kelas
  \end{enumerate}
\end{enumerate}

\textbf{B. Essay}

\begin{enumerate}
  \item Jelaskan perbedaan antara formative dan summative assessment dalam assembly programming!
  
  \item Mengapa rubrik penilaian penting untuk evaluasi kode assembly?
\end{enumerate}

\textbf{C. Practical Challenge}

\begin{enumerate}
  \item Buat comprehensive assessment system:
  \begin{itemize}
    \item Rubrik penilaian untuk berbagai tipe tugas
    \item Template feedback yang konstruktif
    \item Self-assessment checklist untuk semua Sub-CPMK
    \item Portfolio assessment framework
    \item Peer review system dengan guidelines
    \item Exam templates (practical dan written)
    \item Automated testing framework
    \item Documentation untuk assessment process
    \item Training materials untuk assessors
    \item Quality assurance procedures
  \end{itemize}
\end{enumerate}

\textbf{Rubrik Penilaian}: Lihat Lampiran A
\end{asesmen}

% ============================================================
% CHECKLIST KOMPETENSI
% ============================================================

\begin{checklist}
  \item Saya dapat merancang assessment instruments yang efektif
  \item Saya dapat membuat rubrik penilaian yang komprehensif
  \item Saya dapat mengimplementasikan formative assessment
  \item Saya dapat mengembangkan summative assessment
  \item Saya dapat menggunakan self-assessment tools
  \item Saya dapat mengimplementasikan peer review systems
  \item Saya dapat mengevaluasi portfolio assessment
  \item Saya dapat memberikan feedback yang konstruktif
\end{checklist}

% ============================================================
% RANGKUMAN
% ============================================================

\begin{rangkuman}
Bab ini membahas asesmen dan evaluasi dalam assembly programming, termasuk jenis-jenis asesmen, rubrik penilaian, dan self-assessment tools.

\textbf{Poin Kunci:}
\begin{itemize}
  \item Asesmen mengukur pencapaian kompetensi mahasiswa
  \item Formative assessment untuk monitoring progress
  \item Summative assessment untuk evaluasi akhir
  \item Rubrik penilaian memastikan evaluasi yang objektif
  \item Self-assessment tools untuk refleksi diri
  \item Portfolio assessment untuk dokumentasi pencapaian
  \item Peer review untuk collaborative learning
  \item Feedback yang konstruktif untuk improvement
\end{itemize}

\textbf{Kata Kunci}: \asm{Asesmen}, \asm{Evaluasi}, \asm{Rubrik}, \asm{Formative}, \asm{Summative}, \asm{Self-Assessment}, \asm{Portfolio}, \asm{Peer Review}, \asm{Feedback}
\end{rangkuman}
