\documentclass[../main.tex]{subfiles}
\ifSubfilesClassLoaded{\setcounter{chapter}{6}}{}
\begin{document}

\chapter{Prosedur dan Manajemen Stack}

\begin{subcpmk}
  \item Sub-CPMK 4.1: Mengembangkan prosedur dengan parameter passing dan manajemen stack (PUSH, POP, CALL, RET)
  \item Sub-CPMK 2.1: Menulis program assembly sederhana dengan TASM syntax
\end{subcpmk}

% ============================================================
% MATERI POKOK
% ============================================================
% ============================================================
% MATERI POKOK
% ============================================================
\section{Prosedur dan Fungsi}

\subsection{Konsep Prosedur}
Prosedur adalah blok kode yang dapat dipanggil berulang kali untuk modular programming.

\begin{verbatim}
; Prosedur definition
procedure_name PROC
    ; ... body of procedure
    RET
procedure_name ENDP

; Memanggil prosedur
CALL procedure_name
\end{verbatim}

\textbf{Keuntungan prosedur:}
\begin{itemize}
  \item Code reuse - menghindari duplikasi kode
  \item Modular design - memecah program menjadi bagian-bagian kecil
  \item Maintainability - lebih mudah diperbaiki dan diperbarui
  \item Readability - struktur program lebih jelas
\end{itemize}

\subsection{CALL dan RET Instructions}
Instruksi fundamental untuk prosedur management.

\begin{verbatim}
; CALL instruction
CALL procedure_name    ; Push return address, jump ke procedure

; RET instruction  
RET                 ; Pop return address, jump back
RET n               ; Pop return address, add n to SP (cleanup)
\end{verbatim}

\textbf{CALL operation:}
\begin{enumerate}
  \item Push return address (IP) ke stack
  \item Jump ke alamat prosedur
  \item Eksekusi prosedur
  \item RET pop return address dari stack
  \item Jump kembali ke caller
\end{enumerate}

\subsection{Parameter Passing}
Berbagai cara untuk passing parameter ke prosedur.

\subsubsection{Passing via Registers}
\begin{verbatim}
; Caller
MOV AX, param1
MOV BX, param2
CALL calculate
; Result di AX

; Procedure
calculate PROC
    ; AX dan BX berisi parameter
    ADD AX, BX      ; AX = AX + BX
    RET
calculate ENDP
\end{verbatim}

\subsubsection{Passing via Stack}
\begin{verbatim}
; Caller
PUSH param1       ; Push parameter ke stack
PUSH param2
CALL calculate
ADD SP, 4         ; Cleanup stack (2 parameters)

; Procedure
calculate PROC
    PUSH BP        ; Save BP
    MOV BP, SP     ; Setup stack frame
    MOV AX, [BP+4] ; Get parameter 1
    MOV BX, [BP+6] ; Get parameter 2
    ADD AX, BX     ; Calculate
    POP BP         ; Restore BP
    RET
calculate ENDP
\end{verbatim}

\section{Stack Management}

\subsection{Stack Frame Structure}
Organisasi stack untuk prosedur dengan parameter dan local variables.

\begin{verbatim}
; Standard stack frame setup
PUSH BP              ; Save old BP
MOV BP, SP           ; Setup new frame pointer
SUB SP, local_size   ; Reserve space for locals

; Standard stack frame cleanup
MOV SP, BP           ; Remove locals
POP BP               ; Restore old BP
RET                  ; Return to caller
\end{verbatim}

\textbf{Stack layout (high to low address):}
\begin{itemize}
  \item \textbf{Parameters}: [BP+4], [BP+6], [BP+8], ...
  \item \textbf{Return Address}: [BP+2]
  \item \textbf{Old BP}: [BP]
  \item \textbf{Local Variables}: [BP-2], [BP-4], [BP-6], ...
\end{itemize}

\subsection{Local Variables}
Variabel lokal dalam prosedur menggunakan stack space.

\begin{verbatim}
; Procedure dengan local variables
process_data PROC
    PUSH BP              ; Save BP
    MOV BP, SP           ; Setup frame
    SUB SP, 6            ; Reserve 3 words (6 bytes)
    
    ; Local variables
    MOV [BP-2], AX       ; local_var1 = AX
    MOV [BP-4], BX       ; local_var2 = BX
    MOV [BP-6], CX       ; local_var3 = CX
    
    ; ... proses dengan local variables
    MOV AX, [BP-2]       ; Gunakan local_var1
    ADD AX, [BP-4]       ; AX = local_var1 + local_var2
    
    MOV SP, BP           ; Cleanup locals
    POP BP               ; Restore BP
    RET
process_data ENDP
\end{verbatim}

\subsection{Calling Conventions}
Standard untuk parameter passing dan stack management.

\subsubsection{C Calling Convention}
\begin{itemize}
  \item Parameters pushed right-to-left
  \item Caller responsible for stack cleanup
  \item Return value di AX/DX
  \item Registers AX, CX, DX dapat diubah
\end{itemize}

\subsubsection{Pascal Calling Convention}
\begin{itemize}
  \item Parameters pushed left-to-right
  \item Procedure responsible for stack cleanup
  \item Return value di AX/DX
  \item Semua register kecuali BP harus dipreserve
\end{itemize}

\section{Rekursi}

\subsection{Konsep Rekursi}
Prosedur yang memanggil dirinya sendiri.

\begin{verbatim}
; Faktorial rekursif
factorial PROC
    PUSH BP
    MOV BP, SP
    
    MOV AX, [BP+4]       ; Get parameter n
    CMP AX, 1
    JLE base_case        ; if n <= 1
    
    DEC AX               ; n-1
    PUSH AX              ; Push n-1
    CALL factorial       ; factorial(n-1)
    ADD SP, 2            ; Cleanup
    
    MOV BX, [BP+4]       ; Get original n
    MUL BX               ; AX = AX * n
    JMP end_recursion
    
base_case:
    MOV AX, 1            ; factorial(1) = 1
    
end_recursion:
    POP BP
    RET
factorial ENDP
\end{verbatim}

\subsection{Stack Overflow}
Penting untuk menghindari stack overflow dalam rekursi.

\textbf{Penyebab stack overflow:}
\begin{itemize}
  \item Terlalu banyak nested calls
  \item Local variables terlalu besar
  \item Rekursi tanpa base case
  \item Infinite recursion
\end{itemize}

\textbf{Prevention:}
\begin{itemize}
  \item Limit recursion depth
  \item Use iteration when possible
  \item Minimize local variables
  \item Always have proper base case
\end{itemize}

\section{Advanced Stack Operations}

\subsection{Stack Alignment}
Mengoptimasi stack untuk performa lebih baik.

\begin{verbatim}
; Stack alignment untuk word boundaries
PUSH AX              ; AX di stack
AND SP, 0FFFEh       ; Align ke word boundary
; ... operasi yang memerlukan alignment
\end{verbatim}

\subsection{Stack Checking}
Validasi stack untuk debugging.

\begin{verbatim}
; Stack overflow check
check_stack PROC
    CMP SP, stack_limit
    JB stack_overflow     ; Jump if SP < limit
    RET                    ; Stack OK
    
stack_overflow:
    ; Handle overflow error
    INT 3                  ; Breakpoint for debugging
    RET
check_stack ENDP
\end{verbatim}

\subsection{Variable Number of Arguments}
Prosedur dengan jumlah parameter variabel.

\begin{verbatim}
; Printf-style procedure
printf PROC
    PUSH BP
    MOV BP, SP
    
    ; Get format string
    MOV SI, [BP+4]        ; Format string address
    
    ; Process format string and arguments
    ; ... implementation
    
    POP BP
    RET
printf ENDP
\end{verbatim}

% ============================================================
% AKTIVITAS PEMBELAJARAN
% ============================================================

\begin{aktivitas}
  \item \textbf{Procedure Library}: Buat library prosedur untuk operasi matematika (add, subtract, multiply, divide).
  
  \item \textbf{Stack Analysis}: Gunakan TASM debugger untuk trace stack frame setup dan cleanup.
  
  \item \textbf{Recursive Functions}: Implementasikan Fibonacci dan faktorial dengan rekursi.
  
  \item \textbf{Parameter Passing}: Bandingkan register vs stack parameter passing untuk performa.
  
  \item \textbf{Calling Conventions}: Implementasikan prosedur dengan C dan Pascal calling conventions.
  
  \item \textbf{Stack Optimization}: Optimasi stack usage untuk prosedur kompleks.
\end{aktivitas}

% ============================================================
% LATIHAN DAN REFLEKSI
% ============================================================

\begin{latihan}
  \item Buat prosedur untuk sorting array dengan bubble sort menggunakan stack frame.
  
  \item Implementasikan string manipulation library (strlen, strcpy, strcmp) dengan prosedur.
  
  \item Buat program kalkulator scientific dengan prosedur untuk setiap fungsi.
  
  \item Implementasikan binary search tree dengan rekursif prosedur.
  
  \item Buat prosedur untuk konversi bilangan (decimal to binary, hex, octal).
  
  \item Implementasikan linked list operations dengan prosedur dan pointer management.
  
  \item Buat program untuk menghitung statistik (mean, median, mode) dengan modular prosedur.
  
  \item \textbf{Refleksi}: Aspek mana dari prosedur dan stack management yang paling sulit dipahami? Bagaimana Anda mengatasi kesulitan tersebut?
\end{latihan}

% ============================================================
% ASESMEN
% ============================================================

\begin{asesmen}
\textbf{Instrumen Penilaian untuk Sub-CPMK 4.1, 2.1}

\textbf{A. Pilihan Ganda}

\begin{enumerate}
  \item Instruksi untuk memanggil prosedur adalah:
  \begin{enumerate}
    \item JUMP
    \item CALL
    \item RET
    \item PUSH
  \end{enumerate}
  
  \item Register yang digunakan sebagai base pointer untuk stack frame adalah:
  \begin{enumerate}
    \item SP
    \item BP
    \item SI
    \item DI
  \end{enumerate}
  
  \item Untuk menyimpan return address ke stack, instruksi yang digunakan adalah:
  \begin{enumerate}
    \item PUSH IP
    \item CALL
    \item JMP
    \item RET
  \end{enumerate}
  
  \item Dalam C calling convention, yang bertanggung jawab untuk stack cleanup adalah:
  \begin{enumerate}
    \item Caller
    \item Procedure
    \item OS
    \item Compiler
  \end{enumerate}
\end{enumerate}

\textbf{B. Essay}

\begin{enumerate}
  \item Jelaskan perbedaan antara parameter passing via register vs via stack! Berikan kelebihan dan kekurangan masing-masing.
  
  \item Mengapa stack frame management penting dalam prosedur yang kompleks?
\end{enumerate}

\textbf{C. Practical Challenge}

\begin{enumerate}
  \item Buat program text editor dengan modular design:
  \begin{itemize}
    \item Prosedur untuk file operations (open, save, close)
    \item Prosedur untuk text editing (insert, delete, replace)
    \item Prosedur untuk search functionality
    \item Prosedur untuk undo/redo operations
    \item Implementasi dengan proper stack management
    \item Error handling dan validation
    \item User interface dengan menu system
  \end{itemize}
\end{enumerate}

\textbf{Rubrik Penilaian}: Lihat Lampiran A
\end{asesmen}

% ============================================================
% CHECKLIST KOMPETENSI
% ============================================================

\begin{checklist}
  \item Saya dapat membuat dan memanggil prosedur dengan CALL dan RET
  \item Saya dapat mengimplementasikan stack frame dengan benar
  \item Saya dapat melakukan parameter passing via register dan stack
  \item Saya dapat mengelola local variables dalam prosedur
  \item Saya dapat mengimplementasikan prosedur rekursif
  \item Saya memahami calling conventions (C, Pascal)
  \item Saya dapat mencegah dan menangani stack overflow
  \item Saya dapat menganalisis stack dengan TASM debugger
\end{checklist}

% ============================================================
% RANGKUMAN
% ============================================================

\begin{rangkuman}
Bab ini membahas prosedur dan stack management dalam assembly 8086, termasuk parameter passing, local variables, rekursi, dan calling conventions.

\textbf{Poin Kunci:}
\begin{itemize}
  \item Prosedur memungkinkan modular programming dan code reuse
  \item Stack frame management essential untuk parameter dan local variables
  \item CALL dan RET instruksi fundamental untuk prosedur invocation
  \item Parameter passing dapat dilakukan via register atau stack
  \item Local variables menggunakan stack space dengan BP sebagai base pointer
  \item Rekursi memerlukan stack overflow prevention
  \item Calling conventions menentukan parameter passing dan cleanup responsibilities
  \item Stack analysis dengan debugger penting untuk troubleshooting
\end{itemize}

\textbf{Kata Kunci}: \asm{Prosedur}, \asm{Stack}, \asm{CALL}, \asm{RET}, \asm{Parameter Passing}, \asm{Stack Frame}, \asm{Rekursi}, \asm{Calling Convention}, \asm{TASM}, \asm{BP}, \asm{SP}
\end{rangkuman}


\ifSubfilesClassLoaded{
  \renewcommand{\bibname}{Daftar Pustaka}
  \bibliographystyle{plain}
  \bibliography{../references}
}{}
\end{document}
\begin{verbatim}
CALL procedure_name    ; Panggil prosedur
; kode setelah call
RET                  ; Kembali dari prosedur
\end{verbatim}

\section{Stack dan Operasinya}

\subsection{Stack Operations}
\begin{verbatim}
PUSH reg/mem         ; Simpan ke stack, SP--
POP reg/mem          ; Ambil dari stack, SP++
PUSHF               ; Simpan flag register ke stack
POPF                ; Ambil flag register dari stack
\end{verbatim}

\subsection{Stack Frame Layout}
\begin{verbatim}
High Address
+------------------+
|   Parameter      |
+------------------+
|   Return Address  |
+------------------+
|   Saved BP       | <-- BP
+------------------+
|   Local Variables|
+------------------+
Low Address
\end{verbatim}

\section{Parameter Passing}

\subsection{Passing by Register}
\begin{verbatim}
; Caller
MOV AX, param1
MOV BX, param2
CALL calculate_sum

; Procedure
calculate_sum PROC
ADD AX, BX      ; AX = AX + BX
RET
calculate_sum ENDP
\end{verbatim}

\subsection{Passing by Stack}
\begin{verbatim}
; Caller
PUSH param1
PUSH param2
CALL calculate_sum
ADD SP, 4       ; Clean up stack

; Procedure
calculate_sum PROC
PUSH BP         ; Save old BP
MOV BP, SP      ; Set new stack frame
MOV AX, [BP+4]  ; Get param1
ADD AX, [BP+6]  ; Add param2
POP BP          ; Restore BP
RET
calculate_sum ENDP
\end{verbatim}

\section{Local Variables}

\subsection{Local Variables di Stack}
\begin{verbatim}
procedure PROC
PUSH BP         ; Save BP
MOV BP, SP      ; Set stack frame
SUB SP, 4       ; Allocate 2 local variables

; Access local variables
MOV [BP-2], AX  ; local_var1 = AX
MOV [BP-4], BX  ; local_var2 = BX

; Clean up
MOV SP, BP      ; Deallocate locals
POP BP          ; Restore BP
RET
procedure ENDP
\end{verbatim}

\section{Recursive Procedures}

\subsection{Faktorial dengan Rekursi}
\begin{verbatim}
factorial PROC
PUSH BP
MOV BP, SP

MOV AX, [BP+4]  ; Get parameter n
CMP AX, 1
JLE base_case

; Recursive case
DEC AX
PUSH AX
CALL factorial
ADD SP, 2

; Multiply result by n
MOV BX, [BP+4]
MUL BX
JMP end_fact

base_case:
MOV AX, 1

end_fact:
POP BP
RET
factorial ENDP
\end{verbatim}

\section{Standard Calling Conventions}

\subsection{C Calling Convention}
\begin{itemize}
  \item Parameter passed right-to-left via stack
  \item Caller cleans up stack
  \item Return value in AX
  \item Caller-saved registers: AX, CX, DX
  \item Callee-saved registers: BX, SI, DI, BP, DS, ES
\end{itemize}

\subsection{Pascal Calling Convention}
\begin{itemize}
  \item Parameter passed left-to-right via stack
  \item Callee cleans up stack
  \item Return value in AX
\end{itemize}

% ============================================================
% AKTIVITAS PEMBELAJARAN
% ============================================================

\begin{aktivitas}
  \item \textbf{Procedure Creation}: Buat 5 prosedur matematika (tambah, kurang, kali, bagi, pangkat) dengan parameter passing berbeda.
  
  \item \textbf{Stack Analysis}: Gunakan TASM debugger untuk mengamati perubahan stack saat prosedur dipanggil.
  
  \item \textbf{Recursive Functions}: Implementasikan fibonacci dan faktorial dengan rekursi.
  
  \item \textbf{Calling Convention}: Bandingkan C vs Pascal calling convention dengan implementasi yang sama.
  
  \item \textbf{Local Variables}: Buat prosedur dengan local variables dan nested procedure calls.
  
  \item \textbf{Stack Frame Visualization}: Gambar stack frame untuk berbagai skenario parameter passing.
\end{aktivitas}

% ============================================================
% LATIHAN DAN REFLEKSI
% ============================================================

\begin{latihan}
  \item Jelaskan perbedaan antara passing parameter by register vs by stack! Kapan sebaiknya menggunakan masing-masing?
  
  \item Buat prosedur untuk menukar dua bilangan menggunakan:
  \begin{itemize}
    \item Register passing
    \item Stack passing
    \item Bandingkan efisiensi masing-masing
  \end{itemize}
  
  \item Implementasikan prosedur yang:
  \begin{itemize}
    \item Menerima array dan ukuran sebagai parameter
    \item Menghitung rata-rata elemen array
    \item Mengembalikan hasil di AX
    \item Menggunakan local variables untuk temporary storage
  \end{itemize}
  
  \item Buat program kalkulator scientific dengan prosedur untuk setiap operasi.
  
  \item Implementasikan binary search dengan prosedur rekursif.
  
  \item Analisis stack frame untuk nested procedure calls dengan 3 level kedalaman.
  
  \item \textbf{Refleksi}: Konsep mana yang paling sulit dalam manajemen prosedur dan stack? Bagaimana Anda mengatasi kesulitan tersebut?
\end{latihan}

% ============================================================
% ASESMEN
% ============================================================

\begin{asesmen}
\textbf{Instrumen Penilaian untuk Sub-CPMK 5.1, 5.2, 2.1}

\textbf{A. Pilihan Ganda}

\begin{enumerate}
  \item Instruksi untuk memanggil prosedur adalah:
  \begin{enumerate}
    \item JMP
    \item CALL
    \item RET
    \item INT
  \end{enumerate}
  
  \item Register yang digunakan untuk stack pointer adalah:
  \begin{enumerate}
    \item BP
    \item SP
    \item SI
    \item DI
  \end{enumerate}
  
  \item Untuk mengembalikan nilai dari prosedur, register yang umum digunakan adalah:
  \begin{enumerate}
    \item AX
    \item BX
    \item CX
    \item DX
  \end{enumerate}
  
  \item Dalam C calling convention, yang bertanggung jawab membersihkan stack adalah:
  \begin{enumerate}
    \item Caller
    \item Callee
    \item OS
    \item Compiler
  \end{enumerate}
\end{enumerate}

\textbf{B. Essay}

\begin{enumerate}
  \item Jelaskan langkah-langkah yang terjadi saat CALL instruction dieksekusi!
  
  \item Mengapa BP digunakan sebagai base pointer untuk stack frame?
\end{enumerate}

\textbf{C. Practical Challenge}

\begin{enumerate}
  \item Buat program mathematical library:
  \begin{itemize}
    \item Prosedur untuk operasi dasar (+, -, *, /)
    \item Prosedur untuk operasi lanjutan (pow, sqrt, factorial)
    \item Prosedur sorting array
    \item Menggunakan berbagai parameter passing methods
    \item Dokumentasi stack frame untuk setiap prosedur
  \end{itemize}
\end{enumerate}

\textbf{Rubrik Penilaian}: Lihat Lampiran A
\end{asesmen}

% ============================================================
% CHECKLIST KOMPETENSI
% ============================================================

\begin{checklist}
  \item Saya dapat menulis prosedur dengan parameter
  \item Saya dapat menggunakan CALL dan RET untuk prosedur calls
  \item Saya dapat mengimplementasikan stack operations (PUSH, POP)
  \item Saya dapat memahami stack frame layout
  \item Saya dapat menggunakan berbagai parameter passing methods
  \item Saya dapat mengimplementasikan local variables
  \item Saya dapat membuat prosedur rekursif
  \item Saya dapat menganalisis calling conventions
\end{checklist}

% ============================================================
% RANGKUMAN
% ============================================================

\begin{rangkuman}
Bab ini membahas prosedur dan manajemen stack dalam assembly language, termasuk parameter passing, local variables, dan rekursi.

\textbf{Poin Kunci:}
\begin{itemize}
  \item Prosedur memungkinkan modularisasi dan reusability kode
  \item Stack digunakan untuk menyimpan return address, parameter, dan local variables
  \item CALL menyimpan return address ke stack, RET mengambilnya kembali
  \item Parameter dapat dilewatkan melalui register atau stack
  \item Stack frame diorganisasi dengan BP sebagai base pointer
  \item Local variables dialokasikan di dalam stack frame
  \item Rekursi memerlukan manajemen stack yang hati-hati
  \item Calling conventions menentukan standar parameter passing
\end{itemize}

\textbf{Kata Kunci}: \textbf{\texttt{Prosedur}}, \textbf{\texttt{Stack}}, \textbf{\texttt{CALL}}, \textbf{\texttt{RET}}, \textbf{\texttt{PUSH}}, \textbf{\texttt{POP}}, \textbf{\texttt{Parameter}}, \textbf{\texttt{Local Variable}}, \textbf{\texttt{Rekursi}}, \textbf{\texttt{Stack Frame}}, \textbf{\texttt{TASM}}
\end{rangkuman}

\ifSubfilesClassLoaded{
  \renewcommand{\bibname}{Daftar Pustaka}
  \bibliographystyle{plain}
  \bibliography{../references}
}{}
\end{document}
