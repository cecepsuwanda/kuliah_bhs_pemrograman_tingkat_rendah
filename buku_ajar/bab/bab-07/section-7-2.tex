% ============================================================
% AKTIVITAS PEMBELAJARAN
% ============================================================

\begin{aktivitas}
  \item \textbf{Procedure Library}: Buat library prosedur untuk operasi matematika (add, subtract, multiply, divide).
  
  \item \textbf{Stack Analysis}: Gunakan TASM debugger untuk trace stack frame setup dan cleanup.
  
  \item \textbf{Recursive Functions}: Implementasikan Fibonacci dan faktorial dengan rekursi.
  
  \item \textbf{Parameter Passing}: Bandingkan register vs stack parameter passing untuk performa.
  
  \item \textbf{Calling Conventions}: Implementasikan prosedur dengan C dan Pascal calling conventions.
  
  \item \textbf{Stack Optimization}: Optimasi stack usage untuk prosedur kompleks.
\end{aktivitas}

% ============================================================
% LATIHAN DAN REFLEKSI
% ============================================================

\begin{latihan}
  \item Buat prosedur untuk sorting array dengan bubble sort menggunakan stack frame.
  
  \item Implementasikan string manipulation library (strlen, strcpy, strcmp) dengan prosedur.
  
  \item Buat program kalkulator scientific dengan prosedur untuk setiap fungsi.
  
  \item Implementasikan binary search tree dengan rekursif prosedur.
  
  \item Buat prosedur untuk konversi bilangan (decimal to binary, hex, octal).
  
  \item Implementasikan linked list operations dengan prosedur dan pointer management.
  
  \item Buat program untuk menghitung statistik (mean, median, mode) dengan modular prosedur.
  
  \item \textbf{Refleksi}: Aspek mana dari prosedur dan stack management yang paling sulit dipahami? Bagaimana Anda mengatasi kesulitan tersebut?
\end{latihan}

% ============================================================
% ASESMEN
% ============================================================

\begin{asesmen}
\textbf{Instrumen Penilaian untuk Sub-CPMK 4.1, 2.1}

\textbf{A. Pilihan Ganda}

\begin{enumerate}
  \item Instruksi untuk memanggil prosedur adalah:
  \begin{enumerate}
    \item JUMP
    \item CALL
    \item RET
    \item PUSH
  \end{enumerate}
  
  \item Register yang digunakan sebagai base pointer untuk stack frame adalah:
  \begin{enumerate}
    \item SP
    \item BP
    \item SI
    \item DI
  \end{enumerate}
  
  \item Untuk menyimpan return address ke stack, instruksi yang digunakan adalah:
  \begin{enumerate}
    \item PUSH IP
    \item CALL
    \item JMP
    \item RET
  \end{enumerate}
  
  \item Dalam C calling convention, yang bertanggung jawab untuk stack cleanup adalah:
  \begin{enumerate}
    \item Caller
    \item Procedure
    \item OS
    \item Compiler
  \end{enumerate}
\end{enumerate}

\textbf{B. Essay}

\begin{enumerate}
  \item Jelaskan perbedaan antara parameter passing via register vs via stack! Berikan kelebihan dan kekurangan masing-masing.
  
  \item Mengapa stack frame management penting dalam prosedur yang kompleks?
\end{enumerate}

\textbf{C. Practical Challenge}

\begin{enumerate}
  \item Buat program text editor dengan modular design:
  \begin{itemize}
    \item Prosedur untuk file operations (open, save, close)
    \item Prosedur untuk text editing (insert, delete, replace)
    \item Prosedur untuk search functionality
    \item Prosedur untuk undo/redo operations
    \item Implementasi dengan proper stack management
    \item Error handling dan validation
    \item User interface dengan menu system
  \end{itemize}
\end{enumerate}

\textbf{Rubrik Penilaian}: Lihat Lampiran A
\end{asesmen}

% ============================================================
% CHECKLIST KOMPETENSI
% ============================================================

\begin{checklist}
  \item Saya dapat membuat dan memanggil prosedur dengan CALL dan RET
  \item Saya dapat mengimplementasikan stack frame dengan benar
  \item Saya dapat melakukan parameter passing via register dan stack
  \item Saya dapat mengelola local variables dalam prosedur
  \item Saya dapat mengimplementasikan prosedur rekursif
  \item Saya memahami calling conventions (C, Pascal)
  \item Saya dapat mencegah dan menangani stack overflow
  \item Saya dapat menganalisis stack dengan TASM debugger
\end{checklist}

% ============================================================
% RANGKUMAN
% ============================================================

\begin{rangkuman}
Bab ini membahas prosedur dan stack management dalam assembly 8086, termasuk parameter passing, local variables, rekursi, dan calling conventions.

\textbf{Poin Kunci:}
\begin{itemize}
  \item Prosedur memungkinkan modular programming dan code reuse
  \item Stack frame management essential untuk parameter dan local variables
  \item CALL dan RET instruksi fundamental untuk prosedur invocation
  \item Parameter passing dapat dilakukan via register atau stack
  \item Local variables menggunakan stack space dengan BP sebagai base pointer
  \item Rekursi memerlukan stack overflow prevention
  \item Calling conventions menentukan parameter passing dan cleanup responsibilities
  \item Stack analysis dengan debugger penting untuk troubleshooting
\end{itemize}

\textbf{Kata Kunci}: \asm{Prosedur}, \asm{Stack}, \asm{CALL}, \asm{RET}, \asm{Parameter Passing}, \asm{Stack Frame}, \asm{Rekursi}, \asm{Calling Convention}, \asm{TASM}, \asm{BP}, \asm{SP}
\end{rangkuman}
