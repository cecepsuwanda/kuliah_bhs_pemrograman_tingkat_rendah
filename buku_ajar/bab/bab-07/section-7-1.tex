% ============================================================
% MATERI POKOK
% ============================================================
\section{Prosedur dan Fungsi}

\subsection{Konsep Prosedur}
Prosedur adalah blok kode yang dapat dipanggil berulang kali untuk modular programming.

\begin{verbatim}
; Prosedur definition
procedure_name PROC
    ; ... body of procedure
    RET
procedure_name ENDP

; Memanggil prosedur
CALL procedure_name
\end{verbatim}

\textbf{Keuntungan prosedur:}
\begin{itemize}
  \item Code reuse - menghindari duplikasi kode
  \item Modular design - memecah program menjadi bagian-bagian kecil
  \item Maintainability - lebih mudah diperbaiki dan diperbarui
  \item Readability - struktur program lebih jelas
\end{itemize}

\subsection{CALL dan RET Instructions}
Instruksi fundamental untuk prosedur management.

\begin{verbatim}
; CALL instruction
CALL procedure_name    ; Push return address, jump ke procedure

; RET instruction  
RET                 ; Pop return address, jump back
RET n               ; Pop return address, add n to SP (cleanup)
\end{verbatim}

\textbf{CALL operation:}
\begin{enumerate}
  \item Push return address (IP) ke stack
  \item Jump ke alamat prosedur
  \item Eksekusi prosedur
  \item RET pop return address dari stack
  \item Jump kembali ke caller
\end{enumerate}

\subsection{Parameter Passing}
Berbagai cara untuk passing parameter ke prosedur.

\subsubsection{Passing via Registers}
\begin{verbatim}
; Caller
MOV AX, param1
MOV BX, param2
CALL calculate
; Result di AX

; Procedure
calculate PROC
    ; AX dan BX berisi parameter
    ADD AX, BX      ; AX = AX + BX
    RET
calculate ENDP
\end{verbatim}

\subsubsection{Passing via Stack}
\begin{verbatim}
; Caller
PUSH param1       ; Push parameter ke stack
PUSH param2
CALL calculate
ADD SP, 4         ; Cleanup stack (2 parameters)

; Procedure
calculate PROC
    PUSH BP        ; Save BP
    MOV BP, SP     ; Setup stack frame
    MOV AX, [BP+4] ; Get parameter 1
    MOV BX, [BP+6] ; Get parameter 2
    ADD AX, BX     ; Calculate
    POP BP         ; Restore BP
    RET
calculate ENDP
\end{verbatim}

\section{Stack Management}

\subsection{Stack Frame Structure}
Organisasi stack untuk prosedur dengan parameter dan local variables.

\begin{verbatim}
; Standard stack frame setup
PUSH BP              ; Save old BP
MOV BP, SP           ; Setup new frame pointer
SUB SP, local_size   ; Reserve space for locals

; Standard stack frame cleanup
MOV SP, BP           ; Remove locals
POP BP               ; Restore old BP
RET                  ; Return to caller
\end{verbatim}

\textbf{Stack layout (high to low address):}
\begin{itemize}
  \item \textbf{Parameters}: [BP+4], [BP+6], [BP+8], ...
  \item \textbf{Return Address}: [BP+2]
  \item \textbf{Old BP}: [BP]
  \item \textbf{Local Variables}: [BP-2], [BP-4], [BP-6], ...
\end{itemize}

\subsection{Local Variables}
Variabel lokal dalam prosedur menggunakan stack space.

\begin{verbatim}
; Procedure dengan local variables
process_data PROC
    PUSH BP              ; Save BP
    MOV BP, SP           ; Setup frame
    SUB SP, 6            ; Reserve 3 words (6 bytes)
    
    ; Local variables
    MOV [BP-2], AX       ; local_var1 = AX
    MOV [BP-4], BX       ; local_var2 = BX
    MOV [BP-6], CX       ; local_var3 = CX
    
    ; ... proses dengan local variables
    MOV AX, [BP-2]       ; Gunakan local_var1
    ADD AX, [BP-4]       ; AX = local_var1 + local_var2
    
    MOV SP, BP           ; Cleanup locals
    POP BP               ; Restore BP
    RET
process_data ENDP
\end{verbatim}

\subsection{Calling Conventions}
Standard untuk parameter passing dan stack management.

\subsubsection{C Calling Convention}
\begin{itemize}
  \item Parameters pushed right-to-left
  \item Caller responsible for stack cleanup
  \item Return value di AX/DX
  \item Registers AX, CX, DX dapat diubah
\end{itemize}

\subsubsection{Pascal Calling Convention}
\begin{itemize}
  \item Parameters pushed left-to-right
  \item Procedure responsible for stack cleanup
  \item Return value di AX/DX
  \item Semua register kecuali BP harus dipreserve
\end{itemize}

\section{Rekursi}

\subsection{Konsep Rekursi}
Prosedur yang memanggil dirinya sendiri.

\begin{verbatim}
; Faktorial rekursif
factorial PROC
    PUSH BP
    MOV BP, SP
    
    MOV AX, [BP+4]       ; Get parameter n
    CMP AX, 1
    JLE base_case        ; if n <= 1
    
    DEC AX               ; n-1
    PUSH AX              ; Push n-1
    CALL factorial       ; factorial(n-1)
    ADD SP, 2            ; Cleanup
    
    MOV BX, [BP+4]       ; Get original n
    MUL BX               ; AX = AX * n
    JMP end_recursion
    
base_case:
    MOV AX, 1            ; factorial(1) = 1
    
end_recursion:
    POP BP
    RET
factorial ENDP
\end{verbatim}

\subsection{Stack Overflow}
Penting untuk menghindari stack overflow dalam rekursi.

\textbf{Penyebab stack overflow:}
\begin{itemize}
  \item Terlalu banyak nested calls
  \item Local variables terlalu besar
  \item Rekursi tanpa base case
  \item Infinite recursion
\end{itemize}

\textbf{Prevention:}
\begin{itemize}
  \item Limit recursion depth
  \item Use iteration when possible
  \item Minimize local variables
  \item Always have proper base case
\end{itemize}

\section{Advanced Stack Operations}

\subsection{Stack Alignment}
Mengoptimasi stack untuk performa lebih baik.

\begin{verbatim}
; Stack alignment untuk word boundaries
PUSH AX              ; AX di stack
AND SP, 0FFFEh       ; Align ke word boundary
; ... operasi yang memerlukan alignment
\end{verbatim}

\subsection{Stack Checking}
Validasi stack untuk debugging.

\begin{verbatim}
; Stack overflow check
check_stack PROC
    CMP SP, stack_limit
    JB stack_overflow     ; Jump if SP < limit
    RET                    ; Stack OK
    
stack_overflow:
    ; Handle overflow error
    INT 3                  ; Breakpoint for debugging
    RET
check_stack ENDP
\end{verbatim}

\subsection{Variable Number of Arguments}
Prosedur dengan jumlah parameter variabel.

\begin{verbatim}
; Printf-style procedure
printf PROC
    PUSH BP
    MOV BP, SP
    
    ; Get format string
    MOV SI, [BP+4]        ; Format string address
    
    ; Process format string and arguments
    ; ... implementation
    
    POP BP
    RET
printf ENDP
\end{verbatim}
