\documentclass[../main.tex]{subfiles}
\ifSubfilesClassLoaded{\setcounter{chapter}{13}}{}
\begin{document}

\chapter{Evaluasi, Refleksi, dan Integrasi Kompetensi}

Bab ini berisi asesmen komprehensif untuk mengukur pencapaian seluruh CPMK yang telah dipelajari sepanjang mata kuliah Pemrograman Assembly Intel 8086.

\section{Asesmen Akhir Komprehensif}

\subsection{Petunjuk Umum}

Asesmen ini dirancang untuk mengukur pencapaian semua Sub-CPMK dalam assembly programming secara menyeluruh. Kerjakan dengan jujur dan mandiri.

\textbf{Alokasi Waktu}:
\begin{itemize}
  \item Bagian A (Pilihan Ganda): 30 menit
  \item Bagian B (Essay): 45 menit
  \item Bagian C (Analisis Kode): 45 menit
  \item Bagian D (Coding Challenge): 90 menit
\end{itemize}

\subsection{Bagian A: Pilihan Ganda (Sub-CPMK 1.1-1.3)}

\textbf{Petunjuk}: Pilih satu jawaban yang paling tepat.

\begin{enumerate}
  \item Register yang digunakan sebagai base pointer dalam stack frame adalah:
  \begin{enumerate}
    \item SP
    \item BP
    \item SI
    \item DI
  \end{enumerate}
  
  \item Instruksi untuk operasi aritmatika penjumlahan adalah:
  \begin{enumerate}
    \item ADD
    \item SUB
    \item MUL
    \item DIV
  \end{enumerate}
  
  \item Mode addressing yang menggunakan nilai konstan adalah:
  \begin{enumerate}
    \item Immediate
    \item Direct
    \item Register
    \item Indirect
  \end{enumerate}
  
  \item Instruksi yang digunakan untuk unconditional jump adalah:
  \begin{enumerate}
    \item JMP
    \item JZ
    \item JC
    \item JNZ
  \end{enumerate}
  
  \item Untuk implementasi loop dengan counter, register yang paling cocok adalah:
  \begin{enumerate}
    \item AX
    \item BX
    \item CX
    \item DX
  \end{enumerate}
  
  \item Interrupt untuk video services adalah:
  \begin{enumerate}
    \item INT 10h
    \item INT 13h
    \item INT 16h
    \item INT 21h
  \end{enumerate}
  
  \item Fungsi dari instruksi CALL adalah:
  \begin{enumerate}
    \item Melakukan unconditional jump
    \item Memanggil prosedur dan menyimpan return address
    \item Mengalokasikan memori
    \item Membandingkan nilai
  \end{enumerate}
  
  \item Teknik optimasi yang mengganti MUL dengan SHIFT adalah:
  \begin{enumerate}
    \item Loop unrolling
    \item Strength reduction
    \item Register optimization
    \item Code alignment
  \end{enumerate}
  
  \item Dalam TASM debugger, perintah untuk set breakpoint adalah:
  \begin{enumerate}
    \item BP
    \item BC
    \item BL
    \item G
  \end{enumerate}
  
  \item Directive untuk mendefinisikan word (16-bit) data adalah:
  \begin{enumerate}
    \item DB
    \item DW
    \item DD
    \item EQU
  \end{enumerate}
\end{enumerate}

\subsection{Bagian B: Essay (Sub-CPMK 2.1-2.2, 4.1-4.2)}

\textbf{Petunjuk}: Jawab dengan jelas dan lengkap.

\begin{enumerate}
  \item Jelaskan perbedaan antara CALL dan RET instruksi dalam assembly programming! Berikan contoh penggunaan masing-masing.
  
  \item Jelaskan konsep stack frame dan manajemen register dalam prosedur assembly! Mengapa penting untuk preserve register?
  
  \item Jelaskan perbedaan antara formative dan summative assessment dalam pembelajaran assembly programming!
  
  \item Jelaskan keuntungan modular programming dalam pengembangan aplikasi assembly!
  
  \item Jelaskan pentingnya optimasi dalam assembly programming dan berikan contoh teknik optimasi yang umum digunakan!
\end{enumerate}

\subsection{Bagian C: Analisis Kode (Sub-CPMK 3.1-3.2, 4.1)}

\textbf{Petunjuk}: Analisis kode assembly berikut dan jawab pertanyaan.

\begin{verbatim}
; Kode untuk dianalisis
; ===================
DATA SEGMENT
    array DW 10, 20, 30, 40, 50
    size DW 5
    result DW 0
DATA ENDS

CODE SEGMENT
    ASSUME CS:CODE, DS:DATA

start:
    MOV AX, DATA
    MOV DS, AX
    
    MOV CX, [size]
    MOV SI, 0
    MOV BX, 0
    
sum_loop:
    ADD BX, [array+SI]
    ADD SI, 2
    LOOP sum_loop
    
    MOV [result], BX
    
    MOV AH, 4Ch
    INT 21h
CODE ENDS
END start
\end{verbatim}

\textbf{Pertanyaan:}
\begin{enumerate}
  \item Apa fungsi dari kode assembly di atas?
  
  \item Identifikasi potensi bug atau optimasi opportunities dalam kode tersebut!
  
  \item Bagaimana cara mengoptimasi kode tersebut untuk performa yang lebih baik?
  
  \item Jelaskan peran register CX, SI, dan BX dalam program tersebut!
\end{enumerate}

\subsection{Bagian D: Proyek Akhir / Coding Challenge (Sub-CPMK 4.1, 4.2, CPMK-2, CPMK-3, CPMK-4)}

\textbf{Petunjuk}: Pilih salah satu proyek berikut sesuai RPS (silabus) dan implementasikan program assembly lengkap.

\textbf{Pilihan Proyek Akhir (sesuai silabus):}

\textbf{Pilihan 1: Sistem Manajemen Memori Sederhana}
\begin{itemize}
  \item Implementasikan alokasi dan dealokasi blok memori sederhana
  \item Gunakan struktur data linked list atau bitmap untuk tracking memori
  \item Gunakan prosedur modular (CALL/RET) dan manajemen stack
  \item Gunakan interupsi DOS/BIOS untuk output status
\end{itemize}

\textbf{Pilihan 2: Program Editor Teks dengan Fitur Find/Replace}
\begin{itemize}
  \item Implementasikan input teks, penyimpanan buffer, dan tampilan
  \item Fitur find: mencari substring dalam buffer
  \item Fitur replace: mengganti substring dengan string lain
  \item Gunakan prosedur untuk operasi string (manipulasi, perbandingan)
  \item Gunakan INT 21h untuk input/output
\end{itemize}

\textbf{Pilihan 3: Game Sederhana dengan Grafik dan Input Keyboard}
\begin{itemize}
  \item Implementasikan tampilan grafik menggunakan interupsi BIOS (INT 10h)
  \item Gunakan INT 16h untuk input keyboard (arah gerak, aksi)
  \item Game dapat berupa: snake, pong, atau maze sederhana
  \item Gunakan prosedur modular untuk logika game dan rendering
\end{itemize}

\textbf{Pilihan 4: Utility Sistem untuk Monitoring Hardware}
\begin{itemize}
  \item Implementasikan utility yang menampilkan informasi sistem (waktu, tanggal, memori)
  \item Gunakan interupsi BIOS/DOS untuk akses hardware (INT 21h fungsi 2Ch/2Ah, INT 12h, dll.)
  \item Tampilkan hasil dalam format teks yang terbaca
  \item Gunakan prosedur modular untuk tiap fungsi monitoring
\end{itemize}

\textbf{Requirements Umum (semua pilihan):}
\begin{enumerate}
  \item Gunakan sintaks TASM dengan benar
  \item Gunakan prosedur modular dengan parameter passing
  \item Gunakan interupsi untuk input/output
  \item Implementasikan error handling dasar
  \item Dokumentasikan kode dengan komentar yang jelas
\end{enumerate}

\textbf{Evaluasi Criteria:}
\begin{itemize}
  \item Correctness (40\%): Program berfungsi sesuai spesifikasi
  \item Efficiency (20\%): Optimasi dan performa kode
  \item Modularity (20\%): Struktur prosedur yang baik
  \item Documentation (20\%): Komentar dan dokumentasi
\end{itemize}

\textbf{Template Structure (umum):}
\begin{verbatim}
; Template untuk proyek assembly
; =============================
DATA SEGMENT
    ; Data definitions here
DATA ENDS

CODE SEGMENT
    ASSUME CS:CODE, DS:DATA

start:
    MOV AX, DATA
    MOV DS, AX
    ; Main program logic here
    ; Call procedures
    MOV AH, 4Ch
    INT 21h

; Procedures
; ==========
your_proc PROC
    ; Implementation here
    RET
your_proc ENDP

CODE ENDS
END start
\end{verbatim}

\section{Rubrik Penilaian Komprehensif}

\subsection{Rubrik untuk Coding Challenge}

\begin{table}[htbp]
\centering
\setlength{\tabcolsep}{2pt}
\begin{tabular}{|>{\raggedright\arraybackslash}p{1.8cm}|>{\raggedright\arraybackslash}p{2.3cm}|>{\raggedright\arraybackslash}p{2.3cm}|>{\raggedright\arraybackslash}p{2.3cm}|>{\raggedright\arraybackslash}p{2.1cm}|}
\hline
\textbf{Kriteria} & \textbf{Excellent (4)} & \textbf{Good (3)} & \textbf{Fair (2)} & \textbf{Poor (1)} \\
\hline
Correctness & Program berfungsi sempurna sesuai spesifikasi & Berfungsi dengan minor bug & Banyak bug & Tidak berjalan \\
\hline
Modularity & Prosedur terstruktur, code reuse optimal & Struktur prosedur baik & Struktur kurang modular & Monolitik \\
\hline
Dokumentasi & Komentar lengkap, struktur jelas & Dokumentasi adequate & Minimal comments & Tidak ada dokumentasi \\
\hline
Optimasi & Register efisien, strength reduction, loop unrolling & Optimasi diterapkan & Optimasi minimal & Tanpa optimasi \\
\hline
Error Handling & Validasi input, penanganan interrupt tepat & Error handling adequate & Minimal handling & Tidak ada \\
\hline
Code Quality & Clean code, naming konsisten & Good structure & Basic structure & Poor structure \\
\hline
Testing & Pengujian lengkap dengan TASM debugger & Pengujian sebagian & Minimal testing & No testing \\
\hline
\end{tabular}
\caption{Rubrik Penilaian Coding Challenge}
\end{table}

\subsection{Bobot Penilaian}

\begin{table}[htbp]
\centering
\begin{tabular}{|l|c|}
\hline
\textbf{Komponen} & \textbf{Bobot} \\
\hline
Bagian A: Pilihan Ganda & 20\% \\
\hline
Bagian B: Essay & 20\% \\
\hline
Bagian C: Analisis Kode & 20\% \\
\hline
Bagian D: Coding Challenge & 40\% \\
\hline
\textbf{Total} & \textbf{100\%} \\
\hline
\end{tabular}
\caption{Matriks Bobot Penilaian Akhir}
\end{table}

\section{Tinjauan Pencapaian Kompetensi Secara Menyeluruh}

\subsection{Pemetaan Asesmen ke CPMK}

\begin{table}[htbp]
\centering
\begin{tabular}{|p{5cm}|p{8cm}|}
\hline
\textbf{CPMK} & \textbf{Diukur Melalui} \\
\hline
CPMK-1: Memahami konsep register, segmentasi memori, dan instruksi dasar & Pilihan Ganda (1-4), Essay (1-2) \\
\hline
CPMK-2: Merancang program assembly dengan struktur modular & Essay (4), Coding Challenge (Design) \\
\hline
CPMK-3: Mengimplementasikan program assembly dengan TASM & Analisis Kode, Coding Challenge (Implementation) \\
\hline
CPMK-4: Menganalisis dan mengoptimasi kode assembly & Analisis Kode, Coding Challenge (Quality) \\
\hline
\end{tabular}
\caption{Pemetaan Komponen Asesmen ke CPMK}
\end{table}

\subsection{Self-Assessment Checklist}

Sebelum mengerjakan asesmen akhir, pastikan Anda telah menguasai:

\begin{checklist}
  \item Konsep register dan segmentasi memori Intel 8086
  \item Instruksi dasar dan mode addressing
  \item Struktur kontrol dan perulangan dalam assembly
  \item Prosedur dan stack frame management
  \item Input/Output dengan interrupt BIOS dan DOS
  \item Struktur program modular dengan prosedur
  \item Teknik optimasi (strength reduction, loop unrolling)
  \item Debugging dengan TASM debugger
  \item Dokumentasi dan komentar kode
  \item Best practices dalam penulisan kode assembly
\end{checklist}

\section{Rekomendasi Tindak Lanjut Pembelajaran}

\subsection{Untuk Mahasiswa yang Sudah Menguasai Semua CPMK}

Jika Anda telah menguasai semua materi dengan baik, pertimbangkan untuk:

\begin{enumerate}
  \item \textbf{Proyek Portofolio}: Buat aplikasi assembly lengkap yang mendemonstrasikan semua konsep Intel 8086
  \item \textbf{Eksplorasi Advanced Assembly}: Pelajari x86-64 assembly atau ARM assembly
  \item \textbf{System Programming}: Pelajari OS development dan driver programming
  \item \textbf{Optimization Techniques}: Baca buku "Peter Norton's Assembly Language Book" \cite{ref6}
  \item \textbf{Kontribusi Open Source}: Berkontribusi pada proyek system programming open source
  \item \textbf{Sertifikasi}: Persiapkan sertifikasi system programming atau embedded systems
\end{enumerate}

\subsection{Untuk Mahasiswa yang Masih Perlu Perbaikan}

Jika ada CPMK yang belum dikuasai dengan baik:

\begin{enumerate}
  \item \textbf{Review Materi}: Baca ulang bab terkait dengan lebih teliti
  \item \textbf{Praktik Lebih Banyak}: Kerjakan lebih banyak latihan assembly programming
  \item \textbf{Konsultasi}: Diskusikan dengan dosen atau asisten
  \item \textbf{Study Group}: Bentuk kelompok belajar dengan teman
  \item \textbf{Online Resources}: Manfaatkan tutorial online (Intel documentation, assembly tutorials)
  \item \textbf{Coding Practice}: Gunakan emulator 8086 untuk latihan praktik
\end{enumerate}

\subsection{Sumber Belajar Tambahan}

\textbf{Buku}:
\begin{itemize}
  \item "The Intel 8086/8088 Family" - Paul Barry \cite{ref1}
  \item "The 80x86 IBM PC and Compatible Computers" - Mazidi \& Mazidi \cite{ref2}
  \item "The Intel Microprocessors" - Barry Brey \cite{ref3}
  \item "Computer Organization and Design" - Patterson \& Hennessy \cite{ref4}
  \item "Peter Norton's Assembly Language Book" - Norton \& Socha \cite{ref6}
\end{itemize}

\textbf{Online Courses}:
\begin{itemize}
  \item Computer Architecture (Coursera)
  \item Assembly Language Programming (Udemy)
  \item System Programming Fundamentals (Pluralsight)
\end{itemize}

\textbf{Practice Platforms}:
\begin{itemize}
  \item DOSBox dengan TASM untuk praktik assembly
  \item Emulator 8086 untuk visualisasi
  \item Intel Developer Zone untuk dokumentasi resmi
\end{itemize}


% ============================================================
% RANGKUMAN PERJALANAN PEMBELAJARAN
% ============================================================

\begin{rangkuman}
Selamat! Anda telah menyelesaikan perjalanan pembelajaran Pemrograman Assembly Intel 8086. Dari memahami register dan arsitektur hingga menerapkan prosedur, optimasi, dan debugging, Anda telah membangun fondasi yang kuat dalam pemrograman tingkat rendah.

\textbf{Apa yang Telah Anda Pelajari}:
\begin{itemize}
  \item Register dan organisasi memori segmentasi
  \item Instruksi dasar dan mode addressing
  \item Struktur kontrol dan perulangan
  \item Prosedur dan stack frame management
  \item Input/Output dengan interrupt BIOS dan DOS
  \item Modular programming dan library
  \item Optimasi kode (strength reduction, loop unrolling)
  \item Debugging dengan TASM debugger
  \item Best practices dan dokumentasi kode
  \item Asesmen dan evaluasi dalam pembelajaran assembly
\end{itemize}

\textbf{Langkah Selanjutnya}:
Terus praktik, buat proyek nyata, dan jangan berhenti belajar. Assembly adalah fondasi pemahaman hardware dan low-level programming, tetapi masih banyak yang bisa dipelajari dalam dunia system programming!
\end{rangkuman}

\ifSubfilesClassLoaded{
  \renewcommand{\bibname}{Daftar Pustaka}
  \bibliographystyle{plain}
  \bibliography{../references}
}{}
\end{document}
