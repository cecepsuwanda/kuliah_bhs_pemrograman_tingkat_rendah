\documentclass[../main.tex]{subfiles}
\ifSubfilesClassLoaded{\setcounter{chapter}{10}}{}
\begin{document}

\chapter{Asesmen Komprehensif dan Evaluasi}

\begin{subcpmk}
  \item Sub-CPMK 9.1: Mendemonstrasikan kemampuan programming assembly secara komprehensif
  \item Sub-CPMK 9.2: Mengevaluasi performa dan kualitas kode assembly
  \item Sub-CPMK 2.1: Menulis program assembly sederhana dengan TASM syntax
\end{subcpmk}

% ============================================================
% MATERI POKOK
% ============================================================
% ============================================================
% MATERI POKOK
% ============================================================
\section{Jenis-jenis Asesmen}

\subsection{Konsep Asesmen dalam Assembly Programming}
Asesmen adalah proses pengukuran pencapaian kompetensi mahasiswa dalam pemrograman assembly.

\textbf{Tujuan asesmen:}
\begin{itemize}
  \item Mengukur pemahaman konsep assembly
  \item Mengevaluasi kemampuan praktis programming
  \item Memberikan feedback untuk improvement
  \item Menentukan pencapaian Sub-CPMK
\end{itemize}

\subsection{Formative Assessment}
Asesmen formatif untuk monitoring progress pembelajaran.

\begin{verbatim}
; Contoh formatif assessment - debugging exercise
; Student harus menemukan dan memperbaiki bug

; Bug code (diberikan ke mahasiswa)
buggy_code PROC
    MOV AX, [array]  ; Bug: tidak mengecek array bounds
    ADD AX, [array+2]
    ADD AX, [array+4]
    ; ... potensial array overflow
    RET
buggy_code ENDP

; Expected fix (mahasiswa harus menemukan)
fixed_code PROC
    MOV CX, array_size
    MOV SI, 0
    MOV AX, 0
sum_loop:
    CMP SI, CX
    JGE end_sum
    ADD AX, [array+SI]
    ADD SI, 2
    JMP sum_loop
end_sum:
    RET
fixed_code ENDP
\end{verbatim}

\textbf{Formatif assessment types:}
\begin{itemize}
  \item \textbf{Code Review}: Review kode assembly untuk best practices
  \item \textbf{Debugging Exercises}: Temukan dan perbaiki bug
  \item \textbf{Pair Programming}: Kolaborasi debugging
  \item \textbf{Quick Quiz}: Pengetahuan konseptual singkat
\end{itemize}

\subsection{Summative Assessment}
Asesmen sumatif untuk evaluasi akhir kompetensi.

\begin{verbatim}
; Contoh summative assessment - comprehensive project
; Mahasiswa harus implementasikan program lengkap

; Project requirements:
; 1. Text editor dengan fitur:
;    - File operations (open, save, close)
;    - Text editing (insert, delete, replace)
;    - Search functionality
;    - Menu system
; 2. Modular design dengan library
; 3. Documentation lengkap
; 4. Error handling
; 5. Performance optimization
\end{verbatim}

\textbf{Summative assessment components:}
\begin{itemize}
  \item \textbf{Practical Exam}: Implementasi program assembly
  \item \textbf{Written Exam}: Konseptual understanding
  \item \textbf{Project}: Development project lengkap
  \item \textbf{Presentation}: Demo dan explanation
\end{itemize}

\section{Rubrik Penilaian}

\subsection{Rubrik untuk Kode Assembly}
Standar penilaian untuk kualitas kode assembly.

\begin{table}[htbp]
\centering
\begin{tabular}{|p{2cm}|p{3cm}|p{3cm}|p{3cm}|p{2cm}|}
\hline
\textbf{Kriteria} & \textbf{Excellent (4)} & \textbf{Good (3)} & \textbf{Fair (2)} & \textbf{Poor (1)} \\
\hline
Correctness & No bugs, perfect functionality & Minor bugs, good functionality & Some bugs, basic functionality & Major bugs, poor functionality \\
\hline
Efficiency & Optimized, minimal cycles & Good optimization & Basic optimization & No optimization \\
\hline
Readability & Well documented, clear structure & Good documentation & Basic documentation & Poor/no documentation \\
\hline
Modularity & Excellent modular design & Good modular design & Basic modularization & No modularity \\
\hline
Error Handling & Comprehensive error handling & Good error handling & Basic error handling & No error handling \\
\hline
\end{tabular}
\caption{Rubrik Penilaian Kode Assembly}
\end{table}

\subsection{Rubrik untuk Problem Solving}
Standar penilaian untuk kemampuan problem solving.

\begin{table}[htbp]
\centering
\begin{tabular}{|p{2cm}|p{3cm}|p{3cm}|p{3cm}|p{2cm}|}
\hline
\textbf{Kriteria} & \textbf{Excellent (4)} & \textbf{Good (3)} & \textbf{Fair (2)} & \textbf{Poor (1)} \\
\hline
Analysis & Deep understanding of problem & Good understanding & Basic understanding & Poor understanding \\
\hline
Algorithm Design & Optimal algorithm design & Good algorithm design & Basic algorithm design & Poor algorithm design \\
\hline
Implementation & Flawless implementation & Good implementation & Basic implementation & Poor implementation \\
\hline
Optimization & Excellent optimization & Good optimization & Basic optimization & No optimization \\
\hline
Testing & Comprehensive testing & Good testing & Basic testing & No testing \\
\hline
\end{tabular}
\caption{Rubrik Penilaian Problem Solving}
\end{table}

\section{Instrumen Asesmen}

\subsection{Practical Exam Template}
Template untuk ujian praktik assembly programming.

\begin{verbatim}
; Practical Exam Template
; ====================
; Duration: 120 minutes
; Total Points: 100

; Problem 1: String Manipulation (25 points)
; Implementasikan fungsi string reverse dengan optimasi

; Problem 2: Array Processing (25 points)
; Buat program untuk sorting array dengan bubble sort

; Problem 3: File Operations (25 points)
; Implementasikan file reader dengan error handling

; Problem 4: System Programming (25 points)
; Buat program yang menampilkan informasi sistem

; Evaluation Criteria:
; - Correctness: 40%
; - Efficiency: 20%
; - Documentation: 20%
; - Error Handling: 20%
\end{verbatim}

\subsection{Written Exam Template}
Template untuk ujian tertulis konseptual.

\textbf{Sample Questions:}

\begin{enumerate}
  \item Jelaskan perbedaan antara CALL dan RET instruksi! (10 points)
  
  \item Implementasikan algoritma binary search dalam assembly! (20 points)
  
  \item Analisis kode berikut dan identifikasi optimasi opportunities: (15 points)
  \begin{verbatim}
  MOV CX, 100
  MOV SI, 0
  loop_start:
      MOV AL, [array+SI]
      CMP AL, 0
      JE found
      INC SI
      LOOP loop_start
  \end{verbatim}
  
  \item Jelaskan konsep stack frame dan manajemen register dalam prosedur! (15 points)
  
  \item Design modular architecture untuk calculator application! (20 points)
  
  \item Jelaskan optimasi techniques yang dapat diterapkan pada kode assembly! (20 points)
\end{enumerate}

\section{Self-Assessment Tools}

\subsection{Checklist Kompetensi}
Checklist untuk self-assessment mahasiswa.

\textbf{Technical Skills Checklist:}
\begin{itemize}
  \item[$\square$] Saya dapat menulis program assembly dasar
  \item[$\square$] Saya dapat menggunakan instruksi aritmatika dan logika
  \item[$\square$] Saya dapat mengimplementasikan struktur kontrol
  \item[$\square$] Saya dapat membuat prosedur dengan parameter
  \item[$\square$] Saya dapat menggunakan interupsi BIOS/DOS
  \item[$\square$] Saya dapat melakukan debugging dengan TASM
  \item[$\square$] Saya dapat mengoptimasi kode assembly
  \item[$\square$] Saya dapat mengembangkan modular program
\end{itemize}

\textbf{Problem Solving Checklist:}
\begin{itemize}
  \item[$\square$] Saya dapat menganalisis problem requirements
  \item[$\square$] Saya dapat merancang algoritma yang efisien
  \item[$\square$] Saya dapat mengidentifikasi dan memperbaiki bug
  \item[$\square$] Saya dapat mengoptimasi performa kode
  \item[$\square$] Saya dapat mendokumentasikan solusi
  \item[$\square$] Saya dapat menguji dan validasi solusi
\end{itemize}

\subsection{Portfolio Assessment}
Portfolio untuk dokumentasi pencapaian kompetensi.

\textbf{Portfolio Components:}
\begin{itemize}
  \item \textbf{Code Samples}: 5 best assembly programs
  \item \textbf{Projects}: 2-3 complete projects
  \item \textbf{Debugging Logs}: Bug fixing examples
  \item \textbf{Optimization Reports}: Before/after comparisons
  \item \textbf{Documentation}: Complete documentation
  \item \textbf{Reflection}: Learning journey reflection
\end{itemize}

\begin{verbatim}
; Portfolio Entry Template
; =====================
; Project: Calculator Application
; Date: [Current Date]
; Skills Demonstrated:
; - Modular programming
; - File I/O operations
; - User interface design
; - Error handling
; - Performance optimization
; 
; Lessons Learned:
; - Importance of modular design
; - Debugging techniques
; - Optimization strategies
; 
; Future Improvements:
; - Add advanced functions
; - Improve user interface
; - Add configuration options
\end{verbatim}

% ============================================================
% AKTIVITAS PEMBELAJARAN
% ============================================================

\begin{aktivitas}
  \item \textbf{Assessment Design}: Buat rubrik penilaian untuk assembly programming project.
  
  \item \textbf{Peer Review}: Implementasikan sistem peer review untuk kode assembly.
  
  \item \textbf{Self-Assessment}: Gunakan checklist kompetensi untuk self-evaluation.
  
  \item \textbf{Portfolio Development}: Buat portfolio untuk dokumentasi pencapaian.
  
  \item \textbf{Exam Creation}: Rancang ujian praktik dan tertulis untuk assembly programming.
  
  \item \textbf{Feedback System}: Implementasikan sistem feedback yang konstruktif.
\end{aktivitas}

% ============================================================
% LATIHAN DAN REFLEKSI
% ============================================================

\begin{latihan}
  \item Buat rubrik penilaian lengkap untuk text editor assembly project.
  
  \item Rancang ujian praktik 120 menit untuk mengevaluasi kompetensi assembly.
  
  \item Implementasikan sistem peer review dengan template yang standar.
  
  \item Buat checklist self-assessment untuk Sub-CPMK 1.1-1.3.
  
  \item Design portfolio assessment untuk semester project.
  
  \item Buat template feedback untuk memberikan komentar konstruktif.
  
  \item Implementasikan automated testing framework untuk assembly programs.
  
  \item \textbf{Refleksi}: Aspek mana dari assessment yang paling sulit diimplementasikan? Bagaimana Anda mengatasi kesulitan tersebut?
\end{latihan}

% ============================================================
% ASESMEN
% ============================================================

\begin{asesmen}
\textbf{Instrumen Penilaian untuk Sub-CPMK 9.1, 9.2}

\textbf{A. Pilihan Ganda}

\begin{enumerate}
  \item Tujuan utama formative assessment adalah:
  \begin{enumerate}
    \item Evaluasi akhir kompetensi
    \item Monitoring progress pembelajaran
    \item Memberikan nilai akhir
    \item Seleksi mahasiswa terbaik
  \end{enumerate}
  
  \item Komponen penting dalam rubrik penilaian kode assembly adalah:
  \begin{enumerate}
    \item Correctness dan efficiency
    \item Documentation dan modularity
    \item Error handling dan testing
    \item Semua jawaban benar
  \end{enumerate}
  
  \item Self-assessment tool yang paling efektif adalah:
  \begin{enumerate}
    \item Multiple choice quiz
    \item Checklist kompetensi
    \item Essay questions
    \item Practical exam
  \end{enumerate}
  
  \item Portfolio assessment mengevaluasi:
  \begin{enumerate}
    \item Pencapaian kompetensi overtime
    \item Pengetahuan teoretis saja
    \item Kemampuan memorisasi
    \item Kehadiran di kelas
  \end{enumerate}
\end{enumerate}

\textbf{B. Essay}

\begin{enumerate}
  \item Jelaskan perbedaan antara formative dan summative assessment dalam assembly programming!
  
  \item Mengapa rubrik penilaian penting untuk evaluasi kode assembly?
\end{enumerate}

\textbf{C. Practical Challenge}

\begin{enumerate}
  \item Buat comprehensive assessment system:
  \begin{itemize}
    \item Rubrik penilaian untuk berbagai tipe tugas
    \item Template feedback yang konstruktif
    \item Self-assessment checklist untuk semua Sub-CPMK
    \item Portfolio assessment framework
    \item Peer review system dengan guidelines
    \item Exam templates (practical dan written)
    \item Automated testing framework
    \item Documentation untuk assessment process
    \item Training materials untuk assessors
    \item Quality assurance procedures
  \end{itemize}
\end{enumerate}

\textbf{Rubrik Penilaian}: Lihat Lampiran A
\end{asesmen}

% ============================================================
% CHECKLIST KOMPETENSI
% ============================================================

\begin{checklist}
  \item Saya dapat merancang assessment instruments yang efektif
  \item Saya dapat membuat rubrik penilaian yang komprehensif
  \item Saya dapat mengimplementasikan formative assessment
  \item Saya dapat mengembangkan summative assessment
  \item Saya dapat menggunakan self-assessment tools
  \item Saya dapat mengimplementasikan peer review systems
  \item Saya dapat mengevaluasi portfolio assessment
  \item Saya dapat memberikan feedback yang konstruktif
\end{checklist}

% ============================================================
% RANGKUMAN
% ============================================================

\begin{rangkuman}
Bab ini membahas asesmen dan evaluasi dalam assembly programming, termasuk jenis-jenis asesmen, rubrik penilaian, dan self-assessment tools.

\textbf{Poin Kunci:}
\begin{itemize}
  \item Asesmen mengukur pencapaian kompetensi mahasiswa
  \item Formative assessment untuk monitoring progress
  \item Summative assessment untuk evaluasi akhir
  \item Rubrik penilaian memastikan evaluasi yang objektif
  \item Self-assessment tools untuk refleksi diri
  \item Portfolio assessment untuk dokumentasi pencapaian
  \item Peer review untuk collaborative learning
  \item Feedback yang konstruktif untuk improvement
\end{itemize}

\textbf{Kata Kunci}: \asm{Asesmen}, \asm{Evaluasi}, \asm{Rubrik}, \asm{Formative}, \asm{Summative}, \asm{Self-Assessment}, \asm{Portfolio}, \asm{Peer Review}, \asm{Feedback}
\end{rangkuman}


\ifSubfilesClassLoaded{
  \renewcommand{\bibname}{Daftar Pustaka}
  \bibliographystyle{plain}
  \bibliography{../references}
}{}
\end{document}

\subsection{Rubrik Penilaian}
\begin{table}[htbp]
\centering
\begin{tabular}{|l|c|c|c|c|}
\hline
\textbf{Kriteria} & \textbf{A (85-100)} & \textbf{B (70-84)} & \textbf{C (55-69)} & \textbf{D (0-54)} \\
\hline
\textbf{Koreksi Sintaks} & Tidak ada error & 1-2 error & 3-4 error & >4 error \\
\hline
\textbf{Logika Program} & Sempurna & Minor bug & Major bug & Tidak berjalan \\
\hline
\textbf{Struktur Kode} & Modular, rapi & Terstruktur & Kurang terstruktur & Berantakan \\
\hline
\textbf{Optimasi} & Sangat optimal & Optimal & Cukup optimal & Tidak optimal \\
\hline
\textbf{Dokumentasi} & Lengkap & Cukup & Minimal & Tidak ada \\
\hline
\textbf{Error Handling} & Komprehensif & Ada & Minimal & Tidak ada \\
\hline
\end{tabular}
\caption{Rubrik Penilaian Program Assembly}
\end{table}

\section{Praktikum Guidelines}

\subsection{Submission Requirements}
\begin{itemize}
  \item Source code (.asm) yang well-commented
  \item Executable file (.exe) yang berjalan
  \item Documentation file (.txt/.doc)
  \item Screenshot hasil eksekusi
  \item Test report dengan berbagai test case
\end{itemize}

\subsection{Evaluation Criteria}
\begin{enumerate}
  \item \textbf{Functionality (30\%)}: Program memenuhi semua spesifikasi
  \item \textbf{Code Quality (25\%)}: Struktur, readability, maintainability
  \item \textbf{Performance (20\%)}: Efisiensi dan optimasi
  \item \textbf{Error Handling (15\%)}: Robustness dan recovery
  \item \textbf{Documentation (10\%)}: Komentar dan manual
\end{enumerate}

\section{Project Guidelines}

\subsection{Project Topics}
\begin{itemize}
  \item \textbf{System Utilities}: File manager, system monitor
  \item \textbf{Games}: Puzzle, arcade, strategy games
  \item \textbf{Applications}: Text editor, calculator, database
  \item \textbf{Graphics}: Drawing programs, animations
  \item \textbf{Network}: Simple communication protocols
\end{itemize}

\subsection{Project Milestones}
\begin{table}[htbp]
\centering
\begin{tabular}{|l|l|p{5cm}|}
\hline
\textbf{Minggu} & \textbf{Deliverable} & \textbf{Kriteria Sukses} \\
\hline
1-2 & Proposal & Topik disetujui, requirements jelas \\
\hline
3-4 & Design & Arsitektur modular, interface terdefinisi \\
\hline
5-8 & Implementation & Core functionality berjalan \\
\hline
9-10 & Testing & Unit dan integration test selesai \\
\hline
11-12 & Documentation & User manual dan technical docs lengkap \\
\hline
13-14 & Final Integration & Semua modul terintegrasi, demo siap \\
\hline
15-16 & Presentation & Demo dan laporan akhir \\
\hline
\end{tabular}
\caption{Timeline Proyek Akhir}
\end{table}

\section{Exam Format}

\subsection{Ujian Tengah Semester}
\begin{itemize}
  \item \textbf{Teori (40\%)}: Konsep arsitektur, instruksi, addressing
  \item \textbf{Praktik (60\%)}: Debugging, kode completion, optimasi
\end{itemize}

\subsection{Ujian Akhir Semester}
\begin{itemize}
  \item \textbf{Written Exam (50\%)}:
  \begin{itemize}
    \item Essay questions (konsep dan teori)
    \item Code analysis (membaca dan menjelaskan kode)
    \item Problem solving (desain algoritma)
  \end{itemize}
  \item \textbf{Practical Exam (50\%)}:
  \begin{itemize}
    \item Coding on paper (menulis instruksi assembly)
    \item Debugging session (menemukan error dalam kode)
    \item Performance optimization
  \end{itemize}
\end{itemize}

\section{Self-Assessment Tools}

\subsection{Checklist Kompetensi}
\begin{itemize}
  \item \textbf{Basic Skills}: Register usage, instruksi dasar, addressing modes
  \item \textbf{Intermediate Skills}: Prosedur, stack, interrupts, file I/O
  \item \textbf{Advanced Skills}: Optimasi, debugging, modular design
  \item \textbf{Expert Skills}: Project development, system integration
\end{itemize}

\subsection{Portfolio Development}
\begin{itemize}
  \item Kumpulkan semua praktikum dan proyek
  \item Dokumentasikan learning progress
  \item Sertakan reflection untuk setiap milestone
  \item Buat showcase dari best works
\end{itemize}

% ============================================================
% AKTIVITAS PEMBELAJARAN
% ============================================================

\begin{aktivitas}
  \item \textbf{Peer Review}: Review praktikum teman dengan rubrik yang diberikan.
  
  \item \textbf{Self Assessment}: Evaluasi kemampuan pribadi dengan competency checklist.
  
  \item \textbf{Project Planning}: Buat detailed plan untuk proyek akhir dengan milestones.
  
  \item \textbf{Mock Exam}: Simulasi ujian dengan soal-soal latihan.
  
  \item \textbf{Code Review Session}: Presentasi kode untuk feedback dari teman dan dosen.
  
  \item \textbf{Portfolio Building}: Organisasi dan dokumentasi semua karya terbaik.
\end{aktivitas}

% ============================================================
% LATIHAN DAN REFLEKSI
% ============================================================

\begin{latihan}
  \item Buat self-assessment berdasarkan rubrik penilaian untuk:
  \begin{itemize}
    \item Program kalkulator yang sudah dibuat
    \item Proyek text editor
    \item Library fungsi matematika
  \end{itemize}
  
  \item Rancang study plan untuk ujian tengah semester dengan:
  \begin{itemize}
    \item Topik-topik yang perlu dikuasai
    \item Waktu alokasi per topik
    \item Practice exercises untuk setiap topik
  \end{itemize}
  
  \item Buat project proposal untuk proyek akhir dengan:
  \begin{itemize}
    \item Problem statement
    \item Technical requirements
    \item Implementation plan
    \item Success criteria
  \end{itemize}
  
  \item Simulasi practical exam dengan:
  \begin{itemize}
    \item Debug 3 program dengan error berbeda
    \item Optimasi 2 algoritma sorting
    \item Implementasi 1 fungsi kompleks dalam waktu terbatas
  \end{itemize}
  
  \item Buat portfolio showcase dengan:
  \begin{itemize}
    \item 3 best programs dari praktikum
    \item 1 proyek terbaik
    \item Documentation lengkap
    \item Reflection untuk setiap karya
  \end{itemize}
  
  \item \textbf{Refleksi}: Bagaimana Anda mengukur progress belajar assembly programming? Apa indikator sukses yang paling penting?
\end{latihan}

% ============================================================
% ASESMEN
% ============================================================

\begin{asesmen}
\textbf{Instrumen Penilaian untuk Sub-CPMK 9.1, 9.2, 2.1}

\textbf{A. Pilihan Ganda}

\begin{enumerate}
  \item Komponen terbesar dalam penilaian praktikum adalah:
  \begin{enumerate}
    \item Functionality
    \item Code quality
    \item Performance
    \item Documentation
  \end{enumerate}
  
  \item Durasi proyek akhir adalah:
  \begin{enumerate}
    \item 4 minggu
    \item 8 minggu
    \item 12 minggu
    \item 16 minggu
  \end{enumerate}
  
  \item Bobot ujian akhir praktik adalah:
  \begin{enumerate}
    \item 25\%
    \item 50\%
    \item 75\%
    \item 100\%
  \end{enumerate}
  
  \item Tools untuk self-assessment meliputi:
  \begin{enumerate}
    \item Checklist kompetensi
    \item Portfolio
    \item Peer review
    \item Semua benar
  \end{enumerate}
\end{enumerate}

\textbf{B. Essay}

\begin{enumerate}
  \item Jelaskan pentingnya asesmen komprehensif dalam pembelajaran assembly programming!
  
  \item Bagaimana rubrik penilaian membantu meningkatkan kualitas kode mahasiswa?
\end{enumerate}

\textbf{C. Practical Challenge}

\begin{enumerate}
  \item Buat assessment portfolio:
  \begin{itemize}
    \item Pilih 3 program terbaik dari semester ini
    \item Evaluasi masing-masing dengan rubrik lengkap
    \item Tulis improvement plan untuk setiap program
    \item Buat reflection tentang learning journey
    \item Present portfolio dengan demo dan explanation
  \end{itemize}
\end{enumerate}

\textbf{Rubrik Penilaian}: Lihat Lampiran A
\end{asesmen}

% ============================================================
% CHECKLIST KOMPETENSI
% ============================================================

\begin{checklist}
  \item Saya dapat memahami kriteria penilaian untuk assembly programming
  \item Saya dapat melakukan self-assessment dengan objektif
  \item Saya dapat memberikan konstruktif feedback pada kode teman
  \item Saya dapat merencanakan proyek dengan milestones yang jelas
  \item Saya dapat mempersiapkan diri untuk ujian teori dan praktik
  \item Saya dapat membangun portfolio yang showcase kemampuan
  \item Saya dapat mengidentifikasi area untuk improvement
  \item Saya dapat mendemonstrasikan kompetensi assembly secara komprehensif
\end{checklist}

% ============================================================
% RANGKUMAN
% ============================================================

\begin{rangkuman}
Bab ini membahas sistem asesmen komprehensif untuk mata kuliah assembly programming, termasuk praktikum, proyek, dan ujian.

\textbf{Poin Kunci:}
\begin{itemize}
  \item Asesmen mencakup praktikum, proyek, dan ujian teori/praktik
  \item Rubrik penilaian memberikan evaluasi yang objektif dan transparan
  \item Praktikum fokus pada implementasi konsep mingguan
  \item Proyek akhir menguji kemampuan integrasi dan design
  \item Ujian mengukur pemahaman teori dan kemampuan praktik
  \item Self-assessment tools membantu mahasiswa mengukur progress
  \item Portfolio development untuk showcase kemampuan terbaik
  \item Peer review dan feedback untuk continuous improvement
\end{itemize}

\textbf{Kata Kunci}: \textbf{\texttt{Asesmen}}, \textbf{\texttt{Evaluasi}}, \textbf{\texttt{Rubrik}}, \textbf{\texttt{Praktikum}}, \textbf{\texttt{Proyek}}, \textbf{\texttt{Ujian}}, \textbf{\texttt{Self-Assessment}}, \textbf{\texttt{Portfolio}}, \textbf{\texttt{Competency Checklist}}, \textbf{\texttt{TASM}}
\end{rangkuman}

\ifSubfilesClassLoaded{
  \renewcommand{\bibname}{Daftar Pustaka}
  \bibliographystyle{plain}
  \bibliography{../references}
}{}
\end{document}
