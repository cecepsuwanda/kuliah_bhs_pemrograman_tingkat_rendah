% ============================================================
% TASM Development Environment
% ============================================================

% Define asm command for standalone compilation (providecommand avoids conflict when included in main)
\providecommand{\asm}[1]{\textbf{\texttt{#1}}}

\section{TASM Development Environment}

Turbo Assembler (TASM) adalah development environment untuk pemrograman assembly Intel 8086.

\subsection{Komponen TASM}

\begin{itemize}
  \item \textbf{TASM.EXE}: Assembler utama untuk mengompilasi kode assembly
  \item \textbf{TLINK.EXE}: Linker untuk menghubungkan objek files
  \item \textbf{TD.EXE}: Debugger untuk debugging program assembly
  \item \textbf{MAKE.EXE}: Build automation tool
\end{itemize}

\subsection{Workflow Development}

\begin{center}
\textbf{PEMROGRAMAN ASSEMBLY INTEL 8086}\\
$\downarrow$\\
\textbf{DEVELOPMENT CYCLE}\\
$\downarrow$\\
\begin{tabular}{ccc}
Write Code & Compile & Debug
\end{tabular}\\
$\downarrow$\\
\textbf{PRODUKSI}\\
$\downarrow$\\
\begin{tabular}{ccc}
Executable File & Testing & Deployment
\end{tabular}\\
\end{center}

\subsection{TASM Commands}

\begin{verbatim}
; Compile assembly file
TASM /zi program.asm
TLINK /v program.obj

; Run debugger
TD program.exe

; Build dengan makefile
MAKE
\end{verbatim}

\textbf{Switch Options:}
\begin{itemize}
  \item \textbf{/zi}: Include debug information
  \item \textbf{/v}: Include verbose output
  \item \textbf{/c}: Case-sensitive symbols
  \item \textbf{/n}: Generate listing file
\end{itemize}

\textbf{Kata Kunci}: \textbf{\texttt{Register}}, \textbf{\texttt{Addressing Mode}}, \textbf{\texttt{Segmentation}}, \textbf{\texttt{TASM}}, \textbf{\texttt{Debugger}}
