% ============================================================
% TASM Development Environment
% ============================================================

% Define asm command for standalone compilation (providecommand avoids conflict when included in main)
\providecommand{\asm}[1]{\textbf{\texttt{#1}}}

\section{TASM Development Environment}

Turbo Assembler (TASM) adalah development environment untuk pemrograman assembly Intel 8086 \cite{borland1990tasm}. GUI Turbo Assembler (GTASM) mengintegrasikan TASM, TLINK, Turbo Debugger, dan DOSBox dalam satu IDE yang ramah pengguna untuk Windows modern \cite{jones2020}.

\subsection{Instalasi GUI Turbo Assembler (GTASM)}

GTASM cocok untuk pembelajaran assembly pada Windows 7, 8, 10, 11 (32-bit maupun 64-bit). Persyaratan: Microsoft .NET Framework 4.0 atau lebih tinggi, minimal 100 MB ruang penyimpanan.

\textbf{Langkah instalasi:}
\begin{enumerate}
  \item Unduh dari \url{https://github.com/ljnath/GUI-Turbo-Assembler} atau Softpedia
  \item Jalankan \texttt{GTASM\_Setup.exe}; jika ada peringatan keamanan, klik ``More info'' lalu ``Run anyway''
  \item Ikuti wizard (Next $\rightarrow$ pilih direktori $\rightarrow$ Install)
  \item Verifikasi: Buka GTASM, menu Help $\rightarrow$ About untuk memastikan TASM, TLINK, TD, DOSBox terinstal
\end{enumerate}

\begin{figure}[H]
\centering
\includegraphics[width=0.75\textwidth]{gtasm_installer.png}
\caption{Wizard instalasi GUI Turbo Assembler}
\end{figure}

\subsection{Komponen TASM}

\begin{itemize}
  \item \textbf{TASM.EXE}: Assembler utama—mengubah berkas \asm{.ASM} menjadi \asm{.OBJ} (object code). Mengecek sintaks dan menghasilkan kode mesin.
  \item \textbf{TLINK.EXE}: Linker—menghubungkan satu atau lebih \asm{.OBJ} dan library menjadi berkas \asm{.EXE} atau \asm{.COM}. Menyelesaikan referensi eksternal dan relokasi.
  \item \textbf{TD.EXE} (Turbo Debugger): Debugger untuk step-by-step eksekusi, breakpoint, pemeriksaan register dan memori.
  \item \textbf{DOSBox}: Emulator lingkungan DOS untuk menjalankan program assembly pada Windows modern; sudah terintegrasi dalam GTASM.
\end{itemize}

\begin{figure}[H]
\centering
\includegraphics[width=0.75\textwidth]{gtasm_interface.png}
\caption{Antarmuka utama GUI Turbo Assembler}
\end{figure}

\subsection{Struktur Program .COM vs .EXE}

\begin{table}[H]
\centering
\caption{Perbandingan .COM dan .EXE}
\setlength{\tabcolsep}{3pt}
\scriptsize
\begin{tabular}{|p{1.8cm}|p{3.5cm}|p{3.5cm}|}
\hline
\textbf{Aspek} & \textbf{.COM} & \textbf{.EXE} \\
\hline
Header & Tanpa header & Ada header relocation \\
\hline
Ukuran max & $\sim$64 KB & Hingga beberapa MB \\
\hline
Segmen & CS=DS=ES=SS (umumnya sama) & Kode, data, stack terpisah \\
\hline
Entry point & ORG 100h (offset 0100h) & END label\_utama \\
\hline
Model & Cocok untuk program kecil & Cocok untuk program besar \\
\hline
\end{tabular}
\end{table}

Program \asm{.COM}: gunakan \asm{ORG 100h} dan pastikan semua dalam satu segmen. Program \asm{.EXE}: gunakan \asm{.MODEL SMALL}, \asm{.DATA}, \asm{.CODE}, \asm{ASSUME}, dan inisialisasi \asm{DS} dengan \asm{MOV AX, @data} / \asm{MOV DS, AX}.

\subsection{Workflow Development}

\begin{center}
\textbf{PEMROGRAMAN ASSEMBLY INTEL 8086}\\
$\downarrow$\\
\textbf{DEVELOPMENT CYCLE}\\
$\downarrow$\\
{\small
\begin{tabular}{ccc}
Write Code & Compile & Debug
\end{tabular}
}\\
$\downarrow$\\
\textbf{PRODUKSI}\\
$\downarrow$\\
{\small
\begin{tabular}{ccc}
Executable File & Testing & Deployment
\end{tabular}
}\\
\end{center}

\begin{figure}[H]
\centering
\includegraphics[width=0.7\textwidth]{gtasm_build_process.png}
\caption{Proses build dengan GTASM (F9)}
\end{figure}

\subsection{TASM Commands}

\textbf{Command-line (jika tidak menggunakan GTASM):}
\begin{verbatim}
; Compile assembly file
TASM /zi program.asm
TLINK /v program.obj

; Run debugger
TD program.exe

; Build dengan makefile
MAKE
\end{verbatim}

\textbf{Switch Options:}
\begin{itemize}
  \item \textbf{/zi}: Include debug information (untuk breakpoint dan step)
  \item \textbf{/v}: Include verbose output
  \item \textbf{/c}: Case-sensitive symbols
  \item \textbf{/n}: Generate listing file
\end{itemize}

Dengan GTASM, cukup tekan F9 (Build) dan F10 (Run) untuk mengompilasi dan menjalankan program.
