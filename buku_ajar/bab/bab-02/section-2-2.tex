\section{Register dan Organisasi Memori}

\subsection{General Purpose Registers}

Intel 8086 memiliki 8 general-purpose registers 16-bit yang dapat digunakan untuk berbagai operasi:

\begin{verbatim}
; Register breakdown (16-bit)
AX = AH + AL    ; Accumulator (High + Low)
BX = BH + BL    ; Base register
CX = CH + CL    ; Count register
DX = DH + DL    ; Data register
\end{verbatim}

\textbf{Fungsi Spesifik Register:}
\begin{itemize}
  \item \textbf{AX}: Akumulator untuk operasi aritmatika dan I/O
  \item \textbf{BX}: Base register untuk addressing modes
  \item \textbf{CX}: Count register untuk loop dan string operations
  \item \textbf{DX}: Data register untuk aritmatika dan I/O
\end{itemize}

\subsection{Special Purpose Registers}

\begin{verbatim}
; Special purpose registers
SP  - Stack Pointer (menunjuk ke top of stack)
BP  - Base Pointer (menunjuk ke base of stack frame)
SI  - Source Index (index register untuk string operations)
DI  - Destination Index (index register untuk string operations)
IP  - Instruction Pointer (menunjuk ke next instruction)
\end{verbatim}

\textbf{Penggunaan Special Registers:}
\begin{itemize}
  \item \textbf{SP}: Stack management dan procedure calls
  \item \textbf{BP}: Access ke stack frame dan local variables
  \item \textbf{SI/DI}: String operations dan array indexing
  \item \textbf{IP}: Program counter (tidak dapat diakses langsung)
\end{itemize}

\subsection{Segment Registers}

\begin{verbatim}
; Segment registers untuk memory management
CS  - Code Segment (instruction pointer)
DS  - Data Segment (data access)
SS  - Stack Segment (stack operations)
ES  - Extra Segment (data transfer)
\end{verbatim}

\textbf{Segmentation Memory Model:}
\begin{itemize}
  \item \textbf{CS}: Berisi alamat segment untuk kode program
  \item \textbf{DS}: Berisi alamat segment untuk data
  \item \textbf{SS}: Berisi alamat segment untuk stack
  \item \textbf{ES}: Berisi alamat segment untuk string operations
\end{itemize}

\subsection{Perhitungan Alamat Fisik}

Intel 8086 menggunakan model memori segmentasi di mana alamat logis terdiri dari pasangan \textbf{segment:offset}. Alamat fisik dihitung dengan formula:

\begin{center}
\textbf{Alamat Fisik} = (Segment $\times$ 16) + Offset
\end{center}

\begin{contoh}
\textbf{Contoh Perhitungan Alamat:}

Untuk CS:IP = 2000h:1000h:
\begin{itemize}
  \item Alamat fisik = (2000h $\times$ 10h) + 1000h = 20000h + 1000h = 21000h
\end{itemize}

Untuk DS:SI = 3000h:0050h:
\begin{itemize}
  \item Alamat fisik = (3000h $\times$ 10h) + 0050h = 30000h + 50h = 30050h
\end{itemize}
\end{contoh}

\subsection{Contoh Penggunaan Register dalam Instruksi}

Berikut contoh instruksi assembly yang memanfaatkan register:

\begin{verbatim}
; Load immediate value ke AX
MOV AX, 1234h

; Copy dari AX ke BX (register addressing)
MOV BX, AX

; Store ke memory (direct addressing)
MOV [0100h], AX

; Access dengan segment:offset
MOV AX, DS:[SI]
\end{verbatim}
