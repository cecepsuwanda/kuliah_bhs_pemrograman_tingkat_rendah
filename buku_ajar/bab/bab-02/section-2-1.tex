\section{Arsitektur Intel 8086}

Intel 8086 adalah 16-bit processor yang menjadi fondasi untuk arsitektur x86 modern. Processor ini dirancang untuk memberikan keseimbangan antara performa, kompleksitas, dan biaya produksi \cite{ref1}.

\subsection{Evolusi Processor Intel}

\begin{enumerate}
  \item \textbf{1971}: Intel 4004 - 4-bit processor pertama
  \item \textbf{1974}: Intel 8080 - 8-bit processor
  \item \textbf{1978}: Intel 8086 - 16-bit processor (target kita)
  \item \textbf{1982}: Intel 80286 - 16-bit dengan protected mode
  \item \textbf{1985}: Intel 80386 - 32-bit processor
  \item \textbf{1989}: Intel 80486 - 32-bit dengan cache
  \item \textbf{1993}: Pentium - Superscalar architecture
\end{enumerate}

\subsection{Fitur Utama Intel 8086}

\textbf{Arsitektur 16-bit:}
\begin{itemize}
  \item Data bus 16-bit untuk transfer data paralel
  \item Address bus 20-bit untuk 1MB addressable memory
  \item Clock speed: 4.77MHz hingga 10MHz
  \item 29,000 transistors pada chip
\end{itemize}

\textbf{Instruction Set:}
\begin{itemize}
  \item 117 instruksi dasar
  \item Operasi aritmatika, logika, dan transfer data
  \item String processing instructions
  \item Control flow dan loop instructions
  \item I/O dan interrupt instructions
\end{itemize}

\textbf{Mode Operasi:}
\begin{itemize}
  \item \textbf{Real Mode}: Kompatibel dengan 8080/8085
  \item \textbf{Protected Mode}: Tersedia di 80286 ke atas
  \item \textbf{Virtual 8086 Mode}: Tersedia di 80386 ke atas
\end{itemize}
