\section{Arsitektur Intel 8086}

Intel 8086 adalah 16-bit processor yang menjadi fondasi untuk arsitektur x86 modern \cite{ref1}. Mikroprosesor ini diperkenalkan pada Juni 1978 dan merevolusi industri komputer dengan kemampuannya mengalamati hingga 1 MB memori melalui sistem segmentasi \cite{intel_8086_user_manual, wiki_8086}.

\subsection{Spesifikasi Teknis}

\begin{table}[H]
\centering
\caption{Spesifikasi Teknis Intel 8086}
\begin{tabular}{|p{2.2cm}|p{2.5cm}|p{5cm}|}
\hline
\textbf{Aspek} & \textbf{Spesifikasi} & \textbf{Keterangan} \\
\hline
Arsitektur & 16-bit internal/eksternal & Data path dan register 16-bit \\
\hline
Bus Alamat & 20-bit & Pengalamatan hingga 1 MB ($2^{20}$ byte) \\
\hline
Bus Data & 16-bit & Transfer data 2 byte sekaligus \\
\hline
Frekuensi Clock & 5--10 MHz & Tergantung versi prosesor \\
\hline
Transistor & $\sim$29.000 & Proses manufaktur 3-micron HMOS \\
\hline
Konsumsi Daya & $\sim$2.5 watt & Relatif efisien untuk era tersebut \\
\hline
Paket & 40-pin DIP & Dual Inline Package \\
\hline
\end{tabular}
\end{table}

\subsection{Arsitektur Internal: BIU dan EU}

Arsitektur internal Intel 8086 menggunakan desain \textit{pipelined} yang terdiri dari dua unit utama yang bekerja secara paralel \cite{computer_organization_design}:

\begin{figure}[H]
\centering
\includegraphics[width=0.75\textwidth]{8086_architecture.jpg}
\caption{Arsitektur Internal Intel 8086}
\end{figure}

\textbf{Bus Interface Unit (BIU)} bertanggung jawab untuk komunikasi dengan memori dan I/O eksternal:
\begin{itemize}
  \item \textbf{Instruction Queue}: Buffer 6-byte untuk instruksi yang telah diambil
  \item \textbf{Segment Registers}: CS, DS, SS, ES untuk pengalamatan segmentasi
  \item \textbf{Instruction Pointer (IP)}: Menunjuk ke instruksi berikutnya
  \item \textbf{Address Adder}: Menghitung alamat fisik dari segment:offset
  \item \textbf{Bus Control Logic}: Mengatur sinyal kontrol bus (RD, WR, M/IO)
\end{itemize}

\textbf{Execution Unit (EU)} bertanggung jawab untuk eksekusi instruksi:
\begin{itemize}
  \item \textbf{ALU}: Melakukan operasi aritmatika dan logika
  \item \textbf{General Purpose Registers}: AX, BX, CX, DX
  \item \textbf{Index and Pointer Registers}: SI, DI, SP, BP
  \item \textbf{Flag Register}: Menyimpan status hasil operasi
\end{itemize}

\textbf{Prefetch Queue}: Ketika EU mengeksekusi instruksi, BIU secara bersamaan mengambil instruksi berikutnya dari memori ke queue 6-byte, sehingga eksekusi lebih efisien.

\subsection{Evolusi Processor Intel}

\begin{enumerate}
  \item \textbf{1971}: Intel 4004 -- 4-bit processor pertama
  \item \textbf{1974}: Intel 8080 -- 8-bit processor
  \item \textbf{1978}: Intel 8086 -- 16-bit processor (target pembelajaran)
  \item \textbf{1982}: Intel 80286 -- 16-bit dengan protected mode
  \item \textbf{1985}: Intel 80386 -- 32-bit processor
  \item \textbf{1989}: Intel 80486 -- 32-bit dengan cache
  \item \textbf{1993}: Pentium -- Superscalar architecture
\end{enumerate}

\subsection{Instruction Set dan Mode Operasi}

\textbf{Instruction Set:} 117 instruksi dasar untuk operasi aritmatika, logika, transfer data, string, control flow, dan I/O.

\textbf{Mode Operasi:}
\begin{itemize}
  \item \textbf{Real Mode}: Satu-satunya mode pada 8086; kompatibel dengan 8080/8085
  \item \textbf{Protected Mode}: Tersedia di 80286 ke atas
  \item \textbf{Virtual 8086 Mode}: Tersedia di 80386 ke atas
\end{itemize}
