\section{Addressing Modes}

Addressing modes menentukan bagaimana processor mengakses operan untuk instruksi \cite{electronics_hub_8086_addressing}. Pemilihan mode mempengaruhi kecepatan eksekusi, ukuran kode, dan fleksibilitas program.

\subsection{Immediate Addressing}

Mode immediate menggunakan nilai konstan yang disertakan langsung dalam instruksi. Cocok untuk inisialisasi register, konstanta mask, dan nilai yang tidak berubah saat runtime.

\begin{verbatim}
; Immediate addressing - nilai konstan langsung
MOV AX, 5        ; AX = 5 (immediate value)
ADD CX, 100h      ; CX = CX + 256
MOV BX, 'A'        ; BX = ASCII 'A'
\end{verbatim}

\textbf{Karakteristik:}
\begin{itemize}
  \item Nilai konstan langsung dalam instruksi
  \item Tidak memerlukan akses memori untuk operand
  \item Paling cepat untuk operasi konstan
  \item Operand kedua biasanya; format: hex (h), desimal, biner (b), ASCII ('A')
\end{itemize}

\subsection{Register Addressing}

Mode register paling efisien karena operand berada langsung di register. Tidak ada akses memori tambahan sehingga kecepatan eksekusi tertinggi. Digunakan untuk perhitungan sementara dan operasi berulang dalam loop.

\begin{verbatim}
; Register addressing - operan adalah register
MOV AX, BX        ; AX = BX
ADD CX, AX        ; CX = CX + AX
SUB DX, SI        ; DX = DX - SI
\end{verbatim}

\textbf{Karakteristik:}
\begin{itemize}
  \item Operan adalah register general purpose (AX, BX, CX, DX, SI, DI, BP, SP)
  \item Tidak memerlukan akses memori
  \item Sangat cepat dan efisien
  \item Beberapa instruksi memiliki register khusus (mis. MUL menggunakan AX)
\end{itemize}

\subsection{Direct Addressing}

Mode direct menggunakan alamat memori yang ditentukan secara eksplisit. Alamat adalah offset dalam segmen default (biasanya DS). Berguna untuk akses variabel global dengan alamat tetap.

\begin{verbatim}
; Direct addressing - alamat memori langsung
MOV AX, [1234h]    ; AX = memory[DS:1234h]
ADD BX, [data_var]   ; BX = memory[data_var]
MOV CX, [array+10]   ; CX = memory[array+10]
\end{verbatim}

\textbf{Karakteristik:}
\begin{itemize}
  \item Alamat memori langsung dalam instruksi
  \item Akses ke memory yang spesifik (default segment: DS)
  \item Lebih lambat dari register addressing (1 siklus memori)
  \item Segment override: \asm{ES:[addr]}, \asm{CS:[addr]}, \asm{SS:[addr]}
\end{itemize}

\subsection{Register Indirect Addressing}

Mode indirect menggunakan register sebagai pointer ke memori. Register (BX, SI, DI, BP) berisi alamat offset. Fleksibel untuk akses array dan struktur data dinamis karena alamat dapat diubah saat runtime.

\begin{verbatim}
; Register indirect - register berisi alamat
MOV AX, [BX]        ; AX = memory[DS:BX]
MOV CX, [SI]        ; CX = memory[DS:SI]
ADD DX, [DI]        ; DX = memory[ES:DI]
MOV AX, [BP]        ; AX = memory[SS:BP] (stack frame)
\end{verbatim}

\textbf{Karakteristik:}
\begin{itemize}
  \item Register valid: BX, SI, DI, BP
  \item Segment default: DS untuk BX/SI/DI; SS untuk BP
  \item Fleksibel untuk pointer dan loop array
  \item Lebih efisien daripada direct untuk akses berulang
\end{itemize}

\subsection{Based Indexed Addressing}

Mode based indexed menggabungkan base register (BX atau BP) dengan index register (SI atau DI) dan displacement opsional. Sangat fleksibel untuk akses array dua dimensi, struktur data, dan parameter stack.

\begin{verbatim}
; Based indexed addressing - base + index + displacement
MOV AX, [BX+SI]     ; AX = memory[BX+SI]
MOV CX, [BP+DI+4]   ; CX = memory[BP+DI+4] (parameter stack)
ADD DX, [BX+SI+10]  ; DX = memory[BX+SI+10]
\end{verbatim}

\textbf{Karakteristik:}
\begin{itemize}
  \item Kombinasi valid: BX+SI, BX+DI, BP+SI, BP+DI
  \item Displacement opsional: 8-bit atau 16-bit
  \item Sangat fleksibel untuk array 2D dan struktur data
  \item BP+displacement untuk akses parameter prosedur
\end{itemize}

\subsection{Perbandingan dan Pemilihan}

\begin{table}[H]
\centering
\caption{Perbandingan Mode Addressing}
\setlength{\tabcolsep}{3pt}
\scriptsize
\begin{tabular}{|p{1.8cm}|p{1.5cm}|p{2cm}|p{3.2cm}|}
\hline
\textbf{Mode} & \textbf{Kecepatan} & \textbf{Fleksibilitas} & \textbf{Use Case Optimal} \\
\hline
Immediate & Paling cepat & Rendah & Konstanta, inisialisasi \\
\hline
Register & Sangat cepat & Sedang & Perhitungan, loop body \\
\hline
Direct & Cepat & Sedang & Variabel global, alamat tetap \\
\hline
Indirect & Sedang & Tinggi & Array, pointer dinamis \\
\hline
Based Indexed & Sedang & Tertinggi & Array 2D, struktur, stack frame \\
\hline
\end{tabular}
\end{table}

\subsection{Contoh Kombinasi: \asm{MOV AX, [BX+SI+2]}}

Instruksi ini menggunakan based indexed addressing. CPU melakukan langkah berikut:
\begin{enumerate}
  \item Baca nilai BX (base address, misal 1000h)
  \item Baca nilai SI (index, misal 0010h)
  \item Hitung alamat efektif: 1000h + 0010h + 2 = 1012h
  \item Gabung dengan DS: alamat fisik = (DS × 16) + 1012h
  \item Baca 2 byte dari memori (LSB dulu, little-endian) ke AX
\end{enumerate}

Aplikasi: mengakses \texttt{array[BX][SI]} atau field struktur dengan offset 2.
