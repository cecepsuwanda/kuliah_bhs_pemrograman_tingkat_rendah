% ============================================================
% AKTIVITAS PEMBELAJARAN
% ============================================================

\begin{aktivitas}
  \item \textbf{Research Study}: Lakukan penelitian mendalam tentang topik assembly programming yang menarik.
  
  \item \textbf{Documentation Review}: Review dan dokumentasikan kode assembly yang telah dibuat.
  
  \item \textbf{Tool Comparison}: Bandingkan TASM dengan assembler modern (NASM, MASM, FASM).
  
  \item \textbf{Advanced Topics}: Eksplorasi topik advanced seperti memory management dan system programming.
  
  \item \textbf{Knowledge Sharing}: Bagikan pengetahuan dengan mahasiswa lain melalui presentasi.
  
  \item \textbf{Resource Curation}: Kumpulkan dan evaluasi sumber belajar yang berkualitas.
\end{aktivitas}

% ============================================================
% LATIHAN DAN REFLEKSI
% ============================================================

\begin{latihan}
  \item Lakukan penelitian tentang optimasi cache-friendly assembly code dan buat laporan hasilnya.
  
  \item Bandingkan performa TASM dengan assembler modern untuk berbagai jenis program.
  
  \item Implementasikan advanced memory management dengan dynamic allocation.
  
  \item Buat tutorial assembly programming untuk topik yang Anda pilih.
  
  \item Kumpulkan 10 sumber belajar assembly programming yang paling bermanfaat.
  
  \item Buat comparative analysis antara berbagai assembler tools.
  
  \item \textbf{Refleksi}: Topik assembly mana yang paling menantang untuk karir Anda? Bagaimana Anda akan mendalami lebih lanjut?
\end{latihan}

% ============================================================
% ASESMEN
% ============================================================

\begin{asesmen}
\textbf{Instrumen Penilaian untuk Sub-CPMK 11.1, 11.2}

\textbf{A. Pilihan Ganda}

\begin{enumerate}
  \item Sumber referensi resmi untuk Intel 8086 adalah:
  \begin{enumerate}
    \item Intel Developer Zone
    \item Stack Overflow
    \item GitHub repositories
    \item MSDN Library
  \end{enumerate}
  
  \item Tool debugging yang paling modern untuk assembly adalah:
  \begin{enumerate}
    \item GDB
    \item IDA Pro
    \item x64dbg
    \item OllyDbg
  \end{enumerate}
  
  \item Teknik optimasi yang paling advanced adalah:
  \begin{enumerate}
    \item Pipeline optimization
    \item Cache optimization
    \item Register allocation
    \item Instruction scheduling
  \end{enumerate}
  
  \item Memory management yang paling kompleks adalah:
  \begin{enumerate}
    \item Dynamic allocation
    \item Virtual memory management
    \item Garbage collection
    \item Memory protection
  \end{enumerate}
\end{enumerate}

\textbf{B. Essay}

\begin{enumerate}
  \item Jelaskan pentingnya dokumentasi resmi dalam pengembangan assembly programming!
  
  \item Bagaimana research terkini penting untuk tetap update dengan teknologi terkini?
\end{enumerate}

\textbf{C. Practical Challenge}

\begin{enumerate}
  \item Buat research project:
  \begin{itemize}
    \item Topik penelitian yang relevan (dibahas bersama dosen)
    \item Literature review dari sumber terpercaya
    \item Implementasi konsep dalam assembly
    \item Performance analysis dan benchmarking
    \item Penulisan paper untuk publikasi
    \item Presentasi hasil penelitian
    \item Documentation lengkap untuk reproduksi
    \item Future research recommendations
  \end{itemize}
\end{enumerate}

\textbf{Rubrik Penilaian}: Lihat Lampiran A
\end{asesmen}

% ============================================================
% CHECKLIST KOMPETENSI
% ============================================================

\begin{checklist}
  \item Saya dapat mengakses dan memanfaatkan dokumentasi resmi Intel
  \item Saya dapat menggunakan berbagai assembler modern
  \item Saya dapat melakukan research topik assembly programming
  \item Saya dapat mengimplementasikan teknik optimasi advanced
  \item Saya dapat mengimplementasikan memory management
  \item Saya dapat menggunakan debugging tools modern
  \item Saya dapat menulis tutorial dan dokumentasi
  \item Saya dapat berbagi pengetahuan dengan orang lain
\end{checklist}

% ============================================================
% RANGKUMAN
% ============================================================

\begin{rangkuman}
Bab ini membahas materi referensi dan topik advanced dalam assembly programming, termasuk dokumentasi resmi, online resources, dan development tools.

\textbf{Poin Kunci:}
\begin{itemize}
  \item Dokumentasi resmi Intel dan TASM adalah sumber informasi yang otoritatif
  \item Online resources menyediakan pembelajaran yang komprehensif
  \item Modern assembler menawarkan fitur dan kemudahan penggunaan
  \item Advanced topics memperluas kemampuan programming assembly
  \item Research dan development penting untuk inovasi
  \item Community dan forum menyediakan support dan kolaborasi
  \item Continuous learning essential untuk tetap relevan
\end{itemize}

\textbf{Kata Kunci}: \asm{Reference}, \asm{Documentation}, \asm{Online Resources}, \asm{Modern Assembler}, \asm{Debugging Tools}, \asm{Advanced Topics}, \asm{Research}, \asm{Continuous Learning}
\end{rangkuman}
