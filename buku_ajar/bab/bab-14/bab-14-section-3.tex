\section{Tinjauan Pencapaian Kompetensi Secara Menyeluruh}

\subsection{Pemetaan Asesmen ke CPMK}

\begin{table}[htbp]
\centering
\begin{tabular}{|p{5cm}|p{8cm}|}
\hline
\textbf{CPMK} & \textbf{Diukur Melalui} \\
\hline
CPMK-1: Memahami konsep register, segmentasi memori, dan instruksi dasar & Pilihan Ganda (1-4), Essay (1-2) \\
\hline
CPMK-2: Merancang program assembly dengan struktur modular & Essay (4), Coding Challenge (Design) \\
\hline
CPMK-3: Mengimplementasikan program assembly dengan TASM & Analisis Kode, Coding Challenge (Implementation) \\
\hline
CPMK-4: Menganalisis dan mengoptimasi kode assembly & Analisis Kode, Coding Challenge (Quality) \\
\hline
\end{tabular}
\caption{Pemetaan Komponen Asesmen ke CPMK}
\end{table}

\subsection{Self-Assessment Checklist}

Sebelum mengerjakan asesmen akhir, pastikan Anda telah menguasai:

\begin{checklist}
  \item Konsep register dan segmentasi memori Intel 8086
  \item Instruksi dasar dan mode addressing
  \item Struktur kontrol dan perulangan dalam assembly
  \item Prosedur dan stack frame management
  \item Input/Output dengan interrupt BIOS dan DOS
  \item Struktur program modular dengan prosedur
  \item Teknik optimasi (strength reduction, loop unrolling)
  \item Debugging dengan TASM debugger
  \item Dokumentasi dan komentar kode
  \item Best practices dalam penulisan kode assembly
\end{checklist}
