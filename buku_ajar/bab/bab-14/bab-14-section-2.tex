\section{Rubrik Penilaian Komprehensif}

\subsection{Rubrik untuk Coding Challenge}

\begin{table}[htbp]
\centering
\small
\begin{tabular}{|>{\raggedright\arraybackslash}p{2.2cm}|>{\raggedright\arraybackslash}p{2.6cm}|>{\raggedright\arraybackslash}p{2.6cm}|>{\raggedright\arraybackslash}p{2.6cm}|>{\raggedright\arraybackslash}p{2.5cm}|}
\hline
\textbf{Kriteria} & \textbf{Excellent (4)} & \textbf{Good (3)} & \textbf{Fair (2)} & \textbf{Poor (1)} \\
\hline
Correctness & Program berfungsi sempurna sesuai spesifikasi & Berfungsi dengan minor bug & Banyak bug & Tidak berjalan \\
\hline
Modularity & Prosedur terstruktur, code reuse optimal & Struktur prosedur baik & Struktur kurang modular & Monolitik \\
\hline
Documentation & Komentar lengkap, struktur jelas & Dokumentasi adequate & Minimal comments & Tidak ada dokumentasi \\
\hline
Optimization & Register efisien, strength reduction, loop unrolling & Optimasi diterapkan & Optimasi minimal & Tanpa optimasi \\
\hline
Error Handling & Validasi input, penanganan interrupt tepat & Error handling adequate & Minimal handling & Tidak ada \\
\hline
Code Quality & Clean code, naming konsisten & Good structure & Basic structure & Poor structure \\
\hline
Testing & Pengujian lengkap dengan TASM debugger & Pengujian sebagian & Minimal testing & No testing \\
\hline
\end{tabular}
\caption{Rubrik Penilaian Coding Challenge}
\end{table}

\subsection{Bobot Penilaian}

\begin{table}[htbp]
\centering
\begin{tabular}{|l|c|}
\hline
\textbf{Komponen} & \textbf{Bobot} \\
\hline
Bagian A: Pilihan Ganda & 20\% \\
\hline
Bagian B: Essay & 20\% \\
\hline
Bagian C: Analisis Kode & 20\% \\
\hline
Bagian D: Coding Challenge & 40\% \\
\hline
\textbf{Total} & \textbf{100\%} \\
\hline
\end{tabular}
\caption{Matriks Bobot Penilaian Akhir}
\end{table}
