% ============================================================
% AKTIVITAS PEMBELAJARAN
% ============================================================

\begin{aktivitas}
  \item \textbf{Conditional Logic}: Implementasikan program grade converter (A/B/C/D/F) dengan nested IF-THEN-ELSE.
  
  \item \textbf{Loop Optimization}: Bandingkan berbagai teknik loop optimization (unrolling, strength reduction).
  
  \item \textbf{Jump Table}: Buat program calculator dengan jump table untuk operasi aritmatika.
  
  \item \textbf{Procedure Calls}: Implementasikan program dengan prosedur untuk sorting array.
  
  \item \textbf{Stack Analysis}: Gunakan TASM debugger untuk menganalisis stack frame.
  
  \item \textbf{Control Flow}: Buat state machine sederhana dengan jumps.
\end{aktivitas}

% ============================================================
% LATIHAN DAN REFLEKSI
% ============================================================

\begin{latihan}
  \item Buat program untuk menentukan bilangan prima dengan nested loops.
  
  \item Implementasikan binary search menggunakan WHILE loop dan conditional jumps.
  
  \item Buat program untuk menghitung faktorial dengan FOR loop dan prosedur rekursif.
  
  \item Implementasikan bubble sort dengan nested loops dan optimasi.
  
  \item Buat program menu-driven dengan CASE structure untuk berbagai operasi.
  
  \item Implementasikan program untuk membalik string dengan loop dan pointer arithmetic.
  
  \item Buat program untuk menghitung jumlah digit dalam bilangan dengan loop dan conditional logic.
  
  \item \textbf{Refleksi}: Teknik loop mana yang paling sulit dipahami? Bagaimana Anda mengatasi kesulitan tersebut?
\end{latihan}

% ============================================================
% ASESMEN
% ============================================================

\begin{asesmen}
\textbf{Instrumen Penilaian untuk CPMK-3 (kontrol) dan Sub-CPMK 2.1}

\textbf{A. Pilihan Ganda}

\begin{enumerate}
  \item Instruksi yang digunakan untuk unconditional jump adalah:
  \begin{enumerate}
    \item JUMP
    \item JMP
    \item CALL
    \item RET
  \end{enumerate}
  
  \item Untuk implementasi WHILE loop, instruksi yang paling penting adalah:
  \begin{enumerate}
    \item CMP
    \item Jcondition
    \item JMP
    \item LOOP
  \end{enumerate}
  
  \item Register yang umum digunakan sebagai loop counter adalah:
  \begin{enumerate}
    \item CX
    \item BX
    \item DX
    \item AX
  \end{enumerate}
  
  \item Stack pointer yang digunakan untuk prosedur adalah:
  \begin{enumerate}
    \item SP
    \item BP
    \item SI
    \item DI
  \end{enumerate}
\end{enumerate}

\textbf{B. Essay}

\begin{enumerate}
  \item Jelaskan perbedaan antara LOOP dan manual counter dengan DEC/JNZ! Kapan sebaiknya menggunakan masing-masing?
  
  \item Mengapa stack frame management penting dalam prosedur yang kompleks?
\end{enumerate}

\textbf{C. Practical Challenge}

\begin{enumerate}
  \item Buat program text editor:
  \begin{itemize}
    \item Implementasi cursor movement dengan arrow keys
    \item Text insertion dan deletion
    \item Search dan replace functionality
    \item Multiple undo/redo operations
    \item File save dan load operations
    \item Gunakan prosedur untuk modular design
  \end{itemize}
\end{enumerate}

\textbf{Rubrik Penilaian}: Lihat Lampiran A
\end{asesmen}

% ============================================================
% CHECKLIST KOMPETENSI
% ============================================================

\begin{checklist}
  \item Saya dapat mengimplementasikan struktur kontrol (IF-THEN-ELSE, CASE)
  \item Saya dapat menggunakan berbagai jenis jumps (conditional, unconditional)
  \item Saya dapat membuat perulangan (WHILE, FOR, DO-WHILE)
  \item Saya dapat mengoptimasi loops untuk performa lebih baik
  \item Saya dapat mengimplementasikan nested loops
  \item Saya dapat menggunakan CALL dan RET untuk prosedur
  \item Saya dapat mengelola stack frame dengan benar
  \item Saya dapat menganalisis control flow dengan TASM debugger
\end{checklist}

% ============================================================
% RANGKUMAN
% ============================================================

\begin{rangkuman}
Bab ini membahas struktur kontrol dan perulangan dalam assembly 8086, termasuk conditional jumps, berbagai jenis loop, dan manajemen prosedur dengan stack.

\textbf{Poin Kunci:}
\begin{itemize}
  \item Conditional jumps memungkinkan implementasi IF-THEN-ELSE dan CASE structures
  \item LOOP instruksi menyediakan perulangan otomatis dengan counter
  \item Stack frame management essential untuk prosedur dengan parameter dan local variables
  \item Nested loops memungkinkan pemrosesan data multidimensional
  \item CALL dan RET mendukung modular programming dan code reuse
  \item Optimasi loops meningkatkan performa program secara signifikan
  \item Control flow analysis dengan debugger penting untuk troubleshooting
\end{itemize}

\textbf{Kata Kunci}: \asm{Struktur Kontrol}, \asm{Conditional Jump}, \asm{Loop}, \asm{Prosedur}, \asm{Stack Frame}, \asm{CALL}, \asm{RET}, \asm{TASM}, \asm{Optimasi}
\end{rangkuman}
