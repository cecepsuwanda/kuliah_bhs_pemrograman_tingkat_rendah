% ============================================================
% AKTIVITAS PEMBELAJARAN
% ============================================================

\begin{aktivitas}
  \item \textbf{Performance Benchmark}: Bandingkan performa kode sebelum dan sesudah optimasi.
  
  \item \textbf{Debug Session}: Gunakan TASM debugger untuk menemukan dan memperbaiki bug dalam program assembly.
  
  \item \textbf{Optimization Challenge}: Optimasi program sorting untuk kecepatan maksimal.
  
  \item \textbf{Bug Hunt}: Identifikasi dan perbaiki common assembly bugs dalam kode yang diberikan.
  
  \item \textbf{Instruction Timing}: Ukur dan analisis waktu eksekusi berbagai instruksi assembly.
  
  \item \textbf{Memory Analysis}: Analisis pola akses memori untuk cache optimization.
\end{aktivitas}

% ============================================================
% LATIHAN DAN REFLEKSI
% ============================================================

\begin{latihan}
  \item Optimasi program bubble sort dengan loop unrolling dan strength reduction.
  
  \item Debug program rekursif yang mengalami stack overflow.
  
  \item Implementasikan register allocation algorithm untuk program assembly.
  
  \item Buat performance profiler untuk mengukur waktu eksekusi instruksi.
  
  \item Identifikasi dan perbaiki register corruption bug dalam prosedur kompleks.
  
  \item Optimasi program matrix multiplication dengan cache-friendly access patterns.
  
  \item Implementasikan instruction scheduling untuk pipeline optimization.
  
  \item \textbf{Refleksi}: Teknik optimasi mana yang paling sulit dipahami? Bagaimana Anda mengatasi kesulitan tersebut?
\end{latihan}

% ============================================================
% ASESMEN
% ============================================================

\begin{asesmen}
\textbf{Instrumen Penilaian untuk Optimasi dan Debugging (Materi Minggu 15-16), Sub-CPMK 2.1}

\textbf{A. Pilihan Ganda}

\begin{enumerate}
  \item Teknik optimasi yang mengganti MUL dengan SHIFT adalah:
  \begin{enumerate}
    \item Loop unrolling
    \item Strength reduction
    \item Register optimization
    \item Code alignment
  \end{enumerate}
  
  \item Instruksi assembly tercepat untuk operasi aritmatika adalah:
  \begin{enumerate}
    \item MUL
    \item DIV
    \item ADD
    \item SUB
  \end{enumerate}
  
  \item Bug yang paling umum dalam assembly programming adalah:
  \begin{enumerate}
    \item Stack overflow
    \item Register corruption
    \item Off-by-one error
    \item Semua jawaban benar
  \end{enumerate}
  
  \item Untuk debugging assembly dengan TASM, perintah untuk set breakpoint adalah:
  \begin{enumerate}
    \item BP
    \item BC
    \item BL
    \item G
  \end{enumerate}
\end{enumerate}

\textbf{B. Essay}

\begin{enumerate}
  \item Jelaskan perbedaan antara loop unrolling dan strength reduction! Kapan sebaiknya menggunakan masing-masing?
  
  \item Mengapa register preservation penting dalam prosedur assembly?
\end{enumerate}

\textbf{C. Practical Challenge}

\begin{enumerate}
  \item Buat program high-performance calculator:
  \begin{itemize}
    \item Implementasi dengan optimasi maksimal
    \item Gunakan strength reduction untuk operasi matematika
    \item Apply loop unrolling untuk perulangan
    \item Optimasi register usage
    \item Implementasi cache-friendly memory access
    \item Debug dan test dengan TASM debugger
    \item Performance benchmarking dan analysis
    \item Documentation optimasi techniques yang digunakan
  \end{itemize}
\end{enumerate}

\textbf{Rubrik Penilaian}: Lihat Lampiran A
\end{asesmen}

% ============================================================
% CHECKLIST KOMPETENSI
% ============================================================

\begin{checklist}
  \item Saya dapat mengidentifikasi kesempatan optimasi dalam kode assembly
  \item Saya dapat menerapkan strength reduction untuk operasi mahal
  \item Saya dapat mengimplementasikan loop unrolling
  \item Saya dapat mengoptimalkan penggunaan register
  \item Saya dapat menggunakan TASM debugger untuk debugging
  \item Saya dapat mengidentifikasi dan memperbaiki common assembly bugs
  \item Saya dapat menganalisis performa kode assembly
  \item Saya dapat menerapkan advanced optimization techniques
\end{checklist}

% ============================================================
% RANGKUMAN
% ============================================================

\begin{rangkuman}
Bab ini membahas optimasi dan debugging dalam assembly 8086, termasuk teknik optimasi, debugging tools, dan performance analysis.

\textbf{Poin Kunci:}
\begin{itemize}
  \item Optimasi meningkatkan performa tanpa mengubah fungsionalitas
  \item Strength reduction mengganti operasi mahal dengan yang lebih murah
  \item Loop unrolling mengurangi overhead loop
  \item Register optimization meminimalkan akses memori
  \item TASM debugger menyediakan tools debugging yang komprehensif
  \item Common assembly bugs meliputi register corruption dan stack overflow
  \item Performance analysis membantu identifikasi bottleneck
  \item Advanced optimization mencakup code alignment dan instruction scheduling
\end{itemize}

\textbf{Kata Kunci}: \asm{Optimasi}, \asm{Debugging}, \asm{TASM}, \asm{Strength Reduction}, \asm{Loop Unrolling}, \asm{Register Optimization}, \asm{Performance Analysis}, \asm{Breakpoint}, \asm{Instruction Timing}
\end{rangkuman}
