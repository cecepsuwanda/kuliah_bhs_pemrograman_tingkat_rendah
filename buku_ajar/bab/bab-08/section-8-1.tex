% ============================================================
% MATERI POKOK
% ============================================================
\section{Interupsi Sistem}

\subsection{Konsep Interupsi}
Interupsi adalah sinyal yang menghentikan eksekusi program normal untuk menangani event tertentu.

\begin{verbatim}
; Format interupsi
INT interrupt_number    ; Trigger software interrupt
\end{verbatim}

\textbf{Jenis interupsi:}
\begin{itemize}
  \item \textbf{Hardware Interupsi}: Dari perangkat hardware (keyboard, timer, disk)
  \item \textbf{Software Interupsi}: Dipanggil dari program (INT n)
  \item \textbf{Exception}: Interupsi internal CPU (divide by zero, overflow)
\end{itemize}

\subsection{Interrupt Vector Table}
Tabel alamat untuk handler interupsi di memori.

\begin{itemize}
  \item Lokasi: 0000h:0000h - 0000:03FFh (1024 bytes)
  \item 256 entry (0-255), setiap entry 4 bytes (segment:offset)
  \item Entry 0-1Fh: BIOS interrupts
  \item Entry 20h-3Fh: DOS interrupts
  \item Entry 40h-FFh: User-defined interrupts
\end{itemize}

\section{BIOS Interrupts}

\subsection{INT 10h - Video Services}
Layanan untuk operasi video dan display.

\begin{verbatim}
; Set video mode
MOV AH, 00h       ; Function: set video mode
MOV AL, 03h       ; Mode: 80x25 text
INT 10h           ; Call BIOS

; Set cursor position
MOV AH, 02h       ; Function: set cursor
MOV BH, 0         ; Page number
MOV DH, 10        ; Row
MOV DL, 20        ; Column
INT 10h           ; Call BIOS

; Write character
MOV AH, 0Eh       ; Function: write character
MOV AL, 'A'       ; Character
MOV BH, 0         ; Page
MOV BL, 07h       ; Color
INT 10h           ; Call BIOS
\end{verbatim}

\textbf{Fungsi INT 10h utama:}
\begin{itemize}
  \item \textbf{AH=00h}: Set video mode
  \item \textbf{AH=02h}: Set cursor position
  \item \textbf{AH=09h}: Write character/attribute
  \item \textbf{AH=0Eh}: Write character (teletype)
  \item \textbf{AH=13h}: Write string
\end{itemize}

\subsection{INT 13h - Disk Services}
Layanan untuk operasi disk dan floppy.

\begin{verbatim}
; Reset disk system
MOV AH, 00h       ; Function: reset disk
MOV DL, 00h       ; Drive: A:
INT 13h           ; Call BIOS

; Read disk sector
MOV AH, 02h       ; Function: read sector
MOV AL, 01h       ; Number of sectors
MOV CH, 00h       ; Cylinder
MOV CL, 01h       ; Sector
MOV DH, 00h       ; Head
MOV DL, 00h       ; Drive: A:
MOV BX, buffer    ; Buffer address
INT 13h           ; Call BIOS
\end{verbatim}

\subsection{INT 16h - Keyboard Services}
Layanan untuk input keyboard.

\begin{verbatim}
; Get keystroke
MOV AH, 00h       ; Function: get keystroke
INT 16h           ; Call BIOS
; AH = scan code, AL = ASCII code

; Check key status
MOV AH, 01h       ; Function: check key status
INT 16h           ; Call BIOS
; ZF = 0 if key pressed, ZF = 1 if no key
\end{verbatim}

\section{DOS Interrupts}

\subsection{INT 21h - DOS Services}
Layanan sistem operasi DOS untuk file I/O, program control, dan lainnya.

\subsubsection{Character I/O}
\begin{verbatim}
; Display character
MOV AH, 02h       ; Function: display character
MOV DL, 'A'       ; Character
INT 21h           ; Call DOS

; Input character
MOV AH, 01h       ; Function: input character
INT 21h           ; Call DOS
; AL = character input

; Display string
MOV AH, 09h       ; Function: display string
MOV DX, string_addr ; String address (terminated by $)
INT 21h           ; Call DOS

; Input string
MOV AH, 0Ah       ; Function: input string
MOV DX, buffer    ; Buffer address
INT 21h           ; Call DOS
\end{verbatim}

\subsubsection{File Operations}
\begin{verbatim}
; Create file
MOV AH, 3Ch       ; Function: create file
MOV CX, 00h       ; Attributes
MOV DX, filename  ; Filename address
INT 21h           ; Call DOS
; AX = file handle

; Open file
MOV AH, 3Dh       ; Function: open file
MOV AL, 00h       ; Access mode (read)
MOV DX, filename  ; Filename address
INT 21h           ; Call DOS
; AX = file handle

; Read file
MOV AH, 3Fh       ; Function: read file
MOV BX, handle    ; File handle
MOV CX, bytes     ; Number of bytes
MOV DX, buffer    ; Buffer address
INT 21h           ; Call DOS
; AX = bytes read

; Close file
MOV AH, 3Eh       ; Function: close file
MOV BX, handle    ; File handle
INT 21h           ; Call DOS
\end{verbatim}

\subsubsection{Program Control}
\begin{verbatim}
; Terminate program
MOV AH, 4Ch       ; Function: terminate
MOV AL, 00h       ; Return code
INT 21h           ; Call DOS

; Execute program
MOV AH, 4Bh       ; Function: execute
MOV AL, 00h       ; Load and execute
MOV DX, progname  ; Program name
MOV BX, paramblk  ; Parameter block
INT 21h           ; Call DOS
\end{verbatim}

\section{Memory Management}

\subsection{INT 12h - Memory Size}
Mendapatkan ukuran memori sistem.

\begin{verbatim}
; Get memory size
INT 12h           ; Call BIOS
; AX = memory size in KB
\end{verbatim}

\subsection{INT 15h - Extended Memory}
Layanan untuk extended memory dan system information.

\begin{verbatim}
; Get extended memory size
MOV AH, 88h       ; Function: get extended memory
INT 15h           ; Call BIOS
; AX = extended memory size in KB
\end{verbatim}

\section{Time and Date}

\subsection{INT 1Ah - Time Services}
Layanan untuk operasi waktu dan timer.

\begin{verbatim}
; Read system time
INT 1Ah           ; Call BIOS
; CX:DX = timer ticks since midnight

; Set system time
MOV CX, hours     ; Hours
MOV DX, minutes   ; Minutes
MOV AH, 01h       ; Function: set time
INT 1Ah           ; Call BIOS
\end{verbatim}

\subsection{INT 21h - Date Functions}
Fungsi tanggal dalam DOS services.

\begin{verbatim}
; Get system date
MOV AH, 2Ah       ; Function: get date
INT 21h           ; Call DOS
; CX = year, DH = month, DL = day, AL = day of week

; Set system date
MOV AH, 2Bh       ; Function: set date
MOV CX, year      ; Year
MOV DH, month     ; Month
MOV DL, day       ; Day
INT 21h           ; Call DOS
; AL = 00h if success, FFh if invalid
\end{verbatim}
