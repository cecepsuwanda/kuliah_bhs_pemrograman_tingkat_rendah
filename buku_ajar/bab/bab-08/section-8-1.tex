% ============================================================
% MATERI POKOK
% ============================================================
\section{Interupsi Sistem}

\subsection{Konsep Interupsi}
Interupsi adalah sinyal yang menghentikan eksekusi program normal untuk menangani event tertentu.

\begin{verbatim}
; Format interupsi
INT interrupt_number    ; Trigger software interrupt
\end{verbatim}

\textbf{Jenis interupsi:}
\begin{itemize}
  \item \textbf{Hardware Interupsi}: Dari perangkat hardware (keyboard, timer, disk)
  \item \textbf{Software Interupsi}: Dipanggil dari program (INT n)
  \item \textbf{Exception}: Interupsi internal CPU (divide by zero, overflow)
\end{itemize}

\subsection{Interrupt Vector Table (IVT)}

Interrupt Vector Table adalah tabel di memori yang menyimpan alamat handler untuk setiap interupsi \cite{rbil}. Saat software interrupt (\texttt{INT n}) dieksekusi, CPU mengambil alamat handler dari IVT dan memindahkan eksekusi ke handler tersebut.

\begin{itemize}
  \item \textbf{Lokasi}: 0000h:0000h -- 0000h:03FFh (1024 bytes)
  \item \textbf{Struktur}: 256 entry (0--255), setiap entry 4 bytes (segment:offset)
  \item \textbf{Entry 00h--1Fh}: BIOS interrupts (timer, keyboard, disk, video)
  \item \textbf{Entry 20h--3Fh}: DOS interrupts (INT 21h di entry 0x21)
  \item \textbf{Entry 40h--FFh}: User-defined dan reserved
\end{itemize}

\textbf{Langkah eksekusi software interrupt:}
\begin{enumerate}
  \item Push flags, CS, IP ke stack
  \item IF = 0 (disable maskable interrupt)
  \item Baca alamat dari IVT[interrupt\_number * 4]
  \item Jump ke handler (CS:IP)
  \item Handler menjalankan IRET untuk return
\end{enumerate}

\section{BIOS Interrupts}

\subsection{INT 10h - Video Services}
Layanan BIOS untuk operasi video dan display. Berguna untuk set mode teks/grafik, posisi kursor, cetak karakter dengan atribut. Skenario: program yang perlu mengontrol tampilan langsung (menu, game sederhana, utilitas sistem).

\begin{verbatim}
; Set video mode
MOV AH, 00h       ; Function: set video mode
MOV AL, 03h       ; Mode: 80x25 text
INT 10h           ; Call BIOS

; Set cursor position
MOV AH, 02h       ; Function: set cursor
MOV BH, 0         ; Page number
MOV DH, 10        ; Row
MOV DL, 20        ; Column
INT 10h           ; Call BIOS

; Write character
MOV AH, 0Eh       ; Function: write character
MOV AL, 'A'       ; Character
MOV BH, 0         ; Page
MOV BL, 07h       ; Color
INT 10h           ; Call BIOS
\end{verbatim}

\textbf{Fungsi INT 10h utama:}
\begin{itemize}
  \item \textbf{AH=00h}: Set video mode
  \item \textbf{AH=02h}: Set cursor position
  \item \textbf{AH=09h}: Write character/attribute
  \item \textbf{AH=0Eh}: Write character (teletype)
  \item \textbf{AH=13h}: Write string
\end{itemize}

\subsection{INT 13h - Disk Services}
Layanan BIOS untuk operasi disk tingkat rendah (floppy, hard disk). Digunakan untuk bootloader, utilitas disk, atau akses sector langsung. Perhatian: akses low-level dapat merusak data jika salah parameter.

\begin{verbatim}
; Reset disk system
MOV AH, 00h       ; Function: reset disk
MOV DL, 00h       ; Drive: A:
INT 13h           ; Call BIOS

; Read disk sector
MOV AH, 02h       ; Function: read sector
MOV AL, 01h       ; Number of sectors
MOV CH, 00h       ; Cylinder
MOV CL, 01h       ; Sector
MOV DH, 00h       ; Head
MOV DL, 00h       ; Drive: A:
MOV BX, buffer    ; Buffer address
INT 13h           ; Call BIOS
\end{verbatim}

\subsection{INT 16h - Keyboard Services}
Layanan BIOS untuk input keyboard. \asm{AH=00h} menunggu tombol ditekan (blocking); \asm{AH=01h} mengecek status tanpa menunggu (non-blocking). Berguna untuk game, menu interaktif, atau utilitas yang memerlukan scan code.

\begin{verbatim}
; Get keystroke
MOV AH, 00h       ; Function: get keystroke
INT 16h           ; Call BIOS
; AH = scan code, AL = ASCII code

; Check key status
MOV AH, 01h       ; Function: check key status
INT 16h           ; Call BIOS
; ZF = 0 if key pressed, ZF = 1 if no key
\end{verbatim}

\section{DOS Interrupts}

\subsection{INT 21h - DOS Services}
Layanan sistem operasi DOS untuk file I/O, program control, dan lainnya.

\subsubsection{Character I/O}
Fungsi I/O karakter paling sering digunakan. \textbf{AH=02h}: cetak karakter di DL. \textbf{AH=01h}: input karakter (echo ke layar), return di AL. \textbf{AH=09h}: cetak string; DX=offset, string harus diakhiri \texttt{\$}. \textbf{AH=0Ah}: input string; buffer format: byte 0 = max length, byte 1 = actual length (diisi DOS), byte 2+ = karakter. Return: CF=0 sukses; CF=1 error, AX=kode error.

\begin{verbatim}
; Display character
MOV AH, 02h       ; Function: display character
MOV DL, 'A'       ; Character
INT 21h           ; Call DOS

; Input character
MOV AH, 01h       ; Function: input character
INT 21h           ; Call DOS
; AL = character input

; Display string (terminator $)
MOV AH, 09h       ; Function: display string
MOV DX, string_addr ; String address (terminated by $)
INT 21h           ; Call DOS

; Input string - buffer: [max][actual][chars...]
MOV AH, 0Ah       ; Function: input string
MOV DX, buffer    ; Buffer address
INT 21h           ; Call DOS
\end{verbatim}

\subsubsection{File Operations}
Fungsi file menggunakan \textit{handle}: create/open mengembalikan handle di AX (CF=0 sukses); handle tersebut dipakai untuk read/write/close. \textbf{3Ch}: create, CX=atribut (0=normal). \textbf{3Dh}: open, AL=0 read, 1 write, 2 read/write. \textbf{3Fh}: read, BX=handle, CX=bytes, return AX=bytes read. \textbf{3Eh}: close. Selalu cek CF setelah pemanggilan untuk error handling.

\begin{verbatim}
; Create file
MOV AH, 3Ch       ; Function: create file
MOV CX, 00h       ; Attributes (0=normal)
MOV DX, filename  ; Filename address (DS:DX)
INT 21h           ; Call DOS
; CF=0: AX = file handle; CF=1: error

; Open file
MOV AH, 3Dh       ; Function: open file
MOV AL, 00h       ; Access: 0=read, 1=write, 2=read/write
MOV DX, filename  ; Filename address
INT 21h           ; Call DOS
; CF=0: AX = file handle

; Read file
MOV AH, 3Fh       ; Function: read file
MOV BX, handle    ; File handle
MOV CX, bytes     ; Number of bytes
MOV DX, buffer    ; Buffer address
INT 21h           ; Call DOS
; AX = bytes read (0 jika EOF)

; Close file
MOV AH, 3Eh       ; Function: close file
MOV BX, handle    ; File handle
INT 21h           ; Call DOS
\end{verbatim}

\subsubsection{Program Control}
\begin{verbatim}
; Terminate program
MOV AH, 4Ch       ; Function: terminate
MOV AL, 00h       ; Return code
INT 21h           ; Call DOS

; Execute program
MOV AH, 4Bh       ; Function: execute
MOV AL, 00h       ; Load and execute
MOV DX, progname  ; Program name
MOV BX, paramblk  ; Parameter block
INT 21h           ; Call DOS
\end{verbatim}

\section{Memory Management}

\subsection{INT 12h - Memory Size}
Mendapatkan ukuran memori sistem.

\begin{verbatim}
; Get memory size
INT 12h           ; Call BIOS
; AX = memory size in KB
\end{verbatim}

\subsection{INT 15h - Extended Memory}
Layanan untuk extended memory dan system information.

\begin{verbatim}
; Get extended memory size
MOV AH, 88h       ; Function: get extended memory
INT 15h           ; Call BIOS
; AX = extended memory size in KB
\end{verbatim}

\section{Time and Date}

\subsection{INT 1Ah - Time Services}
Layanan untuk operasi waktu dan timer.

\begin{verbatim}
; Read system time
INT 1Ah           ; Call BIOS
; CX:DX = timer ticks since midnight

; Set system time
MOV CX, hours     ; Hours
MOV DX, minutes   ; Minutes
MOV AH, 01h       ; Function: set time
INT 1Ah           ; Call BIOS
\end{verbatim}

\subsection{INT 21h - Date Functions}
Fungsi tanggal dalam DOS services.

\begin{verbatim}
; Get system date
MOV AH, 2Ah       ; Function: get date
INT 21h           ; Call DOS
; CX = year, DH = month, DL = day, AL = day of week

; Set system date
MOV AH, 2Bh       ; Function: set date
MOV CX, year      ; Year
MOV DH, month     ; Month
MOV DL, day       ; Day
INT 21h           ; Call DOS
; AL = 00h if success, FFh if invalid
\end{verbatim}
