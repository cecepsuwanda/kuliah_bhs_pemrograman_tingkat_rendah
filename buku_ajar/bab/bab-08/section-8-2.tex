% ============================================================
% AKTIVITAS PEMBELAJARAN
% ============================================================

\begin{aktivitas}
  \item \textbf{Video Programming}: Buat program yang menggunakan INT 10h untuk membuat animasi sederhana.
  
  \item \textbf{File Operations}: Implementasikan text editor dengan file I/O menggunakan INT 21h.
  
  \item \textbf{Keyboard Handler}: Buat program yang menangani keyboard input dengan INT 16h.
  
  \item \textbf{System Information}: Buat program yang menampilkan informasi sistem (memori, waktu, tanggal).
  
  \item \textbf{Interrupt Analysis}: Gunakan TASM debugger untuk trace interrupt execution.
  
  \item \textbf{Custom Interrupt}: Implementasikan user-defined interrupt handler.
\end{aktivitas}

% ============================================================
% LATIHAN DAN REFLEKSI
% ============================================================

\begin{latihan}
  \item Buat program untuk menampilkan jam digital di pojok kanan atas layar.
  
  \item Implementasikan program untuk membaca dan menulis file teks dengan INT 21h.
  
  \item Buat program calculator dengan input keyboard dan output video.
  
  \item Implementasikan program untuk menampilkan informasi sistem (memori, tanggal, waktu).
  
  \item Buat program text-based game dengan keyboard input dan video output.
  
  \item Implementasikan program untuk copy file dengan progress bar.
  
  \item Buat program untuk mengubah atribut file dan direktori.
  
  \item \textbf{Refleksi}: Interrupt mana yang paling sulit dipahami? Bagaimana Anda mengatasi kesulitan tersebut?
\end{latihan}

% ============================================================
% ASESMEN
% ============================================================

\begin{asesmen}
\textbf{Instrumen Penilaian untuk Sub-CPMK 4.2, 2.1}

\textbf{A. Pilihan Ganda}

\begin{enumerate}
  \item Interrupt untuk video services adalah:
  \begin{enumerate}
    \item INT 10h
    \item INT 13h
    \item INT 16h
    \item INT 21h
  \end{enumerate}
  
  \item Fungsi untuk menampilkan karakter dengan INT 21h adalah:
  \begin{enumerate}
    \item AH = 01h
    \item AH = 02h
    \item AH = 09h
    \item AH = 0Ah
  \end{enumerate}
  
  \item Register yang berisi scan code dari INT 16h adalah:
  \begin{enumerate}
    \item AL
    \item AH
    \item AX
    \item DX
  \end{enumerate}
  
  \item Interrupt untuk disk services adalah:
  \begin{enumerate}
    \item INT 10h
    \item INT 12h
    \item INT 13h
    \item INT 15h
  \end{enumerate}
\end{enumerate}

\textbf{B. Essay}

\begin{enumerate}
  \item Jelaskan perbedaan antara BIOS interrupts dan DOS interrupts! Berikan contoh penggunaan masing-masing.
  
  \item Mengapa interrupt vector table penting dalam sistem operasi?
\end{enumerate}

\textbf{C. Practical Challenge}

\begin{enumerate}
  \item Buat program file manager:
  \begin{itemize}
    \item Display directory listing dengan INT 21h
    \item File operations (create, read, write, delete)
    \item Directory operations (create, remove, change)
    \item File attribute management
    \item Search functionality
    \item User interface dengan menu system
    \item Error handling dan validation
  \end{itemize}
\end{enumerate}

\textbf{Rubrik Penilaian}: Lihat Lampiran A
\end{asesmen}

% ============================================================
% CHECKLIST KOMPETENSI
% ============================================================

\begin{checklist}
  \item Saya dapat menggunakan INT 10h untuk video operations
  \item Saya dapat menggunakan INT 13h untuk disk operations
  \item Saya dapat menggunakan INT 16h untuk keyboard input
  \item Saya dapat menggunakan INT 21h untuk DOS services
  \item Saya dapat melakukan file I/O operations
  \item Saya dapat mengelola memory dengan interrupts
  \item Saya dapat mengakses time dan date services
  \item Saya dapat menganalisis interrupt execution dengan debugger
\end{checklist}

% ============================================================
% RANGKUMAN
% ============================================================

\begin{rangkuman}
Bab ini membahas interupsi sistem dalam assembly 8086, termasuk BIOS interrupts, DOS services, dan system programming.

\textbf{Poin Kunci:}
\begin{itemize}
  \item Interupsi memungkinkan komunikasi dengan hardware dan sistem operasi
  \item BIOS interrupts (INT 10h, 13h, 16h) untuk hardware-level operations
  \item DOS interrupts (INT 21h) untuk sistem operasi services
  \item File I/O, video, keyboard, dan disk operations melalui interrupts
  \item Interrupt vector table mengelola handler addresses
  \item System programming memerlukan pemahaman interrupt mechanisms
  \item Error handling penting untuk interrupt operations
  \item Debugging interrupts membantu troubleshooting system calls
\end{itemize}

\textbf{Kata Kunci}: \asm{Interupsi}, \asm{BIOS}, \asm{DOS}, \asm{INT 10h}, \asm{INT 13h}, \asm{INT 16h}, \asm{INT 21h}, \asm{File I/O}, \asm{Video}, \asm{Keyboard}, \asm{TASM}, \asm{System Programming}
\end{rangkuman}
