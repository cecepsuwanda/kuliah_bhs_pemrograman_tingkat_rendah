\documentclass[../main.tex]{subfiles}
\ifSubfilesClassLoaded{\setcounter{chapter}{2}}{}
\begin{document}

\chapter{Prosesor Intel 8086 dan Register}

\begin{subcpmk}
  \item Sub-CPMK 1.1: Menjelaskan fungsi register AX, BX, CX, DX, SI, DI, BP, SP dalam prosesor 8086
  \item Sub-CPMK 1.3: Mendemonstrasikan organisasi memori segmentasi (CS, DS, ES, SS)
  \item Sub-CPMK 2.1: Menulis program assembly sederhana dengan TASM syntax
\end{subcpmk}

% ============================================================
% MATERI POKOK
% ============================================================
\section{Dasar Bahasa Assembly}

Assembly language adalah bahasa pemrograman tingkat rendah yang paling dekat dengan bahasa mesin. Intel 8086 assembly menggunakan mnemonics yang mudah dibaca untuk mewakili instruksi biner.

\subsection{Struktur Program Assembly}

Program assembly terdiri dari beberapa segmen utama. Direktif \asm{.MODEL SMALL} (TASM/MASM) menentukan model memori: satu segmen kode ($\leq$64 KB), satu segmen data ($\leq$64 KB), satu segmen stack. Assembler kemudian mengatur \asm{ASSUME} dan ukuran segment secara otomatis.

\begin{verbatim}
; Struktur dasar program assembly
; ======================

; Data segment - untuk variabel dan konstanta
DATA SEGMENT
    ; Variabel dan konstanta di sini
DATA ENDS

; Code segment - untuk instruksi program
CODE SEGMENT
    ; Instruksi program di sini
CODE ENDS

; Stack segment - untuk stack operations
STACK SEGMENT
    ; Stack buffer di sini
STACK ENDS

END
\end{verbatim}

\subsection{Program .COM vs .EXE}

\textbf{.COM}: Format sederhana tanpa header, maksimal $\sim$64 KB. Semua segmen (CS, DS, ES, SS) umumnya sama. Entry point di offset \asm{0100h} (setelah PSP). Gunakan \asm{ORG 100h}. Cocok untuk program kecil.

\textbf{.EXE}: Ada header, segmen terpisah untuk kode, data, stack. Gunakan \asm{.MODEL SMALL}, \asm{.DATA}, \asm{.CODE}, \asm{ASSUME}. Inisialisasi \asm{DS} wajib: \asm{MOV AX, @data} / \asm{MOV DS, AX}. Cocok untuk program besar dengan banyak variabel.

\subsection{Assembler Directives}

Directives adalah instruksi untuk assembler, bukan untuk processor:

\begin{verbatim}
; Data definition directives
DB  'A'           ; Define byte (8-bit)
DW  1234h         ; Define word (16-bit)
DD  12345678h      ; Define double word (32-bit)
EQU  MAX_SIZE 100   ; Define constant
ORG  100h          ; Set origin address
\end{verbatim}

\textbf{Common Directives:}
\begin{itemize}
  \item \textbf{DB}: Define byte (8-bit)
  \item \textbf{DW}: Define word (16-bit)
  \item \textbf{DD}: Define double word (32-bit)
  \item \textbf{EQU}: Equate constant
  \item \textbf{ORG}: Set origin address
  \item \textbf{SEGMENT}: Define segment
  \item \textbf{END}: End of program
\end{itemize}

\subsection{Syntax Assembly}

\textbf{Format Instruksi:}
\begin{verbatim}
[label:] mnemonic [operand1[, operand2[, operand3]]
\end{verbatim}

\textbf{Contoh Penggunaan:}
\begin{verbatim}
        MOV AX, BX     ; Transfer BX ke AX
        ADD CX, 5      ; Tambah 5 ke CX
        JMP loop_start  ; Jump ke loop_start
        CMP AX, 0      ; Bandingkan AX dengan 0
        JZ  zero_found  ; Jump jika zero
\end{verbatim}

\subsection{Label dan Organisasi Kode}

Label digunakan untuk menandai lokasi dalam kode sehingga dapat direferensi oleh instruksi jump dan call:

\begin{verbatim}
; Contoh penggunaan label
start:
    MOV AX, 0
    JMP loop_start

loop_start:
    INC AX
    CMP AX, 10
    JL loop_start
    RET
\end{verbatim}

\begin{table}[htbp]
\centering
\begin{tabular}{|p{3cm}|p{5cm}|p{5cm}|}
\hline
\textbf{Aspek} & \textbf{Label} & \textbf{Address} \\
\hline
Definisi & Penanda lokasi dalam kode & Alamat memori fisik \\
\hline
Penggunaan & Target jump, call, referensi data & Diperoleh saat linking \\
\hline
Scope & Lokal atau global & Bergantung segment \\
\hline
Format & Nama diikuti titik dua (:) & Segment:Offset (hex) \\
\hline
\end{tabular}
\caption{Hubungan Label dan Address}
\end{table}

\section{Tipe Data dan Direktif}

\subsection{Tipe Data Dasar}

Intel 8086 mendukung berbagai tipe data untuk operasi yang berbeda:

\textbf{Integer Types:}
\begin{itemize}
  \item \textbf{Byte}: 8-bit unsigned (0-255)
  \item \textbf{Word}: 16-bit unsigned (0-65535)
  \item \textbf{Double Word}: 32-bit (0-4294967295)
\end{itemize}

\textbf{String Types:}
\begin{itemize}
  \item \textbf{ASCII String}: Array of bytes terminated dengan null (0)
  \item \textbf{Pascal String}: String dengan length prefix
\end{itemize}

\subsection{Data Definition Directives}

\begin{verbatim}
; Integer data definition
byte_var   DB  10h      ; 8-bit value
word_var   DW  1234h     ; 16-bit value
dword_var  DD  12345678h ; 32-bit value

; String definition
str_hello  DB 'Hello World', 0
str_name   DB 'John Doe', 0
str_age   DB 25, 0

; Array definition
numbers   DW 10, 20, 30, 40, 50
matrix    DW 1, 2, 3, 4
buffer    DB 256 DUP(0)  ; 256 bytes buffer
\end{verbatim}

\subsection{Constant Definition}

\begin{verbatim}
; Constant definitions
MAX_SIZE EQU 100
PI       EQU 3.14159
NEW_LINE  EQU 0Dh, 0Ah
TRUE     EQU 1
FALSE    EQU 0
\end{verbatim}

\subsection{Memory Layout}

\begin{verbatim}
; Memory organization example
DATA SEGMENT
    ; Constants
    MAX_SIZE EQU 100
    
    ; Variables
    array1 DW 10 DUP(0)
    array2 DW MAX_SIZE DUP(0)
    string1 DB 'Hello', 0
    string2 DB 'World', 0
    
    ; Aligned data
    aligned_var DW 0
DATA ENDS
\end{verbatim}

\section{Instruksi Dasar}

Instruksi dasar adalah fundamental untuk pemrograman assembly Intel 8086. Setiap instruksi memetakan hampir satu-ke-satu ke kode mesin. Pemahaman kategori instruksi membantu merancang alur program: transfer data untuk inisialisasi dan pengiriman hasil; aritmatika untuk perhitungan; logika untuk masking dan pengujian bit; kontrol alur untuk percabangan dan perulangan.

\subsection{Data Transfer Instructions}

Digunakan untuk memindahkan data antar register, memori, dan konstanta. Inisialisasi register, menyimpan hasil ke memori, dan menyiapkan parameter untuk prosedur/interupsi.

\begin{verbatim}
; MOV - Move data (tidak mengubah flag)
MOV AX, BX        ; AX = BX
MOV CX, 1234h     ; CX = 4660
MOV [mem_var], AX ; memory[mem_var] = AX
MOV DX, [SI]      ; DX = memory[SI]

; XCHG - Exchange data (swap tanpa register temp)
XCHG AX, BX       ; Swap AX dan BX
XCHG CX, DX       ; Swap CX dan DX
\end{verbatim}

\subsection{Arithmetic Instructions}

Untuk perhitungan numerik. Semua instruksi aritmatika mengubah flag (ZF, CF, SF, OF). \asm{CMP} membandingkan tanpa menyimpan hasil—hanya mengatur flag untuk conditional jump.

\begin{verbatim}
; ADD - Addition
ADD AX, BX        ; AX = AX + BX
ADD CX, 5         ; CX = CX + 5
ADD [mem], 10     ; memory[mem] = memory[mem] + 10

; SUB - Subtraction
SUB AX, BX        ; AX = AX - BX
SUB CX, 5         ; CX = CX - 5

; INC/DEC - Increment/Decrement (INC tidak mengubah CF)
INC AX            ; AX = AX + 1
DEC BX            ; BX = BX - 1

; MUL/DIV - Multiply/Divide (AX/DX implisit)
MUL BX            ; DX:AX = AX * BX (16-bit)
DIV BX            ; AX / BX -> AL=quotient, AH=remainder
\end{verbatim}

\subsection{Logic Instructions}

Untuk manipulasi bit: masking (AND), setting bit (OR), toggle (XOR), clear register (XOR reg,reg). \asm{TEST} seperti AND tapi tidak menyimpan hasil—hanya mengatur flag.

\begin{verbatim}
; AND - Logical AND (masking, test bit)
AND AX, 0Fh       ; AX = AX AND 15 (ambil 4 bit rendah)

; OR - Logical OR (set bit)
OR AX, 80h        ; AX = AX OR 128 (set bit 7)

; XOR - Logical XOR (toggle, clear)
XOR AX, AX        ; AX = 0 (clear register, 2 byte)

; NOT - Logical NOT (bitwise complement)
NOT AX            ; AX = ~AX

; SHL/SHR - Shift (perkalian/pembagian dengan 2)
SHL AX, 1         ; AX = AX << 1 (AX * 2)
SHR AX, 1         ; AX = AX >> 1 (unsigned / 2)
\end{verbatim}

\subsection{Control Flow Instructions}

Mengendalikan urutan eksekusi. \asm{JMP} unconditional; \asm{JZ}, \asm{JNZ}, \asm{JC}, \asm{JNC} conditional berdasarkan flag. Biasanya didahului \asm{CMP} atau \asm{TEST}.

\begin{verbatim}
; JMP - Unconditional jump
JMP label_name    ; Jump ke label_name

; Conditional jumps (berdasarkan flag)
JZ  zero_flag     ; Jump if Zero (ZF = 1)
JNZ not_zero      ; Jump if Not Zero (ZF = 0)
JC  carry_flag    ; Jump if Carry (CF = 1)
JNC no_carry      ; Jump if No Carry (CF = 0)
\end{verbatim}

\subsection{Contoh Kombinasi Instruksi}

\begin{verbatim}
; Inisialisasi dan loop: jumlahkan 5 nilai dari array
    XOR AX, AX        ; AX = 0 (akumulator)
    MOV CX, 5         ; CX = counter
    MOV SI, 0         ; SI = indeks
loop_start:
    ADD AX, [array+SI] ; AX += array[SI]
    ADD SI, 2          ; SI += 2 (word = 2 byte)
    LOOP loop_start   ; CX--; jump jika CX != 0
    MOV [jumlah], AX   ; simpan hasil
\end{verbatim}

\section{Contoh Program Sederhana}

Program assembly sederhana untuk memahami konsep dasar pemrograman Intel 8086.

\subsection{Program Hello World}

\begin{verbatim}
; Hello World Program
; =================
DATA SEGMENT
    message DB 'Hello, World!', 0Dh, 0Ah, '$'
DATA ENDS

CODE SEGMENT
    ASSUME CS:CODE, DS:DATA

start:
    ; Initialize DS
    MOV AX, DATA
    MOV DS, AX
    
    ; Display message
    MOV AH, 09h        ; DOS print string function
    MOV DX, OFFSET message
    INT 21h
    
    ; Exit program
    MOV AH, 4Ch        ; DOS exit function
    INT 21h

CODE ENDS
END start
\end{verbatim}

\subsection{Program Kalkulator Sederhana}

\begin{verbatim}
; Simple Calculator
; ================
DATA SEGMENT
    num1 DW 10
    num2 DW 20
    result DW 0
    msg_result DB 'Result: ', '$'
DATA ENDS

CODE SEGMENT
    ASSUME CS:CODE, DS:DATA

start:
    ; Initialize DS
    MOV AX, DATA
    MOV DS, AX
    
    ; Load numbers
    MOV AX, num1
    ADD AX, num2        ; AX = num1 + num2
    MOV result, AX
    
    ; Display result message
    MOV AH, 09h
    MOV DX, OFFSET msg_result
    INT 21h
    
    ; Display result (simplified)
    MOV AX, result
    ; Add code to display AX value
    
    ; Exit program
    MOV AH, 4Ch
    INT 21h

CODE ENDS
END start
\end{verbatim}

\subsection{Program Loop Counter}

\begin{verbatim}
; Loop Counter Program
; ===================
DATA SEGMENT
    counter DW 0
    max_count DW 10
    msg_count DB 'Count: ', '$'
DATA ENDS

CODE SEGMENT
    ASSUME CS:CODE, DS:DATA

start:
    ; Initialize DS
    MOV AX, DATA
    MOV DS, AX
    
    ; Initialize counter
    MOV CX, max_count
    MOV counter, 0

count_loop:
    ; Increment counter
    INC counter
    
    ; Display count (simplified)
    MOV AH, 09h
    MOV DX, OFFSET msg_count
    INT 21h
    
    ; Loop until CX = 0
    LOOP count_loop
    
    ; Exit program
    MOV AH, 4Ch
    INT 21h

CODE ENDS
END start
\end{verbatim}

\section{Dasar Debugging}

Debugging adalah proses menemukan dan memperbaiki error dalam program assembly.

\subsection{Common Assembly Errors}

\textbf{Syntax Errors:}
\begin{itemize}
  \item \textbf{Invalid Mnemonic}: Instruksi tidak dikenal
  \item \textbf{Invalid Operand}: Format operand salah
  \item \textbf{Undefined Symbol}: Label atau variabel tidak didefinisikan
  \item \textbf{Duplicate Definition}: Label didefinisikan dua kali
\end{itemize}

\textbf{Logic Errors:}
\begin{itemize}
  \item \textbf{Register Corruption}: Register tidak dipreserve
  \item \textbf{Stack Overflow}: Rekursi terlalu dalam
  \item \textbf{Off-by-One}: Loop count salah
  \item \textbf{Memory Access}: Akses memori tidak valid
\end{itemize}

\subsection{Debugging Techniques}

\begin{verbatim}
; Debugging with comments
; ===================
; Add comments to explain logic
MOV AX, BX        ; Copy BX to AX
ADD AX, 5         ; Add 5 to AX
CMP AX, 0         ; Compare with zero
JZ  zero_found    ; Jump if zero

; Use registers for debugging
MOV DX, 1         ; Debug flag = 1
; ... code ...
CMP DX, 1         ; Check debug flag
JNE  skip_debug   ; Skip if not debugging
\end{verbatim}

\subsection{TASM Debugger Commands}

\begin{verbatim}
; Common debugger commands
; =====================
; BP address      - Set breakpoint
; BC number       - Clear breakpoint
; BL              - List breakpoints
; G               - Run program
; T               - Trace single instruction
; P               - Procedure step
; R register      - Show register value
; ?               - Show help
\end{verbatim}

\subsection{Debugging Strategy}

\begin{enumerate}
  \item \textbf{Compile with Debug Info}: Gunakan /zi switch
  \item \textbf{Set Breakpoints}: Set di lokasi strategis
  \item \textbf{Step Through}: Eksekusi per instruksi
  \item \textbf{Watch Registers}: Monitor register values
  \item \textbf{Check Memory}: Inspect memory content
  \item \textbf{Verify Logic}: Pastikan algoritma benar
\end{enumerate}

\section{Contoh Program Lanjutan}

Program assembly yang lebih kompleks untuk mendemonstrasikan konsep lanjutan.

\subsection{Program Array Processing}

\begin{verbatim}
; Array Processing Program
; =======================
DATA SEGMENT
    array DW 10, 20, 30, 40, 50, 60, 70, 80, 90, 100
    size DW 10
    sum DW 0
    average DW 0
    max DW 0
    min DW 0
    msg_sum DB 'Sum: ', '$'
    msg_avg DB 'Average: ', '$'
    msg_max DB 'Max: ', '$'
    msg_min DB 'Min: ', '$'
DATA ENDS

CODE SEGMENT
    ASSUME CS:CODE, DS:DATA

start:
    ; Initialize DS
    MOV AX, DATA
    MOV DS, AX
    
    ; Calculate sum and find min/max
    MOV CX, [size]
    MOV SI, 0
    MOV BX, [array]      ; BX = first element (initial max)
    MOV DX, [array]      ; DX = first element (initial min)
    MOV AX, 0            ; AX = sum accumulator
    
sum_loop:
    ADD AX, [array+SI]   ; Add to sum
    CMP [array+SI], BX    ; Compare with current max
    JLE not_max          ; Jump if less or equal
    MOV BX, [array+SI]   ; Update max
    
not_max:
    CMP [array+SI], DX    ; Compare with current min
    JGE not_min          ; Jump if greater or equal
    MOV DX, [array+SI]   ; Update min
    
not_min:
    ADD SI, 2            ; Move to next element
    LOOP sum_loop
    
    ; Store results
    MOV [sum], AX
    MOV [max], BX
    MOV [min], DX
    
    ; Calculate average (sum / size)
    XOR DX, DX           ; Clear DX for division
    MOV CX, [size]
    DIV CX               ; AX = AX / CX
    MOV [average], AX
    
    ; Exit program
    MOV AH, 4Ch
    INT 21h

CODE ENDS
END start
\end{verbatim}

\subsection{Program String Operations}

\begin{verbatim}
; String Operations Program
; =========================
DATA SEGMENT
    source_str DB 'Hello World', 0
    dest_str DB 50 DUP(0)
    length DW 0
    msg_copy DB 'String copied: ', '$'
DATA ENDS

CODE SEGMENT
    ASSUME CS:CODE, DS:DATA

start:
    ; Initialize DS
    MOV AX, DATA
    MOV DS, AX
    
    ; Calculate string length
    MOV SI, OFFSET source_str
    MOV CX, 0
    
length_loop:
    MOV AL, [SI]
    CMP AL, 0            ; Check for null terminator
    JE length_done
    INC CX
    INC SI
    JMP length_loop
    
length_done:
    MOV [length], CX
    
    ; Copy string
    MOV SI, OFFSET source_str
    MOV DI, OFFSET dest_str
    MOV CX, [length]
    INC CX               ; Include null terminator
    
copy_loop:
    MOV AL, [SI]
    MOV [DI], AL
    INC SI
    INC DI
    LOOP copy_loop
    
    ; Exit program
    MOV AH, 4Ch
    INT 21h

CODE ENDS
END start
\end{verbatim}

\subsection{Ringkasan Operasi String}

Program di atas mendemonstrasikan penggunaan instruksi LOOP, MOV dengan indeks SI/DI, dan operasi pada string null-terminated. Instruksi \texttt{MOVSB} dan \texttt{MOVSW} dapat digunakan untuk mengoptimasi operasi copy dengan prefix \texttt{REP}.

\section{Teknik Manajemen Memori}

Manajemen memori yang efektif sangat penting dalam pemrograman assembly.

\subsection{Stack Management}

Stack digunakan untuk temporary storage dan procedure calls.

\begin{verbatim}
; Stack Management Example
; =======================
DATA SEGMENT
    buffer DW 100 DUP(0)
    count DW 0
DATA ENDS

CODE SEGMENT
    ASSUME CS:CODE, DS:DATA

start:
    ; Initialize DS and SS
    MOV AX, DATA
    MOV DS, AX
    MOV AX, DATA
    MOV SS, AX
    
    ; Initialize stack pointer
    MOV SP, OFFSET buffer + 200  ; Top of stack
    
    ; Push values onto stack
    MOV AX, 10
    PUSH AX
    MOV BX, 20
    PUSH BX
    MOV CX, 30
    PUSH CX
    
    ; Pop values from stack (LIFO order)
    POP DX      ; DX = 30
    POP DX      ; DX = 20
    POP DX      ; DX = 10
    
    ; Exit program
    MOV AH, 4Ch
    INT 21h

CODE ENDS
END start
\end{verbatim}

\subsection{Dynamic Memory Allocation}

\begin{verbatim}
; Dynamic Memory Allocation
; ========================
DATA SEGMENT
    heap_start DW 0
    heap_size DW 1024
    allocated DW 0
DATA ENDS

CODE SEGMENT
    ASSUME CS:CODE, DS:DATA

start:
    ; Initialize DS
    MOV AX, DATA
    MOV DS, AX
    
    ; Simulate memory allocation
    MOV AX, [heap_size]
    MOV [allocated], AX
    
    ; Use allocated memory
    MOV SI, [heap_start]
    MOV [SI], 1234h
    
    ; Exit program
    MOV AH, 4Ch
    INT 21h

CODE ENDS
END start
\end{verbatim}

\subsection{Memory Segmentation}

\begin{verbatim}
; Memory Segmentation Example
; ===========================
DATA SEGMENT
    data_var1 DW 100
    data_var2 DW 200
DATA ENDS

CODE SEGMENT
    ASSUME CS:CODE, DS:DATA

start:
    ; Initialize DS
    MOV AX, DATA
    MOV DS, AX
    
    ; Access data using different addressing modes
    MOV AX, data_var1        ; Direct addressing
    MOV BX, OFFSET data_var2  ; Get address
    MOV CX, [BX]             ; Indirect addressing
    
    ; Use ES for extra segment
    MOV AX, DATA
    MOV ES, AX
    MOV DX, ES:data_var1    ; Access via ES
    
    ; Exit program
    MOV AH, 4Ch
    INT 21h

CODE ENDS
END start
\end{verbatim}

\begin{catatan}
Program di atas mendemonstrasikan penggunaan segment register ES untuk mengakses data. Instruksi \texttt{MOV AX, ES:data\_var1} menunjukkan cara explicit segment override ketika mengakses memori melalui segment selain default DS.
\end{catatan}

\section{Proyek Assembly Lengkap}

Proyek lengkap untuk mendemonstrasikan semua konsep assembly programming yang telah dipelajari.

\subsection{Program Kalkulator Scientific}

\begin{verbatim}
; Scientific Calculator Program
; =============================
DATA SEGMENT
    ; Input buffer
    input_buffer DB 50 DUP(0)
    input_length DW 0
    
    ; Operation codes
    OP_ADD EQU 1
    OP_SUB EQU 2
    OP_MUL EQU 3
    OP_DIV EQU 4
    OP_MOD EQU 5
    OP_POW EQU 6
    
    ; Variables
    operand1 DW 0
    operand2 DW 0
    result DW 0
    operation DW 0
    
    ; Messages
    msg_welcome DB '=== Scientific Calculator ===', 0Dh, 0Ah, '$'
    msg_prompt1 DB 'Enter first number: ', '$'
    msg_prompt2 DB 'Enter second number: ', '$'
    msg_op DB 'Enter operation (+,-,*,/,%,^): ', '$'
    msg_result DB 'Result: ', '$'
    msg_error DB 'Error: Invalid operation', 0Dh, 0Ah, '$'
    msg_continue DB 'Continue? (y/n): ', '$'
    
    ; Constants
    MAX_INPUT EQU 50
DATA ENDS

CODE SEGMENT
    ASSUME CS:CODE, DS:DATA

start:
    ; Initialize DS
    MOV AX, DATA
    MOV DS, AX
    
    ; Display welcome message
    MOV AH, 09h
    MOV DX, OFFSET msg_welcome
    INT 21h
    
main_loop:
    ; Get first operand
    CALL get_number
    MOV [operand1], AX
    
    ; Get operation
    CALL get_operation
    MOV [operation], BX
    
    ; Get second operand
    CALL get_number
    MOV [operand2], AX
    
    ; Perform calculation
    CALL calculate
    MOV [result], AX
    
    ; Display result
    CALL display_result
    
    ; Ask to continue
    CALL ask_continue
    CMP AL, 'y'
    JE main_loop
    
    ; Exit program
    MOV AH, 4Ch
    INT 21h

; Procedure to get number input
get_number PROC
    ; Display prompt
    MOV AH, 09h
    MOV DX, OFFSET msg_prompt1
    INT 21h
    
    ; Read input (simplified)
    MOV AH, 0Ah
    MOV DX, OFFSET input_buffer
    INT 21h
    
    ; Convert string to number (simplified)
    MOV SI, OFFSET input_buffer + 2
    MOV AX, 0
    MOV CX, 10
    
convert_loop:
    MOV BL, [SI]
    CMP BL, 0Dh
    JE convert_done
    
    SUB BL, '0'        ; Convert ASCII to digit
    MOV BH, 0
    MUL CX             ; AX = AX * 10
    ADD AX, BX         ; Add digit
    
    INC SI
    JMP convert_loop
    
convert_done:
    RET
get_number ENDP

; Procedure to get operation
get_operation PROC
    ; Display prompt
    MOV AH, 09h
    MOV DX, OFFSET msg_op
    INT 21h
    
    ; Read character
    MOV AH, 01h
    INT 21h
    
    ; Convert to operation code
    CMP AL, '+'
    JE op_add
    CMP AL, '-'
    JE op_sub
    CMP AL, '*'
    JE op_mul
    CMP AL, '/'
    JE op_div
    CMP AL, '%'
    JE op_mod
    CMP AL, '^'
    JE op_pow
    
    ; Invalid operation
    MOV BX, 0
    JMP op_done
    
op_add: MOV BX, OP_ADD
    JMP op_done
op_sub: MOV BX, OP_SUB
    JMP op_done
op_mul: MOV BX, OP_MUL
    JMP op_done
op_div: MOV BX, OP_DIV
    JMP op_done
op_mod: MOV BX, OP_MOD
    JMP op_done
op_pow: MOV BX, OP_POW
    
op_done:
    RET
get_operation ENDP

; Procedure to perform calculation
calculate PROC
    MOV AX, [operand1]
    MOV BX, [operand2]
    MOV CX, [operation]
    
    CMP CX, OP_ADD
    JE calc_add
    CMP CX, OP_SUB
    JE calc_sub
    CMP CX, OP_MUL
    JE calc_mul
    CMP CX, OP_DIV
    JE calc_div
    CMP CX, OP_MOD
    JE calc_mod
    CMP CX, OP_POW
    JE calc_pow
    
    ; Invalid operation
    MOV AX, 0
    JMP calc_done
    
calc_add:
    ADD AX, BX
    JMP calc_done
    
calc_sub:
    SUB AX, BX
    JMP calc_done
    
calc_mul:
    MUL BX
    JMP calc_done
    
calc_div:
    XOR DX, DX
    DIV BX
    JMP calc_done
    
calc_mod:
    XOR DX, DX
    DIV BX
    MOV AX, DX
    JMP calc_done
    
calc_pow:
    ; Power calculation (simplified)
    MOV CX, BX
    MOV BX, AX
    MOV AX, 1
    
power_loop:
    MUL BX
    LOOP power_loop
    JMP calc_done
    
calc_done:
    RET
calculate ENDP

; Procedure to display result
display_result PROC
    ; Display result message
    MOV AH, 09h
    MOV DX, OFFSET msg_result
    INT 21h
    
    ; Display number (simplified)
    MOV AX, [result]
    CALL display_number
    
    ; New line
    MOV AH, 02h
    MOV DL, 0Dh
    INT 21h
    MOV DL, 0Ah
    INT 21h
    
    RET
display_result ENDP

; Procedure to display number
display_number PROC
    ; Convert number to string and display
    ; (Implementation would be more complex)
    RET
display_number ENDP

; Procedure to ask continue
ask_continue PROC
    ; Display prompt
    MOV AH, 09h
    MOV DX, OFFSET msg_continue
    INT 21h
    
    ; Read response
    MOV AH, 01h
    INT 21h
    
    RET
ask_continue ENDP

CODE ENDS
END start
\end{verbatim}

\subsection{Program Features}

\textbf{Supported Operations:}
\begin{itemize}
  \item \textbf{Arithmetic}: Addition, Subtraction, Multiplication, Division
  \item \textbf{Advanced}: Modulo, Power operations
  \item \textbf{Input}: Number parsing and validation
  \item \textbf{Output}: Formatted result display
  \item \textbf{Looping}: Continuous calculation mode
\end{itemize}

\textbf{Programming Concepts Demonstrated:}
\begin{itemize}
  \item Modular programming with procedures
  \item Stack-based parameter passing
  \item Memory management and segmentation
  \item String processing and conversion
  \item User input/output handling
  \item Error handling and validation
\end{itemize}


% ============================================================
% AKTIVITAS PEMBELAJARAN
% ============================================================

\begin{aktivitas}
  \item \textbf{Register Exploration}: Gunakan TASM debugger untuk mengamati perubahan nilai register AX, BX, CX, DX saat menjalankan instruksi MOV, ADD, SUB. Dokumentasikan setiap step.
  
  \item \textbf{Segment Register Analysis}: Buat program yang mendemonstrasikan penggunaan CS, DS, ES, SS. Amati bagaimana nilai berubah saat program dieksekusi.
  
  \item \textbf{Flag Register Investigation}: Implementasikan operasi aritmatika dan logika yang mempengaruhi flag register (Zero, Carry, Sign, Overflow). Analisis perubahan flag.
  
  \item \textbf{Stack Operations}: Buat program yang menggunakan PUSH dan POP untuk menyimpan dan mengambil data dari stack. Visualisasikan perubahan SP.
  
  \item \textbf{Index Register Practice}: Implementasikan program string manipulation yang menggunakan SI dan DI untuk source dan destination index.
  
  \item \textbf{Code Review}: Analisis kode assembly teman dan identifikasi penggunaan register yang efisien vs tidak efisien.
\end{aktivitas}

% ============================================================
% LATIHAN DAN REFLEKSI
% ============================================================

\begin{latihan}
  \item Jelaskan perbedaan antara register AX, BX, CX, DX! Berikan contoh instruksi yang paling cocok untuk masing-masing register.
  
  \item Apa fungsi dari register 8-bit (AH, AL, BH, BL, CH, CL, DH, DL)? Kapan sebaiknya menggunakan register 8-bit vs 16-bit?
  
  \item Jelaskan perbedaan antara register index (SI, DI) dan register pointer (BP, SP)! Berikan contoh penggunaan masing-masing.
  
  \item Buat program assembly yang:
  \begin{itemize}
    \item Menginisialisasi AX = 1234h, BX = 5678h
    \item Menjumlahkan AX dan BX, simpan hasil di CX
    \item Menggunakan register segment DS untuk mengakses data
    \item Menampilkan hasil ke layar
  \end{itemize}
  
  \item Jelaskan bagaimana flag register bekerja! Analisis flag yang berubah setelah operasi ADD, SUB, AND, OR.
  
  \item Implementasikan program stack yang:
  \begin{itemize}
    \item Menggunakan PUSH untuk menyimpan 3 nilai ke stack
    \item Menggunakan POP untuk mengambil nilai dalam urutan terbalik
    \item Menampilkan perubahan SP setiap operasi
  \end{itemize}
  
  \item Buat program string copy yang menggunakan SI sebagai source index dan DI sebagai destination index.
  
  \item \textbf{Refleksi}: Manakah register yang paling sering Anda gunakan dan mengapa? Apa tantangan terbesar dalam memahami register 8086?
\end{latihan}

% ============================================================
% ASESMEN
% ============================================================

\begin{asesmen}
\textbf{Instrumen Penilaian untuk Sub-CPMK 1.1, 1.3, 2.1}

\textbf{A. Pilihan Ganda}

\begin{enumerate}
  \item Register yang secara khusus digunakan untuk operasi perkalian dan pembagian adalah:
  \begin{enumerate}
    \item AX
    \item BX
    \item CX
    \item DX
  \end{enumerate}
  
  \item Register yang digunakan sebagai stack pointer adalah:
  \begin{enumerate}
    \item BP
    \item SP
    \item SI
    \item DI
  \end{enumerate}
  
  \item Untuk mengakses data string sebagai source, register yang digunakan adalah:
  \begin{enumerate}
    \item SI
    \item DI
    \item BP
    \item SP
  \end{enumerate}
  
  \item Register segment yang menunjuk ke area kode program adalah:
  \begin{enumerate}
    \item CS
    \item DS
    \item ES
    \item SS
  \end{enumerate}
  
  \item Instruksi yang digunakan untuk mendorong nilai ke stack adalah:
  \begin{enumerate}
    \item PUSH
    \item POP
    \item MOV
    \item ADD
  \end{enumerate}
  
  \item Perbedaan \asm{EQU} dan variabel (\asm{DW}, \asm{DB}) yang benar adalah:
  \begin{enumerate}
    \item EQU tidak mengalokasi memori; variabel mengalokasi memori
    \item EQU dan variabel sama saja
    \item Variabel tidak mengalokasi memori
    \item EQU hanya untuk string
  \end{enumerate}
  
  \item Format buffer untuk INT 21h AH=0Ah (input string): byte 0 menyimpan:
  \begin{enumerate}
    \item Max length (maksimal karakter yang dapat dibaca)
    \item Actual length (jumlah karakter yang dibaca)
    \item Karakter pertama
    \item Terminator string
  \end{enumerate}
  
  \item Program .COM menggunakan entry point:
  \begin{enumerate}
    \item \asm{ORG 100h} dengan CS=DS=ES=SS
    \item \asm{END label} dengan segmen terpisah
    \item \asm{.MODEL LARGE}
    \item Tidak ada entry point
  \end{enumerate}
\end{enumerate}

\textbf{B. Essay}

\begin{enumerate}
  \item Jelaskan perbedaan fungsi antara register AX dan register CX! Berikan contoh program yang menggunakan kedua register tersebut.
  
  \item Mengapa register BP sering digunakan untuk mengakses parameter dalam prosedur? Jelaskan dengan contoh.
  
  \item Jelaskan perbedaan \asm{EQU} dan variabel (\asm{DW})! Kapan masing-masing sebaiknya digunakan?
  
  \item Jelaskan format buffer untuk INT 21h AH=0Ah! Apa fungsi byte 0, byte 1, dan byte 2+?
\end{enumerate}

\textbf{C. Practical Challenge}

\begin{enumerate}
  \item Buat program assembly yang mengimplementasikan kalkulator sederhana:
  \begin{itemize}
    \item Input dua bilangan dari user (menggunakan interupsi DOS)
    \item Simpan bilangan pertama di AX, kedua di BX
    \item Lakukan operasi penjumlahan, simpan hasil di CX
    \item Tampilkan hasil ke layar
    \item Gunakan register DS untuk mengakses data program
    \item Dokumentasikan penggunaan register dengan TASM debugger
  \end{itemize}
  
  \item Buat deklarasi buffer untuk input string dengan INT 21h AH=0Ah yang dapat menerima maksimal 80 karakter. Jelaskan format buffer yang Anda gunakan.
\end{enumerate}

\textbf{Rubrik Penilaian}: Lihat Lampiran A
\end{asesmen}

% ============================================================
% CHECKLIST KOMPETENSI
% ============================================================

\begin{checklist}
  \item Saya dapat menjelaskan fungsi register AX, BX, CX, DX
  \item Saya dapat membedakan register 8-bit dan 16-bit
  \item Saya dapat menggunakan register index (SI, DI) untuk operasi string
  \item Saya dapat menggunakan register pointer (BP, SP) untuk stack dan parameter
  \item Saya memahami fungsi register segment (CS, DS, ES, SS)
  \item Saya dapat menganalisis flag register setelah operasi aritmatika
  \item Saya dapat mengimplementasikan operasi stack (PUSH/POP)
  \item Saya dapat menulis program assembly dengan TASM syntax
  \item Saya dapat menggunakan register secara efisien dalam program
\end{checklist}

% ============================================================
% RANGKUMAN
% ============================================================

\begin{rangkuman}
Bab ini membahas struktur prosesor Intel 8086 dan organisasi register, termasuk register general purpose, register index, register pointer, register segment, dan flag register.

\textbf{Poin Kunci:}
\begin{itemize}
  \item Register general purpose (AX, BX, CX, DX) digunakan untuk operasi data umum
  \item AX khusus untuk aritmatika, CX untuk counter, DX untuk I/O dan aritmatika
  \item Register index (SI, DI) digunakan untuk operasi string dan array
  \item Register pointer (BP, SP) mengelola stack frame dan parameter prosedur
  \item Register segment (CS, DS, ES, SS) mengorganisasi memori segmentasi
  \item Flag register menunjukkan status hasil operasi (Zero, Carry, Sign, Overflow)
  \item TASM menyediakan tools untuk debugging register dan instruksi
  \item Penggunaan register yang efisien kunci untuk optimasi program assembly
\end{itemize}

\textbf{Kata Kunci}: \textbf{\texttt{Register}}, \textbf{\texttt{AX}}, \textbf{\texttt{BX}}, \textbf{\texttt{CX}}, \textbf{\texttt{DX}}, \textbf{\texttt{SI}}, \textbf{\texttt{DI}}, \textbf{\texttt{BP}}, \textbf{\texttt{SP}}, \textbf{\texttt{Segment Register}}, \textbf{\texttt{Flag Register}}, \textbf{\texttt{TASM}}, \textbf{\texttt{Stack}}
\end{rangkuman}

\ifSubfilesClassLoaded{
  \renewcommand{\bibname}{Daftar Pustaka}
  \bibliographystyle{plain}
  \bibliography{../references}
}{}
\end{document}
