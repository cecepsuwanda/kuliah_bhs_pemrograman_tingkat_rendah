\documentclass[../main.tex]{subfiles}
\ifSubfilesClassLoaded{\setcounter{chapter}{5}}{}
\begin{document}

\chapter{Struktur Kontrol dan Perulangan}

\begin{subcpmk}
  \item Mendukung CPMK-3 (operasi kontrol): Mengimplementasikan struktur kontrol (IF-THEN-ELSE, CASE) dan perulangan (FOR, WHILE, DO-WHILE) — Materi Minggu 9-10
  \item Sub-CPMK 2.1: Menulis program assembly sederhana dengan TASM syntax
\end{subcpmk}

% ============================================================
% MATERI POKOK
% ============================================================
% ============================================================
% MATERI POKOK
% ============================================================
\section{Struktur Kontrol}

\subsection{Conditional Jumps}
Instruksi jump yang dieksekusi berdasarkan kondisi flag register.

\begin{verbatim}
; Unconditional jump
JMP label        ; Selalu jump ke label

; Conditional jumps - berdasarkan flag
JZ  target       ; Jump if Zero (ZF = 1)
JNZ target       ; Jump if Not Zero (ZF = 0)
JC  target       ; Jump if Carry (CF = 1)
JNC target       ; Jump if No Carry (CF = 0)
JS  target       ; Jump if Sign (SF = 1)
JNS target       ; Jump if Not Sign (SF = 0)
JO  target       ; Jump if Overflow (OF = 1)
JNO target       ; Jump if No Overflow (OF = 0)
\end{verbatim}

\textbf{Contoh penggunaan:}
\begin{verbatim}
CMP AX, BX      ; Bandingkan AX dengan BX, set flags
JZ  equal        ; Jump jika AX = BX
JG  greater       ; Jump jika AX > BX (signed)
JAE above_equal   ; Jump jika AX >= BX (unsigned)
\end{verbatim}

\subsection{IF-THEN-ELSE Structure}
Implementasi struktur kondisional dengan jumps.

\begin{verbatim}
; IF-THEN structure
CMP AX, 0
JNE not_zero
    MOV CX, 1      ; Eksekusi jika AX != 0
    JMP end_if
not_zero:
    MOV CX, 0      ; Eksekusi jika AX = 0
end_if:
; Lanjutkan program
\end{verbatim}

\subsection{CASE/SWITCH Structure}
Implementasi struktur pemilihan dengan multiple jumps.

\begin{verbatim}
; CASE structure dengan jump table
MOV AL, input_char    ; Load input
SUB AL, 'A'        ; Bandingkan dengan 'A'
JE  case_A          ; Jump jika sama
SUB AL, 'B'        ; Bandingkan dengan 'B'
JE  case_B          ; Jump jika sama
SUB AL, 'C'        ; Bandingkan dengan 'C'
JE  case_C          ; Jump jika sama
JMP default_case     ; Default case
\end{verbatim}

\section{Perulangan (Loops)}

\subsection{LOOP Instruction}
Instruksi LOOP untuk perulangan dengan CX counter.

\begin{verbatim}
; Basic loop
MOV CX, 10        ; Set counter = 10
loop_start:
; ... instruksi yang diulang
LOOP loop_start    ; CX--, jump jika CX != 0
\end{verbatim}

\textbf{Variasi LOOP:}
\begin{itemize}
  \item \textbf{LOOP}: Loop tanpa kondisi
  \item \textbf{LOOPE}: Loop jika ZF = 1
  \item \textbf{LOOPNE}: Loop jika ZF = 0
  \item \textbf{LOOPC}: Loop jika CF = 1
  \item \textbf{LOOPNC}: Loop jika CF = 0
\end{itemize}

\subsection{WHILE Loop}
Implementasi perulangan dengan kondisi awal.

\begin{verbatim}
; WHILE structure
MOV CX, array_size
MOV SI, 0
while_loop:
    CMP SI, CX
    JAE end_while    ; Jump jika SI >= CX
    
    ; ... proses array[SI]
    INC SI
    JMP while_loop
end_while:
\end{verbatim}

\subsection{FOR Loop}
Implementasi perulangan dengan counter terkontrol.

\begin{verbatim}
; FOR loop (0 to n-1)
MOV CX, n
MOV SI, 0
for_loop:
    ; ... proses array[SI]
    INC SI
    LOOP for_loop    ; CX--, loop jika CX != 0
\end{verbatim}

\subsection{DO-WHILE Loop}
Implementasi perulangan dengan test di akhir.

\begin{verbatim}
; DO-WHILE structure
MOV CX, 0
do_while_loop:
    ; ... proses yang diulang
    INC CX
    CMP CX, limit
    JL do_while_loop ; Loop jika CX < limit
\end{verbatim}

\subsection{Nested Loops}
Perulangan bersarang untuk multidimensional array.

\begin{verbatim}
; Nested loop example
MOV CX, rows       ; Outer loop counter
outer_loop:
    MOV DX, cols       ; Inner loop counter
    inner_loop:
        ; ... proses array[row][col]
        INC DX
        LOOP inner_loop
    INC SI
    LOOP outer_loop
\end{verbatim}

\textbf{Optimasi nested loops:}
\begin{itemize}
  \item Gunakan register berbeda untuk inner/outer loop
  \item Minimalkan instruksi dalam inner loop
  \item Pertimbangkan loop unrolling
  \item Hindari akses memori berulang kali
\end{itemize}

\section{Prosedur Kontrol Lanjutan}

\subsection{CALL dan RET}
Instruksi untuk prosedur dan return value.

\begin{verbatim}
; Procedure call
CALL procedure_name    ; Push IP, jump ke procedure
; ... eksekusi prosedur
RET                 ; Pop IP, return ke caller

; Function dengan parameter
MOV AX, param1
MOV BX, param2
CALL calculate
; Result di AX
\end{verbatim}

\subsection{Stack Frame Management}
Manajemen stack frame untuk prosedur kompleks.

\begin{verbatim}
; Standard procedure entry
PUSH BP              ; Simpan BP lama
MOV BP, SP           ; Setup stack frame
SUB SP, local_size   ; Reserve space untuk local variables

; Akses parameter
MOV AX, [BP+4]      ; Parameter pertama
MOV BX, [BP+6]      ; Parameter kedua

; Akses local variables
MOV CX, [BP-2]      ; Local variable pertama
MOV DX, [BP-4]      ; Local variable kedua

; Standard procedure exit
MOV SP, BP           ; Cleanup stack frame
POP BP               ; Restore BP lama
RET                 ; Return
\end{verbatim}

\textbf{Stack Layout:}
\begin{itemize}
  \item \textbf{Parameter}: Diakses dengan positive offset dari BP
  \item \textbf{Local Variables}: Diakses dengan negative offset dari BP
  \item \textbf{Return Address}: Disimpan di stack sebelum CALL
  \item \textbf{Saved BP}: BP lama disimpan di stack
\end{itemize}

% ============================================================
% AKTIVITAS PEMBELAJARAN
% ============================================================

\begin{aktivitas}
  \item \textbf{Conditional Logic}: Implementasikan program grade converter (A/B/C/D/F) dengan nested IF-THEN-ELSE.
  
  \item \textbf{Loop Optimization}: Bandingkan berbagai teknik loop optimization (unrolling, strength reduction).
  
  \item \textbf{Jump Table}: Buat program calculator dengan jump table untuk operasi aritmatika.
  
  \item \textbf{Procedure Calls}: Implementasikan program dengan prosedur untuk sorting array.
  
  \item \textbf{Stack Analysis}: Gunakan TASM debugger untuk menganalisis stack frame.
  
  \item \textbf{Control Flow}: Buat state machine sederhana dengan jumps.
\end{aktivitas}

% ============================================================
% LATIHAN DAN REFLEKSI
% ============================================================

\begin{latihan}
  \item Buat program untuk menentukan bilangan prima dengan nested loops.
  
  \item Implementasikan binary search menggunakan WHILE loop dan conditional jumps.
  
  \item Buat program untuk menghitung faktorial dengan FOR loop dan prosedur rekursif.
  
  \item Implementasikan bubble sort dengan nested loops dan optimasi.
  
  \item Buat program menu-driven dengan CASE structure untuk berbagai operasi.
  
  \item Implementasikan program untuk membalik string dengan loop dan pointer arithmetic.
  
  \item Buat program untuk menghitung jumlah digit dalam bilangan dengan loop dan conditional logic.
  
  \item \textbf{Refleksi}: Teknik loop mana yang paling sulit dipahami? Bagaimana Anda mengatasi kesulitan tersebut?
\end{latihan}

% ============================================================
% ASESMEN
% ============================================================

\begin{asesmen}
\textbf{Instrumen Penilaian untuk CPMK-3 (kontrol) dan Sub-CPMK 2.1}

\textbf{A. Pilihan Ganda}

\begin{enumerate}
  \item Instruksi yang digunakan untuk unconditional jump adalah:
  \begin{enumerate}
    \item JUMP
    \item JMP
    \item CALL
    \item RET
  \end{enumerate}
  
  \item Untuk implementasi WHILE loop, instruksi yang paling penting adalah:
  \begin{enumerate}
    \item CMP
    \item Jcondition
    \item JMP
    \item LOOP
  \end{enumerate}
  
  \item Register yang umum digunakan sebagai loop counter adalah:
  \begin{enumerate}
    \item CX
    \item BX
    \item DX
    \item AX
  \end{enumerate}
  
  \item Stack pointer yang digunakan untuk prosedur adalah:
  \begin{enumerate}
    \item SP
    \item BP
    \item SI
    \item DI
  \end{enumerate}
\end{enumerate}

\textbf{B. Essay}

\begin{enumerate}
  \item Jelaskan perbedaan antara LOOP dan manual counter dengan DEC/JNZ! Kapan sebaiknya menggunakan masing-masing?
  
  \item Mengapa stack frame management penting dalam prosedur yang kompleks?
\end{enumerate}

\textbf{C. Practical Challenge}

\begin{enumerate}
  \item Buat program text editor:
  \begin{itemize}
    \item Implementasi cursor movement dengan arrow keys
    \item Text insertion dan deletion
    \item Search dan replace functionality
    \item Multiple undo/redo operations
    \item File save dan load operations
    \item Gunakan prosedur untuk modular design
  \end{itemize}
\end{enumerate}

\textbf{Rubrik Penilaian}: Lihat Lampiran A
\end{asesmen}

% ============================================================
% CHECKLIST KOMPETENSI
% ============================================================

\begin{checklist}
  \item Saya dapat mengimplementasikan struktur kontrol (IF-THEN-ELSE, CASE)
  \item Saya dapat menggunakan berbagai jenis jumps (conditional, unconditional)
  \item Saya dapat membuat perulangan (WHILE, FOR, DO-WHILE)
  \item Saya dapat mengoptimasi loops untuk performa lebih baik
  \item Saya dapat mengimplementasikan nested loops
  \item Saya dapat menggunakan CALL dan RET untuk prosedur
  \item Saya dapat mengelola stack frame dengan benar
  \item Saya dapat menganalisis control flow dengan TASM debugger
\end{checklist}

% ============================================================
% RANGKUMAN
% ============================================================

\begin{rangkuman}
Bab ini membahas struktur kontrol dan perulangan dalam assembly 8086, termasuk conditional jumps, berbagai jenis loop, dan manajemen prosedur dengan stack.

\textbf{Poin Kunci:}
\begin{itemize}
  \item Conditional jumps memungkinkan implementasi IF-THEN-ELSE dan CASE structures
  \item LOOP instruksi menyediakan perulangan otomatis dengan counter
  \item Stack frame management essential untuk prosedur dengan parameter dan local variables
  \item Nested loops memungkinkan pemrosesan data multidimensional
  \item CALL dan RET mendukung modular programming dan code reuse
  \item Optimasi loops meningkatkan performa program secara signifikan
  \item Control flow analysis dengan debugger penting untuk troubleshooting
\end{itemize}

\textbf{Kata Kunci}: \asm{Struktur Kontrol}, \asm{Conditional Jump}, \asm{Loop}, \asm{Prosedur}, \asm{Stack Frame}, \asm{CALL}, \asm{RET}, \asm{TASM}, \asm{Optimasi}
\end{rangkuman}


% ============================================================
% AKTIVITAS PEMBELAJARAN
% ============================================================

\begin{aktivitas}
  \item \textbf{Conditional Logic}: Buat program yang mengimplementasikan grade calculation (A, B, C, D, E) dengan IF-THEN-ELSE.
  
  \item \textbf{Menu System}: Implementasikan menu-driven program dengan struktur CASE menggunakan jump table.
  
  \item \textbf{Loop Patterns}: Buat program yang mendemonstrasikan 4 jenis loop (FOR, WHILE, DO-WHILE, nested).
  
  \item \textbf{Flag Analysis}: Gunakan TASM debugger untuk mengamati flag register sebelum dan sesudah instruksi CMP.
  
  \item \textbf{Performance Comparison}: Bandingkan kecepatan LOOP vs DEC/JNZ untuk implementasi loop.
  
  \item \textbf{Complex Logic}: Implementasikan program sorting dengan nested loop dan conditional statements.
\end{aktivitas}

% ============================================================
% LATIHAN DAN REFLEKSI
% ============================================================

\begin{latihan}
  \item Jelaskan perbedaan antara instruksi JG dan JA! Kapan sebaiknya menggunakan masing-masing?
  
  \item Buat program untuk menentukan bilangan terbesar dari 3 bilangan menggunakan nested IF.
  
  \item Implementasikan program yang:
  \begin{itemize}
    \item Menerima input angka 1-7 dari user
    \item Menampilkan nama hari yang sesuai
    \item Menggunakan struktur CASE dengan jump table
    \item Handle input invalid
  \end{itemize}
  
  \item Buat program untuk menghitung faktorial menggunakan FOR loop.
  
  \item Implementasikan program untuk mencari bilangan prima dengan nested loop.
  
  \item Bandingkan implementasi WHILE vs DO-WHILE untuk kasus yang sama.
  
  \item \textbf{Refleksi}: Struktur kontrol mana yang paling sulit diimplementasikan dalam assembly? Bagaimana Anda mengatasi kesulitan tersebut?
\end{latihan}

% ============================================================
% ASESMEN
% ============================================================

\begin{asesmen}
\textbf{Instrumen Penilaian untuk Sub-CPMK 4.1, 4.2, 2.1}

\textbf{A. Pilihan Ganda}

\begin{enumerate}
  \item Instruksi yang digunakan untuk lompat jika carry flag set adalah:
  \begin{enumerate}
    \item JC
    \item JNC
    \item JZ
    \item JNZ
  \end{enumerate}
  
  \item Untuk implementasi FOR loop dengan counter, register yang paling cocok adalah:
  \begin{enumerate}
    \item AX
    \item BX
    \item CX
    \item DX
  \end{enumerate}
  
  \item Instruksi yang secara otomatis mendekrement counter dan melakukan jump adalah:
  \begin{enumerate}
    \item JMP
    \item DEC
    \item LOOP
    \item JNZ
  \end{enumerate}
  
  \item Jump yang cocok untuk unsigned comparison adalah:
  \begin{enumerate}
    \item JG
    \item JL
    \item JA
    \item JGE
  \end{enumerate}
\end{enumerate}

\textbf{B. Essay}

\begin{enumerate}
  \item Jelaskan perbedaan antara implementasi WHILE dan DO-WHILE dalam assembly language!
  
  \item Mengapa jump table lebih efisien untuk implementasi CASE dibandingkan dengan nested IF?
\end{enumerate}

\textbf{C. Practical Challenge}

\begin{enumerate}
  \item Buat program student grade system:
  \begin{itemize}
    \item Input nilai 0-100
    \item Konversi ke grade (A: 85-100, B: 70-84, C: 55-69, D: 40-54, E: 0-39)
    \item Tampilkan grade dan keterangan
    \item Handle input validation
    \item Gunakan struktur kontrol yang efisien
  \end{itemize}
\end{enumerate}

\textbf{Rubrik Penilaian}: Lihat Lampiran A
\end{asesmen}

% ============================================================
% CHECKLIST KOMPETENSI
% ============================================================

\begin{checklist}
  \item Saya dapat mengimplementasikan struktur IF-THEN-ELSE
  \item Saya dapat menggunakan jump instructions berdasarkan flag register
  \item Saya dapat mengimplementasikan struktur CASE dengan jump table
  \item Saya dapat membuat berbagai jenis loop (FOR, WHILE, DO-WHILE)
  \item Saya dapat menggunakan instruksi LOOP khusus
  \item Saya dapat mengimplementasikan nested loop
  \item Saya dapat memilih instruksi jump yang tepat untuk kasus tertentu
  \item Saya dapat menganalisis flag register untuk conditional logic
\end{checklist}

% ============================================================
% RANGKUMAN
% ============================================================

\begin{rangkuman}
Bab ini membahas struktur kontrol dan perulangan dalam assembly language, termasuk implementasi conditional statements dan berbagai jenis loop.

\textbf{Poin Kunci:}
\begin{itemize}
  \item Instruksi jump memungkinkan kontrol alur program berdasarkan kondisi
  \item Flag register (ZF, CF, SF, OF) menentukan kondisi jump
  \item IF-THEN-ELSE diimplementasikan dengan CMP dan conditional jump
  \item Struktur CASE lebih efisien dengan jump table
  \item Berbagai jenis loop dapat diimplementasikan dengan counter dan jump
  \item Instruksi LOOP khusus untuk implementasi loop yang efisien
  \item Nested loop memerlukan manajemen counter yang hati-hati
  \item Pemilihan instruksi jump mempengaruhi kinerja dan ukuran kode
\end{itemize}

\textbf{Kata Kunci}: \textbf{\texttt{Jump}}, \textbf{\texttt{Conditional}}, \textbf{\texttt{IF-THEN-ELSE}}, \textbf{\texttt{CASE}}, \textbf{\texttt{LOOP}}, \textbf{\texttt{Flag Register}}, \textbf{\texttt{CMP}}, \textbf{\texttt{TASM}}, \textbf{\texttt{Control Flow}}
\end{rangkuman}

\ifSubfilesClassLoaded{
  \renewcommand{\bibname}{Daftar Pustaka}
  \bibliographystyle{plain}
  \bibliography{../references}
}{}
\end{document}
