% ============================================================
% MATERI POKOK
% ============================================================
\section{Operasi Aritmatika}

\subsection{Penjumlahan dan Pengurangan}
Instruksi ADD dan SUB adalah operasi aritmatika dasar dalam assembly 8086.

\begin{verbatim}
; Addition examples
ADD AX, BX      ; AX = AX + BX, flags terpengaruh
ADD CX, 10       ; CX = CX + 10 (immediate)
ADD [SI], AL     ; memori[SI] = memori[SI] + AL

; Subtraction examples  
SUB DX, CX      ; DX = DX - CX, flags terpengaruh
SUB AX, 5       ; AX = AX - 5 (immediate)
SUB [DI], BL     ; memori[DI] = memori[DI] - BL
\end{verbatim}

\textbf{Flags yang terpengaruh:}
\begin{itemize}
  \item \textbf{ZF (Zero Flag)}: Set jika hasil = 0
  \item \textbf{SF (Sign Flag)}: Set jika hasil negatif (MSB = 1)
  \item \textbf{CF (Carry Flag)}: Set jika ada borrow (SUB) atau carry (ADD)
  \item \textbf{OF (Overflow Flag)}: Set jika overflow signed
  \item \textbf{AF (Auxiliary Flag)}: Set jika carry dari bit 3 ke bit 4
\end{itemize}

\subsection{Perkalian dan Pembagian}
Instruksi MUL dan DIV untuk operasi perkalian dan pembagian.

\begin{verbatim}
; Unsigned multiplication
MUL BL          ; AX = AL * BL (8-bit x 8-bit = 16-bit)
MUL CX          ; DX:AX = AX * CX (16-bit x 16-bit = 32-bit)

; Signed multiplication  
IMUL BX         ; DX:AX = AX * BX (signed)

; Unsigned division
DIV BL          ; AL = AX / BL, AH = AX % BL
DIV CX          ; AX = DX:AX / CX, DX = DX:AX % CX

; Signed division
IDIV BX         ; AX = DX:AX / BX (signed), DX = DX:AX % BX (signed)
\end{verbatim}

\textbf{Catatan penting:}
\begin{itemize}
  \item MUL selalu menggunakan A register (AL/AX) sebagai operand implisit
  \item Hasil 8-bit: AL, 16-bit: AX, 32-bit: DX:AX
  \item DIV error (divide by zero) trigger interrupt 0
  \item Overflow pada MUL/DIV set CF dan OF
\end{itemize}

\subsection{Increment dan Decrement}
Instruksi INC dan DEC untuk operasi +1 dan -1.

\begin{verbatim}
; Increment
INC AX          ; AX = AX + 1
INC [SI]        ; memori[SI] = memori[SI] + 1
INC BL          ; BL = BL + 1

; Decrement  
DEC CX          ; CX = CX - 1
DEC [DI]        ; memori[DI] = memori[DI] - 1
DEC BH          ; BH = BH - 1
\end{verbatim}

\textbf{Pengaruh flags:}
\begin{itemize}
  \item INC/DEC mempengaruhi OF, SF, ZF, AF
  \item Tidak mempengaruhi CF (kecuali DEC word menjadi 0)
\end{itemize}

\subsection{Negasi dan Komplemen}
Instruksi NEG dan NOT untuk operasi negasi.

\begin{verbatim}
; Two's complement (negasi)
NEG AX          ; AX = -AX (two's complement)
NEG BL          ; BL = -BL

; One's complement (komplemen bitwise)
NOT AX          ; AX = ~AX (bitwise NOT)
NOT BL          ; BL = ~BL
\end{verbatim}

\textbf{Perbedaan NEG vs NOT:}
\begin{itemize}
  \item NEG: arithmetic negation (-n)
  \item NOT: bitwise complement (~n)
  \item NEG = NOT + 1
\end{itemize}

\section{Operasi Logika}

\subsection{AND Operation}
Bitwise AND untuk masking dan testing bits.

\begin{verbatim}
; Basic AND
AND AX, BX      ; AX = AX & BX
AND AL, 0Fh     ; Mask lower nibble (keep 4 LSB)
AND BL, 01h     ; Test bit 0 (result 0 atau 1)

; Memory operations
AND [SI], 80h    ; memori[SI] = memori[SI] & 80h
AND WORD PTR [DI], 0FF00h ; Mask upper byte
\end{verbatim}

\textbf{Aplikasi AND:}
\begin{itemize}
  \item Clear bits: AND dengan 0
  \item Test bits: AND dengan mask, cek ZF
  \item Extract bits: AND dengan mask spesifik
\end{itemize}

\subsection{OR Operation}
Bitwise OR untuk setting bits.

\begin{verbatim}
; Basic OR
OR AX, BX       ; AX = AX | BX
OR AL, 80h      ; Set bit 7
OR BL, 01h      ; Set bit 0

; Memory operations
OR [SI], 0Fh    ; Set lower nibble
OR WORD PTR [DI], 8000h ; Set bit 15
\end{verbatim}

\textbf{Aplikasi OR:}
\begin{itemize}
  \item Set bits ke 1
  \item Combine flag bits
  \item Create bit patterns
\end{itemize}

\subsection{XOR Operation}
Bitwise XOR untuk toggling dan enkripsi sederhana.

\begin{verbatim}
; Basic XOR
XOR AX, BX      ; AX = AX ^ BX
XOR AL, 01h     ; Toggle bit 0
XOR BL, 0FFh    ; Toggle semua bits

; Clear register (XOR dengan diri sendiri)
XOR AX, AX       ; AX = 0
XOR CX, CX       ; CX = 0

; Memory operations
XOR [SI], 80h    ; Toggle bit 7 di memori
\end{verbatim}

\textbf{Aplikasi XOR:}
\begin{itemize}
  \item Toggle bits
  \item Simple encryption/decryption
  \item Clear registers
  \item Parity checking
\end{itemize}

\subsection{TEST Instruction}
TEST seperti AND tetapi tidak mengubah hasil, hanya mengubah flags.

\begin{verbatim}
; Test individual bits
TEST AL, 01h     ; Test bit 0, set ZF jika bit 0
TEST AH, 80h     ; Test bit 7

; Test multiple bits
TEST AX, 0F0Fh   ; Test lower byte, set ZF jika 0
TEST BX, 8000h   ; Test bit 15

; Test untuk jump condition
TEST CX, CX       ; Test apakah CX = 0, set ZF
JZ zero           ; Jump jika CX = 0
\end{verbatim}

\textbf{Keuntungan TEST:}
\begin{itemize}
  \item Tidak mengubah data asli
  \item Efisien untuk conditional testing
  \item Sering digunakan sebelum conditional jump
\end{itemize}

\section{Operasi Shift dan Rotate}

\subsection{Logical Shift}
SHL dan SHR untuk pergeseran bit dengan mempertahankan sign.

\begin{verbatim}
; Shift Left (SHL)
SHL AX, 1       ; AX = AX << 1, CF = bit yang keluar
SHL BX, CL       ; BX = BX << CL, CF = bit yang keluar

; Shift Right (SHR) 
SHR AX, 1       ; AX = AX >> 1 (unsigned), CF = bit yang keluar
SHR BX, CL       ; BX = BX >> CL (unsigned)
\end{verbatim}

\textbf{Pengaruh flags:}
\begin{itemize}
  \item CF berisi bit yang digeser keluar
  \item OF tergantung pada shift left (SHL)
  \item ZF set jika hasil = 0
\end{itemize}

\subsection{Arithmetic Shift}
SAL dan SAR untuk pergeseran dengan mempertahankan sign.

\begin{verbatim}
; Shift Arithmetic Left (SAL) - sama dengan SHL
SAL AX, 1       ; AX = AX << 1 (signed)

; Shift Arithmetic Right (SAR)
SAR AX, 1       ; AX = AX >> 1 (signed), sign bit preserved
SAR BX, CL       ; BX = BX >> CL (signed)
\end{verbatim}

\textbf{Perbedaan SHR vs SAR:}
\begin{itemize}
  \item SHR: mengisi MSB dengan 0 (unsigned shift)
  \item SAR: mempertahankan sign bit (signed shift)
  \item SAR mempertahankan tanda bilangan negatif
\end{itemize}

\subsection{Rotate Operations}
RCL, RCR, ROL, ROR untuk rotasi bit.

\begin{verbatim}
; Rotate Left (ROL)
ROL AL, 1       ; Rotate left 1 bit, CF = bit yang keluar
ROL AX, CL       ; Rotate left CL bits

; Rotate Right (ROR)  
ROR AL, 1       ; Rotate right 1 bit
ROR AX, CL       ; Rotate right CL bits

; Rotate through Carry (RCL/RCR)
RCL AL, 1       ; Rotate left through carry
RCR AL, 1       ; Rotate right through carry
\end{verbatim}

\textbf{Aplikasi Rotate:}
\begin{itemize}
  \item Circular buffer
  \item Cryptography
  \item Bit manipulation
  \item Graphics operations
\end{itemize}
