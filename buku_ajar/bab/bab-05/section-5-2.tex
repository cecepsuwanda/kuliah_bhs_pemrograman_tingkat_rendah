% ============================================================
% AKTIVITAS PEMBELAJARAN
% ============================================================

\begin{aktivitas}
  \item \textbf{Arithmetic Calculator}: Implementasikan kalkulator dengan operasi ADD, SUB, MUL, DIV.
  
  \item \textbf{Logic Gates}: Simulasi logic gates (AND, OR, XOR, NOT) menggunakan instruksi assembly.
  
  \item \textbf{Bit Manipulation}: Buat program untuk mengekstrak, mengatur, dan menguji bit individual.
  
  \item \textbf{Shift Operations}: Implementasikan program enkripsi sederhana dengan shift dan rotate.
  
  \item \textbf{Flag Analysis}: Gunakan TASM debugger untuk mengamati perubahan flag register.
  
  \item \textbf{Performance Testing}: Bandingkan kecepatan berbagai operasi aritmatika dan logika.
\end{aktivitas}

% ============================================================
% LATIHAN DAN REFLEKSI
% ============================================================

\begin{latihan}
  \item Buat program untuk menghitung faktorial menggunakan MUL dan loop.
  
  \item Implementasikan program untuk mengkonversi bilangan desimal ke biner dengan shift operations.
  
  \item Buat program yang mengecek apakah suatu bilangan adalah pangkat dari 2 menggunakan TEST instruksi.
  
  \item Implementasikan program enkripsi Caesar cipher dengan XOR operations.
  
  \item Buat program untuk menghitung jumlah bit yang set (population count) dalam sebuah register.
  
  \item Implementasikan swap dua variabel tanpa menggunakan temporary variable (hanya XOR).
  
  \item Buat program untuk membagi bilangan besar dengan DIV dan handle overflow.
  
  \item \textbf{Refleksi}: Operasi logika mana yang paling sulit dipahami? Bagaimana Anda mengatasi kesulitan tersebut?
\end{latihan}

% ============================================================
% ASESMEN
% ============================================================

\begin{asesmen}
\textbf{Instrumen Penilaian untuk Sub-CPMK 3.1, 3.2}

\textbf{A. Pilihan Ganda}

\begin{enumerate}
  \item Instruksi untuk perkalian unsigned 8-bit adalah:
  \begin{enumerate}
    \item MUL
    \item IMUL
    \item DIV
    \item IDIV
  \end{enumerate}
  
  \item Flag yang tidak terpengaruh oleh instruksi AND adalah:
  \begin{enumerate}
    \item CF (Carry Flag)
    \item OF (Overflow Flag)
    \item ZF (Zero Flag)
    \item SF (Sign Flag)
  \end{enumerate}
  
  \item Instruksi untuk shift right yang mempertahankan sign bit adalah:
  \begin{enumerate}
    \item SHR
    \item SHL
    \item SAR
    \item SAL
  \end{enumerate}
  
  \item Hasil dari XOR AX, AX adalah:
  \begin{enumerate}
    \item AX
    \item 0
    \item 0FFFFh
    \item Tidak terdefinisi
  \end{enumerate}
\end{enumerate}

\textbf{B. Essay}

\begin{enumerate}
  \item Jelaskan perbedaan antara instruksi TEST dan CMP! Kapan sebaiknya menggunakan masing-masing?
  
  \item Mengapa instruksi MUL memiliki format yang berbeda dengan instruksi ADD dalam hal operand?
\end{enumerate}

\textbf{C. Practical Challenge}

\begin{enumerate}
  \item Buat program kalkulator scientific:
  \begin{itemize}
    \item Support operasi dasar (+, -, *, /)
    \item Implementasikan fungsi trigonometri dengan lookup table
    \item Gunakan operasi logika untuk validasi input
    \item Handle overflow dan underflow
    \item Tampilkan hasil dengan format yang baik
  \end{itemize}
\end{enumerate}

\textbf{Rubrik Penilaian}: Lihat Lampiran A
\end{asesmen}

% ============================================================
% CHECKLIST KOMPETENSI
% ============================================================

\begin{checklist}
  \item Saya dapat mengimplementasikan operasi aritmatika (ADD, SUB, MUL, DIV)
  \item Saya dapat menerapkan operasi logika (AND, OR, XOR, NOT, TEST)
  \item Saya dapat menggunakan shift dan rotate operations
  \item Saya memahami pengaruh operasi terhadap flag register
  \item Saya dapat menangani overflow dan underflow
  \item Saya dapat mengoptimasi operasi bit manipulation
  \item Saya dapat menggunakan TASM debugger untuk tracing operasi
  \item Saya dapat menerapkan konsep two's complement
\end{checklist}

% ============================================================
% RANGKUMAN
% ============================================================

\begin{rangkuman}
Bab ini membahas operasi aritmatika dan logika dalam assembly 8086, termasuk instruksi dasar, flag manipulation, dan teknik bit manipulation.

\textbf{Poin Kunci:}
\begin{itemize}
  \item Operasi aritmatika meliputi ADD, SUB, MUL, DIV, INC, DEC, NEG
  \item Operasi logika meliputi AND, OR, XOR, NOT, TEST
  \item Shift operations: SHL, SHR, SAL, SAR untuk pergeseran bit
  \item Rotate operations: ROL, ROR, RCL, RCR untuk rotasi bit
  \item Flag register penting untuk conditional operations dan error detection
  \item Bit manipulation essential untuk encryption, compression, dan graphics
  \item Overflow handling critical untuk numerical computations
  \item TASM debugger membantu analisis operasi langkah demi langkah
\end{itemize}

\textbf{Kata Kunci}: \asm{Aritmatika}, \asm{Logika}, \asm{ADD}, \asm{SUB}, \asm{MUL}, \asm{DIV}, \asm{AND}, \asm{OR}, \asm{XOR}, \asm{NOT}, \asm{TEST}, \asm{SHL}, \asm{SHR}, \asm{SAL}, \asm{SAR}, \asm{ROL}, \asm{ROR}, \asm{Flag}, \asm{TASM}
\end{rangkuman}
