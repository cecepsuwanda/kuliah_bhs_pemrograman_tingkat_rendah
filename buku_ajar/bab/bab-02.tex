\documentclass[../main.tex]{subfiles}
\ifSubfilesClassLoaded{\setcounter{chapter}{1}}{}
\begin{document}

\chapter{Arsitektur Komputer dan Konsep Assembly}

\begin{subcpmk}
  \item Melakukan konversi antar sistem bilangan (desimal, biner, heksadesimal)
  \item Menjelaskan representasi data (Two's complement, Endianness)
  \item Sub-CPMK 1.1: Menjelaskan fungsi register AX, BX, CX, DX, SI, DI, BP, SP dalam prosesor 8086
  \item Sub-CPMK 1.2: Mengidentifikasi mode addressing (immediate, direct, indirect, indexed)
  \item Sub-CPMK 1.3: Mendemonstrasikan organisasi memori segmentasi (CS, DS, ES, SS)
\end{subcpmk}

% ============================================================
% MATERI POKOK
% ============================================================
% ============================================================
% PETA KONSEP / BAGAN STRUKTUR MATERI
% ============================================================
\section{Peta Konsep dan Struktur Materi}

Peta konsep berikut menggambarkan hierarki dan hubungan antar konsep utama dalam pemrograman Assembly Intel 8086 yang akan dipelajari dalam buku ajar ini:

\begin{figure}[H]
\centering
\begin{tikzpicture}[
  node distance=0.8cm and 1.2cm,
  box/.style={rectangle, draw, rounded corners=2pt, fill=blue!8, minimum width=2.8cm, minimum height=0.6cm, font=\small, align=center},
  subbox/.style={rectangle, draw, rounded corners=1pt, fill=green!8, minimum width=2.2cm, minimum height=0.5cm, font=\footnotesize, align=center},
  arrow/.style={->, >=stealth, thick}
]
  % Level 1: Arsitektur
  \node[box] (arch) {Arsitektur Intel 8086};
  
  % Level 2: Register, Addressing, Segmentasi
  \node[subbox, below left=of arch] (reg) {Register\\AX, BX, CX, DX\\SI, DI, BP, SP};
  \node[subbox, below=of arch] (addr) {Mode Addressing\\Immediate, Direct\\Indirect, Indexed};
  \node[subbox, below right=of arch] (seg) {Segmentasi Memori\\CS, DS, ES, SS\\Segment:Offset};
  
  % Level 3: Instruksi, TASM
  \node[box, below=2.2cm of arch] (inst) {Instruksi Dasar\\Data Transfer, Aritmatika\\Logika, Control Flow};
  \node[box, below=3.5cm of arch] (tasm) {TASM / Turbo Assembler\\Assemble, Link, Debug};
  
  % Arrows
  \draw[arrow] (arch) -- (reg);
  \draw[arrow] (arch) -- (addr);
  \draw[arrow] (arch) -- (seg);
  \draw[arrow] (reg) -- (inst);
  \draw[arrow] (addr) -- (inst);
  \draw[arrow] (seg) -- (inst);
  \draw[arrow] (inst) -- (tasm);
\end{tikzpicture}
\caption{Peta Konsep Pemrograman Assembly Intel 8086}
\label{fig:peta-konsep}
\end{figure}

\textbf{Alur Pembelajaran:} Fondasi dimulai dari arsitektur 8086 (BIU, EU, register). Pemahaman register dan mode addressing memungkinkan penggunaan instruksi assembly yang benar. Konsep segmentasi memori dibutuhkan untuk pengalamatan dan organisasi program. Seluruh konsep diintegrasikan melalui TASM untuk pengembangan program assembly yang fungsional.

\section{Sistem Bilangan dan Representasi Data}

\subsection{Mengapa Sistem Bilangan Penting dalam Assembly?}

Dalam pemrograman assembly, pemahaman sistem bilangan sangat krusial karena \cite{wiki_binary_number}:
\begin{itemize}
  \item Komputer bekerja dengan sistem biner (0 dan 1)
  \item Alamat memori dan register dinyatakan dalam heksadesimal
  \item Debugging dan analisis kode memerlukan konversi antar sistem bilangan
  \item Optimasi memerlukan pemahaman representasi data di level bit
\end{itemize}

\subsection{Sistem Bilangan Dasar}

\subsubsection{Sistem Bilangan Biner (basis 2)}

Biner menggunakan dua digit (0 dan 1). Nilai tempatnya adalah pangkat 2: \(2^3, 2^2, 2^1, 2^0, \ldots\)

Contoh: \(101010_2 = 1\cdot 2^5 + 0\cdot 2^4 + 1\cdot 2^3 + 0\cdot 2^2 + 1\cdot 2^1 + 0\cdot 2^0 = 42_{10}\)

\subsubsection{Sistem Bilangan Heksadesimal (basis 16)}

Heksadesimal menggunakan digit 0--9 dan A--F (A=10, B=11, \ldots, F=15). Setiap digit heksadesimal memetakan tepat ke 4 bit (\textit{nibble}) \cite{wiki_hexadecimal}.

\begin{table}[H]
\centering
\caption{Tabel Konversi Sistem Bilangan}
\begin{tabular}{|c|c|c|}
\hline
\textbf{Desimal} & \textbf{Biner} & \textbf{Heksadesimal} \\
\hline
0 & 0000 & 0 \\
1 & 0001 & 1 \\
2 & 0010 & 2 \\
3 & 0011 & 3 \\
4 & 0100 & 4 \\
5 & 0101 & 5 \\
6 & 0110 & 6 \\
7 & 0111 & 7 \\
8 & 1000 & 8 \\
9 & 1001 & 9 \\
10 & 1010 & A \\
11 & 1011 & B \\
12 & 1100 & C \\
13 & 1101 & D \\
14 & 1110 & E \\
15 & 1111 & F \\
\hline
\end{tabular}
\end{table}

\subsection{Konversi Antar Sistem Bilangan}

Metode konversi sistematis berikut sangat membantu dalam debugging dan analisis kode assembly \cite{wiki_binary_number}:

\subsubsection{Desimal ke Biner}
Gunakan metode pembagian berulang oleh 2 dan catat sisa pembagiannya. Urutan sisa dari bawah ke atas membentuk bilangan biner.

\begin{contoh}
\textbf{Konversi \(214_{10}\) ke biner:}

\begin{table}[H]
\centering
\caption{Konversi Desimal ke Biner}
\begin{tabular}{|c|c|c|c|}
\hline
\textbf{Langkah} & \textbf{Operasi} & \textbf{Hasil Bagi} & \textbf{Sisa} \\
\hline
1 & 214 $\div$ 2 & 107 & 0 \\
2 & 107 $\div$ 2 & 53 & 1 \\
3 & 53 $\div$ 2 & 26 & 1 \\
4 & 26 $\div$ 2 & 13 & 0 \\
5 & 13 $\div$ 2 & 6 & 1 \\
6 & 6 $\div$ 2 & 3 & 0 \\
7 & 3 $\div$ 2 & 1 & 1 \\
8 & 1 $\div$ 2 & 0 & 1 \\
\hline
\end{tabular}
\end{table}

Membaca sisa dari bawah ke atas: \(11010110_2\). Jadi, \(214_{10} = 11010110_2\).
\end{contoh}

\subsubsection{Biner ke Desimal}
Kalikan setiap digit biner dengan \(2^{\text{posisi}}\) (posisi dimulai dari 0 di kanan), lalu jumlahkan.

\begin{contoh}
\textbf{Konversi \(11010110_2\) ke desimal:}

\begin{align*}
11010110_2 &= (1 \times 2^7) + (1 \times 2^6) + (0 \times 2^5) + (1 \times 2^4) + (0 \times 2^3) + (1 \times 2^2) + (1 \times 2^1) + (0 \times 2^0) \\
&= 128 + 64 + 0 + 16 + 0 + 4 + 2 + 0 = 214_{10}
\end{align*}
\end{contoh}

\subsubsection{Desimal ke Heksadesimal}
Metode pembagian berulang oleh 16. Jika sisa lebih dari 9, gunakan huruf A--F untuk 10--15.

\begin{contoh}
\textbf{Konversi \(255_{10}\) ke heksadesimal:}

\begin{table}[H]
\centering
\caption{Konversi Desimal ke Heksadesimal}
\begin{tabular}{|c|c|c|c|c|}
\hline
\textbf{Langkah} & \textbf{Operasi} & \textbf{Hasil Bagi} & \textbf{Sisa} & \textbf{Representasi} \\
\hline
1 & 255 $\div$ 16 & 15 & 15 & F \\
2 & 15 $\div$ 16 & 0 & 15 & F \\
\hline
\end{tabular}
\end{table}

Membaca sisa dari bawah ke atas: \texttt{FF}h. Jadi, \(255_{10} = \mathrm{FF}_{16}\).
\end{contoh}

\subsubsection{Heksadesimal ke Biner dan Biner ke Heksadesimal}
Heksadesimal ke Biner: Setiap digit heksa = 4 bit biner (gunakan tabel di atas). Contoh: \(\mathrm{D6}_{16} = 1101\,0110_2\).

Biner ke Heksadesimal: Kelompokkan 4 bit dari kanan ke kiri, petakan ke digit heksa. Contoh: \(11010110_2 = 1101\,0110 = \mathrm{D6}_{16}\).

\subsection{Operasi Aritmatika Biner}

Operasi aritmatika biner fundamental untuk memahami perilaku ALU dan flag register (carry, overflow) dalam assembly \cite{wiki_binary_number}.

\subsubsection{Penjumlahan Biner}
Aturan dasar: \(0+0=0\), \(0+1=1\), \(1+1=10_2\) (tulis 0, simpan carry 1). Carry dari bit sebelumnya ditambahkan ke digit berikutnya.

\begin{contoh}
\textbf{Penjumlahan \(1011_2 + 1101_2\):}

\begin{tabular}{ccccc}
  & 1 & 0 & 1 & 1 \\
+ & 1 & 1 & 0 & 1 \\
\hline
1 & 1 & 0 & 0 & 0 \\
\end{tabular}

Langkah: \(1+1=10\) (carry 1), \(1+0+1=10\) (carry 1), \(0+1+1=10\) (carry 1), \(1+1+1=11\). Hasil: \(11000_2\) (11 + 13 = 24).
\end{contoh}

\subsubsection{Pengurangan Biner}
Aturan: \(0-0=0\), \(1-0=1\), \(1-1=0\), \(0-1=1\) (dengan \textit{borrow} dari posisi kiri). Borrow mempengaruhi flag CF (Carry Flag) pada instruksi SUB.

\begin{contoh}
\textbf{Pengurangan \(10010_2 - 01101_2\)} (18 - 13 = 5):

\begin{tabular}{cccccc}
1 & 0 & 0 & 1 & 0 & \\
- & 0 & 1 & 1 & 0 & 1 \\
\hline
  & 0 & 0 & 1 & 0 & 1 \\
\end{tabular}

Hasil: \(00101_2 = 5_{10}\).
\end{contoh}

\subsection{Representasi Data dalam Komputer}

\subsubsection{Satuan Data Dasar}

\begin{itemize}
  \item \textbf{Bit}: Unit terkecil, bernilai 0 atau 1
  \item \textbf{Byte}: 8 bit; satuan standar untuk karakter
  \item \textbf{Word}: Pada 8086, word = 16 bit (2 byte)
  \item \textbf{Double Word}: 32 bit (4 byte)
\end{itemize}

\subsubsection{Two's Complement (Bilangan Bertanda)}

Dalam assembly, instruksi seperti \asm{ADD}, \asm{SUB}, \asm{IMUL} bekerja dengan bilangan bertanda. Komputer merepresentasikan bilangan negatif dengan \textbf{Two's Complement} \cite{wiki_two_complement}:
\begin{enumerate}
  \item Tulis bilangan positif dalam biner
  \item Flip semua bit (0 jadi 1, 1 jadi 0)
  \item Tambahkan 1 ke hasilnya
\end{enumerate}

Contoh: \(-5\) dalam 8-bit:
\begin{itemize}
  \item \(+5 = 00000101\)
  \item Flip: \(11111010\)
  \item \(+1\): \(11111011\) = \(-5\)
\end{itemize}

\subsubsection{Endianness (Little-Endian pada 8086)}

Pemahaman endianness penting saat mengakses memori dengan \asm{MOV [addr], AX} atau saat debugging dengan TASM. \textbf{Little-Endian}: Byte LSB (Least Significant Byte) disimpan di alamat memori terkecil. Pada 8086, word \(0x1234\) disimpan sebagai: alamat 1000: 0x34, alamat 1001: 0x12 \cite{wiki_endianness}.

\subsubsection{Rentang Nilai}

\begin{table}[H]
\centering
\caption{Rentang nilai integer}
\begin{tabular}{|l|l|l|}
\hline
\textbf{Ukuran} & \textbf{Unsigned} & \textbf{Signed (two's complement)} \\
\hline
8-bit & 0 -- 255 & $-128$ -- 127 \\
16-bit & 0 -- 65535 & $-32768$ -- 32767 \\
\hline
\end{tabular}
\end{table}

\subsection{Aplikasi Assembly}

Assembly language diterapkan dalam: embedded systems, bootloader, BIOS/UEFI, driver, kernel, keamanan siber, dan optimasi performa kritis.

\section{Arsitektur Intel 8086}

Intel 8086 adalah 16-bit processor yang menjadi fondasi untuk arsitektur x86 modern. Processor ini dirancang untuk memberikan keseimbangan antara performa, kompleksitas, dan biaya produksi \cite{ref1}.

\subsection{Evolusi Processor Intel}

\begin{enumerate}
  \item \textbf{1971}: Intel 4004 - 4-bit processor pertama
  \item \textbf{1974}: Intel 8080 - 8-bit processor
  \item \textbf{1978}: Intel 8086 - 16-bit processor (target kita)
  \item \textbf{1982}: Intel 80286 - 16-bit dengan protected mode
  \item \textbf{1985}: Intel 80386 - 32-bit processor
  \item \textbf{1989}: Intel 80486 - 32-bit dengan cache
  \item \textbf{1993}: Pentium - Superscalar architecture
\end{enumerate}

\subsection{Fitur Utama Intel 8086}

\textbf{Arsitektur 16-bit:}
\begin{itemize}
  \item Data bus 16-bit untuk transfer data paralel
  \item Address bus 20-bit untuk 1MB addressable memory
  \item Clock speed: 4.77MHz hingga 10MHz
  \item 29,000 transistors pada chip
\end{itemize}

\textbf{Instruction Set:}
\begin{itemize}
  \item 117 instruksi dasar
  \item Operasi aritmatika, logika, dan transfer data
  \item String processing instructions
  \item Control flow dan loop instructions
  \item I/O dan interrupt instructions
\end{itemize}

\textbf{Mode Operasi:}
\begin{itemize}
  \item \textbf{Real Mode}: Kompatibel dengan 8080/8085
  \item \textbf{Protected Mode}: Tersedia di 80286 ke atas
  \item \textbf{Virtual 8086 Mode}: Tersedia di 80386 ke atas
\end{itemize}

\section{Register dan Organisasi Memori}

\begin{figure}[H]
\centering
\includegraphics[width=0.85\textwidth]{8086_registers.png}
\caption{Register-Register Intel 8086}
\end{figure}

\subsection{General Purpose Registers}

Intel 8086 memiliki 8 general-purpose registers 16-bit yang dapat digunakan untuk berbagai operasi:

\begin{verbatim}
; Register breakdown (16-bit)
AX = AH + AL    ; Accumulator (High + Low)
BX = BH + BL    ; Base register
CX = CH + CL    ; Count register
DX = DH + DL    ; Data register
\end{verbatim}

\textbf{Fungsi Spesifik Register:}
\begin{itemize}
  \item \textbf{AX}: Akumulator utama—digunakan untuk operasi aritmatika (\asm{ADD}, \asm{SUB}), I/O (\asm{IN}/\asm{OUT}), dan return value. MUL/DIV menggunakan AX secara implisit.
  \item \textbf{BX}: Base register—untuk addressing modes \asm{[BX]}, \asm{[BX+SI]}. Pointer ke struktur data dan array.
  \item \textbf{CX}: Count register—counter untuk \asm{LOOP} (decrement otomatis) dan \asm{REP} (string ops). Juga untuk shift count (\asm{SHL AX, CL}).
  \item \textbf{DX}: Data register—extended accumulator untuk MUL/DIV (DX:AX), I/O port $>$255, dividend high untuk DIV.
\end{itemize}

\subsection{Special Purpose Registers}

\begin{verbatim}
; Special purpose registers
SP  - Stack Pointer (menunjuk ke top of stack)
BP  - Base Pointer (menunjuk ke base of stack frame)
SI  - Source Index (index register untuk string operations)
DI  - Destination Index (index register untuk string operations)
IP  - Instruction Pointer (menunjuk ke next instruction)
\end{verbatim}

\textbf{Penggunaan Special Registers:}
\begin{itemize}
  \item \textbf{SP}: Stack management—diubah otomatis oleh PUSH, POP, CALL, RET. Tidak diubah manual kecuali setup/cleanup.
  \item \textbf{BP}: Base pointer untuk stack frame—akses parameter (\asm{[BP+4]}) dan variabel lokal (\asm{[BP-2]}). Default segment: SS.
  \item \textbf{SI/DI}: String operations (MOVS, CMPS, LODS, STOS)—SI default DS, DI default ES. Juga untuk array indexing.
  \item \textbf{IP}: Program counter (tidak dapat diakses langsung; diubah oleh JMP, CALL, RET).
\end{itemize}

\subsection{Segment Registers}

\begin{verbatim}
; Segment registers untuk memory management
CS  - Code Segment (instruction pointer)
DS  - Data Segment (data access)
SS  - Stack Segment (stack operations)
ES  - Extra Segment (data transfer)
\end{verbatim}

\textbf{Segmentation Memory Model:}
\begin{itemize}
  \item \textbf{CS}: Berisi alamat segment untuk kode program
  \item \textbf{DS}: Berisi alamat segment untuk data
  \item \textbf{SS}: Berisi alamat segment untuk stack
  \item \textbf{ES}: Berisi alamat segment untuk string operations
\end{itemize}

\subsection{Segmentasi Memori}

\begin{figure}[H]
\centering
\includegraphics[width=0.8\textwidth]{8086_memory_segmentation.png}
\caption{Organisasi Memori Segmentasi Intel 8086}
\end{figure}

\subsection{Perhitungan Alamat Fisik}

Intel 8086 menggunakan model memori segmentasi di mana alamat logis terdiri dari pasangan \textbf{segment:offset}. Alamat fisik dihitung dengan formula:

\begin{center}
\textbf{Alamat Fisik} = (Segment $\times$ 16) + Offset
\end{center}

\begin{contoh}
\textbf{Contoh Perhitungan Alamat:}

Untuk CS:IP = 2000h:1000h:
\begin{itemize}
  \item Alamat fisik = (2000h $\times$ 10h) + 1000h = 20000h + 1000h = 21000h
\end{itemize}

Untuk DS:SI = 3000h:0050h:
\begin{itemize}
  \item Alamat fisik = (3000h $\times$ 10h) + 0050h = 30000h + 50h = 30050h
\end{itemize}
\end{contoh}

\subsection{Contoh Penggunaan Register dalam Instruksi}

Berikut contoh instruksi assembly yang memanfaatkan register:

\begin{verbatim}
; Load immediate value ke AX
MOV AX, 1234h

; Copy dari AX ke BX (register addressing)
MOV BX, AX

; Store ke memory (direct addressing)
MOV [0100h], AX

; Access dengan segment:offset
MOV AX, DS:[SI]
\end{verbatim}

\textbf{Contoh SI/DI untuk string:} \asm{MOV SI, OFFSET src} / \asm{MOV DI, OFFSET dst} lalu \asm{REP MOVSB} untuk menyalin string. \textbf{Contoh BP untuk stack frame:} \asm{PUSH BP} / \asm{MOV BP, SP} lalu \asm{MOV AX, [BP+4]} untuk akses parameter yang di-push sebelum CALL.

\section{Instruction Set Dasar}

Intel 8086 memiliki instruction set yang kaya dan komprehensif untuk berbagai operasi pemrograman.

\subsection{Kategori Instruksi}

\textbf{Data Transfer Instructions:}
\begin{itemize}
  \item \textbf{MOV}: Transfer data antar register/memory
  \item \textbf{PUSH/POP}: Stack operations
  \item \textbf{XCHG}: Exchange data antar register
  \item \textbf{LEA}: Load effective address
  \item \textbf{LDS/LES}: Load pointer dari memory
\end{itemize}

\textbf{Arithmetic Instructions:}
\begin{itemize}
  \item \textbf{ADD/SUB}: Penjumlahan dan pengurangan
  \item \textbf{MUL/DIV}: Perkalian dan pembagian
  \item \textbf{INC/DEC}: Increment dan decrement
  \item \textbf{NEG}: Negasi (two's complement)
  \item \textbf{CMP}: Comparison (set flags)
\end{itemize}

\textbf{Logic Instructions:}
\begin{itemize}
  \item \textbf{AND/OR/XOR}: Operasi logika bitwise
  \item \textbf{NOT}: Negasi bitwise
  \item \textbf{TEST}: Test bits (set flags)
  \item \textbf{SHL/SHR}: Bit shift operations
  \item \textbf{ROR/ROL}: Bit rotate operations
\end{itemize}

\subsection{Syntax Instruksi}

\begin{verbatim}
; Format dasar instruksi assembly
[label:] mnemonic [operand1[, operand2[, operand3]]

; Contoh penggunaan
MOV AX, BX        ; AX = BX
ADD AX, 5         ; AX = AX + 5
SUB CX, COUNT      ; CX = CX - COUNT
JMP label_name     ; Jump ke label_name
\end{verbatim}

\textbf{Operands:}
\begin{itemize}
  \item \textbf{Register}: AX, BX, CX, DX, SI, DI, BP, SP
  \item \textbf{Immediate}: Konstanta numerik (5, 0FFh, 'A')
  \item \textbf{Memory}: [address], [BX+offset], [SI]
  \item \textbf{Implicit}: Tidak ada operand (CLD, STI, NOP)
\end{itemize}

\subsection{Flag Register}

Flag register (FLAGS) menyimpan status hasil operasi dan sangat penting untuk percabangan (conditional jump) \cite{intel_8086_user_manual}:

\begin{figure}[H]
\centering
\includegraphics[width=0.75\textwidth]{8086_flags_register.png}
\caption{Flag Register Intel 8086}
\end{figure}

\textbf{Flag Status Penting:}
\begin{itemize}
  \item \textbf{CF (Carry Flag)}: Set jika ada carry out atau borrow; digunakan untuk unsigned comparison
  \item \textbf{ZF (Zero Flag)}: Set jika hasil operasi = 0; digunakan untuk JZ, JNZ
  \item \textbf{SF (Sign Flag)}: Set jika hasil negatif (MSB = 1); untuk signed comparison
  \item \textbf{OF (Overflow Flag)}: Set jika signed overflow; untuk JO, JNO
  \item \textbf{PF (Parity Flag)}: Set jika jumlah bit 1 genap (even parity)
  \item \textbf{AF (Auxiliary Flag)}: Set untuk carry/borrow di nibble rendah; BCD adjust
  \item \textbf{DF (Direction Flag)}: 0 = increment (SI/DI naik), 1 = decrement untuk string ops
  \item \textbf{IF (Interrupt Flag)}: 0 = disable maskable interrupt, 1 = enable
  \item \textbf{TF (Trap Flag)}: Single-step mode untuk debugging
\end{itemize}

\textbf{Implikasi pada Arsitektur}: Operasi aritmatika (ADD, SUB, MUL, DIV) dan logika (AND, OR, XOR) memengaruhi flag. Instruksi CMP mengubah flag tanpa menyimpan hasil. Conditional jump (JE, JNE, JL, JG, dll.) memeriksa flag untuk mengambil keputusan percabangan.

\section{Addressing Modes}

Addressing modes menentukan bagaimana processor mengakses operan untuk instruksi \cite{electronics_hub_8086_addressing}. Pemilihan mode mempengaruhi kecepatan eksekusi, ukuran kode, dan fleksibilitas program.

\subsection{Immediate Addressing}

Mode immediate menggunakan nilai konstan yang disertakan langsung dalam instruksi. Cocok untuk inisialisasi register, konstanta mask, dan nilai yang tidak berubah saat runtime.

\begin{verbatim}
; Immediate addressing - nilai konstan langsung
MOV AX, 5        ; AX = 5 (immediate value)
ADD CX, 100h      ; CX = CX + 256
MOV BX, 'A'        ; BX = ASCII 'A'
\end{verbatim}

\textbf{Karakteristik:}
\begin{itemize}
  \item Nilai konstan langsung dalam instruksi
  \item Tidak memerlukan akses memori untuk operand
  \item Paling cepat untuk operasi konstan
  \item Operand kedua biasanya; format: hex (h), desimal, biner (b), ASCII ('A')
\end{itemize}

\subsection{Register Addressing}

Mode register paling efisien karena operand berada langsung di register. Tidak ada akses memori tambahan sehingga kecepatan eksekusi tertinggi. Digunakan untuk perhitungan sementara dan operasi berulang dalam loop.

\begin{verbatim}
; Register addressing - operan adalah register
MOV AX, BX        ; AX = BX
ADD CX, AX        ; CX = CX + AX
SUB DX, SI        ; DX = DX - SI
\end{verbatim}

\textbf{Karakteristik:}
\begin{itemize}
  \item Operan adalah register general purpose (AX, BX, CX, DX, SI, DI, BP, SP)
  \item Tidak memerlukan akses memori
  \item Sangat cepat dan efisien
  \item Beberapa instruksi memiliki register khusus (mis. MUL menggunakan AX)
\end{itemize}

\subsection{Direct Addressing}

Mode direct menggunakan alamat memori yang ditentukan secara eksplisit. Alamat adalah offset dalam segmen default (biasanya DS). Berguna untuk akses variabel global dengan alamat tetap.

\begin{verbatim}
; Direct addressing - alamat memori langsung
MOV AX, [1234h]    ; AX = memory[DS:1234h]
ADD BX, [data_var]   ; BX = memory[data_var]
MOV CX, [array+10]   ; CX = memory[array+10]
\end{verbatim}

\textbf{Karakteristik:}
\begin{itemize}
  \item Alamat memori langsung dalam instruksi
  \item Akses ke memory yang spesifik (default segment: DS)
  \item Lebih lambat dari register addressing (1 siklus memori)
  \item Segment override: \asm{ES:[addr]}, \asm{CS:[addr]}, \asm{SS:[addr]}
\end{itemize}

\subsection{Register Indirect Addressing}

Mode indirect menggunakan register sebagai pointer ke memori. Register (BX, SI, DI, BP) berisi alamat offset. Fleksibel untuk akses array dan struktur data dinamis karena alamat dapat diubah saat runtime.

\begin{verbatim}
; Register indirect - register berisi alamat
MOV AX, [BX]        ; AX = memory[DS:BX]
MOV CX, [SI]        ; CX = memory[DS:SI]
ADD DX, [DI]        ; DX = memory[ES:DI]
MOV AX, [BP]        ; AX = memory[SS:BP] (stack frame)
\end{verbatim}

\textbf{Karakteristik:}
\begin{itemize}
  \item Register valid: BX, SI, DI, BP
  \item Segment default: DS untuk BX/SI/DI; SS untuk BP
  \item Fleksibel untuk pointer dan loop array
  \item Lebih efisien daripada direct untuk akses berulang
\end{itemize}

\subsection{Based Indexed Addressing}

Mode based indexed menggabungkan base register (BX atau BP) dengan index register (SI atau DI) dan displacement opsional. Sangat fleksibel untuk akses array dua dimensi, struktur data, dan parameter stack.

\begin{verbatim}
; Based indexed addressing - base + index + displacement
MOV AX, [BX+SI]     ; AX = memory[BX+SI]
MOV CX, [BP+DI+4]   ; CX = memory[BP+DI+4] (parameter stack)
ADD DX, [BX+SI+10]  ; DX = memory[BX+SI+10]
\end{verbatim}

\textbf{Karakteristik:}
\begin{itemize}
  \item Kombinasi valid: BX+SI, BX+DI, BP+SI, BP+DI
  \item Displacement opsional: 8-bit atau 16-bit
  \item Sangat fleksibel untuk array 2D dan struktur data
  \item BP+displacement untuk akses parameter prosedur
\end{itemize}

\subsection{Perbandingan dan Pemilihan}

\begin{table}[H]
\centering
\caption{Perbandingan Mode Addressing}
\setlength{\tabcolsep}{3pt}
\scriptsize
\begin{tabular}{|p{1.8cm}|p{1.5cm}|p{2cm}|p{3.2cm}|}
\hline
\textbf{Mode} & \textbf{Kecepatan} & \textbf{Fleksibilitas} & \textbf{Use Case Optimal} \\
\hline
Immediate & Paling cepat & Rendah & Konstanta, inisialisasi \\
\hline
Register & Sangat cepat & Sedang & Perhitungan, loop body \\
\hline
Direct & Cepat & Sedang & Variabel global, alamat tetap \\
\hline
Indirect & Sedang & Tinggi & Array, pointer dinamis \\
\hline
Based Indexed & Sedang & Tertinggi & Array 2D, struktur, stack frame \\
\hline
\end{tabular}
\end{table}

\subsection{Contoh Kombinasi: \asm{MOV AX, [BX+SI+2]}}

Instruksi ini menggunakan based indexed addressing. CPU melakukan langkah berikut:
\begin{enumerate}
  \item Baca nilai BX (base address, misal 1000h)
  \item Baca nilai SI (index, misal 0010h)
  \item Hitung alamat efektif: 1000h + 0010h + 2 = 1012h
  \item Gabung dengan DS: alamat fisik = (DS × 16) + 1012h
  \item Baca 2 byte dari memori (LSB dulu, little-endian) ke AX
\end{enumerate}

Aplikasi: mengakses \texttt{array[BX][SI]} atau field struktur dengan offset 2.

% ============================================================
% TASM Development Environment
% ============================================================

% Define asm command for standalone compilation (providecommand avoids conflict when included in main)
\providecommand{\asm}[1]{\textbf{\texttt{#1}}}

\section{TASM Development Environment}

Turbo Assembler (TASM) adalah development environment untuk pemrograman assembly Intel 8086 \cite{borland1990tasm}. GUI Turbo Assembler (GTASM) mengintegrasikan TASM, TLINK, Turbo Debugger, dan DOSBox dalam satu IDE yang ramah pengguna untuk Windows modern \cite{jones2020}.

\subsection{Instalasi GUI Turbo Assembler (GTASM)}

GTASM cocok untuk pembelajaran assembly pada Windows 7, 8, 10, 11 (32-bit maupun 64-bit). Persyaratan: Microsoft .NET Framework 4.0 atau lebih tinggi, minimal 100 MB ruang penyimpanan.

\textbf{Langkah instalasi:}
\begin{enumerate}
  \item Unduh dari \url{https://github.com/ljnath/GUI-Turbo-Assembler} atau Softpedia
  \item Jalankan \texttt{GTASM\_Setup.exe}; jika ada peringatan keamanan, klik ``More info'' lalu ``Run anyway''
  \item Ikuti wizard (Next $\rightarrow$ pilih direktori $\rightarrow$ Install)
  \item Verifikasi: Buka GTASM, menu Help $\rightarrow$ About untuk memastikan TASM, TLINK, TD, DOSBox terinstal
\end{enumerate}

\begin{figure}[H]
\centering
\includegraphics[width=0.75\textwidth]{gtasm_installer.png}
\caption{Wizard instalasi GUI Turbo Assembler}
\end{figure}

\subsection{Komponen TASM}

\begin{itemize}
  \item \textbf{TASM.EXE}: Assembler utama—mengubah berkas \asm{.ASM} menjadi \asm{.OBJ} (object code). Mengecek sintaks dan menghasilkan kode mesin.
  \item \textbf{TLINK.EXE}: Linker—menghubungkan satu atau lebih \asm{.OBJ} dan library menjadi berkas \asm{.EXE} atau \asm{.COM}. Menyelesaikan referensi eksternal dan relokasi.
  \item \textbf{TD.EXE} (Turbo Debugger): Debugger untuk step-by-step eksekusi, breakpoint, pemeriksaan register dan memori.
  \item \textbf{DOSBox}: Emulator lingkungan DOS untuk menjalankan program assembly pada Windows modern; sudah terintegrasi dalam GTASM.
\end{itemize}

\begin{figure}[H]
\centering
\includegraphics[width=0.75\textwidth]{gtasm_interface.png}
\caption{Antarmuka utama GUI Turbo Assembler}
\end{figure}

\subsection{Struktur Program .COM vs .EXE}

\begin{table}[H]
\centering
\caption{Perbandingan .COM dan .EXE}
\setlength{\tabcolsep}{3pt}
\scriptsize
\begin{tabular}{|p{1.8cm}|p{3.5cm}|p{3.5cm}|}
\hline
\textbf{Aspek} & \textbf{.COM} & \textbf{.EXE} \\
\hline
Header & Tanpa header & Ada header relocation \\
\hline
Ukuran max & $\sim$64 KB & Hingga beberapa MB \\
\hline
Segmen & CS=DS=ES=SS (umumnya sama) & Kode, data, stack terpisah \\
\hline
Entry point & ORG 100h (offset 0100h) & END label\_utama \\
\hline
Model & Cocok untuk program kecil & Cocok untuk program besar \\
\hline
\end{tabular}
\end{table}

Program \asm{.COM}: gunakan \asm{ORG 100h} dan pastikan semua dalam satu segmen. Program \asm{.EXE}: gunakan \asm{.MODEL SMALL}, \asm{.DATA}, \asm{.CODE}, \asm{ASSUME}, dan inisialisasi \asm{DS} dengan \asm{MOV AX, @data} / \asm{MOV DS, AX}.

\subsection{Workflow Development}

\begin{center}
\textbf{PEMROGRAMAN ASSEMBLY INTEL 8086}\\
$\downarrow$\\
\textbf{DEVELOPMENT CYCLE}\\
$\downarrow$\\
{\small
\begin{tabular}{ccc}
Write Code & Compile & Debug
\end{tabular}
}\\
$\downarrow$\\
\textbf{PRODUKSI}\\
$\downarrow$\\
{\small
\begin{tabular}{ccc}
Executable File & Testing & Deployment
\end{tabular}
}\\
\end{center}

\begin{figure}[H]
\centering
\includegraphics[width=0.7\textwidth]{gtasm_build_process.png}
\caption{Proses build dengan GTASM (F9)}
\end{figure}

\subsection{TASM Commands}

\textbf{Command-line (jika tidak menggunakan GTASM):}
\begin{verbatim}
; Compile assembly file
TASM /zi program.asm
TLINK /v program.obj

; Run debugger
TD program.exe

; Build dengan makefile
MAKE
\end{verbatim}

\textbf{Switch Options:}
\begin{itemize}
  \item \textbf{/zi}: Include debug information (untuk breakpoint dan step)
  \item \textbf{/v}: Include verbose output
  \item \textbf{/c}: Case-sensitive symbols
  \item \textbf{/n}: Generate listing file
\end{itemize}

Dengan GTASM, cukup tekan F9 (Build) dan F10 (Run) untuk mengompilasi dan menjalankan program.


% ============================================================
% AKTIVITAS PEMBELAJARAN
% ============================================================

\begin{aktivitas}
  \item \textbf{Konversi dan Representasi}: Lakukan konversi 5 bilangan desimal ke biner dan heksadesimal. Verifikasi dengan TASM debugger untuk nilai di register.
  
  \item \textbf{Eksplorasi Register}: Gunakan TASM debugger untuk mengamati nilai register AX, BX, CX, DX saat program dieksekusi. Dokumentasikan perubahan nilai setiap instruksi.
  
  \item \textbf{Mode Addressing}: Buat program assembly yang mendemonstrasikan 5 mode addressing berbeda (immediate, direct, register, indirect, indexed). Bandingkan efisiensi masing-masing.
  
  \item \textbf{Organisasi Memori}: Visualisasikan segmentasi memori dengan diagram. Tunjukkan bagaimana CS:IP dan DS:SI mengakses lokasi memori yang berbeda.
  
  \item \textbf{Praktik TASM}: Instalasi dan konfigurasi GUI Turbo Assembler. Buat program sederhana "Hello World" dalam assembly dan jalankan dengan debugger.
  
  \item \textbf{Analisis Instruksi}: Pilih 10 instruksi assembly dasar. Analisis bit pattern dan format encoding masing-masing instruksi.
\end{aktivitas}

% ============================================================
% LATIHAN DAN REFLEKSI
% ============================================================

\begin{latihan}
  \item Konversikan \(85_{10}\) ke biner dan heksadesimal. Jelaskan langkah konversinya.
  
  \item Jelaskan representasi Two's complement untuk \(-10\) dalam 8-bit dan mengapa Little-Endian penting saat debugging alamat memori di 8086!
  
  \item Jelaskan perbedaan antara register general purpose (AX, BX, CX, DX) dan register index (SI, DI)! Berikan contoh penggunaan masing-masing.
  
  \item Apa fungsi dari segment register (CS, DS, ES, SS) dalam organisasi memori 8086? Jelaskan dengan contoh alamat fisik.
  
  \item Bandingkan kelebihan dan kekurangan setiap mode addressing! Kapan sebaiknya menggunakan mode immediate vs indirect?
  
  \item Buat diagram yang menunjukkan hubungan antara register, ALU, dan memori dalam arsitektur 8086.
  
  \item Jelaskan bagaimana cara kerja stack pointer (SP) dan base pointer (BP) dalam manajemen memori stack.
  
  \item Hitung alamat fisik dari CS:IP = 2000h:1000h dan DS:SI = 3000h:0050h. Jelaskan perbedaan segmentasi.
  
  \item Buat program assembly sederhana yang menggunakan minimal 5 register berbeda dan jelaskan fungsinya.
  
  \item \textbf{Refleksi}: Bagaimana pemahaman Anda tentang arsitektur komputer sebelum dan sesudah mempelajari bab ini? Konsep mana yang paling menantik?
\end{latihan}

% ============================================================
% ASESMEN
% ============================================================

\begin{asesmen}
\textbf{Instrumen Penilaian untuk Sub-CPMK 1.1, 1.2, 1.3 dan Sistem Bilangan}

\textbf{A. Pilihan Ganda}

\begin{enumerate}
  \item Hasil konversi \(214_{10}\) ke biner adalah:
  \begin{enumerate}
    \item \(10101010_2\)
    \item \(11010110_2\)
    \item \(11001100_2\)
    \item \(10110010_2\)
  \end{enumerate}
  
  \item Hasil konversi \(255_{10}\) ke heksadesimal adalah:
  \begin{enumerate}
    \item F5h
    \item FFh
    \item FEh
    \item 0Fh
  \end{enumerate}
  
  \item Register yang digunakan sebagai counter dalam operasi loop adalah:
  \begin{enumerate}
    \item AX
    \item BX
    \item CX
    \item DX
  \end{enumerate}
  
  \item Pada Intel 8086, word \(0x1234\) disimpan di memori dengan little-endian sebagai:
  \begin{enumerate}
    \item Alamat rendah: 0x12, alamat tinggi: 0x34
    \item Alamat rendah: 0x34, alamat tinggi: 0x12
    \item Hanya 0x1234 di satu alamat
    \item 0x12 dan 0x34 di alamat terpisah acak
  \end{enumerate}
  
  \item Two's complement untuk \(-5\) dalam 8-bit adalah:
  \begin{enumerate}
    \item \(11111011_{2}\)
    \item \(00000101_{2}\)
    \item \(11111010_{2}\)
    \item \(10000101_{2}\)
  \end{enumerate}
  
  \item Mode addressing yang menggunakan nilai konstan langsung dalam instruksi adalah:
  \begin{enumerate}
    \item Immediate
    \item Direct
    \item Register
    \item Indirect
  \end{enumerate}
  
  \item Alamat fisik dihitung dengan formula:
  \begin{enumerate}
    \item Segment + Offset
    \item Segment × 16 + Offset
    \item Segment × 10 + Offset
    \item Segment - Offset
  \end{enumerate}
  
  \item Register yang digunakan untuk menyimpan alamat return dari prosedur adalah:
  \begin{enumerate}
    \item SP
    \item BP
    \item IP
    \item SI
  \end{enumerate}
  
  \item Komponen yang diintegrasikan dalam GUI Turbo Assembler (GTASM) untuk menjalankan program assembly di Windows modern adalah:
  \begin{enumerate}
    \item TASM, TLINK, TD, DOSBox
    \item Hanya TASM dan TLINK
    \item Visual Studio dan MASM
    \item GCC dan GDB
  \end{enumerate}
  
  \item Perbedaan program .COM dan .EXE yang benar adalah:
  \begin{enumerate}
    \item .COM maksimal $\sim$64 KB, entry point ORG 100h; .EXE memiliki header dan segmen terpisah
    \item .COM dan .EXE identik
    \item .EXE lebih kecil dari .COM
    \item .COM memerlukan header relocation
  \end{enumerate}
\end{enumerate}

\textbf{B. Essay}

\begin{enumerate}
  \item Jelaskan metode konversi desimal ke biner dengan pembagian berulang! Berikan contoh singkat untuk \(42_{10}\).
  
  \item Jelaskan mengapa pemahaman Two's complement dan Endianness penting dalam debugging program assembly!
  
  \item Jelaskan perbedaan antara register AX dan AL! Berikan contoh instruksi yang menggunakan masing-masing register.
  
  \item Bandingkan mode addressing direct dan indirect dalam hal kecepatan akses dan fleksibilitas!
  
  \item Jelaskan langkah CPU saat mengeksekusi instruksi \asm{MOV AX, [BX+SI+2]}! (Asumsikan BX=1000h, SI=0010h, DS=2000h).
  
  \item Kapan sebaiknya menggunakan mode addressing immediate vs based indexed? Berikan contoh use case!
\end{enumerate}

\textbf{C. Practical Challenge}

\begin{enumerate}
  \item Lakukan operasi penjumlahan biner: \(1011_2 + 1101_2\). Tunjukkan langkah carry dan hasilnya dalam biner dan desimal.
  
  \item Konversikan \(11010110_2\) ke desimal dan ke heksadesimal. Tunjukkan perhitungannya.
  
  \item Buat program assembly yang:
  \begin{itemize}
    \item Menggunakan register AX untuk menyimpan nilai 1234h
    \item Memindahkan nilai ke register BX dengan mode addressing register
    \item Menyimpan nilai ke memori dengan mode addressing direct
    \item Menampilkan nilai tersebut ke layar menggunakan interupsi DOS
  \end{itemize}
  
  \item Verifikasi instalasi GTASM: buka menu Help $\rightarrow$ About, pastikan TASM, TLINK, TD, dan DOSBox terinstal. Dokumentasikan dengan screenshot.
  
  \item Dokumentasikan setiap langkah dengan screenshot dari TASM debugger.
\end{enumerate}

\textbf{Rubrik Penilaian}: Lihat Lampiran A
\end{asesmen}

% ============================================================
% CHECKLIST KOMPETENSI
% ============================================================

\begin{checklist}
  \item Saya dapat melakukan konversi desimal ke biner dan heksadesimal
  \item Saya dapat melakukan konversi biner ke desimal dan heksadesimal
  \item Saya dapat melakukan operasi penjumlahan dan pengurangan biner
  \item Saya dapat menjelaskan Two's complement dan representasi bilangan negatif
  \item Saya dapat menjelaskan Endianness (little-endian) pada 8086
  \item Saya dapat menjelaskan fungsi register AX, BX, CX, DX dalam prosesor 8086
  \item Saya dapat membedakan register index (SI, DI) dan register pointer (BP, SP)
  \item Saya dapat mengidentifikasi 5 mode addressing berbeda
  \item Saya dapat menghitung alamat fisik dari alamat logis segment:offset
  \item Saya memahami organisasi memori segmentasi 8086
  \item Saya dapat menggunakan TASM untuk membuat program assembly sederhana
  \item Saya dapat menganalisis instruksi assembly dasar
  \item Saya dapat mendemonstrasikan penggunaan register dalam program assembly
\end{checklist}

% ============================================================
% RANGKUMAN
% ============================================================

\begin{rangkuman}
Bab ini membahas landasan teori arsitektur komputer dan konsep assembly language, termasuk sistem bilangan, representasi data, struktur prosesor Intel 8086, organisasi register, mode addressing, dan manajemen memori segmentasi.

\textbf{Poin Kunci:}
\begin{itemize}
  \item Sistem bilangan: biner, heksadesimal, konversi antar basis, operasi aritmatika biner
  \item Representasi data: bit, byte, word, Two's complement, Endianness, rentang nilai
  \item Intel 8086 memiliki register general purpose (AX, BX, CX, DX) dan register khusus (SI, DI, BP, SP)
  \item Mode addressing menentukan cara instruksi mengakses data (immediate, direct, indirect, indexed)
  \item Memori segmentasi menggunakan segment register (CS, DS, ES, SS) dan offset
  \item Alamat fisik dihitung dengan formula: Segment × 16 + Offset
  \item TASM menyediakan lingkungan development untuk assembly language programming
  \item Pemahaman arsitektur adalah fondasi untuk pemrograman tingkat rendah
\end{itemize}

\textbf{Kata Kunci}: \asm{Register}, \asm{Addressing Mode}, \asm{Segmentasi}, \asm{Intel 8086}, \asm{TASM}, \asm{Assembly Language}, \asm{ALU}, \asm{Memori}, \asm{Biner}, \asm{Heksadesimal}, \asm{Two's Complement}, \asm{Endianness}
\end{rangkuman}

\ifSubfilesClassLoaded{
  \renewcommand{\bibname}{Daftar Pustaka}
  \bibliographystyle{plain}
  \bibliography{../references}
}{}
\end{document}
