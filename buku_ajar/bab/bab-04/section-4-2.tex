% ============================================================
% AKTIVITAS PEMBELAJARAN
% ============================================================

\begin{aktivitas}
  \item \textbf{Instruction Analysis}: Pilih 10 instruksi assembly berbeda. Analisis format, operand, dan flag yang terpengaruh.
  
  \item \textbf{Addressing Mode Practice}: Buat program yang mendemonstrasikan 7 mode addressing berbeda dengan operasi yang sama.
  
  \item \textbf{Directif Assembler}: Gunakan DB, DW, DD untuk mendefinisikan data. Bandingkan ukuran dan penggunaan memori.
  
  \item \textbf{Instruction Timing}: Ukur waktu eksekusi berbagai instruksi menggunakan TASM profiler.
  
  \item \textbf{Code Optimization}: Tulis program yang sama dengan 3 pendekatan addressing berbeda. Bandingkan efisiensi.
  
  \item \textbf{Debug Session}: Gunakan TASM debugger untuk trace eksekusi instruksi per instruksi.
\end{aktivitas}

% ============================================================
% LATIHAN DAN REFLEKSI
% ============================================================

\begin{latihan}
  \item Jelaskan perbedaan antara mode addressing immediate dan direct! Berikan contoh penggunaan masing-masing.
  
  \item Kapan sebaiknya menggunakan register indirect vs based indexed addressing?
  
  \item Buat program yang:
  \begin{itemize}
    \item Mendefinisikan array bilangan dengan DW
    \item Mengakses elemen array dengan berbagai mode addressing
    \item Menghitung jumlah semua elemen
    \item Menyimpan hasil di memori dengan DD
  \end{itemize}
  
  \item Analisis instruksi MOV AX, [BX+SI+2]. Identifikasi mode addressing dan langkah akses memori.
  
  \item Buat tabel perbandingan semua mode addressing (sintaks, kecepatan, fleksibilitas, use case).
  
  \item Implementasikan program string search yang menggunakan berbagai addressing mode.
  
  \item \textbf{Refleksi}: Mode addressing mana yang paling sulit dipahami dan mengapa? Bagaimana Anda mengatasi kesulitan tersebut?
\end{latihan}

% ============================================================
% ASESMEN
% ============================================================

\begin{asesmen}
\textbf{Instrumen Penilaian untuk Sub-CPMK 1.2, 2.1, 2.2}

\textbf{A. Pilihan Ganda}

\begin{enumerate}
  \item Mode addressing yang menggunakan nilai konstan dalam instruksi adalah:
  \begin{enumerate}
    \item Immediate
    \item Direct
    \item Register
    \item Indirect
  \end{enumerate}
  
  \item Instruksi untuk mendefinisikan word (16-bit) data adalah:
  \begin{enumerate}
    \item DB
    \item DW
    \item DD
    \item EQU
  \end{enumerate}
  
  \item Register yang tidak dapat digunakan sebagai base address adalah:
  \begin{enumerate}
    \item BX
    \item BP
    \item SI
    \item DI
  \end{enumerate}
  
  \item Instruksi yang tidak mempengaruhi flag register adalah:
  \begin{enumerate}
    \item ADD
    \item SUB
    \item MOV
    \item AND
  \end{enumerate}
\end{enumerate}

\textbf{B. Essay}

\begin{enumerate}
  \item Jelaskan kelebihan dan kekurangan setiap mode addressing! Berikan contoh kasus penggunaan optimal.
  
  \item Mengapa instruksi LEA berbeda dengan MOV dalam hal pengalamatan?
  
  \item Berdasarkan tabel perbandingan addressing modes, pilih mode yang paling tepat untuk: (a) inisialisasi register dengan konstanta 100, (b) akses variabel global \asm{result}, (c) akses array 2D \texttt{array[row][col]}. Jelaskan alasannya!
  
  \item Jelaskan langkah CPU saat mengeksekusi \asm{MOV AX, [BX+SI+2]}! (Asumsikan BX=1000h, SI=0010h, DS=3000h). Hitung alamat efektif dan alamat fisik.
\end{enumerate}

\textbf{C. Practical Challenge}

\begin{enumerate}
  \item Buat program array processing:
  \begin{itemize}
    \item Definisikan array 10 bilangan dengan DW
    \item Hitung rata-rata, minimum, maksimum
    \item Gunakan minimal 4 mode addressing berbeda (termasuk based indexed \asm{[BX+SI]} atau \asm{[BX+SI+offset]})
    \item Tampilkan hasil dengan interupsi DOS
    \item Dokumentasikan setiap addressing mode yang digunakan
  \end{itemize}
\end{enumerate}

\textbf{Rubrik Penilaian}: Lihat Lampiran A
\end{asesmen}

% ============================================================
% CHECKLIST KOMPETENSI
% ============================================================

\begin{checklist}
  \item Saya dapat mengidentifikasi 7 mode addressing berbeda
  \item Saya dapat memilih mode addressing yang tepat untuk kasus tertentu
  \item Saya dapat menggunakan direktif assembler (DB, DW, DD, EQU, ORG)
  \item Saya dapat menganalisis instruksi assembly dan operand-nya
  \item Saya dapat mengoptimasi kode dengan pemilihan addressing mode
  \item Saya dapat menulis program assembly dengan sintaks TASM yang benar
  \item Saya dapat menggunakan TASM debugger untuk trace instruksi
  \item Saya memahami hubungan antara instruksi dan flag register
\end{checklist}

% ============================================================
% RANGKUMAN
% ============================================================

\begin{rangkuman}
Bab ini membahas set instruksi Intel 8086 dan mode addressing, termasuk instruksi data transfer, aritmatika, logika, serta berbagai cara mengakses data dalam memori.

\textbf{Poin Kunci:}
\begin{itemize}
  \item Set instruksi 8086 meliputi data transfer, aritmatika, logika, dan kontrol
  \item Mode addressing menentukan cara instruksi mengakses operand
  \item Immediate addressing paling cepat, based indexed paling fleksibel
  \item Direktif assembler mendefinisikan data dan konstanta program
  \item Pemilihan addressing mode mempengaruhi ukuran kode dan kecepatan
  \item TASM menyediakan tools untuk analisis dan optimasi instruksi
  \item Pemahaman instruksi dan addressing fundamental untuk pemrograman assembly
\end{itemize}

\textbf{Kata Kunci}: \textbf{\texttt{Instruksi}}, \textbf{\texttt{Addressing Mode}}, \textbf{\texttt{MOV}}, \textbf{\texttt{ADD}}, \textbf{\texttt{SUB}}, \textbf{\texttt{AND}}, \textbf{\texttt{OR}}, \textbf{\texttt{DB}}, \textbf{\texttt{DW}}, \textbf{\texttt{DD}}, \textbf{\texttt{TASM}}, \textbf{\texttt{Opcode}}
\end{rangkuman}
