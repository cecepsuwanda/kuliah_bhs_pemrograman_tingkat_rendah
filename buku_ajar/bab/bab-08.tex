\documentclass[../main.tex]{subfiles}
\ifSubfilesClassLoaded{\setcounter{chapter}{7}}{}
\begin{document}

\chapter{Interupsi Sistem dan BIOS/DOS}

\begin{subcpmk}
  \item Sub-CPMK 4.2: Menangani interupsi BIOS dan DOS (INT 10h, INT 21h) untuk input/output dan layanan sistem
  \item Sub-CPMK 2.1: Menulis program assembly sederhana dengan TASM syntax
\end{subcpmk}

% ============================================================
% MATERI POKOK
% ============================================================
% ============================================================
% MATERI POKOK
% ============================================================
\section{Interupsi Sistem}

\subsection{Konsep Interupsi}
Interupsi adalah sinyal yang menghentikan eksekusi program normal untuk menangani event tertentu.

\begin{verbatim}
; Format interupsi
INT interrupt_number    ; Trigger software interrupt
\end{verbatim}

\textbf{Jenis interupsi:}
\begin{itemize}
  \item \textbf{Hardware Interupsi}: Dari perangkat hardware (keyboard, timer, disk)
  \item \textbf{Software Interupsi}: Dipanggil dari program (INT n)
  \item \textbf{Exception}: Interupsi internal CPU (divide by zero, overflow)
\end{itemize}

\subsection{Interrupt Vector Table}
Tabel alamat untuk handler interupsi di memori.

\begin{itemize}
  \item Lokasi: 0000h:0000h - 0000:03FFh (1024 bytes)
  \item 256 entry (0-255), setiap entry 4 bytes (segment:offset)
  \item Entry 0-1Fh: BIOS interrupts
  \item Entry 20h-3Fh: DOS interrupts
  \item Entry 40h-FFh: User-defined interrupts
\end{itemize}

\section{BIOS Interrupts}

\subsection{INT 10h - Video Services}
Layanan untuk operasi video dan display.

\begin{verbatim}
; Set video mode
MOV AH, 00h       ; Function: set video mode
MOV AL, 03h       ; Mode: 80x25 text
INT 10h           ; Call BIOS

; Set cursor position
MOV AH, 02h       ; Function: set cursor
MOV BH, 0         ; Page number
MOV DH, 10        ; Row
MOV DL, 20        ; Column
INT 10h           ; Call BIOS

; Write character
MOV AH, 0Eh       ; Function: write character
MOV AL, 'A'       ; Character
MOV BH, 0         ; Page
MOV BL, 07h       ; Color
INT 10h           ; Call BIOS
\end{verbatim}

\textbf{Fungsi INT 10h utama:}
\begin{itemize}
  \item \textbf{AH=00h}: Set video mode
  \item \textbf{AH=02h}: Set cursor position
  \item \textbf{AH=09h}: Write character/attribute
  \item \textbf{AH=0Eh}: Write character (teletype)
  \item \textbf{AH=13h}: Write string
\end{itemize}

\subsection{INT 13h - Disk Services}
Layanan untuk operasi disk dan floppy.

\begin{verbatim}
; Reset disk system
MOV AH, 00h       ; Function: reset disk
MOV DL, 00h       ; Drive: A:
INT 13h           ; Call BIOS

; Read disk sector
MOV AH, 02h       ; Function: read sector
MOV AL, 01h       ; Number of sectors
MOV CH, 00h       ; Cylinder
MOV CL, 01h       ; Sector
MOV DH, 00h       ; Head
MOV DL, 00h       ; Drive: A:
MOV BX, buffer    ; Buffer address
INT 13h           ; Call BIOS
\end{verbatim}

\subsection{INT 16h - Keyboard Services}
Layanan untuk input keyboard.

\begin{verbatim}
; Get keystroke
MOV AH, 00h       ; Function: get keystroke
INT 16h           ; Call BIOS
; AH = scan code, AL = ASCII code

; Check key status
MOV AH, 01h       ; Function: check key status
INT 16h           ; Call BIOS
; ZF = 0 if key pressed, ZF = 1 if no key
\end{verbatim}

\section{DOS Interrupts}

\subsection{INT 21h - DOS Services}
Layanan sistem operasi DOS untuk file I/O, program control, dan lainnya.

\subsubsection{Character I/O}
\begin{verbatim}
; Display character
MOV AH, 02h       ; Function: display character
MOV DL, 'A'       ; Character
INT 21h           ; Call DOS

; Input character
MOV AH, 01h       ; Function: input character
INT 21h           ; Call DOS
; AL = character input

; Display string
MOV AH, 09h       ; Function: display string
MOV DX, string_addr ; String address (terminated by $)
INT 21h           ; Call DOS

; Input string
MOV AH, 0Ah       ; Function: input string
MOV DX, buffer    ; Buffer address
INT 21h           ; Call DOS
\end{verbatim}

\subsubsection{File Operations}
\begin{verbatim}
; Create file
MOV AH, 3Ch       ; Function: create file
MOV CX, 00h       ; Attributes
MOV DX, filename  ; Filename address
INT 21h           ; Call DOS
; AX = file handle

; Open file
MOV AH, 3Dh       ; Function: open file
MOV AL, 00h       ; Access mode (read)
MOV DX, filename  ; Filename address
INT 21h           ; Call DOS
; AX = file handle

; Read file
MOV AH, 3Fh       ; Function: read file
MOV BX, handle    ; File handle
MOV CX, bytes     ; Number of bytes
MOV DX, buffer    ; Buffer address
INT 21h           ; Call DOS
; AX = bytes read

; Close file
MOV AH, 3Eh       ; Function: close file
MOV BX, handle    ; File handle
INT 21h           ; Call DOS
\end{verbatim}

\subsubsection{Program Control}
\begin{verbatim}
; Terminate program
MOV AH, 4Ch       ; Function: terminate
MOV AL, 00h       ; Return code
INT 21h           ; Call DOS

; Execute program
MOV AH, 4Bh       ; Function: execute
MOV AL, 00h       ; Load and execute
MOV DX, progname  ; Program name
MOV BX, paramblk  ; Parameter block
INT 21h           ; Call DOS
\end{verbatim}

\section{Memory Management}

\subsection{INT 12h - Memory Size}
Mendapatkan ukuran memori sistem.

\begin{verbatim}
; Get memory size
INT 12h           ; Call BIOS
; AX = memory size in KB
\end{verbatim}

\subsection{INT 15h - Extended Memory}
Layanan untuk extended memory dan system information.

\begin{verbatim}
; Get extended memory size
MOV AH, 88h       ; Function: get extended memory
INT 15h           ; Call BIOS
; AX = extended memory size in KB
\end{verbatim}

\section{Time and Date}

\subsection{INT 1Ah - Time Services}
Layanan untuk operasi waktu dan timer.

\begin{verbatim}
; Read system time
INT 1Ah           ; Call BIOS
; CX:DX = timer ticks since midnight

; Set system time
MOV CX, hours     ; Hours
MOV DX, minutes   ; Minutes
MOV AH, 01h       ; Function: set time
INT 1Ah           ; Call BIOS
\end{verbatim}

\subsection{INT 21h - Date Functions}
Fungsi tanggal dalam DOS services.

\begin{verbatim}
; Get system date
MOV AH, 2Ah       ; Function: get date
INT 21h           ; Call DOS
; CX = year, DH = month, DL = day, AL = day of week

; Set system date
MOV AH, 2Bh       ; Function: set date
MOV CX, year      ; Year
MOV DH, month     ; Month
MOV DL, day       ; Day
INT 21h           ; Call DOS
; AL = 00h if success, FFh if invalid
\end{verbatim}

% ============================================================
% AKTIVITAS PEMBELAJARAN
% ============================================================

\begin{aktivitas}
  \item \textbf{Video Programming}: Buat program yang menggunakan INT 10h untuk membuat animasi sederhana.
  
  \item \textbf{File Operations}: Implementasikan text editor dengan file I/O menggunakan INT 21h.
  
  \item \textbf{Keyboard Handler}: Buat program yang menangani keyboard input dengan INT 16h.
  
  \item \textbf{System Information}: Buat program yang menampilkan informasi sistem (memori, waktu, tanggal).
  
  \item \textbf{Interrupt Analysis}: Gunakan TASM debugger untuk trace interrupt execution.
  
  \item \textbf{Custom Interrupt}: Implementasikan user-defined interrupt handler.
\end{aktivitas}

% ============================================================
% LATIHAN DAN REFLEKSI
% ============================================================

\begin{latihan}
  \item Buat program untuk menampilkan jam digital di pojok kanan atas layar.
  
  \item Implementasikan program untuk membaca dan menulis file teks dengan INT 21h.
  
  \item Buat program calculator dengan input keyboard dan output video.
  
  \item Implementasikan program untuk menampilkan informasi sistem (memori, tanggal, waktu).
  
  \item Buat program text-based game dengan keyboard input dan video output.
  
  \item Implementasikan program untuk copy file dengan progress bar.
  
  \item Buat program untuk mengubah atribut file dan direktori.
  
  \item \textbf{Refleksi}: Interrupt mana yang paling sulit dipahami? Bagaimana Anda mengatasi kesulitan tersebut?
\end{latihan}

% ============================================================
% ASESMEN
% ============================================================

\begin{asesmen}
\textbf{Instrumen Penilaian untuk Sub-CPMK 4.2, 2.1}

\textbf{A. Pilihan Ganda}

\begin{enumerate}
  \item Interrupt untuk video services adalah:
  \begin{enumerate}
    \item INT 10h
    \item INT 13h
    \item INT 16h
    \item INT 21h
  \end{enumerate}
  
  \item Fungsi untuk menampilkan karakter dengan INT 21h adalah:
  \begin{enumerate}
    \item AH = 01h
    \item AH = 02h
    \item AH = 09h
    \item AH = 0Ah
  \end{enumerate}
  
  \item Register yang berisi scan code dari INT 16h adalah:
  \begin{enumerate}
    \item AL
    \item AH
    \item AX
    \item DX
  \end{enumerate}
  
  \item Interrupt untuk disk services adalah:
  \begin{enumerate}
    \item INT 10h
    \item INT 12h
    \item INT 13h
    \item INT 15h
  \end{enumerate}
  
  \item Format buffer untuk INT 21h AH=0Ah: byte 1 (setelah max length) diisi oleh DOS dengan:
  \begin{enumerate}
    \item Actual length (jumlah karakter yang dibaca)
    \item Max length
    \item Karakter pertama
    \item Terminator CR
  \end{enumerate}
  
  \item Setelah INT 21h file operations (create/open), cara mengecek apakah operasi berhasil adalah:
  \begin{enumerate}
    \item Cek CF: CF=0 sukses, CF=1 error
    \item Cek ZF
    \item Cek AX saja
    \item Tidak perlu pengecekan
  \end{enumerate}
\end{enumerate}

\textbf{B. Essay}

\begin{enumerate}
  \item Jelaskan perbedaan antara BIOS interrupts dan DOS interrupts! Berikan contoh penggunaan masing-masing.
  
  \item Mengapa interrupt vector table penting dalam sistem operasi?
  
  \item Kapan sebaiknya menggunakan INT 10h vs INT 21h untuk output teks/karakter? Jelaskan skenario penggunaannya!
  
  \item Jelaskan format buffer untuk INT 21h AH=0Ah dan cara mengecek error (CF) setelah operasi file (create, open)!
\end{enumerate}

\textbf{C. Practical Challenge}

\begin{enumerate}
  \item Buat program file manager:
  \begin{itemize}
    \item Display directory listing dengan INT 21h
    \item File operations (create, read, write, delete) dengan error handling (cek CF setelah create/open)
    \item Directory operations (create, remove, change)
    \item File attribute management
    \item Search functionality
    \item User interface dengan menu system
    \item Error handling dan validation
  \end{itemize}
\end{enumerate}

\textbf{Rubrik Penilaian}: Lihat Lampiran A
\end{asesmen}

% ============================================================
% CHECKLIST KOMPETENSI
% ============================================================

\begin{checklist}
  \item Saya dapat menggunakan INT 10h untuk video operations
  \item Saya dapat menggunakan INT 13h untuk disk operations
  \item Saya dapat menggunakan INT 16h untuk keyboard input
  \item Saya dapat menggunakan INT 21h untuk DOS services
  \item Saya dapat melakukan file I/O operations
  \item Saya dapat mengelola memory dengan interrupts
  \item Saya dapat mengakses time dan date services
  \item Saya dapat menganalisis interrupt execution dengan debugger
\end{checklist}

% ============================================================
% RANGKUMAN
% ============================================================

\begin{rangkuman}
Bab ini membahas interupsi sistem dalam assembly 8086, termasuk BIOS interrupts, DOS services, dan system programming.

\textbf{Poin Kunci:}
\begin{itemize}
  \item Interupsi memungkinkan komunikasi dengan hardware dan sistem operasi
  \item BIOS interrupts (INT 10h, 13h, 16h) untuk hardware-level operations
  \item DOS interrupts (INT 21h) untuk sistem operasi services
  \item File I/O, video, keyboard, dan disk operations melalui interrupts
  \item Interrupt vector table mengelola handler addresses
  \item System programming memerlukan pemahaman interrupt mechanisms
  \item Error handling penting untuk interrupt operations
  \item Debugging interrupts membantu troubleshooting system calls
\end{itemize}

\textbf{Kata Kunci}: \asm{Interupsi}, \asm{BIOS}, \asm{DOS}, \asm{INT 10h}, \asm{INT 13h}, \asm{INT 16h}, \asm{INT 21h}, \asm{File I/O}, \asm{Video}, \asm{Keyboard}, \asm{TASM}, \asm{System Programming}
\end{rangkuman}


\ifSubfilesClassLoaded{
  \renewcommand{\bibname}{Daftar Pustaka}
  \bibliographystyle{plain}
  \bibliography{../references}
}{}
\end{document}
  \item \textbf{Software Interrupts}: INT instruction calls
\end{itemize}

\section{DOS Interrupts (INT 21h)}

\subsection{DOS Function Calls}
\begin{verbatim}
MOV AH, function_number
; Set parameters in registers
INT 21h              ; Call DOS function
\end{verbatim}

\subsection{Common DOS Functions}

\subsubsection{Display String (AH=09h)}
\begin{verbatim}
MOV AH, 09h
MOV DX, offset string
INT 21h
string DB 'Hello World$'
\end{verbatim}

\subsubsection{Read Character (AH=01h)}
\begin{verbatim}
MOV AH, 01h
INT 21h              ; Character in AL
\end{verbatim}

\subsubsection{Read String (AH=0Ah)}
\begin{verbatim}
MOV AH, 0Ah
MOV DX, offset buffer
INT 21h
buffer DB 80, ?, 80 DUP(?)
\end{verbatim}

\subsubsection{Exit Program (AH=4Ch)}
\begin{verbatim}
MOV AH, 4Ch
INT 21h              ; Exit to DOS
\end{verbatim}

\section{BIOS Interrupts}

\subsection{Video Services (INT 10h)}

\subsubsection{Set Video Mode (AH=00h)}
\begin{verbatim}
MOV AH, 00h
MOV AL, mode        ; 03h = 80x25 text
INT 10h
\end{verbatim}

\subsubsection{Position Cursor (AH=02h)}
\begin{verbatim}
MOV AH, 02h
MOV BH, 0          ; Page number
MOV DH, row         ; Row (0-24)
MOV DL, col         ; Column (0-79)
INT 10h
\end{verbatim}

\subsubsection{Write Character (AH=09h)}
\begin{verbatim}
MOV AH, 09h
MOV AL, character
MOV BH, 0          ; Page
MOV BL, attribute   ; Color
MOV CX, count       ; Repeat count
INT 10h
\end{verbatim}

\subsection{Keyboard Services (INT 16h)}

\subsubsection{Check for Key (AH=01h)}
\begin{verbatim}
MOV AH, 01h
INT 16h
JZ no_key          ; ZF=1 if no key
; Key available in AX
\end{verbatim}

\subsubsection{Read Key (AH=00h)}
\begin{verbatim}
MOV AH, 00h
INT 16h              ; Key in AX (scan code in AH, ASCII in AL)
\end{verbatim}

\section{File Handling dengan DOS}

\subsection{Create File (AH=3Ch)}
\begin{verbatim}
MOV AH, 3Ch
MOV CX, 0          ; Normal attributes
MOV DX, offset filename
INT 21h
JC error            ; CF=1 if error
MOV handle, AX      ; Save file handle
\end{verbatim}

\subsection{Open File (AH=3Dh)}
\begin{verbatim}
MOV AH, 3Dh
MOV AL, access_mode  ; 0=read, 1=write, 2=both
MOV DX, offset filename
INT 21h
JC error
MOV handle, AX
\end{verbatim}

\subsection{Read File (AH=3Fh)}
\begin{verbatim}
MOV AH, 3Fh
MOV BX, handle
MOV CX, bytes_to_read
MOV DX, offset buffer
INT 21h
JC error
MOV bytes_read, AX
\end{verbatim}

\subsection{Write File (AH=40h)}
\begin{verbatim}
MOV AH, 40h
MOV BX, handle
MOV CX, bytes_to_write
MOV DX, offset buffer
INT 21h
JC error
MOV bytes_written, AX
\end{verbatim}

\subsection{Close File (AH=3Eh)}
\begin{verbatim}
MOV AH, 3Eh
MOV BX, handle
INT 21h
JC error
\end{verbatim}

\section{Memory Management}

\subsection{Allocate Memory (AH=48h)}
\begin{verbatim}
MOV AH, 48h
MOV BX, paragraphs   ; 1 paragraph = 16 bytes
INT 21h
JC error
MOV memory_block, AX
\end{verbatim}

\subsection{Free Memory (AH=49h)}
\begin{verbatim}
MOV AH, 49h
MOV ES, memory_block
INT 21h
JC error
\end{verbatim}

% ============================================================
% AKTIVITAS PEMBELAJARAN
% ============================================================

\begin{aktivitas}
  \item \textbf{Text Editor}: Buat text editor sederhana dengan BIOS dan DOS interrupts.
  
  \item \textbf{File Manager}: Implementasikan program file manager dengan create, read, write, delete operations.
  
  \item \textbf{Screen Manipulation}: Buat program yang menggambar border dan menu menggunakan video BIOS.
  
  \item \textbf{Input Validation}: Buat program dengan keyboard input validation menggunakan BIOS interrupts.
  
  \item \textbf{Memory Display}: Tampilkan memory map dan alokasi memori menggunakan DOS functions.
  
  \item \textbf{System Information}: Buat program yang menampilkan informasi sistem menggunakan BIOS calls.
\end{aktivitas}

% ============================================================
% LATIHAN DAN REFLEKSI
% ============================================================

\begin{latihan}
  \item Jelaskan perbedaan antara DOS interrupts dan BIOS interrupts! Kapan sebaiknya menggunakan masing-masing?
  
  \item Buat program yang:
  \begin{itemize}
    \item Menampilkan menu di layar
    \item Menerima input pilihan dari user
    \item Menjalankan fungsi sesuai pilihan
    \item Menggunakan BIOS untuk display dan DOS untuk I/O
  \end{itemize}
  
  \item Implementasikan program text viewer:
  \begin{itemize}
    \item Buka file teks
    \item Tampilkan isi file dengan paging
    \item Support scroll up/down
    \item Handle file not found error
  \end{itemize}
  
  \item Buat program untuk menghitung karakter dalam file teks.
  
  \item Implementasikan program copy file dengan progress indicator.
  
  \item Buat program yang menampilkan informasi tanggal dan waktu sistem.
  
  \item \textbf{Refleksi}: Interrupt mana yang paling sering Anda gunakan dan mengapa? Apa tantangan dalam error handling?
\end{latihan}

% ============================================================
% ASESMEN
% ============================================================

\begin{asesmen}
\textbf{Instrumen Penilaian untuk Sub-CPMK 6.1, 6.2, 2.1}

\textbf{A. Pilihan Ganda}

\begin{enumerate}
  \item Interrupt untuk display string di DOS adalah:
  \begin{enumerate}
    \item INT 10h
    \item INT 16h
    \item INT 20h
    \item INT 21h
  \end{enumerate}
  
  \item Fungsi untuk membaca karakter dari keyboard di DOS adalah:
  \begin{enumerate}
    \item AH = 01h
    \item AH = 02h
    \item AH = 09h
    \item AH = 4Ch
  \end{enumerate}
  
  \item Interrupt untuk video services adalah:
  \begin{enumerate}
    \item INT 10h
    \item INT 16h
    \item INT 21h
    \item INT 33h
  \end{enumerate}
  
  \item Register yang digunakan untuk function number di DOS interrupts adalah:
  \begin{enumerate}
    \item AX
    \item BX
    \item CX
    \item DX
  \end{enumerate}
\end{enumerate}

\textbf{B. Essay}

\begin{enumerate}
  \item Jelaskan langkah-langkah yang terjadi saat software interrupt dieksekusi!
  
  \item Mengapa BIOS interrupts lebih cepat untuk operasi video dibandingkan DOS interrupts?
\end{enumerate}

\textbf{C. Practical Challenge}

\begin{enumerate}
  \item Buat program notepad sederhana:
  \begin{itemize}
    \item Create new file
    \item Open existing file
    \item Edit text dengan insert/delete
    \item Save file
    \item Basic menu interface
    \item Error handling untuk semua operasi
  \end{itemize}
\end{enumerate}

\textbf{Rubrik Penilaian}: Lihat Lampiran A
\end{asesmen}

% ============================================================
% CHECKLIST KOMPETENSI
% ============================================================

\begin{checklist}
  \item Saya dapat menggunakan DOS interrupts untuk I/O operations
  \item Saya dapat menggunakan BIOS interrupts untuk video dan keyboard
  \item Saya dapat mengimplementasikan file handling dengan DOS
  \item Saya dapat mengelola memory allocation
  \item Saya dapat membuat program dengan user interface
  \item Saya dapat handle error conditions dengan interrupts
  \item Saya dapat memilih interrupt yang tepat untuk kasus tertentu
  \item Saya dapat mengoptimasi program dengan interrupt calls
\end{checklist}

% ============================================================
% RANGKUMAN
% ============================================================

\begin{rangkuman}
Bab ini membahas interupsi sistem dan fungsi BIOS/DOS untuk interaksi dengan hardware dan sistem operasi.

\textbf{Poin Kunci:}
\begin{itemize}
  \item Interrupts menyediakan antarmuka ke sistem operasi dan hardware
  \item DOS interrupts (INT 21h) untuk file I/O dan program control
  \item BIOS interrupts (INT 10h, 16h) untuk video dan keyboard langsung
  \item Function number ditentukan di register AH
  \item Parameter dan hasil dikirim melalui register tertentu
  \item Error handling dengan checking carry flag (CF)
  \item File handling memerlukan manajemen file handle
  \item Memory allocation untuk dynamic memory management
\end{itemize}

\textbf{Kata Kunci}: \textbf{\texttt{Interrupt}}, \textbf{\texttt{DOS}}, \textbf{\texttt{BIOS}}, \textbf{\texttt{INT 21h}}, \textbf{\texttt{INT 10h}}, \textbf{\texttt{File Handling}}, \textbf{\texttt{Memory Management}}, \textbf{\texttt{System Calls}}, \textbf{\texttt{TASM}}
\end{rangkuman}

\ifSubfilesClassLoaded{
  \renewcommand{\bibname}{Daftar Pustaka}
  \bibliographystyle{plain}
  \bibliography{../references}
}{}
\end{document}
