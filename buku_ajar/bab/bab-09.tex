\documentclass[../main.tex]{subfiles}
\ifSubfilesClassLoaded{\setcounter{chapter}{8}}{}
\begin{document}

\chapter{Optimasi dan Debugging}

\begin{subcpmk}
  \item Mendukung CPMK-2 dan CPMK-4: Mengoptimasi kode assembly dan menggunakan TASM debugger untuk troubleshooting — Materi Minggu 15-16
  \item Sub-CPMK 2.1: Menulis program assembly sederhana dengan TASM syntax
\end{subcpmk}

% ============================================================
% MATERI POKOK
% ============================================================
% ============================================================
% MATERI POKOK
% ============================================================
\section{Optimasi Kode Assembly}

\subsection{Konsep Optimasi}
Optimasi adalah proses meningkatkan performa kode assembly tanpa mengubah fungsionalitas.

\textbf{Tujuan optimasi:}
\begin{itemize}
  \item Mengurangi ukuran kode
  \item Meningkatkan kecepatan eksekusi
  \item Mengurangi penggunaan memori
  \item Meningkatkan efisiensi register usage
\end{itemize}

\subsection{Strength Reduction}
Mengganti operasi mahal dengan operasi yang lebih murah.

\begin{verbatim}
; Before: expensive multiplication
MOV AX, 10
MUL BX          ; AX = AX * BX

; After: strength reduction with shifts
MOV AX, BX
SHL AX, 1       ; AX = BX * 2
SHL AX, 2       ; AX = BX * 8
ADD AX, BX      ; AX = BX * 10
\end{verbatim}

\textbf{Contoh strength reduction:}
\begin{itemize}
  \item MUL n $\rightarrow$ SHIFT + ADD (untuk n = $2^k \pm 1$)
  \item DIV n $\rightarrow$ SHIFT (untuk n = $2^k$)
  \item MUL 2 → SHL 1
  \item DIV 2 → SHR 1
\end{itemize}

\subsection{Loop Unrolling}
Mengurangi overhead loop dengan menggabungkan beberapa iterasi.

\begin{verbatim}
; Before: normal loop
MOV CX, 100
loop_start:
    MOV [SI], AL
    INC SI
    LOOP loop_start

; After: loop unrolling (4x)
MOV CX, 25
loop_start:
    MOV [SI], AL
    MOV [SI+1], AL
    MOV [SI+2], AL
    MOV [SI+3], AL
    ADD SI, 4
    LOOP loop_start
\end{verbatim}

\textbf{Keuntungan loop unrolling:}
\begin{itemize}
  \item Mengurangi overhead LOOP instruction
  \item Meningkatkan instruction-level parallelism
  \item Mengurangi branch misprediction
\end{itemize}

\subsection{Register Optimization}
Mengoptimalkan penggunaan register untuk akses memori minimal.

\begin{verbatim}
; Before: frequent memory access
MOV AX, [array+0]
ADD AX, [array+2]
ADD AX, [array+4]
ADD AX, [array+6]

; After: register optimization
MOV SI, array
MOV AX, [SI]
ADD AX, [SI+2]
ADD AX, [SI+4]
ADD AX, [SI+6]
\end{verbatim}

\textbf{Strategi register optimization:}
\begin{itemize}
  \item Gunakan register untuk frequently accessed variables
  \item Minimalkan register spills ke memori
  \item Pertimbangkan register allocation
  \item Gunakan register addressing modes
\end{itemize}

\section{Debugging Techniques}

\subsection{GUI Turbo Assembler (GTASM) dan TASM Debugger}

GUI Turbo Assembler (GTASM) adalah IDE yang mengintegrasikan TASM, TLINK, Turbo Debugger (TD), dan DOSBox \cite{jones2020}. Cocok untuk pembelajaran assembly pada Windows modern.

\begin{figure}[H]
\centering
\includegraphics[width=0.85\textwidth]{gtasm_interface.png}
\caption{Antarmuka GUI Turbo Assembler}
\end{figure}

\begin{figure}[H]
\centering
\includegraphics[width=0.8\textwidth]{gtasm_debugger.png}
\caption{Turbo Debugger -- Register dan Memory Inspection}
\end{figure}

\subsection{TASM Debugger Overview}
TASM menyediakan debugger untuk analisis program assembly.

\begin{verbatim}
; Compile dengan debug information
TASM /zi program.asm
TLINK /v program.obj

; Run debugger
TD program.exe
\end{verbatim}

\textbf{Fitur TASM debugger:}
\begin{itemize}
  \item Breakpoint setting
  \item Step-by-step execution
  \item Register dan memory inspection
  \item Call stack tracing
  \item Variable watching
\end{itemize}

\subsection{Breakpoint Debugging}
Menghentikan eksekusi pada titik tertentu untuk analisis.

\begin{verbatim}
; Set breakpoint di TASM debugger
BP address        ; Set breakpoint
BC number         ; Clear breakpoint
BL                ; List breakpoints
G [address]       ; Go/run to breakpoint
\end{verbatim}

\textbf{Strategi breakpoint:}
\begin{itemize}
  \item Set breakpoint pada suspicious code
  \item Gunakan conditional breakpoints
  \item Monitor register changes
  \item Trace program flow
\end{itemize}

\subsection{Common Assembly Bugs}

\subsubsection{Register Corruption}
\begin{verbatim}
; Bug: register tidak dipreserve
procedure PROC
    MOV AX, data    ; Mengubah AX tanpa menyimpan
    ; ... proses
    RET             ; AX berubah tanpa sepengetahuan caller
procedure ENDP

; Fix: preserve register
procedure PROC
    PUSH AX         ; Simpan AX
    MOV AX, data    ; Gunakan AX
    ; ... proses
    POP AX          ; Restore AX
    RET
procedure ENDP
\end{verbatim}

\subsubsection{Stack Overflow}
\begin{verbatim}
; Bug: rekursi tanpa base case
factorial PROC
    PUSH BP
    MOV BP, SP
    MOV AX, [BP+4]
    ; Tidak ada base case!
    DEC AX
    PUSH AX
    CALL factorial
    ADD SP, 2
    ; ... stack overflow!
factorial ENDP

; Fix: tambah base case
factorial PROC
    PUSH BP
    MOV BP, SP
    MOV AX, [BP+4]
    CMP AX, 1
    JLE base_case
    DEC AX
    PUSH AX
    CALL factorial
    ADD SP, 2
    JMP end_fact
base_case:
    MOV AX, 1
end_fact:
    POP BP
    RET
factorial ENDP
\end{verbatim}

\subsubsection{Off-by-One Errors}
\begin{verbatim}
; Bug: off-by-one dalam loop
MOV CX, 10        ; Count = 10
MOV SI, 0
loop_start:
    MOV [SI], AL   ; Akses array[0] sampai array[9]
    INC SI
    LOOP loop_start ; 10 iterasi, benar

; Bug: salah count
MOV CX, 11        ; Count = 11
; ... akan akses array[10] (out of bounds)
\end{verbatim}

\section{Performance Analysis}

\subsection{Instruction Timing}
Menganalisis waktu eksekusi instruksi.

\begin{verbatim}
; Instruction cycles (8086)
ADD reg, reg     : 2-3 cycles
ADD mem, reg     : 16-24 cycles
MUL reg         : 70-77 cycles
DIV reg         : 80-90 cycles
JMP label       : 15 cycles
Jcondition label: 4-16 cycles
LOOP label      : 17 cycles
\end{verbatim}

\textbf{Optimasi berdasarkan timing:}
\begin{itemize}
  \item Gunakan register operations daripada memory
  \item Minimalkan jumps dalam tight loops
  \item Gunakan instruksi yang lebih cepat
  \item Pertimbangkan instruction pipelining
\end{itemize}

\subsection{Memory Access Patterns}
Mengoptimasi pola akses memori untuk cache efficiency.

\begin{verbatim}
; Bad: random access
MOV AX, [random_addr1]
MOV BX, [random_addr2]
MOV CX, [random_addr3]

; Good: sequential access
MOV SI, array_start
MOV AX, [SI]
MOV BX, [SI+2]
MOV CX, [SI+4]
\end{verbatim}

\textbf{Prinsip cache-friendly access:}
\begin{itemize}
  \item Sequential access lebih baik daripada random
  \item Localitas spasial (spatial locality)
  \item Localitas temporal (temporal locality)
  \item Minimalkan cache misses
\end{itemize}

\section{Advanced Optimization}

\subsection{Code Alignment}
Mengalign kode untuk optimal fetch.

\begin{verbatim}
; Align code ke word boundaries
ALIGN 2          ; Align ke word boundary
procedure PROC
    ; ... kode prosedur
procedure ENDP
\end{verbatim}

\subsection{Instruction Scheduling}
Mengatur instruksi untuk optimal pipeline utilization.

\begin{verbatim}
; Before: dependent instructions
MOV AX, [mem]
ADD AX, BX       ; Tunggu AX dari memori

; After: independent instructions
MOV CX, [mem2]   ; Independent dari AX
MOV AX, [mem]    ; Load AX
ADD AX, BX       ; Gunakan AX
\end{verbatim}

% ============================================================
% AKTIVITAS PEMBELAJARAN
% ============================================================

\begin{aktivitas}
  \item \textbf{Performance Benchmark}: Bandingkan performa kode sebelum dan sesudah optimasi.
  
  \item \textbf{Debug Session}: Gunakan TASM debugger untuk menemukan dan memperbaiki bug dalam program assembly.
  
  \item \textbf{Optimization Challenge}: Optimasi program sorting untuk kecepatan maksimal.
  
  \item \textbf{Bug Hunt}: Identifikasi dan perbaiki common assembly bugs dalam kode yang diberikan.
  
  \item \textbf{Instruction Timing}: Ukur dan analisis waktu eksekusi berbagai instruksi assembly.
  
  \item \textbf{Memory Analysis}: Analisis pola akses memori untuk cache optimization.
\end{aktivitas}

% ============================================================
% LATIHAN DAN REFLEKSI
% ============================================================

\begin{latihan}
  \item Optimasi program bubble sort dengan loop unrolling dan strength reduction.
  
  \item Debug program rekursif yang mengalami stack overflow.
  
  \item Implementasikan register allocation algorithm untuk program assembly.
  
  \item Buat performance profiler untuk mengukur waktu eksekusi instruksi.
  
  \item Identifikasi dan perbaiki register corruption bug dalam prosedur kompleks.
  
  \item Optimasi program matrix multiplication dengan cache-friendly access patterns.
  
  \item Implementasikan instruction scheduling untuk pipeline optimization.
  
  \item \textbf{Refleksi}: Teknik optimasi mana yang paling sulit dipahami? Bagaimana Anda mengatasi kesulitan tersebut?
\end{latihan}

% ============================================================
% ASESMEN
% ============================================================

\begin{asesmen}
\textbf{Instrumen Penilaian untuk Optimasi dan Debugging (Materi Minggu 15-16), Sub-CPMK 2.1}

\textbf{A. Pilihan Ganda}

\begin{enumerate}
  \item Teknik optimasi yang mengganti MUL dengan SHIFT adalah:
  \begin{enumerate}
    \item Loop unrolling
    \item Strength reduction
    \item Register optimization
    \item Code alignment
  \end{enumerate}
  
  \item Instruksi assembly tercepat untuk operasi aritmatika adalah:
  \begin{enumerate}
    \item MUL
    \item DIV
    \item ADD
    \item SUB
  \end{enumerate}
  
  \item Bug yang paling umum dalam assembly programming adalah:
  \begin{enumerate}
    \item Stack overflow
    \item Register corruption
    \item Off-by-one error
    \item Semua jawaban benar
  \end{enumerate}
  
  \item Untuk debugging assembly dengan TASM, perintah untuk set breakpoint adalah:
  \begin{enumerate}
    \item BP
    \item BC
    \item BL
    \item G
  \end{enumerate}
\end{enumerate}

\textbf{B. Essay}

\begin{enumerate}
  \item Jelaskan perbedaan antara loop unrolling dan strength reduction! Kapan sebaiknya menggunakan masing-masing?
  
  \item Mengapa register preservation penting dalam prosedur assembly?
\end{enumerate}

\textbf{C. Practical Challenge}

\begin{enumerate}
  \item Buat program high-performance calculator:
  \begin{itemize}
    \item Implementasi dengan optimasi maksimal
    \item Gunakan strength reduction untuk operasi matematika
    \item Apply loop unrolling untuk perulangan
    \item Optimasi register usage
    \item Implementasi cache-friendly memory access
    \item Debug dan test dengan TASM debugger
    \item Performance benchmarking dan analysis
    \item Documentation optimasi techniques yang digunakan
  \end{itemize}
\end{enumerate}

\textbf{Rubrik Penilaian}: Lihat Lampiran A
\end{asesmen}

% ============================================================
% CHECKLIST KOMPETENSI
% ============================================================

\begin{checklist}
  \item Saya dapat mengidentifikasi kesempatan optimasi dalam kode assembly
  \item Saya dapat menerapkan strength reduction untuk operasi mahal
  \item Saya dapat mengimplementasikan loop unrolling
  \item Saya dapat mengoptimalkan penggunaan register
  \item Saya dapat menggunakan TASM debugger untuk debugging
  \item Saya dapat mengidentifikasi dan memperbaiki common assembly bugs
  \item Saya dapat menganalisis performa kode assembly
  \item Saya dapat menerapkan advanced optimization techniques
\end{checklist}

% ============================================================
% RANGKUMAN
% ============================================================

\begin{rangkuman}
Bab ini membahas optimasi dan debugging dalam assembly 8086, termasuk teknik optimasi, debugging tools, dan performance analysis.

\textbf{Poin Kunci:}
\begin{itemize}
  \item Optimasi meningkatkan performa tanpa mengubah fungsionalitas
  \item Strength reduction mengganti operasi mahal dengan yang lebih murah
  \item Loop unrolling mengurangi overhead loop
  \item Register optimization meminimalkan akses memori
  \item TASM debugger menyediakan tools debugging yang komprehensif
  \item Common assembly bugs meliputi register corruption dan stack overflow
  \item Performance analysis membantu identifikasi bottleneck
  \item Advanced optimization mencakup code alignment dan instruction scheduling
\end{itemize}

\textbf{Kata Kunci}: \asm{Optimasi}, \asm{Debugging}, \asm{TASM}, \asm{Strength Reduction}, \asm{Loop Unrolling}, \asm{Register Optimization}, \asm{Performance Analysis}, \asm{Breakpoint}, \asm{Instruction Timing}
\end{rangkuman}


\ifSubfilesClassLoaded{
  \renewcommand{\bibname}{Daftar Pustaka}
  \bibliographystyle{plain}
  \bibliography{../references}
}{}
\end{document}
\begin{verbatim}
; Slow version
MOV AX, 1
ADD AX, BX
ADD AX, CX
ADD AX, DX

; Optimized version
LEA AX, [BX+1]  ; If possible
ADD AX, CX
ADD AX, DX
\end{verbatim}

\subsection{Loop Optimization}
\begin{verbatim}
; Unoptimized loop
MOV CX, 100
loop_start:
; 5 instructions
DEC CX
JNZ loop_start

; Optimized loop
MOV CX, 25
loop_start:
; 20 instructions (unrolled)
DEC CX
JNZ loop_start
\end{verbatim}

\subsection{Memory Access Optimization}
\begin{verbatim}
; Slow - multiple memory accesses
MOV AX, [array]
ADD AX, [array+2]
ADD AX, [array+4]

; Optimized - single access with pointer
MOV SI, offset array
MOV AX, [SI]
ADD AX, [SI+2]
ADD AX, [SI+4]
\end{verbatim}

\section{TASM Debugger}

\subsection{Starting Debugger}
\begin{verbatim}
tasm program.asm
tlink program.obj
td program.exe
\end{verbatim}

\subsection{Debugger Commands}
\begin{itemize}
  \item \textbf{F7}: Trace into (step by step)
  \item \textbf{F8}: Step over (skip procedures)
  \item \textbf{F4}: Run to cursor
  \item \textbf{F9}: Run program
  \item \textbf{Alt+V}: View registers
  \item \textbf{Alt+M}: View memory
  \item \textbf{Alt+W}: Watch variables
\end{itemize}

\subsection{Breakpoints}
\begin{verbatim}
; Set breakpoint at label
; Press F2 at code line
; Or use Ctrl+B to set by address
\end{verbatim}

\section{Debugging Techniques}

\subsection{Register Analysis}
\begin{verbatim}
; Check register values
; Use Alt+V to view all registers
; Look for unexpected values
\end{verbatim}

\subsection{Memory Inspection}
\begin{verbatim}
; View memory content
; Use Alt+M to open memory window
; Enter address to examine
; Look for corruption
\end{verbatim}

\subsection{Step-by-Step Execution}
\begin{verbatim}
; Use F7 for detailed tracing
; Watch register changes
; Identify logic errors
\end{verbatim}

\section{Common Bugs and Solutions}

\subsection{Off-by-One Errors}
\begin{verbatim}
; Bug
MOV CX, array_size
loop_start:
MOV AL, [SI]    ; Access array[SI]
INC SI
DEC CX
JNZ loop_start    ; One extra iteration

; Fix
MOV CX, array_size
DEC CX           ; Adjust count
loop_start:
MOV AL, [SI]
INC SI
DEC CX
JNZ loop_start
\end{verbatim}

\subsection{Stack Corruption}
\begin{verbatim}
; Bug - unbalanced stack
PUSH AX
PUSH BX
; Missing POP for AX
POP BX
RET

; Fix - balanced stack
PUSH AX
PUSH BX
POP BX
POP AX
RET
\end{verbatim}

\subsection{Memory Alignment Issues}
\begin{verbatim}
; Bug - unaligned access
MOV AX, [odd_address]  ; Slow on some systems

; Fix - aligned access
MOV AX, [aligned_address]
\end{verbatim}

\section{Performance Measurement}

\subsection{Timing Analysis}
\begin{verbatim}
; Use BIOS timer for timing
MOV AH, 00h     ; Get system timer
INT 1Ah
MOV start_time, DX

; Code to measure
; ...

MOV AH, 00h
INT 1Ah
MOV end_time, DX

; Calculate difference
SUB end_time, start_time
\end{verbatim}

\subsection{Code Profiling}
\begin{itemize}
  \item Identifikasi hotspots
  \item Ukur execution time
  \item Prioritaskan optimasi
\end{itemize}

\section{Advanced Optimization}

\subsection{Lookup Tables}
\begin{verbatim}
; Instead of calculation
MOV AL, input
SHL AL, 1        ; Multiply by 2
MOV BX, offset table
XLAT               ; AL = [BX+AL]

table DB result0, result1, result2, ...
\end{verbatim}

\subsection{Bit Manipulation}
\begin{verbatim}
; Instead of division by 2
SHR AX, 1

; Instead of multiplication by 2
SHL AX, 1

; Instead of modulo power of 2
AND AX, 3        ; Modulo 8
\end{verbatim}

% ============================================================
% AKTIVITAS PEMBELAJARAN
% ============================================================

\begin{aktivitas}
  \item \textbf{Code Optimization}: Ambil program yang ada dan optimasi untuk kecepatan.
  
  \item \textbf{Debug Session}: Gunakan TASM debugger untuk menemukan bug dalam program yang disediakan.
  
  \item \textbf{Performance Analysis}: Ukur waktu eksekusi berbagai algoritma sorting.
  
  \item \textbf{Bug Hunt}: Cari dan perbaiki 5 common bugs dalam kode assembly.
  
  \item \textbf{Optimization Challenge}: Optimasi loop dengan berbagai teknik.
  
  \item \textbf{Memory Analysis}: Gunakan debugger untuk menganalisis penggunaan memori.
\end{aktivitas}

% ============================================================
% LATIHAN DAN REFLEKSI
% ============================================================

\begin{latihan}
  \item Jelaskan perbedaan antara trace into (F7) dan step over (F8)! Kapan sebaiknya menggunakan masing-masing?
  
  \item Optimasi program berikut:
  \begin{verbatim}
MOV CX, 1000
loop_start:
MOV AX, [array]
ADD AX, [array+2]
MOV [result], AX
ADD SI, 4
DEC CX
JNZ loop_start
  \end{verbatim}
  
  \item Debug program yang memiliki bug:
  \begin{itemize}
    \item Infinite loop
    \item Stack overflow
    \item Memory corruption
    \item Wrong calculation result
  \end{itemize}
  
  \item Implementasikan program untuk menghitung execution time fungsi.
  
  \item Buat lookup table untuk konversi ASCII ke EBCDIC.
  
  \item Optimasi bubble sort dengan berbagai teknik optimasi.
  
  \item \textbf{Refleksi}: Teknik debugging mana yang paling efektif untuk Anda? Bagaimana Anda mengembangkan debugging skills?
\end{latihan}

% ============================================================
% ASESMEN
% ============================================================

\begin{asesmen}
\textbf{Instrumen Penilaian untuk Sub-CPMK 7.1, 7.2, 2.1}

\textbf{A. Pilihan Ganda}

\begin{enumerate}
  \item Shortcut key untuk trace into di TASM debugger adalah:
  \begin{enumerate}
    \item F5
    \item F7
    \item F8
    \item F9
  \end{enumerate}
  
  \item Untuk mengoptimasi perkalian dengan 2, gunakan:
  \begin{enumerate}
    \item MUL
    \item ADD
    \item SHL
    \item SHR
  \end{enumerate}
  
  \item Teknik untuk mengurangi loop overhead adalah:
  \begin{enumerate}
    \item Loop unrolling
    \item Register caching
    \item Memory alignment
    \item Lookup table
  \end{enumerate}
  
  \item Untuk melihat register values di debugger adalah:
  \begin{enumerate}
    \item Alt+V
    \item Alt+M
    \item Alt+W
    \item Alt+B
  \end{enumerate}
\end{enumerate}

\textbf{B. Essay}

\begin{enumerate}
  \item Jelaskan strategi optimasi yang dapat diterapkan pada program assembly!
  
  \item Mengapa debugging lebih sulit di assembly dibandingkan high-level language?
\end{enumerate}

\textbf{C. Practical Challenge}

\begin{enumerate}
  \item Optimasi dan debug program sorting:
  \begin{itemize}
    \item Implementasikan 3 algoritma sorting (bubble, quick, merge)
    \item Ukur performance masing-masing
    \item Optimasi algoritma tercepat
    \item Debug semua bugs
    \item Dokumentasikan teknik optimasi yang digunakan
  \end{itemize}
\end{enumerate}

\textbf{Rubrik Penilaian}: Lihat Lampiran A
\end{asesmen}

% ============================================================
% CHECKLIST KOMPETENSI
% ============================================================

\begin{checklist}
  \item Saya dapat mengoptimasi kode assembly untuk performa
  \item Saya dapat menggunakan TASM debugger secara efektif
  \item Saya dapat mengidentifikasi dan memperbaiki common bugs
  \item Saya dapat menggunakan breakpoints dan watch variables
  \item Saya dapat menganalisis performance kode
  \item Saya dapat menerapkan teknik optimasi lanjutan
  \item Saya dapat melakukan step-by-step debugging
  \item Saya dapat mengukur execution time program
\end{checklist}

% ============================================================
% RANGKUMAN
% ============================================================

\begin{rangkuman}
Bab ini membahas teknik optimasi dan debugging untuk program assembly, termasuk penggunaan TASM debugger dan strategi performa.

\textbf{Poin Kunci:}
\begin{itemize}
  \item Optimasi meliputi register usage, instruction selection, dan loop efficiency
  \item TASM debugger menyediakan tools untuk debugging yang powerful
  \item Step-by-step execution membantu identifikasi logic errors
  \item Breakpoints memungkinkan analisis kode spesifik
  \item Common bugs meliputi off-by-one, stack corruption, memory issues
  \item Performance measurement penting untuk optimasi yang efektif
  \item Advanced optimization menggunakan lookup tables dan bit manipulation
  \item Debugging skills essential untuk assembly programming
\end{itemize}

\textbf{Kata Kunci}: \textbf{\texttt{Optimasi}}, \textbf{\texttt{Debugging}}, \textbf{\texttt{TASM Debugger}}, \textbf{\texttt{Performance}}, \textbf{\texttt{Breakpoint}}, \textbf{\texttt{Register Optimization}}, \textbf{\texttt{Loop Unrolling}}, \textbf{\texttt{Memory Alignment}}, \textbf{\texttt{Profiling}}
\end{rangkuman}

\ifSubfilesClassLoaded{
  \renewcommand{\bibname}{Daftar Pustaka}
  \bibliographystyle{plain}
  \bibliography{../references}
}{}
\end{document}
