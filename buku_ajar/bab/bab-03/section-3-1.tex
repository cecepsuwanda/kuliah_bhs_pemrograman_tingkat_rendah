\section{Dasar Bahasa Assembly}

Assembly language adalah bahasa pemrograman tingkat rendah yang paling dekat dengan bahasa mesin. Intel 8086 assembly menggunakan mnemonics yang mudah dibaca untuk mewakili instruksi biner.

\subsection{Struktur Program Assembly}

Program assembly terdiri dari beberapa segmen utama. Direktif \asm{.MODEL SMALL} (TASM/MASM) menentukan model memori: satu segmen kode ($\leq$64 KB), satu segmen data ($\leq$64 KB), satu segmen stack. Assembler kemudian mengatur \asm{ASSUME} dan ukuran segment secara otomatis.

\begin{verbatim}
; Struktur dasar program assembly
; ======================

; Data segment - untuk variabel dan konstanta
DATA SEGMENT
    ; Variabel dan konstanta di sini
DATA ENDS

; Code segment - untuk instruksi program
CODE SEGMENT
    ; Instruksi program di sini
CODE ENDS

; Stack segment - untuk stack operations
STACK SEGMENT
    ; Stack buffer di sini
STACK ENDS

END
\end{verbatim}

\subsection{Program .COM vs .EXE}

\textbf{.COM}: Format sederhana tanpa header, maksimal $\sim$64 KB. Semua segmen (CS, DS, ES, SS) umumnya sama. Entry point di offset \asm{0100h} (setelah PSP). Gunakan \asm{ORG 100h}. Cocok untuk program kecil.

\textbf{.EXE}: Ada header, segmen terpisah untuk kode, data, stack. Gunakan \asm{.MODEL SMALL}, \asm{.DATA}, \asm{.CODE}, \asm{ASSUME}. Inisialisasi \asm{DS} wajib: \asm{MOV AX, @data} / \asm{MOV DS, AX}. Cocok untuk program besar dengan banyak variabel.

\subsection{Assembler Directives}

Directives adalah instruksi untuk assembler, bukan untuk processor:

\begin{verbatim}
; Data definition directives
DB  'A'           ; Define byte (8-bit)
DW  1234h         ; Define word (16-bit)
DD  12345678h      ; Define double word (32-bit)
EQU  MAX_SIZE 100   ; Define constant
ORG  100h          ; Set origin address
\end{verbatim}

\textbf{Common Directives:}
\begin{itemize}
  \item \textbf{DB}: Define byte (8-bit)
  \item \textbf{DW}: Define word (16-bit)
  \item \textbf{DD}: Define double word (32-bit)
  \item \textbf{EQU}: Equate constant
  \item \textbf{ORG}: Set origin address
  \item \textbf{SEGMENT}: Define segment
  \item \textbf{END}: End of program
\end{itemize}

\subsection{Syntax Assembly}

\textbf{Format Instruksi:}
\begin{verbatim}
[label:] mnemonic [operand1[, operand2[, operand3]]
\end{verbatim}

\textbf{Contoh Penggunaan:}
\begin{verbatim}
        MOV AX, BX     ; Transfer BX ke AX
        ADD CX, 5      ; Tambah 5 ke CX
        JMP loop_start  ; Jump ke loop_start
        CMP AX, 0      ; Bandingkan AX dengan 0
        JZ  zero_found  ; Jump jika zero
\end{verbatim}

\subsection{Label dan Organisasi Kode}

Label digunakan untuk menandai lokasi dalam kode sehingga dapat direferensi oleh instruksi jump dan call:

\begin{verbatim}
; Contoh penggunaan label
start:
    MOV AX, 0
    JMP loop_start

loop_start:
    INC AX
    CMP AX, 10
    JL loop_start
    RET
\end{verbatim}

\begin{table}[htbp]
\centering
\begin{tabular}{|p{3cm}|p{5cm}|p{5cm}|}
\hline
\textbf{Aspek} & \textbf{Label} & \textbf{Address} \\
\hline
Definisi & Penanda lokasi dalam kode & Alamat memori fisik \\
\hline
Penggunaan & Target jump, call, referensi data & Diperoleh saat linking \\
\hline
Scope & Lokal atau global & Bergantung segment \\
\hline
Format & Nama diikuti titik dua (:) & Segment:Offset (hex) \\
\hline
\end{tabular}
\caption{Hubungan Label dan Address}
\end{table}
