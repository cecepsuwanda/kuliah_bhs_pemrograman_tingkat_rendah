\section{Contoh Program Sederhana}

Program assembly sederhana untuk memahami konsep dasar pemrograman Intel 8086.

\subsection{Program Hello World}

\begin{verbatim}
; Hello World Program
; =================
DATA SEGMENT
    message DB 'Hello, World!', 0Dh, 0Ah, '$'
DATA ENDS

CODE SEGMENT
    ASSUME CS:CODE, DS:DATA

start:
    ; Initialize DS
    MOV AX, DATA
    MOV DS, AX
    
    ; Display message
    MOV AH, 09h        ; DOS print string function
    MOV DX, OFFSET message
    INT 21h
    
    ; Exit program
    MOV AH, 4Ch        ; DOS exit function
    INT 21h

CODE ENDS
END start
\end{verbatim}

\subsection{Program Kalkulator Sederhana}

\begin{verbatim}
; Simple Calculator
; ================
DATA SEGMENT
    num1 DW 10
    num2 DW 20
    result DW 0
    msg_result DB 'Result: ', '$'
DATA ENDS

CODE SEGMENT
    ASSUME CS:CODE, DS:DATA

start:
    ; Initialize DS
    MOV AX, DATA
    MOV DS, AX
    
    ; Load numbers
    MOV AX, num1
    ADD AX, num2        ; AX = num1 + num2
    MOV result, AX
    
    ; Display result message
    MOV AH, 09h
    MOV DX, OFFSET msg_result
    INT 21h
    
    ; Display result (simplified)
    MOV AX, result
    ; Add code to display AX value
    
    ; Exit program
    MOV AH, 4Ch
    INT 21h

CODE ENDS
END start
\end{verbatim}

\subsection{Program Loop Counter}

\begin{verbatim}
; Loop Counter Program
; ===================
DATA SEGMENT
    counter DW 0
    max_count DW 10
    msg_count DB 'Count: ', '$'
DATA ENDS

CODE SEGMENT
    ASSUME CS:CODE, DS:DATA

start:
    ; Initialize DS
    MOV AX, DATA
    MOV DS, AX
    
    ; Initialize counter
    MOV CX, max_count
    MOV counter, 0

count_loop:
    ; Increment counter
    INC counter
    
    ; Display count (simplified)
    MOV AH, 09h
    MOV DX, OFFSET msg_count
    INT 21h
    
    ; Loop until CX = 0
    LOOP count_loop
    
    ; Exit program
    MOV AH, 4Ch
    INT 21h

CODE ENDS
END start
\end{verbatim}
