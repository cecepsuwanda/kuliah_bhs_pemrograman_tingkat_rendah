\section{Tipe Data dan Direktif}

\subsection{Tipe Data Dasar}

Intel 8086 mendukung berbagai tipe data untuk operasi yang berbeda:

\textbf{Integer Types:}
\begin{itemize}
  \item \textbf{Byte}: 8-bit unsigned (0-255)
  \item \textbf{Word}: 16-bit unsigned (0-65535)
  \item \textbf{Double Word}: 32-bit (0-4294967295)
\end{itemize}

\textbf{String Types:}
\begin{itemize}
  \item \textbf{ASCII String}: Array of bytes terminated dengan null (0)
  \item \textbf{Pascal String}: String dengan length prefix
\end{itemize}

\subsection{Data Definition Directives}

\begin{verbatim}
; Integer data definition
byte_var   DB  10h      ; 8-bit value
word_var   DW  1234h     ; 16-bit value
dword_var  DD  12345678h ; 32-bit value

; String definition
str_hello  DB 'Hello World', 0
str_name   DB 'John Doe', 0
str_age   DB 25, 0

; Array definition
numbers   DW 10, 20, 30, 40, 50
matrix    DW 1, 2, 3, 4
buffer    DB 256 DUP(0)  ; 256 bytes buffer
\end{verbatim}

\subsection{Constant Definition}

\begin{verbatim}
; Constant definitions
MAX_SIZE EQU 100
PI       EQU 3.14159
NEW_LINE  EQU 0Dh, 0Ah
TRUE     EQU 1
FALSE    EQU 0
\end{verbatim}

\subsection{Memory Layout}

\begin{verbatim}
; Memory organization example
DATA SEGMENT
    ; Constants
    MAX_SIZE EQU 100
    
    ; Variables
    array1 DW 10 DUP(0)
    array2 DW MAX_SIZE DUP(0)
    string1 DB 'Hello', 0
    string2 DB 'World', 0
    
    ; Aligned data
    aligned_var DW 0
DATA ENDS
\end{verbatim}
