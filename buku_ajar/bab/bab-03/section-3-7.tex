\section{Teknik Manajemen Memori}

Manajemen memori yang efektif sangat penting dalam pemrograman assembly.

\subsection{Stack Management}

Stack digunakan untuk temporary storage dan procedure calls.

\begin{verbatim}
; Stack Management Example
; =======================
DATA SEGMENT
    buffer DW 100 DUP(0)
    count DW 0
DATA ENDS

CODE SEGMENT
    ASSUME CS:CODE, DS:DATA

start:
    ; Initialize DS and SS
    MOV AX, DATA
    MOV DS, AX
    MOV AX, DATA
    MOV SS, AX
    
    ; Initialize stack pointer
    MOV SP, OFFSET buffer + 200  ; Top of stack
    
    ; Push values onto stack
    MOV AX, 10
    PUSH AX
    MOV BX, 20
    PUSH BX
    MOV CX, 30
    PUSH CX
    
    ; Pop values from stack (LIFO order)
    POP DX      ; DX = 30
    POP DX      ; DX = 20
    POP DX      ; DX = 10
    
    ; Exit program
    MOV AH, 4Ch
    INT 21h

CODE ENDS
END start
\end{verbatim}

\subsection{Dynamic Memory Allocation}

\begin{verbatim}
; Dynamic Memory Allocation
; ========================
DATA SEGMENT
    heap_start DW 0
    heap_size DW 1024
    allocated DW 0
DATA ENDS

CODE SEGMENT
    ASSUME CS:CODE, DS:DATA

start:
    ; Initialize DS
    MOV AX, DATA
    MOV DS, AX
    
    ; Simulate memory allocation
    MOV AX, [heap_size]
    MOV [allocated], AX
    
    ; Use allocated memory
    MOV SI, [heap_start]
    MOV [SI], 1234h
    
    ; Exit program
    MOV AH, 4Ch
    INT 21h

CODE ENDS
END start
\end{verbatim}

\subsection{Memory Segmentation}

\begin{verbatim}
; Memory Segmentation Example
; ===========================
DATA SEGMENT
    data_var1 DW 100
    data_var2 DW 200
DATA ENDS

CODE SEGMENT
    ASSUME CS:CODE, DS:DATA

start:
    ; Initialize DS
    MOV AX, DATA
    MOV DS, AX
    
    ; Access data using different addressing modes
    MOV AX, data_var1        ; Direct addressing
    MOV BX, OFFSET data_var2  ; Get address
    MOV CX, [BX]             ; Indirect addressing
    
    ; Use ES for extra segment
    MOV AX, DATA
    MOV ES, AX
    MOV DX, ES:data_var1    ; Access via ES
    
    ; Exit program
    MOV AH, 4Ch
    INT 21h

CODE ENDS
END start
\end{verbatim}

\begin{catatan}
Program di atas mendemonstrasikan penggunaan segment register ES untuk mengakses data. Instruksi \texttt{MOV AX, ES:data\_var1} menunjukkan cara explicit segment override ketika mengakses memori melalui segment selain default DS.
\end{catatan}
