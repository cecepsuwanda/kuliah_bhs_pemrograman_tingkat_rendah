\section{Dasar Debugging}

Debugging adalah proses menemukan dan memperbaiki error dalam program assembly.

\subsection{Common Assembly Errors}

\textbf{Syntax Errors:}
\begin{itemize}
  \item \textbf{Invalid Mnemonic}: Instruksi tidak dikenal
  \item \textbf{Invalid Operand}: Format operand salah
  \item \textbf{Undefined Symbol}: Label atau variabel tidak didefinisikan
  \item \textbf{Duplicate Definition}: Label didefinisikan dua kali
\end{itemize}

\textbf{Logic Errors:}
\begin{itemize}
  \item \textbf{Register Corruption}: Register tidak dipreserve
  \item \textbf{Stack Overflow}: Rekursi terlalu dalam
  \item \textbf{Off-by-One}: Loop count salah
  \item \textbf{Memory Access}: Akses memori tidak valid
\end{itemize}

\subsection{Debugging Techniques}

\begin{verbatim}
; Debugging with comments
; ===================
; Add comments to explain logic
MOV AX, BX        ; Copy BX to AX
ADD AX, 5         ; Add 5 to AX
CMP AX, 0         ; Compare with zero
JZ  zero_found    ; Jump if zero

; Use registers for debugging
MOV DX, 1         ; Debug flag = 1
; ... code ...
CMP DX, 1         ; Check debug flag
JNE  skip_debug   ; Skip if not debugging
\end{verbatim}

\subsection{TASM Debugger Commands}

\begin{verbatim}
; Common debugger commands
; =====================
; BP address      - Set breakpoint
; BC number       - Clear breakpoint
; BL              - List breakpoints
; G               - Run program
; T               - Trace single instruction
; P               - Procedure step
; R register      - Show register value
; ?               - Show help
\end{verbatim}

\subsection{Debugging Strategy}

\begin{enumerate}
  \item \textbf{Compile with Debug Info}: Gunakan /zi switch
  \item \textbf{Set Breakpoints}: Set di lokasi strategis
  \item \textbf{Step Through}: Eksekusi per instruksi
  \item \textbf{Watch Registers}: Monitor register values
  \item \textbf{Check Memory}: Inspect memory content
  \item \textbf{Verify Logic}: Pastikan algoritma benar
\end{enumerate}
