\documentclass[../main.tex]{subfiles}
\ifSubfilesClassLoaded{\setcounter{chapter}{11}}{}
\begin{document}

\chapter{Integrasi dan Refleksi Pembelajaran}

\begin{subcpmk}
  \item Sub-CPMK 10.1: Mengintegrasikan semua konsep assembly programming
  \item Sub-CPMK 10.2: Merefleksikan pembelajaran dan rencana pengembangan
\end{subcpmk}

% ============================================================
% MATERI POKOK
% ============================================================
% ============================================================
% MATERI POKOK
% ============================================================
\section{Integrasi Konsep Assembly}

\subsection{Konsep Integrasi}
Integrasi adalah proses menggabungkan semua konsep assembly programming yang telah dipelajari menjadi aplikasi yang komprehensif.

\textbf{Komponen integrasi:}
\begin{itemize}
  \item \textbf{Processor Architecture}: Registers, addressing modes, segmentation
  \item \textbf{Instruction Set}: Data transfer, arithmetic, logic, control flow
  \item \textbf{Programming Constructs}: Procedures, loops, conditional statements
  \item \textbf{System Programming}: Interrupts, BIOS/DOS services
  \item \textbf{Optimization}: Performance tuning, debugging techniques
  \item \textbf{Project Development}: Modular design, library development
\end{itemize}

\subsection{Integrated Application Example}
Contoh aplikasi yang menggabungkan semua konsep assembly.

\begin{verbatim}
; Integrated Application: Student Management System
; ==================================================
; Features:
; 1. Student data management (arrays, file I/O)
; 2. Grade calculation (arithmetic operations)
; 3. Report generation (string operations, file output)
; 4. User interface (BIOS/DOS interrupts)
; 5. Data validation (conditional logic)
; 6. Performance optimization (register usage, loops)

; Main program structure
MAIN PROC
    ; Initialize system
    CALL initialize_system
    
    ; Load student data
    CALL load_student_data
    
    ; Display menu
    CALL display_menu
    
    ; Process user input
    CALL process_menu_selection
    
    ; Generate reports
    CALL generate_reports
    
    ; Cleanup and exit
    CALL cleanup_and_exit
MAIN ENDP

; Supporting procedures
initialize_system PROC
    ; Set up video mode
    MOV AH, 00h
    MOV AL, 03h
    INT 10h
    
    ; Initialize data structures
    MOV SI, student_array
    MOV CX, MAX_STUDENTS
    CALL initialize_student_array
    RET
initialize_system ENDP

load_student_data PROC
    ; Open data file
    MOV AH, 3Dh
    MOV AL, 00h
    MOV DX, student_data_file
    INT 21h
    JC file_error
    
    MOV file_handle, AX
    
    ; Read student records
    MOV CX, MAX_STUDENTS
    MOV SI, student_array
read_loop:
    CALL read_student_record
    ADD SI, STUDENT_SIZE
    LOOP read_loop
    
    ; Close file
    MOV AH, 3Eh
    MOV BX, file_handle
    INT 21h
    RET
load_student_data ENDP
\end{verbatim}

\section{Refleksi Pembelajaran}

\subsection{Perjalanan Pembelajaran Assembly}
Refleksi tentang proses pembelajaran assembly programming dari dasar hingga advanced.

\textbf{Tahapan pembelajaran:}
\begin{enumerate}
  \item \textbf{Foundation (Minggu 1-2)}: Konsep dasar processor dan assembly
  \item \textbf{Core Skills (Minggu 3-6)}: Instruksi, addressing, operasi dasar
  \item \textbf{Advanced Concepts (Minggu 7-10)}: Prosedur, interrupts, optimasi
  \item \textbf{Application (Minggu 11-12)}: Project development, best practices
  \item \textbf{Mastery (Minggu 13-14)}: Integration, assessment, reflection
\end{enumerate}

\subsection{Pencapaian Kompetensi}
Evaluasi pencapaian Sub-CPMK selama pembelajaran.

\begin{table}[htbp]
\centering
\begin{tabular}{|p{3cm}|p{2cm}|p{3cm}|p{3cm}|}
\hline
\textbf{Sub-CPMK} & \textbf{Target} & \textbf{Pencapaian} & \textbf{Action} \\
\hline
1.1: Processor Architecture & 85\% & 90\% & Maintain \\
\hline
1.2: Addressing Modes & 80\% & 85\% & Practice \\
\hline
2.1: TASM Syntax & 90\% & 95\% & Maintain \\
\hline
2.2: Assembler Directives & 75\% & 80\% & Review \\
\hline
3.1: Arithmetic Operations & 85\% & 90\% & Maintain \\
\hline
3.2: Logical Operations & 80\% & 85\% & Practice \\
\hline
4.1: Control Structures & 85\% & 90\% & Maintain \\
\hline
4.2: Procedures & 80\% & 85\% & Practice \\
\hline
\end{tabular}
\caption{Pencapaian Sub-CPMK}
\end{table}

\section{Best Practices Summary}

\subsection{Programming Best Practices}
Ringkasan praktik terbaik dalam assembly programming.

\textbf{Code Quality:}
\begin{itemize}
  \item Gunakan meaningful variable dan procedure names
  \item Dokumentasikan kode dengan komentar yang jelas
  \item Implementasikan error handling yang komprehensif
  \item Gunakan modular design untuk maintainability
  \item Optimasi kode untuk performa yang baik
\end{itemize}

\textbf{Development Process:}
\begin{itemize}
  \item Plan sebelum coding (requirements, design)
  \item Test secara bertahap (unit, integration, system)
  \item Debug secara sistematis dengan TASM debugger
  \item Review kode untuk best practices
  \item Document untuk maintainability
\end{itemize}

\subsection{Common Pitfalls dan Solutions}
Masalah umum dan solusinya dalam assembly programming.

\begin{table}[htbp]
\centering
\begin{tabular}{|p{4cm}|p{5cm}|p{4cm}|}
\hline
\textbf{Common Pitfall} & \textbf{Description} & \textbf{Solution} \\
\hline
Register Corruption & Mengubah register tanpa menyimpan nilai asli & Gunakan PUSH/POP untuk preservation \\
\hline
Stack Overflow & Rekursi tanpa base case atau terlalu dalam & Tambah base case, limit depth \\
\hline
Off-by-One Error & Loop count atau array index salah & Validasi boundary conditions \\
\hline
Memory Leaks & Tidak cleanup allocated resources & Implement proper cleanup \\
\hline
Poor Performance & Inefficient algorithms atau data structures & Use profiling, optimize \\
\hline
Hard to Debug & Kode tidak terorganisir dengan baik & Modular design, good documentation \\
\hline
\end{tabular}
\caption{Common Pitfalls dan Solutions}
\end{table}

\section{Career Pathways}

\subsection{Career Opportunities}
Peluang karir untuk mahasiswa dengan keahlian assembly programming.

\textbf{Technical Roles:}
\begin{itemize}
  \item \textbf{Systems Programmer}: Low-level system development
  \item \textbf{Embedded Systems Engineer}: Device programming
  \item \textbf{Firmware Developer}: Hardware-software interface
  \item \textbf{Security Researcher}: Reverse engineering, vulnerability analysis
  \item \textbf{Performance Engineer}: System optimization
\end{itemize}

\textbf{Industry Applications:}
\begin{itemize}
  \item \textbf{Automotive}: ECU programming, control systems
  \item \textbf{Aerospace}: Flight control systems, avionics
  \item \textbf{Medical Devices}: Equipment programming, diagnostics
  \item \textbf{IoT}: Device firmware, sensor programming
  \item \textbf{Gaming}: Engine development, optimization
\end{itemize}

\subsection{Continuous Learning}
Strategi untuk pembelajaran berkelanjutan.

\textbf{Advanced Topics:}
\begin{itemize}
  \item Modern processor architectures (x86-64, ARM)
  \item Operating system internals
  \item Compiler design and construction
  \item Computer architecture and organization
  \item Parallel and distributed computing
\end{itemize}

\textbf{Learning Resources:}
\begin{itemize}
  \item Intel Developer Manuals
  \item AMD Processor Programming Guides
  \item Open source assembly projects
  \item Online communities and forums
  \item Technical conferences and workshops
\end{itemize}

% ============================================================
% AKTIVITAS PEMBELAJARAN
% ============================================================

\begin{aktivitas}
  \item \textbf{Integration Project}: Buat aplikasi lengkap yang menggabungkan semua konsep assembly.
  
  \item \textbf{Self-Reflection}: Lakukan refleksi pembelajaran dan identifikasi area improvement.
  
  \item \textbf{Best Practices Review}: Review kode assembly yang telah dibuat untuk best practices.
  
  \item \textbf{Career Planning}: Rancang career path berdasarkan keahlian assembly programming.
  
  \item \textbf{Portfolio Completion}: Finalisasi portfolio dengan semua project terbaik.
  
  \item \textbf{Knowledge Sharing}: Bagikan pengalaman dan tips dengan mahasiswa lain.
\end{aktivitas}

% ============================================================
% LATIHAN DAN REFLEKSI
% ============================================================

\begin{latihan}
  \item Buat aplikasi database management system yang mengintegrasikan semua konsep assembly.
  
  \item Lakukan self-assessment untuk semua Sub-CPMK dan buat action plan untuk improvement.
  
  \item Identifikasi dan dokumentasikan 5 best practices yang telah Anda pelajari.
  
  \item Rancang career development plan untuk 5 tahun ke depan.
  
  \item Buat comprehensive portfolio dengan semua project dan pencapaian.
  
  \item Tulis reflection essay tentang perjalanan pembelajaran assembly programming.
  
  \item Buat knowledge sharing presentation untuk mahasiswa angkatan berikutnya.
  
  \item \textbf{Refleksi}: Konsep assembly mana yang paling bermanfaat untuk karir Anda? Bagaimana Anda mengaplikasikannya?
\end{latihan}

% ============================================================
% ASESMEN
% ============================================================

\begin{asesmen}
\textbf{Instrumen Penilaian untuk Sub-CPMK 10.1, 10.2}

\textbf{A. Pilihan Ganda}

\begin{enumerate}
  \item Tujuan utama integrasi konsep assembly adalah:
  \begin{enumerate}
    \item Membuat program yang kompleks
    \item Menggabungkan semua konsep yang dipelajari
    \item Meningkatkan performa program
    \item Menunjukkan kemampuan individual
  \end{enumerate}
  
  \item Komponen penting dalam refleksi pembelajaran adalah:
  \begin{enumerate}
    \item Identifikasi kekuatan dan kelemahan
    \item Membuat action plan untuk improvement
    \item Mengevaluasi pencapaian kompetensi
    \item Semua jawaban benar
  \end{enumerate}
  
  \item Best practice yang paling penting dalam assembly programming adalah:
  \begin{enumerate}
    \item Code documentation
    \item Performance optimization
    \item Error handling
    \item Modular design
  \end{enumerate}
  
  \item Career pathway yang paling relevan untuk assembly programming adalah:
  \begin{enumerate}
    \item Web development
    \item Mobile development
    \item Systems programming
    \item Database administration
  \end{enumerate}
\end{enumerate}

\textbf{B. Essay}

\begin{enumerate}
  \item Jelaskan bagaimana integrasi konsep assembly penting untuk pengembangan karir!
  
  \item Mengapa refleksi pembelajaran penting untuk continuous improvement?
\end{enumerate}

\textbf{C. Practical Challenge}

\begin{enumerate}
  \item Buat capstone project:
  \begin{itemize}
    \item Integrated application yang menggabungkan semua konsep assembly
    \item Comprehensive documentation dan user manual
    \item Performance analysis dan optimization report
    \item Testing framework dengan unit tests
    \item Deployment guide dan installation instructions
    \item Maintenance manual dan troubleshooting guide
    \item Knowledge transfer documentation
    \item Presentation materials untuk demo
    \item Code review checklist dan quality assurance
    \item Future enhancement recommendations
  \end{itemize}
\end{enumerate}

\textbf{Rubrik Penilaian}: Lihat Lampiran A
\end{asesmen}

% ============================================================
% CHECKLIST KOMPETENSI
% ============================================================

\begin{checklist}
  \item Saya dapat mengintegrasikan semua konsep assembly programming
  \item Saya dapat melakukan refleksi pembelajaran yang mendalam
  \item Saya dapat mengidentifikasi best practices dalam assembly programming
  \item Saya dapat merancang career development yang realistis
  \item Saya dapat membuat portfolio yang komprehensif
  \item Saya dapat berbagi pengetahuan dengan orang lain
  \item Saya dapat merencanakan continuous learning
\end{checklist}

% ============================================================
% RANGKUMAN
% ============================================================

\begin{rangkuman}
Bab ini membahas integrasi dan refleksi dalam assembly programming, termasuk penggabungan konsep, evaluasi pembelajaran, dan persiapan karir.

\textbf{Poin Kunci:}
\begin{itemize}
  \item Integrasi menggabungkan semua konsep assembly menjadi aplikasi komprehensif
  \item Refleksi penting untuk continuous improvement
  \item Best practices memastikan kode berkualitas dan maintainable
  \item Career pathways tersedia untuk berbagai industri
  \item Continuous learning essential untuk tetap relevan
  \item Portfolio mendokumentasikan pencapaian kompetensi
  \item Knowledge sharing memperluas dampak pembelajaran
  \item Action plan memastikan development berkelanjutan
\end{itemize}

\textbf{Kata Kunci}: \asm{Integrasi}, \asm{Refleksi}, \asm{Best Practices}, \asm{Career Path}, \asm{Continuous Learning}, \asm{Portfolio}, \asm{Knowledge Sharing}, \asm{Action Plan}
\end{rangkuman}


\ifSubfilesClassLoaded{
  \renewcommand{\bibname}{Daftar Pustaka}
  \bibliographystyle{plain}
  \bibliography{../references}
}{}
\end{document}
  \item Text editor dengan syntax highlighting
\end{itemize}

\section{Refleksi Pembelajaran}

\subsection{Learning Journey}
\begin{itemize}
  \item Konsep paling menantik
  \item Breakthrough moments
  \item Areas untuk improvement
  \item Future learning goals
\end{itemize}

\subsection{Career Pathways}
\begin{itemize}
  \item System programming
  \item Embedded systems
  \item Security research
  \item Compiler development
\end{itemize}

\section{Best Practices Summary}

\subsection{Code Quality}
\begin{itemize}
  \item Modular design
  \item Error handling
  \item Documentation
  \item Testing
\end{itemize}

\subsection{Performance}
\begin{itemize}
  \item Register optimization
  \item Memory efficiency
  \item Algorithm selection
  \item Debugging skills
\end{itemize}

\ifSubfilesClassLoaded{
  \renewcommand{\bibname}{Daftar Pustaka}
  \bibliographystyle{plain}
  \bibliography{../references}
}{}
\end{document}
