\documentclass[../main.tex]{subfiles}
\ifSubfilesClassLoaded{\setcounter{chapter}{9}}{}
\begin{document}

\chapter{Pengembangan Proyek Assembly}

\begin{subcpmk}
  \item Sub-CPMK 8.1: Merancang proyek assembly yang kompleks
  \item Sub-CPMK 8.2: Mengimplementasikan integrasi modul dan library
  \item Sub-CPMK 2.1: Menulis program assembly sederhana dengan TASM syntax
\end{subcpmk}

% ============================================================
% MATERI POKOK
% ============================================================
% ============================================================
% MATERI POKOK
% ============================================================
\section{Project Development Lifecycle}

\subsection{Konsep Project Development}
Project development dalam assembly programming memerlukan pendekatan sistematis untuk menghasilkan program yang efisien dan maintainable.

\textbf{Tahapan project development:}
\begin{itemize}
  \item \textbf{Requirements Analysis}: Memahami kebutuhan dan constraints
  \item \textbf{Design}: Merancang arsitektur dan algoritma
  \item \textbf{Implementation}: Menulis kode assembly
  \item \textbf{Testing}: Debugging dan validasi
  \item \textbf{Optimization}: Meningkatkan performa
  \item \textbf{Documentation}: Menulis dokumentasi
\end{itemize}

\subsection{Project Planning}
Perencanaan yang baik adalah kunci keberhasilan project assembly.

\begin{verbatim}
; Project structure template
; -----------------------
; main.asm          - Entry point
; constants.inc     - Constants dan definitions
; procedures.inc    - Procedure declarations
; data.inc          - Data definitions
; utils.asm         - Utility procedures
; io.asm            - Input/output procedures
; math.asm          - Mathematical operations
\end{verbatim}

\textbf{Komponen project planning:}
\begin{itemize}
  \item Definisi scope dan requirements
  \item Pemilihan algoritma yang tepat
  \item Estimasi waktu dan resources
  \item Risk assessment
  \item Milestone definition
\end{itemize}

\section{Modular Programming}

\subsection{Modular Design Principles}
Membagi program menjadi modul-modul yang terpisah untuk maintainability.

\begin{verbatim}
; Module structure example
; math.asm - Mathematical operations module

PUBLIC add_numbers, subtract_numbers, multiply_numbers
EXTERN print_result

; Code segment
CODE SEGMENT
    ASSUME CS:CODE, DS:DATA

add_numbers PROC
    ; Implementation
    RET
add_numbers ENDP

subtract_numbers PROC
    ; Implementation
    RET
subtract_numbers ENDP

multiply_numbers PROC
    ; Implementation
    RET
multiply_numbers ENDP

CODE ENDS
END
\end{verbatim}

\textbf{Keuntungan modular programming:}
\begin{itemize}
  \item Code reuse dan maintainability
  \item Parallel development
  \item Easier testing dan debugging
  \item Better organization
\end{itemize}

\subsection{Interface Design}
Mendefinisikan interface yang jelas antar modul.

\begin{verbatim}
; Interface definitions
; io_interface.inc

; Input procedures
PUBLIC read_char, read_string, read_number
read_char PROC NEAR
    ; Returns character in AL
    RET
read_char ENDP

read_string PROC NEAR
    ; DS:DX = buffer address
    RET
read_string ENDP

; Output procedures
PUBLIC print_char, print_string, print_number
print_char PROC NEAR
    ; AL = character to print
    RET
print_char ENDP
\end{verbatim}

\section{Library Development}

\subsection{Creating Reusable Libraries}
Mengembangkan library untuk fungsi-fungsi yang sering digunakan.

\begin{verbatim}
; Standard library structure
; stdlib.asm

; String operations
PUBLIC strlen, strcpy, strcmp, strcat
strlen PROC NEAR
    PUSH SI
    MOV SI, DX        ; DX = string address
    MOV CX, 0
count_loop:
    CMP BYTE PTR [SI], 0
    JE end_count
    INC CX
    INC SI
    JMP count_loop
end_count:
    MOV AX, CX        ; Return length in AX
    POP SI
    RET
strlen ENDP

strcpy PROC NEAR
    ; DS:SI = source, ES:DI = destination
    PUSH SI
    PUSH DI
copy_loop:
    MOV AL, [SI]
    MOV [DI], AL
    CMP AL, 0
    JE end_copy
    INC SI
    INC DI
    JMP copy_loop
end_copy:
    POP DI
    POP SI
    RET
strcpy ENDP
\end{verbatim}

\subsection{Library Organization}
Mengorganisir library untuk kemudahan penggunaan.

\textbf{Kategori library:}
\begin{itemize}
  \item \textbf{String Library}: strlen, strcpy, strcmp, strcat
  \item \textbf{Math Library}: add, subtract, multiply, divide, power
  \item \textbf{I/O Library}: \texttt{read\_char}, \texttt{print\_string}, file operations
  \item \textbf{Utility Library}: memory operations, conversions
  \item \textbf{Graphics Library}: pixel operations, line drawing
\end{itemize}

\section{Best Practices}

\subsection{Code Organization}
Praktik terbaik untuk organisasi kode assembly.

\begin{verbatim}
; Standard file organization
; ------------------------
; 1. Header comments
; 2. Include files
; 3. Constants and equates
; 4. Data segment
; 5. Code segment
; 6. Procedures
; 7. Entry point

; Example structure
; main.asm

; ============================================================
; PROGRAM: Calculator Application
; AUTHOR: Your Name
; DATE: Current Date
; DESCRIPTION: Simple calculator with basic operations
; ============================================================

INCLUDE constants.inc
INCLUDE procedures.inc

; Data segment
DATA SEGMENT
    prompt_msg DB 'Enter expression: $'
    result_msg DB 'Result: $'
    buffer DB 80 DUP(?)
DATA ENDS

; Code segment
CODE SEGMENT
    ASSUME CS:CODE, DS:DATA

start:
    ; Initialize DS
    MOV AX, DATA
    MOV DS, AX
    
    ; Main program logic
    CALL main_loop
    
    ; Exit program
    MOV AH, 4Ch
    INT 21h

main_loop PROC
    ; Implementation
    RET
main_loop ENDP

CODE ENDS
END start
\end{verbatim}

\subsection{Naming Conventions}
Konvensi penamaan untuk konsistensi kode.

\textbf{Recommended conventions:}
\begin{itemize}
  \item \textbf{Procedures}: camelCase (calculateSum, printResult)
  \item \textbf{Variables}: snake\_case (user\_input, buffer\_ptr)
  \item \textbf{Constants}: UPPER\_CASE (MAX\_SIZE, ERROR\_CODE)
  \item \textbf{Labels}: descriptive\_name (loop\_start, error\_handler)
\end{itemize}

\subsection{Documentation Standards}
Standar dokumentasi untuk kode assembly.

\begin{verbatim}
; Procedure documentation template
; -------------------------------------------------
; Procedure: calculateSum
; Purpose: Calculate sum of two numbers
; Input: AX = first number, BX = second number
; Output: AX = sum
; Modifies: AX, flags
; Calls: None
; -------------------------------------------------
calculateSum PROC
    ADD AX, BX       ; Calculate sum
    RET
calculateSum ENDP
\end{verbatim}

\section{Version Control}

\subsection{Source Code Management}
Mengelola versi kode assembly dengan version control.

\textbf{Best practices:}
\begin{itemize}
  \item Gunakan meaningful commit messages
  \item Commit frequently dengan logical units
  \item Gunakan branching untuk features
  \item Review code sebelum merge
  \item Maintain changelog
\end{itemize}

\subsection{Backup and Recovery}
Strategi backup dan recovery untuk project assembly.

\begin{verbatim}
; Backup strategy
; -------------
; 1. Daily automatic backup
; 2. Weekly full backup
; 3. Version control repository
; 4. Offsite backup
; 5. Documentation backup
\end{verbatim}

% ============================================================
% AKTIVITAS PEMBELAJARAN
% ============================================================

\begin{aktivitas}
  \item \textbf{Project Planning}: Buat project plan untuk aplikasi calculator dengan modular design.
  
  \item \textbf{Library Development}: Kembangkan string library dengan fungsi strlen, strcpy, strcmp.
  
  \item \textbf{Modular Design}: Implementasikan text editor dengan modular architecture.
  
  \item \textbf{Code Organization}: Reorganize existing code mengikuti best practices.
  
  \item \textbf{Documentation}: Tulis dokumentasi lengkap untuk assembly project.
  
  \item \textbf{Version Control}: Setup version control untuk assembly project development.
\end{aktivitas}

% ============================================================
% LATIHAN DAN REFLEKSI
% ============================================================

\begin{latihan}
  \item Buat project plan untuk game sederhana dengan modular architecture.
  
  \item Kembangkan math library dengan operasi dasar dan fungsi trigonometri.
  
  \item Implementasikan file I/O library dengan error handling yang baik.
  
  \item Buat documentation template untuk assembly project.
  
  \item Design modular architecture untuk database management system.
  
  \item Implementasikan graphics library untuk drawing operations.
  
  \item Buat build system untuk multi-module assembly project.
  
  \item \textbf{Refleksi}: Aspek mana dari project development yang paling menantang? Bagaimana Anda mengatasi kesulitan tersebut?
\end{latihan}

% ============================================================
% ASESMEN
% ============================================================

\begin{asesmen}
\textbf{Instrumen Penilaian untuk Sub-CPMK 8.1, 8.2}

\textbf{A. Pilihan Ganda}

\begin{enumerate}
  \item Tujuan utama modular programming adalah:
  \begin{enumerate}
    \item Meningkatkan ukuran kode
    \item Memudahkan maintainability dan reuse
    \item Mengurangi performa
    \item Memperumit testing
  \end{enumerate}
  
  \item Directive untuk membuat procedure visible ke modul lain adalah:
  \begin{enumerate}
    \item EXTERN
    \item PUBLIC
    \item GLOBAL
    \item EXPORT
  \end{enumerate}
  
  \item Bagian pertama yang harus ada dalam assembly file adalah:
  \begin{enumerate}
    \item Data segment
    \item Code segment
    \item Header comments
    \item Include files
  \end{enumerate}
  
  \item Naming convention yang direkomendasikan untuk procedures adalah:
  \begin{enumerate}
    \item UPPERCASE
    \item lowercase
    \item camelCase
    \item snake\_case
  \end{enumerate}
  
  \item Urutan tahapan project development lifecycle yang benar adalah:
  \begin{enumerate}
    \item Requirements Analysis $\rightarrow$ Design $\rightarrow$ Implementation $\rightarrow$ Testing $\rightarrow$ Optimization $\rightarrow$ Documentation
    \item Implementation $\rightarrow$ Testing $\rightarrow$ Design
    \item Design $\rightarrow$ Implementation $\rightarrow$ Requirements
    \item Documentation $\rightarrow$ Implementation $\rightarrow$ Testing
  \end{enumerate}
  
  \item Pada tahap Requirements Analysis untuk project assembly, yang dianalisis antara lain:
  \begin{enumerate}
    \item Kebutuhan fungsional, constraints (ukuran .COM 64KB, interupsi DOS/BIOS)
    \item Hanya nama file
    \item Hanya warna tampilan
    \item Tidak ada yang spesifik
  \end{enumerate}
\end{enumerate}

\textbf{B. Essay}

\begin{enumerate}
  \item Jelaskan keuntungan modular programming dalam assembly development!
  
  \item Mengapa documentation penting dalam assembly programming?
  
  \item Jelaskan tahapan Design untuk project assembly! Apa yang perlu dirancang dalam hal register usage, model memori, dan struktur prosedur?
  
  \item Mengapa Testing dengan Turbo Debugger penting dalam project development assembly?
\end{enumerate}

\textbf{C. Practical Challenge}

\begin{enumerate}
  \item Buat comprehensive project:
  \begin{itemize}
    \item Text editor dengan modular architecture
    \item String library (strlen, strcpy, strcmp, strcat)
    \item File I/O operations
    \item Search dan replace functionality
    \item Multiple file handling
    \item Configuration management
    \item Complete documentation
    \item Build system dengan makefile
    \item Version control integration
    \item Testing framework
  \end{itemize}
  
  \item Untuk proyek yang dikerjakan, buat penjelasan singkat (1–2 kalimat) per tahap project development lifecycle: Requirements, Design, Implementation, Testing, Optimization, Documentation.
\end{enumerate}

\textbf{Rubrik Penilaian}: Lihat Lampiran A
\end{asesmen}

% ============================================================
% CHECKLIST KOMPETENSI
% ============================================================

\begin{checklist}
  \item Saya dapat merancang project assembly dengan struktur yang baik
  \item Saya dapat mengimplementasikan modular programming
  \item Saya dapat mengembangkan reusable libraries
  \item Saya dapat mengorganisir kode assembly dengan best practices
  \item Saya dapat menulis dokumentasi yang komprehensif
  \item Saya dapat menggunakan version control untuk assembly project
  \item Saya dapat membuat build system untuk multi-module project
  \item Saya dapat menerapkan quality assurance dalam assembly development
\end{checklist}

% ============================================================
% RANGKUMAN
% ============================================================

\begin{rangkuman}
Bab ini membahas project development dalam assembly programming, termasuk modular programming, library development, dan best practices.

\textbf{Poin Kunci:}
\begin{itemize}
  \item Project development lifecycle memerlukan perencanaan sistematis
  \item Modular programming meningkatkan maintainability dan code reuse
  \item Library development menyediakan fungsi-fungsi yang reusable
  \item Best practices memastikan kode yang organized dan maintainable
  \item Documentation penting untuk long-term maintenance
  \item Version control memungkinkan collaboration dan tracking
  \item Code organization mengikuti standar industri
  \item Quality assurance memastikan deliverables yang berkualitas
\end{itemize}

\textbf{Kata Kunci}: \asm{Project Development}, \asm{Modular Programming}, \asm{Library}, \asm{Documentation}, \asm{Version Control}, \asm{Best Practices}, \asm{Code Organization}, \asm{Build System}
\end{rangkuman}


\ifSubfilesClassLoaded{
  \renewcommand{\bibname}{Daftar Pustaka}
  \bibliographystyle{plain}
  \bibliography{../references}
}{}
\end{document}

\subsection{Desain Arsitektur}
\begin{itemize}
  \item Breakdown menjadi modul-modul independen
  \item Definisikan interface antar modul
  \item Rancang struktur data yang efisien
  \item Pertimbangkan optimasi awal
\end{itemize}

\section{Struktur Proyek}

\subsection{Modular Programming}
\begin{verbatim}
; Main program structure
.MODEL SMALL
.STACK 100h
.DATA
; Global variables
.CODE
MAIN PROC
    ; Initialize
    CALL init_modules
    
    ; Main loop
main_loop:
    CALL process_input
    CALL update_display
    JMP main_loop
    
    ; Cleanup
    CALL cleanup_modules
    MOV AH, 4Ch
    INT 21h
MAIN ENDP

; Module implementations
include "input.asm"
include "display.asm"
include "utils.asm"
END MAIN
\end{verbatim}

\subsection{Library Development}
\begin{verbatim}
; Library file: mathlib.asm
PUBLIC add_numbers, multiply_numbers
.MODEL SMALL
.CODE

add_numbers PROC
    ; Add two numbers
    ; Input: AX = num1, BX = num2
    ; Output: AX = result
    ADD AX, BX
    RET
add_numbers ENDP

multiply_numbers PROC
    ; Multiply two numbers
    ; Input: AX = num1, BX = num2
    ; Output: DX:AX = result
    MUL BX
    RET
multiply_numbers ENDP
END
\end{verbatim}

\section{Development Best Practices}

\subsection{Code Organization}
\begin{itemize}
  \item Gunakan meaningful labels dan variable names
  \item Tambahkan komentar untuk logika kompleks
  \item Pisahkan data, code, dan stack segments
  \item Gunakan consistent indentation
\end{itemize}

\subsection{Error Handling}
\begin{verbatim}
; Error handling framework
ERROR_NO_MEMORY     EQU 1
ERROR_FILE_NOT_FOUND EQU 2
ERROR_INVALID_INPUT EQU 3

handle_error PROC
    ; Input: AX = error code
    CMP AX, 0
    JE no_error
    
    ; Display error message
    MOV BX, AX
    SHL BX, 1        ; Multiply by 2 for word table
    MOV SI, offset error_table
    ADD SI, BX
    MOV DX, [SI]
    MOV AH, 09h
    INT 21h
    
    MOV AH, 4Ch
    INT 21h
    
no_error:
    RET
handle_error ENDP

error_table DW offset msg_no_memory
              DW offset msg_file_not_found
              DW offset msg_invalid_input
\end{verbatim}

\section{Testing Strategies}

\subsection{Unit Testing}
\begin{verbatim}
; Test framework for math functions
test_add PROC
    ; Test case 1: 5 + 3 = 8
    MOV AX, 5
    MOV BX, 3
    CALL add_numbers
    CMP AX, 8
    JNE test_failed
    
    ; Test case 2: 0 + 0 = 0
    MOV AX, 0
    MOV BX, 0
    CALL add_numbers
    CMP AX, 0
    JNE test_failed
    
    MOV AH, 09h
    MOV DX, offset test_passed_msg
    INT 21h
    RET
    
test_failed:
    MOV AH, 09h
    MOV DX, offset test_failed_msg
    INT 21h
    RET
test_add ENDP
\end{verbatim}

\subsection{Integration Testing}
\begin{itemize}
  \item Test interface antar modul
  \item Verifikasi data flow
  \item Test error propagation
  \item Validasi resource management
\end{itemize}

\section{Documentation}

\subsection{Code Documentation}
\begin{itemize}
  \item Fungsi description dan parameters
  \item Register usage conventions
  \item Memory layout documentation
  \item Algorithm explanation
\end{itemize}

\subsection{User Manual}
\begin{itemize}
  \item Installation instructions
  \item Usage examples
  \item Troubleshooting guide
  \item Technical specifications
\end{itemize}

% ============================================================
% AKTIVITAS PEMBELAJARAN
% ============================================================

\begin{aktivitas}
  \item \textbf{Project Planning}: Rancang proyek kalkulator scientific dengan modul-modul terpisah.
  
  \item \textbf{Library Development}: Buat library untuk operasi string dan matematika.
  
  \item \textbf{Integration Exercise}: Gabungkan beberapa modul menjadi aplikasi lengkap.
  
  \item \textbf{Code Review}: Review dan refactor proyek teman dengan best practices.
  
  \item \textbf{Testing Implementation}: Implementasikan unit test untuk fungsi-fungsi kompleks.
  
  \item \textbf{Documentation Practice}: Tulis dokumentasi lengkap untuk proyek assembly.
\end{aktivitas}

% ============================================================
% LATIHAN DAN REFLEKSI
% ============================================================

\begin{latihan}
  \item Rancang arsitektur untuk proyek text editor dengan fitur:
  \begin{itemize}
    \item File operations (open, save, close)
    \item Text editing (insert, delete, replace)
    \item Search functionality
    \item Menu interface
  \end{itemize}
  
  \item Implementasikan library untuk:
  \begin{itemize}
    \item String manipulation (length, copy, compare, concatenate)
    \item Array operations (sort, search, min/max)
    \item Input validation
  \end{itemize}
  
  \item Buat proyek database sederhana dengan:
  \begin{itemize}
    \item Record management
    \item Search functionality
    \item Data persistence
    \item Error handling
  \end{itemize}
  
  \item Develop game sederhana (snake atau puzzle) dengan:
  \begin{itemize}
    \item Graphics display
    \item Keyboard input
    \item Game logic
    \item Score tracking
  \end{itemize}
  
  \item Implementasikan system monitor dengan:
  \begin{itemize}
    \item Memory usage display
    \item Process information
    \item System statistics
    \item Real-time updates
  \end{itemize}
  
  \item \textbf{Refleksi}: Tantangan terbesar dalam mengembangkan proyek assembly? Bagaimana Anda mengatasi kompleksitas?
\end{latihan}

% ============================================================
% ASESMEN
% ============================================================

\begin{asesmen}
\textbf{Instrumen Penilaian untuk Sub-CPMK 8.1, 8.2, 2.1}

\textbf{A. Pilihan Ganda}

\begin{enumerate}
  \item Direktif untuk membuat library yang dapat digunakan oleh program lain adalah:
  \begin{enumerate}
    \item PUBLIC
    \item EXTRN
    \item GLOBAL
    \item EXTERNAL
  \end{enumerate}
  
  \item Untuk mengorganisasi proyek besar, pendekatan terbaik adalah:
  \begin{enumerate}
    \item Monolithic programming
    \item Modular programming
    \item Sequential programming
    \item Linear programming
  \end{enumerate}
  
  \item Error handling yang baik harus:
  \begin{enumerate}
    \item Mengabaikan semua error
    \item Exit program immediately
    \item Log dan recover gracefully
    \item Display technical details
  \end{enumerate}
  
  \item Testing untuk memvalidasi interface antar modul disebut:
  \begin{enumerate}
    \item Unit testing
    \item Integration testing
    \item System testing
    \item Acceptance testing
  \end{enumerate}
\end{enumerate}

\textbf{B. Essay}

\begin{enumerate}
  \item Jelaskan pentingnya modular programming dalam pengembangan proyek assembly!
  
  \item Mengapa documentation penting dalam assembly programming?
\end{enumerate}

\textbf{C. Practical Challenge}

\begin{enumerate}
  \item Kembangkan proyek lengkap:
  \begin{itemize}
    \item Pilih salah satu: text editor, game, atau database system
    \item Implementasikan dengan modular approach
    \item Gunakan library untuk fungsi reusable
    \item Include comprehensive error handling
    \item Tulis dokumentasi lengkap
    \item Present dengan demo dan technical explanation
  \end{itemize}
\end{enumerate}

\textbf{Rubrik Penilaian}: Lihat Lampiran A
\end{asesmen}

% ============================================================
% CHECKLIST KOMPETENSI
% ============================================================

\begin{checklist}
  \item Saya dapat merancang arsitektur proyek assembly yang kompleks
  \item Saya dapat mengimplementasikan modular programming
  \item Saya dapat mengembangkan library yang reusable
  \item Saya dapat mengimplementasikan error handling yang robust
  \item Saya dapat melakukan unit dan integration testing
  \item Saya dapat menulis dokumentasi teknis yang lengkap
  \item Saya dapat mengelola dependensi antar modul
  \item Saya dapat mengoptimasi proyek untuk performa
\end{checklist}

% ============================================================
% RANGKUMAN
% ============================================================

\begin{rangkuman}
Bab ini membahas pengembangan proyek assembly yang kompleks, termasuk perencanaan, modular design, testing, dan documentation.

\textbf{Poin Kunci:}
\begin{itemize}
  \item Perencanaan proyek essential untuk keberhasilan implementasi
  \item Modular programming memungkinkan pengembangan yang terstruktur
  \item Library development meningkatkan code reusability
  \item Error handling critical untuk robust applications
  \item Testing memastikan kualitas dan reliability
  \item Documentation penting untuk maintainability
  \item Best practices meningkatkan code quality
  \item Integration testing validasi kesesuaian antar modul
\end{itemize}

\textbf{Kata Kunci}: \textbf{\texttt{Project Development}}, \textbf{\texttt{Modular Programming}}, \textbf{\texttt{Library}}, \textbf{\texttt{Error Handling}}, \textbf{\texttt{Testing}}, \textbf{\texttt{Documentation}}, \textbf{\texttt{Integration}}, \textbf{\texttt{TASM}}, \textbf{\texttt{Best Practices}}
\end{rangkuman}

\ifSubfilesClassLoaded{
  \renewcommand{\bibname}{Daftar Pustaka}
  \bibliographystyle{plain}
  \bibliography{../references}
}{}
\end{document}
