\chapter{Percabangan dan Loop: CMP, JE, JNE, JL, JG, LOOP; Interupsi Mouse (INT 33h)}

\section{Tujuan Pembelajaran}
Setelah mengikuti pertemuan ini, mahasiswa diharapkan dapat:
\begin{itemize}
\item Menggunakan instruksi CMP untuk perbandingan
\item Menggunakan instruksi percabangan (JE, JNE, JL, JG)
\item Menggunakan instruksi LOOP untuk perulangan
\item Memahami interupsi mouse INT 33h
\item Mampu membuat program dengan struktur kontrol
\end{itemize}

\section{Materi Pembelajaran}

\subsection{Instruksi Perbandingan (CMP)}
\begin{itemize}
\item Sintaks instruksi CMP
\item Cara kerja CMP
\item Pengaruh terhadap flag register
\item Perbandingan dengan operasi aritmatika
\end{itemize}

\subsection{Instruksi Percabangan}
\begin{itemize}
\item Instruksi JE (Jump if Equal)
\item Instruksi JNE (Jump if Not Equal)
\item Instruksi JL (Jump if Less)
\item Instruksi JG (Jump if Greater)
\item Instruksi lainnya (JLE, JGE, JB, JA, etc.)
\item Conditional jumps dan flag register
\end{itemize}

\subsection{Instruksi Perulangan (LOOP)}
\begin{itemize}
\item Instruksi LOOP
\item Instruksi LOOPZ/LOOPE
\item Instruksi LOOPNZ/LOOPNE
\item Penggunaan register CX
\item Kontrol perulangan
\end{itemize}

\subsection{Interupsi Mouse INT 33h}
\begin{itemize}
\item Fungsi 00h: Initialize Mouse
\item Fungsi 01h: Show Mouse Cursor
\item Fungsi 02h: Hide Mouse Cursor
\item Fungsi 03h: Get Mouse Position and Button Status
\item Fungsi 04h: Set Mouse Cursor Position
\item Fungsi 05h: Get Button Press Information
\end{itemize}

\section{Praktikum}
\begin{enumerate}
\item Program percabangan sederhana
\item Program perulangan dengan LOOP
\item Program kombinasi percabangan dan perulangan
\item Program interaksi dengan mouse
\item Program game sederhana dengan mouse
\end{enumerate}

\section{Contoh Kode}
\begin{verbatim}
; Program percabangan, loop, dan mouse
TITLE Percabangan dan Loop
.MODEL SMALL
.STACK 100h

.DATA
    pesan1 DB 'Bilangan sama$'
    pesan2 DB 'Bilangan berbeda$'
    pesan3 DB 'Loop selesai$'
    bil1 DW 10
    bil2 DW 10

.CODE
START:
    MOV AX, @DATA
    MOV DS, AX
    
    ; Percabangan
    MOV AX, bil1
    CMP AX, bil2
    JE sama
    ; Jika tidak sama
    MOV AH, 09h
    MOV DX, OFFSET pesan2
    INT 21h
    JMP lanjut
    
sama:
    MOV AH, 09h
    MOV DX, OFFSET pesan1
    INT 21h
    
lanjut:
    ; Perulangan
    MOV CX, 5
ulang:
    ; Kode yang diulang
    DEC CX
    JNZ ulang
    
    ; Inisialisasi mouse
    MOV AX, 00h
    INT 33h
    CMP AX, 0
    JE no_mouse
    
    ; Tampilkan cursor mouse
    MOV AX, 01h
    INT 33h
    
    ; Baca posisi mouse
    MOV AX, 03h
    INT 33h
    ; BX = button status, CX = X position, DX = Y position
    
no_mouse:
    MOV AH, 4Ch
    INT 21h
END START
\end{verbatim}

\section{Latihan}
\begin{enumerate}
\item Buat program yang membandingkan dua bilangan
\item Buat program yang menghitung faktorial menggunakan LOOP
\item Buat program yang menampilkan angka 1-10 menggunakan perulangan
\item Buat program yang merespon klik mouse
\end{enumerate}

\section{Tugas}
\begin{itemize}
\item Buat program kalkulator dengan menu percabangan
\item Implementasikan algoritma sorting sederhana dengan perulangan
\item Buat program game tebak angka dengan mouse
\item Dokumentasikan penggunaan instruksi percabangan dan perulangan
\end{itemize}

\section{Referensi}
\begin{itemize}
\item Hyde, Randall. \textit{The Art of Assembly Language}, 2nd ed., No Starch Press, 2010.
\item Partoharsojo, Hartono. \textit{Tuntunan Praktis Pemrograman Assembly}, Penerbit Informatika.
\end{itemize}

