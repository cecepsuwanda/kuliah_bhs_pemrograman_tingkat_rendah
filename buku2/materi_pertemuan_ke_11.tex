\chapter{Pemrograman Modular dan Makro}

\section{Tujuan Pembelajaran}
Mahasiswa mampu:
\begin{itemize}
    \item Menjelaskan prinsip pemrograman modular serta manfaatnya terhadap kualitas perangkat lunak.
    \item Menggunakan \texttt{PROC}/\texttt{ENDP} untuk membangun prosedur (dengan/ tanpa parameter dan nilai balik) dengan disiplin penyimpanan register dan \textit{stack frame}.
    \item Membuat dan memanfaatkan makro (\texttt{MACRO}/\texttt{ENDM}) dengan parameter, termasuk bentuk bersyarat sederhana.
    \item Membedakan penggunaan prosedur vs. makro berdasarkan kebutuhan \textit{runtime}, ukuran kode, dan pemeliharaan.
    \item Mengorganisasi program multi-berkas dengan \texttt{INCLUDE}, modul, dan pustaka prosedur.
\end{itemize}

\section{Pendahuluan}
Pemrograman modular mendorong pemisahan program menjadi komponen-komponen kecil dengan tanggung jawab jelas \cite{susanto1995belajar}. Dalam assembly, modularitas dicapai melalui prosedur (\texttt{PROC}) dan dapat diperkuat dengan makro untuk mengurangi pengulangan pola kode \cite{hyde2010art}. Perbedaan fundamental: prosedur dieksekusi di \textit{runtime} (overhead \texttt{CALL}/\texttt{RET}), sedangkan makro diperluas saat perakitan (\textit{compile-time}), menghasilkan duplikasi kode yang dapat menambah ukuran biner \cite{nopi2003tutorial}.

\section{Konsep Pemrograman Modular}
\subsection{Definisi dan keuntungan}
Modularitas meningkatkan \textit{readability}, \textit{testability}, dan \textit{reusability}. Komponen yang terdefinisi baik mempercepat \textit{debugging} dan memudahkan perawatan.

\subsubsection{Prinsip modularitas}
\begin{itemize}
  \item \textbf{Cohesion}: setiap modul fokus pada satu tanggung jawab.
  \item \textbf{Coupling}: minimalkan ketergantungan antar modul.
  \item \textbf{Kontrak}: dokumentasikan antarmuka (parameter, return, register yang dimodifikasi).
\end{itemize}

\subsubsection{Organisasi kode}
Pisahkan deklarasi data, prosedur utilitas, dan logika utama. Gunakan penamaan konsisten dan komentar yang menerangkan tujuan.

\subsection{Prosedur dengan PROC/ENDP}
\subsubsection{Sintaks dan deklarasi}
\begin{verbatim}
NamaProsedur PROC [NEAR|FAR]
  ; body
  RET [n]
NamaProsedur ENDP
\end{verbatim}
Gunakan \texttt{NEAR} untuk segmen yang sama; \texttt{FAR} bila antar segmen.

\subsubsection{Parameter dan local variables}
Konvensi umum: parameter lewat stack; bentuk \textit{prologue/epilogue} standar:
\begin{verbatim}
push bp
mov  bp, sp
sub  sp, local_size   ; alokasikan lokal (opsional)
; akses param: [bp+4], [bp+6], ...
; akses lokal: [bp-2], [bp-4], ...
mov  sp, bp
pop  bp
ret  n     ; atau ret, caller cleans
\end{verbatim}

\subsubsection{Return value dan konvensi pemanggilan}
Kembalikan nilai di \texttt{AX} (umum) atau di memori. Tetapkan siapa yang menyimpan register: \textit{caller-saves} (penelepon menyimpan) atau \textit{callee-saves} (terpanggil menyimpan). Dokumentasikan jelas.

\subsection{Makro (MACRO)}
\subsubsection{Definisi dan sintaks}
\begin{verbatim}
NamaMakro MACRO [param1[, param2, ...]]
  ; ekspansi
ENDM
\end{verbatim}
Makro memperluas teks pada saat perakitan. Parameter menggantikan token dalam ekspansi.

\subsubsection{Makro berparameter dan bersyarat}
Gunakan arahan perakitan bersyarat (bergantung assembler) untuk variasi ekspansi. Perhatikan bahwa makro tidak memiliki \textit{runtime context} seperti stack kecuali disimulasikan.

\subsubsection{Perbedaan makro vs. prosedur}
Makro: tanpa overhead \texttt{CALL}/\texttt{RET}, namun memperbesar ukuran biner jika sering digunakan. Prosedur: lebih kecil di biner jika dipanggil berulang, namun memiliki overhead panggilan.

\section{Pendalaman}\label{sec:modular-pendalaman}
\subsection{Organisasi proyek multi-berkas}
Pisahkan modul berdasarkan domain (mis. \texttt{io.asm}, \texttt{math.asm}, \texttt{ui.asm}) dan satu berkas utama \texttt{main.asm}. Gunakan \texttt{PUBLIC}/\texttt{EXTRN} (TASM/MASM) untuk mengekspor/mengimpor simbol lintas-berkas. Pastikan kesesuaian prototipe (ukuran parameter) dan konvensi pemanggilan. \cite{borland1990tasm}

\subsection{Makro tingkat lanjut}
Makro dengan parameter mendukung substitusi token dan kondisi perakitan (bergantung assembler). Hindari makro yang mengubah konteks tak terduga (mis., register) tanpa mendokumentasikan efek samping. Untuk debug, sediakan makro \texttt{TRACE} yang dapat diaktif/nonaktifkan \textit{compile-time}. \cite{nasm_manual}

\subsection{Pedoman API prosedur}
Dokumentasikan: parameter (urutan, ukuran, interpretasi bertanda/tak bertanda), return value, register yang dimodifikasi, dan prasyarat segmen (\texttt{DS}/\texttt{ES}). Sertakan kasus tepi dan perilaku error (mis., mengembalikan kode di \texttt{AX}). \cite{osdev_wiki}

\subsection{Strategi pengujian}
Bangun program uji kecil per modul yang memverifikasi kontrak rutinitas. Untuk rutinitas murni (tanpa I/O), validasi tabel-kasus. Gunakan debugger (TD) untuk memeriksa state sebelum/sesudah panggilan. \cite{borland1990tasm}

\subsection{Organisasi Program Modular}
\subsubsection{Struktur file dan include}
Kumpulkan makro umum dalam berkas \texttt{.INC}; sertakan dengan \texttt{INCLUDE}. Simpan prosedur utilitas dalam modul terpisah untuk \textit{link}-time reuse.

\subsubsection{Library routines dan dependencies}
Buat pustaka prosedur (sekumpulan \texttt{.OBJ}) dan tautkan seperlunya. Dokumentasikan ketergantungan dan urutan inisialisasi.

\section{Praktikum}
\begin{enumerate}
  \item Bangun prosedur \texttt{print\_str} (\texttt{AH=09h}) dengan prologue/epilogue rapi.
  \item Prosedur berparameter: \texttt{add16(a,b)} mengembalikan hasil di \texttt{AX}; \texttt{RET 4} (callee cleans) vs \texttt{ADD SP,4} (caller cleans).
  \item Makro sederhana: \texttt{PRINT{\$} pesan} yang memanggil \texttt{INT 21h} untuk string berakhiran \texttt{\$}.
  \item Makro berparameter: \texttt{SWAP var1, var2} untuk menukar dua word.
  \item Susun program modular yang memanfaatkan makro dan prosedur pada beberapa berkas.
\end{enumerate}

\section{Contoh Kode}
\begin{verbatim}
; Program demonstrasi prosedur dan makro
TITLE Pemrograman Modular
.MODEL SMALL
.STACK 100h

.DATA
    pesan1 DB 'Hello from procedure!$'
    pesan2 DB 'Hello from macro!$'
    nilai1 DW 15
    nilai2 DW 25
    hasil  DW ?

; Makro untuk menampilkan pesan
TAMPILKAN_PESAN MACRO pesan
    PUSH AX
    PUSH DX
    MOV  AH, 09h
    MOV  DX, OFFSET pesan
    INT  21h
    POP  DX
    POP  AX
ENDM

; Makro untuk pertukaran nilai
TUKAR_NILAI MACRO var1, var2
    PUSH AX
    MOV  AX, var1
    XCHG AX, var2
    MOV  var1, AX
    POP  AX
ENDM

.CODE
START:
    MOV AX, @DATA
    MOV DS, AX
    
    ; Panggil prosedur
    CALL TAMPILKAN_HELLO
    
    ; Gunakan makro
    TAMPILKAN_PESAN pesan2
    
    ; Pertukaran nilai menggunakan makro
    TUKAR_NILAI nilai1, nilai2
    
    ; Panggil prosedur dengan parameter
    PUSH nilai1
    PUSH nilai2
    CALL TAMBAHKAN
    ADD  SP, 4
    
    MOV AH, 4Ch
    INT 21h

; Prosedur tanpa parameter
TAMPILKAN_HELLO PROC
    PUSH AX
    PUSH DX
    MOV  AH, 09h
    MOV  DX, OFFSET pesan1
    INT  21h
    POP  DX
    POP  AX
    RET
TAMPILKAN_HELLO ENDP

; Prosedur dengan parameter via stack
TAMBAHKAN PROC
    PUSH BP
    MOV  BP, SP
    PUSH AX
    PUSH BX
    MOV  AX, [BP+6]  ; Parameter kedua
    MOV  BX, [BP+4]  ; Parameter pertama
    ADD  AX, BX
    MOV  hasil, AX
    POP  BX
    POP  AX
    POP  BP
    RET
TAMBAHKAN ENDP

END START
\end{verbatim}

\section{Latihan}
\begin{enumerate}
  \item Buat prosedur \texttt{mul8(a,b)} yang mengembalikan hasil di \texttt{AX} dan perhatikan flags.
  \item Buat makro \texttt{PRINTC ch} untuk mencetak satu karakter via \texttt{AH=02h} dan \texttt{DL}.
  \item Buat prosedur \texttt{sort\_word\_asc(ptr,n)} (bubble/selection) dengan kontrak parameter jelas.
  \item Buat makro \texttt{ASSERT cond, msg} yang mencetak pesan jika kondisi gagal (gunakan \texttt{IF} assembler jika tersedia).
\end{enumerate}

\section{Tugas}
\begin{itemize}
  \item \textbf{Library string}: Implementasikan \texttt{strlen}, \texttt{strcpy}, \texttt{strcmp} untuk satu konvensi terminator; gabungkan menjadi pustaka.
  \item \textbf{Makro debugging}: Makro untuk cetak register (AX,BX,CX,DX) dalam heksadesimal; gunakan pada titik-titik penting program.
  \item \textbf{Kalkulator modular}: Gunakan prosedur operasi dan makro UI untuk membangun kalkulator teks dengan menu.
  \item \textbf{Dokumentasi}: Tulis panduan singkat kapan memilih prosedur vs. makro pada proyek Anda.
\end{itemize}

\section{Referensi}
% Bibliography is handled by the main document
