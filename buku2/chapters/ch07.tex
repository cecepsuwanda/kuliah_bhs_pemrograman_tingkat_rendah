\documentclass[../main.tex]{subfiles}
\begin{document}
\chapter{Instruksi Logika dan Operasi Bit}

    \section{Tujuan Pembelajaran}
        Mahasiswa mampu:
        \begin{itemize}
            \item Menerapkan instruksi logika \texttt{AND}, \texttt{OR}, \texttt{XOR}, \texttt{NOT} pada data 8/16-bit.
            \item Memahami operasi bitwise, masking, set/clear bit, dan kasus penggunaan praktis.
            \item Menggunakan layanan BIOS video \texttt{INT 10h} (fungsi \texttt{02h}, \texttt{09h}, \texttt{0Ah}, \texttt{0Eh}) untuk output teks dan kontrol kursor.
            \item Menampilkan teks di layar dengan kontrol posisi dan atribut warna.
        \end{itemize}

    \section{Pendahuluan}
        Instruksi logika menyediakan manipulasi bit tingkat rendah yang fundamental untuk kontrol perangkat keras, pengolahan bendera (flag), dan optimasi \cite{8086_instruction_set_reference}. Bab ini berfokus pada operasi bit dan logika; materi I/O teks BIOS dipindahkan ke bab khusus I/O teks.

        Operasi logika dan bitwise pada Intel 8086 merupakan fondasi penting dalam pemrograman assembly yang memungkinkan:
        \begin{itemize}
            \item Manipulasi bit individual untuk kontrol perangkat keras
            \item Implementasi algoritma enkripsi sederhana
            \item Optimasi operasi aritmatika melalui shift dan rotate
            \item Pengolahan data dengan efisiensi tinggi
            \item Kontrol flag register untuk conditional branching
        \end{itemize}

        Bab ini akan membahas secara detail instruksi logika dasar, operasi shift/rotate, dan aplikasi praktis dalam pemrograman assembly.

    \section{Instruksi Logika}
        \subsection{Operasi Logika Dasar}
            \subsubsection{AND}
Operasi konjungsi bitwise: bit hasil 1 jika dan hanya jika kedua operand 1. Umum untuk \textit{masking} menghapus bit tertentu.

\textbf{Tabel Kebenaran AND:}
\begin{center}
    \begin{tabular}{|c|c|c|}
        \hline
        A & B & A AND B \\
        \hline
        0 & 0 & 0 \\
        0 & 1 & 0 \\
        1 & 0 & 0 \\
        1 & 1 & 1 \\
        \hline
    \end{tabular}
\end{center}

\begin{lstlisting}[language={[x86masm]Assembler}, caption=Instruksi AND, label={lst:and-examples}]
; Contoh AND untuk masking
mov ax, 0A5Fh        ; AX = 101001011111b
and ax, 00FFh        ; AX = AX AND 0000000011111111b = 005Fh
; Hasil: hanya 8 bit rendah yang tersisa

; Contoh AND untuk clear bit
mov al, 11111111b    ; AL = 11111111b
and al, 11110111b    ; AL = AL AND 11110111b = 11110111b
; Hasil: bit 3 di-clear (dihitung dari 0)

; Contoh AND untuk ekstraksi bit
mov ax, 1234h        ; AX = 0001001000110100b
and ax, 0008h        ; AX = AX AND 0000000000001000b
; Hasil: AX = 0008h jika bit 3 set, 0000h jika bit 3 clear

; Contoh AND dengan register
mov ax, 0F0F0h       ; AX = 0F0F0h
mov bx, 00FFh        ; BX = 00FFh
and ax, bx           ; AX = 0F0F0h AND 00FFh = 00F0h
\end{lstlisting}

\textbf{Flags}: memperbarui \texttt{ZF}, \texttt{SF}, \texttt{PF}; \texttt{CF} dan \texttt{OF} direset ke 0.

            \subsubsection{OR}
Disjungsi bitwise: bit hasil 1 jika salah satu operand 1. Berguna untuk menyetel bit spesifik.

\textbf{Tabel Kebenaran OR:}
\begin{center}
    \begin{tabular}{|c|c|c|}
        \hline
        A & B & A OR B \\
        \hline
        0 & 0 & 0 \\
        0 & 1 & 1 \\
        1 & 0 & 1 \\
        1 & 1 & 1 \\
        \hline
    \end{tabular}
\end{center}

\begin{lstlisting}[language={[x86masm]Assembler}, caption=Instruksi OR, label={lst:or-examples}]
; Contoh OR untuk set bit
mov al, 00110000b    ; AL = 00110000b
or  al, 00000101b    ; AL = AL OR 00000101b = 00110101b
; Hasil: bit 0 dan 2 di-set

; Contoh OR untuk menggabungkan nilai
mov ax, 1234h        ; AX = 0001001000110100b
or  ax, 8000h        ; AX = AX OR 1000000000000000b = 9234h
; Hasil: bit 15 di-set (bit tanda)

; Contoh OR untuk set multiple bit
mov bl, 00000000b    ; BL = 00000000b
or  bl, 11110000b    ; BL = BL OR 11110000b = 11110000b
; Hasil: bit 4-7 di-set

; Contoh OR dengan register
mov ax, 0F000h       ; AX = 0F000h
mov bx, 00FFh        ; BX = 00FFh
or  ax, bx           ; AX = 0F000h OR 00FFh = 0F0FFh

; Contoh OR untuk flags kombinasi
mov al, 00000001b    ; AL = 00000001b (bit 0 set)
or  al, 00000100b    ; AL = AL OR 00000100b = 00000101b
; Hasil: bit 0 dan 2 set
\end{lstlisting}

\textbf{Flags}: \texttt{ZF}, \texttt{SF}, \texttt{PF} diperbarui; \texttt{CF}, \texttt{OF} = 0.

            \subsubsection{XOR}
Eksklusif-or: 1 jika operand berbeda; \textit{involutif} (\texttt{x xor k xor k = x}). Biasa untuk enkripsi sederhana dan pembersihan register cepat.

\textbf{Tabel Kebenaran XOR:}
\begin{center}
    \begin{tabular}{|c|c|c|}
        \hline
        A & B & A XOR B \\
        \hline
        0 & 0 & 0 \\
        0 & 1 & 1 \\
        1 & 0 & 1 \\
        1 & 1 & 0 \\
        \hline
    \end{tabular}
\end{center}

\begin{lstlisting}[language={[x86masm]Assembler}, caption=Instruksi XOR, label={lst:xor-examples}]
; Contoh XOR untuk clear register (lebih cepat dari MOV)
xor ax, ax           ; AX = AX XOR AX = 0
xor bx, bx           ; BX = BX XOR BX = 0

; Contoh XOR untuk toggle bit
mov al, 00001000b    ; AL = 00001000b (bit 3 set)
xor al, 00001000b    ; AL = AL XOR 00001000b = 00000000b (bit 3 clear)
xor al, 00001000b    ; AL = AL XOR 00001000b = 00001000b (bit 3 set lagi)

; Contoh XOR untuk enkripsi sederhana
mov al, 'A'          ; AL = 01000001b (ASCII 'A')
mov bl, 0FFh         ; BL = 11111111b (kunci enkripsi)
xor al, bl           ; AL = AL XOR BL = 10111110b (hasil enkripsi)
xor al, bl           ; AL = AL XOR BL = 01000001b (dekripsi kembali ke 'A')

; Contoh XOR untuk swap nilai tanpa temporary variable
mov ax, 1234h        ; AX = 1234h
mov bx, 5678h        ; BX = 5678h
xor ax, bx           ; AX = 1234h XOR 5678h = 444Ch
xor bx, ax           ; BX = 5678h XOR 444Ch = 1234h
xor ax, bx           ; AX = 444Ch XOR 1234h = 5678h
; Hasil: AX = 5678h, BX = 1234h (nilai tertukar)

; Contoh XOR untuk masking
mov ax, 0F0F0h       ; AX = 0F0F0h
xor ax, 0FFFFh       ; AX = 0F0F0h XOR 0FFFFh = 0F0Fh
; Hasil: bit 0-7 di-inversi, bit 8-15 tetap
\end{lstlisting}

\textbf{Flags}: mirip \texttt{AND}/\texttt{OR}; \texttt{CF}, \texttt{OF} = 0.

            \subsubsection{NOT}
Komplemen bitwise (unary). Sering dipakai untuk inversi \textit{mask}.

\textbf{Tabel Kebenaran NOT:}
\begin{center}
    \begin{tabular}{|c|c|}
        \hline
        A & NOT A \\
        \hline
        0 & 1 \\
        1 & 0 \\
        \hline
    \end{tabular}
\end{center}

\begin{lstlisting}[language={[x86masm]Assembler}, caption=Instruksi NOT, label={lst:not-examples}]
; Contoh NOT untuk inversi mask
mov bl, 11110000b    ; BL = 11110000b
not bl               ; BL = NOT BL = 00001111b
; Hasil: bit 0-3 menjadi 1, bit 4-7 menjadi 0

; Contoh NOT untuk two's complement (bersama dengan INC)
mov ax, 1234h        ; AX = 1234h
not ax               ; AX = NOT AX = EDCBh
inc ax               ; AX = AX + 1 = EDCDh
; Hasil: AX = -1234h (two's complement)

; Contoh NOT untuk clear semua bit
mov al, 0FFh         ; AL = 11111111b
not al               ; AL = NOT AL = 00000000b

; Contoh NOT untuk set semua bit
mov al, 00h          ; AL = 00000000b
not al               ; AL = NOT AL = 11111111b

; Contoh NOT dengan register 16-bit
mov ax, 0AAAAh       ; AX = 1010101010101010b
not ax               ; AX = NOT AX = 5555h = 0101010101010101b

; Contoh NOT untuk mask inversi
mov bl, 00111100b    ; BL = 00111100b (mask untuk bit 2-5)
not bl               ; BL = NOT BL = 11000011b (mask untuk bit 0-1,6-7)
\end{lstlisting}

\textbf{Flags}: tidak terpengaruh (kecuali implementasi spesifik; pada 8086, flags tidak terdefinisi diubah untuk NOT—anggap tidak relevan).

            \subsubsection{Instruksi SHIFT dan ROTATE}
Operasi geser dan rotasi penting untuk manipulasi bit tingkat rendah, multiplikasi/divisi basis 2, serta ekstraksi bit-field.

\begin{itemize}
    \item \textbf{SHL/SAL}: geser ke kiri; mengalikan tak bertanda dengan 2 per langkah; bit keluar ke \texttt{CF}.
    \item \textbf{SHR radi}: geser ke kanan tak bertanda; membagi dengan 2 per langkah; bit keluar ke \texttt{CF}.
    \item \textbf{SAR}: geser kanan bertanda; mempertahankan bit tanda; membagi bertanda dengan pembulatan ke bawah.
    \item \textbf{ROL/ROR}: rotasi tanpa melalui carry; \textbf{RCL/RCR}: rotasi melalui carry (memakai \texttt{CF} sebagai bit tambahan).
\end{itemize}

\begin{lstlisting}[language={[x86masm]Assembler}, caption=Instruksi SHIFT, label={lst:shift-examples}]
; Contoh SHL (Shift Left) - perkalian dengan 2
mov ax, 0001h        ; AX = 0001h (1 desimal)
shl ax, 1            ; AX = AX << 1 = 0002h (2 desimal), CF=0
shl ax, 2            ; AX = AX << 2 = 0008h (8 desimal), CF=0

; Contoh SHR (Shift Right) - pembagian dengan 2
mov ax, 0010h        ; AX = 0010h (16 desimal)
shr ax, 1            ; AX = AX >> 1 = 0008h (8 desimal), CF=0
shr ax, 2            ; AX = AX >> 2 = 0002h (2 desimal), CF=0

; Contoh SAR (Shift Arithmetic Right) - pembagian bertanda
mov ax, 0FFFEh       ; AX = 0FFFEh (-2 dalam two's complement)
sar ax, 1            ; AX = AX >> 1 = 0FFFFh (-1), CF=0 (mempertahankan bit tanda)

; Contoh SHL dengan count > 1
mov ax, 0003h        ; AX = 0003h (3 desimal)
shl ax, 3            ; AX = AX << 3 = 0018h (24 desimal), CF=0
; Setara dengan perkalian dengan 2^3 = 8

; Contoh SHR untuk ekstraksi bit
mov ax, 1234h        ; AX = 0001001000110100b
shr ax, 4            ; AX = AX >> 4 = 0123h (ekstrak 4 bit tinggi)
; Bit 0-3 dari hasil adalah bit 4-7 dari nilai asli
\end{lstlisting}

\begin{lstlisting}[language={[x86masm]Assembler}, caption=Instruksi ROTATE, label={lst:rotate-examples}]
; Contoh ROL (Rotate Left)
mov ax, 8001h        ; AX = 1000000000000001b
rol ax, 1            ; AX = 0000000000000011b, CF=1 (bit 15 masuk ke CF dan bit 0)

; Contoh ROR (Rotate Right)
mov ax, 0003h        ; AX = 0000000000000011b
ror ax, 1            ; AX = 1000000000000001b, CF=1 (bit 0 masuk ke CF dan bit 15)

; Contoh RCL (Rotate through Carry Left)
mov ax, 8000h        ; AX = 1000000000000000b
stc                  ; Set CF = 1
rcl ax, 1            ; AX = 0000000000000001b, CF=1 (CF masuk ke bit 0, bit 15 ke CF)

; Contoh RCR (Rotate through Carry Right)
mov ax, 0001h        ; AX = 0000000000000001b
stc                  ; Set CF = 1
rcr ax, 1            ; AX = 1000000000000000b, CF=1 (CF masuk ke bit 15, bit 0 ke CF)

; Contoh rotasi dengan count > 1
mov ax, 1234h        ; AX = 1234h
rol ax, 4            ; AX = 2341h (rotasi 4 bit ke kiri)
ror ax, 4            ; AX = 1234h (kembali ke nilai semula)
\end{lstlisting}

            \subsubsection{Ekstraksi dan Packing Bit-field}
Untuk mengekstrak bidang 3-bit mulai dari posisi \(p\): geser kanan \(p\) lalu mask \texttt{AND} dengan \texttt{0b111}. Untuk mengemas, geser nilai ke posisi yang diinginkan lalu gabungkan dengan \texttt{OR}.

\begin{lstlisting}[language={[x86masm]Assembler}, caption=Ekstraksi dan Packing Bit-field, label={lst:bitfield-examples}]
; Contoh ekstraksi bit-field 3-bit dari posisi 4
mov ax, 1234h        ; AX = 0001001000110100b
shr ax, 4            ; AX = 0000000100100011b (geser 4 bit ke kanan)
and ax, 0007h        ; AX = AX AND 0000000000000111b = 0003h
; Hasil: bit-field 3-bit dari posisi 4-6 = 011b = 3

; Contoh packing multiple bit-field ke dalam satu word
mov ax, 0000h        ; AX = 0 (start dengan clean word)
mov bx, 0005h        ; BX = 5 (nilai untuk field 1)
mov cx, 0003h        ; CX = 3 (nilai untuk field 2)
mov dx, 0007h        ; DX = 7 (nilai untuk field 3)

; Pack field 1 (5) ke bit 0-2
or ax, bx            ; AX = AX OR 5 = 0005h

; Pack field 2 (3) ke bit 3-5
shl cx, 3            ; CX = 3 << 3 = 0018h
or ax, cx            ; AX = AX OR 0018h = 001Dh

; Pack field 3 (7) ke bit 6-8
shl dx, 6            ; DX = 7 << 6 = 01C0h
or ax, dx            ; AX = AX OR 01C0h = 01DDh

; Hasil: AX = 01DDh = 0000000111011101b
; Field 1 (bit 0-2): 101b = 5
; Field 2 (bit 3-5): 011b = 3  
; Field 3 (bit 6-8): 111b = 7
\end{lstlisting}

            \subsubsection{Instruksi TEST}
Instruksi \texttt{TEST} melakukan operasi \texttt{AND} tanpa menyimpan hasil, hanya untuk menguji flag. Sangat efisien untuk testing bit.

\begin{lstlisting}[language={[x86masm]Assembler}, caption=Instruksi TEST, label={lst:test-examples}]
; Contoh TEST untuk uji bit individual
mov al, value        ; AL = nilai yang akan diuji
test al, 00001000b   ; Test bit 3 (hitung dari 0)
jz bit3_clear        ; Jika ZF=1 (bit 3 clear), lompat ke bit3_clear
jnz bit3_set         ; Jika ZF=0 (bit 3 set), lompat ke bit3_set

bit3_clear:
    ; Bit 3 clear
    jmp end_test

bit3_set:
    ; Bit 3 set
    jmp end_test

; Contoh TEST untuk uji multiple bit
test al, 11110000b   ; Test bit 4-7 (semua harus set)
jnz all_upper_set    ; Jika tidak semua set, ZF=0

; Contoh TEST untuk uji zero
test ax, ax          ; Test apakah AX = 0
jz zero_detected     ; Jika AX = 0, ZF=1

; Contoh TEST untuk uji parity
test al, 00000001b   ; Test bit 0 (parity bit)
jp even_parity       ; Jika PF=1 (jumlah bit 1 genap)
jnp odd_parity       ; Jika PF=0 (jumlah bit 1 ganjil)
\end{lstlisting}

            \subsubsection{Operasi bitwise dan aplikasinya}
\begin{itemize}
    \item \textbf{Masking}: \texttt{AND} untuk \textit{clear}, \texttt{OR} untuk \textit{set}, \texttt{XOR} untuk \textit{toggle}.
    \item \textbf{Packing/unpacking} field bit dalam satu byte/word.
    \item \textbf{Uji bit}: kombinasikan \texttt{TEST} (seperti \texttt{AND} tanpa menyimpan hasil) dan percabangan pada \texttt{ZF}.
\end{itemize}

        \subsection{Aplikasi Instruksi Logika}
            \subsubsection{Manipulasi bit dan masking}
\begin{lstlisting}[language={[x86masm]Assembler}, caption=Manipulasi Bit dan Masking, label={lst:bit-manipulation}]
; Contoh manipulasi bit individual
mov al, 11111111b    ; AL = 11111111b (semua bit set)

; Clear 2 bit terendah (bit 0-1)
and al, 11111100b    ; AL = 11111100b

; Set bit 7 (bit tertinggi)
or  al, 10000000b    ; AL = 11111100b OR 10000000b = 11111100b (sudah set)

; Toggle bit 4
xor al, 00010000b    ; AL = 11111100b XOR 00010000b = 11101100b

; Clear bit 3
and al, 11110111b    ; AL = 11101100b AND 11110111b = 11100100b

; Set multiple bit (bit 1-3)
or  al, 00001110b    ; AL = 11100100b OR 00001110b = 11101110b

; Contoh masking untuk ekstraksi
mov ax, 1234h        ; AX = 1234h
and ax, 0FF00h       ; AX = AX AND 0FF00h = 1200h (ekstrak byte tinggi)
shr ax, 8            ; AX = AX >> 8 = 0012h (pindahkan ke posisi rendah)
\end{lstlisting}

\subsubsection{Enkripsi sederhana (XOR)}
\begin{lstlisting}[language={[x86masm]Assembler}, caption=Enkripsi XOR Sederhana, label={lst:xor-encryption}]
; Program enkripsi/dekripsi XOR sederhana
org 100h

start:
    mov ax, data
    mov ds, ax
    
    ; Enkripsi string dengan kunci
    mov si, offset pesan      ; SI = alamat pesan
    mov di, offset enkripsi   ; DI = alamat buffer enkripsi
    mov cx, panjang_pesan     ; CX = panjang pesan
    mov bl, kunci             ; BL = kunci enkripsi
    
encrypt_loop:
    mov al, [si]              ; AL = karakter dari pesan
    xor al, bl                ; AL = AL XOR kunci (enkripsi)
    mov [di], al              ; Simpan hasil enkripsi
    inc si                    ; SI = SI + 1
    inc di                    ; DI = DI + 1
    loop encrypt_loop         ; Ulangi sampai CX = 0
    
    ; Dekripsi (XOR lagi dengan kunci yang sama)
    mov si, offset enkripsi   ; SI = alamat data terenkripsi
    mov di, offset dekripsi   ; DI = alamat buffer dekripsi
    mov cx, panjang_pesan     ; CX = panjang pesan
    
decrypt_loop:
    mov al, [si]              ; AL = karakter terenkripsi
    xor al, bl                ; AL = AL XOR kunci (dekripsi)
    mov [di], al              ; Simpan hasil dekripsi
    inc si                    ; SI = SI + 1
    inc di                    ; DI = DI + 1
    loop decrypt_loop         ; Ulangi sampai CX = 0
    
    ; Keluar dari program
    mov ah, 4Ch
    int 21h

data segment
    pesan         db 'HELLO WORLD', 0
    panjang_pesan dw 11
    kunci         db 0AAh
    enkripsi      db 11 dup(?)
    dekripsi      db 11 dup(?)
data ends

end start
\end{lstlisting}

\textit{Catatan}: XOR cipher lemah dan hanya untuk demonstrasi. Untuk keamanan nyata, gunakan algoritma enkripsi yang lebih kuat.

\subsubsection{Operasi set dan clear bit, perbandingan bit}
Gunakan \texttt{BT}/\texttt{BTS}/\texttt{BTR} pada prosesor lebih baru; pada 8086, gunakan \texttt{AND}/\texttt{OR}/\texttt{XOR} manual. Perbandingan bit dilakukan dengan \texttt{TEST}.

\begin{lstlisting}[language={[x86masm]Assembler}, caption=Set/Clear Bit dan Bit Testing, label={lst:bit-operations}]
; Fungsi untuk set bit pada posisi tertentu
; Input: AL = nilai, BL = posisi bit (0-7)
; Output: AL = nilai dengan bit di-set
set_bit:
    push cx
    mov cl, bl           ; CL = posisi bit
    mov ch, 1            ; CH = mask bit
    shl ch, cl           ; CH = 1 << posisi
    or al, ch            ; AL = AL OR mask
    pop cx
    ret

; Fungsi untuk clear bit pada posisi tertentu
; Input: AL = nilai, BL = posisi bit (0-7)
; Output: AL = nilai dengan bit di-clear
clear_bit:
    push cx
    mov cl, bl           ; CL = posisi bit
    mov ch, 1            ; CH = mask bit
    shl ch, cl           ; CH = 1 << posisi
    not ch               ; CH = NOT mask (inversi)
    and al, ch           ; AL = AL AND inversi mask
    pop cx
    ret

; Fungsi untuk test bit pada posisi tertentu
; Input: AL = nilai, BL = posisi bit (0-7)
; Output: ZF = 1 jika bit clear, ZF = 0 jika bit set
test_bit:
    push cx
    mov cl, bl           ; CL = posisi bit
    mov ch, 1            ; CH = mask bit
    shl ch, cl           ; CH = 1 << posisi
    test al, ch          ; Test bit dengan mask
    pop cx
    ret

; Contoh penggunaan
mov al, 01010101b       ; AL = 01010101b
mov bl, 3               ; BL = posisi bit 3
call set_bit            ; Set bit 3: AL = 01011101b

mov bl, 1               ; BL = posisi bit 1
call clear_bit          ; Clear bit 1: AL = 01011101b AND 11111101b = 01011101b

mov bl, 2               ; BL = posisi bit 2
call test_bit           ; Test bit 2: ZF = 0 (bit 2 set)
\end{lstlisting}

\section{Program Lengkap dengan Instruksi Logika}

\subsection{Program Bit Manipulator}
\begin{lstlisting}[language={[x86masm]Assembler}, caption=Program Bit Manipulator Lengkap, label={lst:bit-manipulator-program}]
org 100h

start:
    mov ax, data
    mov ds, ax
    
    ; Program demonstrasi operasi bitwise
    ; Input: nilai_awal = 0xAA55
    ; Output: berbagai manipulasi bit
    
    ; 1. Tampilkan nilai awal
    mov ax, nilai_awal        ; AX = 0xAA55 = 1010101001010101b
    
    ; 2. Operasi AND untuk masking
    mov bx, ax                ; BX = nilai awal
    and bx, 0x00FF            ; BX = BX AND 0x00FF = 0x0055
    mov hasil_and, bx         ; Simpan hasil AND
    
    ; 3. Operasi OR untuk set bit
    mov cx, ax                ; CX = nilai awal
    or  cx, 0x8000            ; CX = CX OR 0x8000 = 0x2A55
    mov hasil_or, cx          ; Simpan hasil OR
    
    ; 4. Operasi XOR untuk toggle bit
    mov dx, ax                ; DX = nilai awal
    xor dx, 0xFFFF            ; DX = DX XOR 0xFFFF = 0x55AA
    mov hasil_xor, dx         ; Simpan hasil XOR
    
    ; 5. Operasi NOT untuk inversi
    mov si, ax                ; SI = nilai awal
    not si                    ; SI = NOT SI = 0x55AA
    mov hasil_not, si         ; Simpan hasil NOT
    
    ; 6. Operasi SHIFT untuk perkalian/pembagian
    mov di, ax                ; DI = nilai awal
    shl di, 2                 ; DI = DI << 2 = 0xA954 (perkalian dengan 4)
    mov hasil_shl, di         ; Simpan hasil SHL
    
    mov bp, ax                ; BP = nilai awal
    shr bp, 2                 ; BP = BP >> 2 = 0x2A95 (pembagian dengan 4)
    mov hasil_shr, bp         ; Simpan hasil SHR
    
    ; 7. Operasi ROTATE
    mov ax, nilai_awal        ; AX = nilai awal
    rol ax, 4                 ; AX = ROL(AX, 4) = 0xA55A
    mov hasil_rol, ax         ; Simpan hasil ROL
    
    mov ax, nilai_awal        ; AX = nilai awal
    ror ax, 4                 ; AX = ROR(AX, 4) = 0x5AA5
    mov hasil_ror, ax         ; Simpan hasil ROR
    
    ; Keluar dari program
    mov ah, 4Ch
    int 21h

data segment
    nilai_awal  dw 0AA55h
    hasil_and   dw ?
    hasil_or    dw ?
    hasil_xor   dw ?
    hasil_not   dw ?
    hasil_shl   dw ?
    hasil_shr   dw ?
    hasil_rol   dw ?
    hasil_ror   dw ?
data ends

end start
\end{lstlisting}

\subsection{Program Bit Counter}
\begin{lstlisting}[language={[x86masm]Assembler}, caption=Program Penghitung Bit, label={lst:bit-counter-program}]
org 100h

start:
    mov ax, data
    mov ds, ax
    
    ; Program untuk menghitung jumlah bit 1 dalam suatu nilai
    ; Input: nilai_input = 0x5A3C
    ; Output: jumlah bit 1
    
    mov ax, nilai_input       ; AX = nilai yang akan dihitung
    mov bx, 0                 ; BX = counter untuk jumlah bit 1
    mov cx, 16                ; CX = jumlah bit yang akan dicek (16-bit)
    
count_bits:
    test ax, 1                ; Test bit 0
    jz bit_is_zero            ; Jika bit 0 = 0, lompat ke bit_is_zero
    inc bx                    ; Jika bit 0 = 1, increment counter
    
bit_is_zero:
    shr ax, 1                 ; Shift AX ke kanan untuk cek bit berikutnya
    loop count_bits           ; Ulangi sampai CX = 0
    
    ; Simpan hasil
    mov jumlah_bit_1, bx      ; Simpan jumlah bit 1
    
    ; Keluar dari program
    mov ah, 4Ch
    int 21h

data segment
    nilai_input   dw 5A3Ch    ; 0101101000111100b = 12 bit 1
    jumlah_bit_1  dw ?
data ends

end start
\end{lstlisting}

\subsection{Program Bit Pattern Generator}
\begin{lstlisting}[language={[x86masm]Assembler}, caption=Program Generator Pola Bit, label={lst:bit-pattern-program}]
org 100h

start:
    mov ax, data
    mov ds, ax
    
    ; Program untuk menghasilkan pola bit tertentu
    ; Menghasilkan pola: 1010101010101010b
    
    mov ax, 0                 ; AX = 0 (start clean)
    mov bx, 0                 ; BX = 0 (counter)
    mov cx, 8                 ; CX = 8 (jumlah pasangan bit)
    
generate_pattern:
    ; Set bit 0 (bit pertama dari pasangan)
    or ax, 1                  ; Set bit 0
    shl ax, 1                 ; Shift left untuk bit berikutnya
    
    ; Set bit 0 lagi (bit kedua dari pasangan)
    or ax, 1                  ; Set bit 0
    shl ax, 1                 ; Shift left untuk pasangan berikutnya
    
    loop generate_pattern     ; Ulangi 8 kali
    
    ; Hasil: AX = 1010101010101010b = 0xAAAA
    mov pola_bit, ax          ; Simpan pola bit
    
    ; Keluar dari program
    mov ah, 4Ch
    int 21h

data segment
    pola_bit dw ?
data ends

end start
\end{lstlisting}

\section{Kesimpulan}
Bab ini telah membahas secara detail instruksi logika dan operasi bit pada Intel 8086. Materi yang telah dipelajari meliputi:

\begin{itemize}
    \item \textbf{Instruksi Logika Dasar}: AND, OR, XOR, NOT dengan tabel kebenaran dan contoh penggunaan
    \item \textbf{Instruksi Shift dan Rotate}: SHL/SAL, SHR, SAR, ROL, ROR, RCL, RCR untuk manipulasi bit dan operasi aritmatika
    \item \textbf{Instruksi TEST}: Untuk testing bit tanpa mengubah nilai operand
    \item \textbf{Ekstraksi dan Packing Bit-field}: Teknik untuk memanipulasi field bit dalam data
    \item \textbf{Aplikasi Praktis}: Manipulasi bit, enkripsi XOR, set/clear bit, dan bit testing
    \item \textbf{Program Lengkap}: Contoh implementasi program dengan instruksi logika
\end{itemize}

Pemahaman yang baik tentang operasi logika dan bitwise akan memungkinkan:
\begin{itemize}
    \item Implementasi algoritma yang efisien
    \item Kontrol perangkat keras tingkat rendah
    \item Optimasi performa program
    \item Pengembangan sistem embedded dan driver
    \item Pemahaman mendalam tentang representasi data dalam komputer
\end{itemize}

Operasi bitwise merupakan fondasi penting dalam pemrograman assembly yang akan digunakan secara ekstensif dalam bab-bab berikutnya, terutama dalam implementasi struktur data, algoritma sorting, dan manipulasi string.

% Bagian I/O teks dipindahkan ke bab I/O Teks Dasar

\end{document}
