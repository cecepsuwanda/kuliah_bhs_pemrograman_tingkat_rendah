\documentclass[../main.tex]{subfiles}
\begin{document}
\chapter{Pemrosesan File Dasar}

    \section{Tujuan Pembelajaran}
        Mahasiswa mampu:
        \begin{itemize}
            \item Menjelaskan konsep berkas (file), sistem berkas, dan \textit{handle}-based I/O pada DOS.
            \item Menggunakan \texttt{INT 21h} untuk membuat, membuka, menutup, membaca, menulis, dan menghapus berkas.
            \item Menerapkan pola \textit{buffered I/O} sederhana dan menangani kesalahan berdasarkan kode error.
            \item Menyusun program utilitas file dasar seperti penyalinan dan pembacaan konten.
        \end{itemize}

    \section{Pendahuluan}
        Pada DOS, layanan \texttt{INT 21h} menyediakan antarmuka sistem berkas berbasiskan \textit{file handle}. Operasi utama meliputi pembuatan/ pembukaan berkas, pembacaan/penulisan data, dan penutupan. Keberhasilan/kegagalan operasi diindikasikan oleh \textit{carry flag} (CF) dan kode error di \texttt{AX} saat CF diset. Materi ini ditempatkan setelah percabangan dan perulangan agar contoh dapat menggunakan kontrol alur yang tepat.

    \section{Konsep File Handling}
        \subsection{Definisi file dan file system}
            File adalah sekumpulan byte pada media penyimpanan, diorganisasikan oleh sistem berkas (FAT pada DOS). Identifikasi via nama (8.3) dan atribut (read-only, hidden, system, archive).

        \subsubsection{File handle dan mode akses}
            Setelah berkas dibuka/dibuat, DOS mengembalikan \textit{handle} (bilangan kecil) untuk digunakan pada operasi lanjutan. Mode akses: baca (\texttt{AL=0}), tulis (\texttt{AL=1}), baca/tulis (\texttt{AL=2}); lampiran (append) dapat disimulasikan dengan memindah \textit{file pointer} ke akhir.

        \subsubsection{Atribut dan perizinan}
            Atribut ditentukan pada pembuatan atau dapat diubah menggunakan fungsi lain (di luar cakupan dasar). Gunakan atribut normal (\texttt{CX=0}) untuk kebanyakan kasus.

        \subsubsection{Error handling}
            CF=1 menunjukkan kesalahan; \texttt{AX} memuat kode (mis. file tidak ditemukan, akses ditolak). Tanggapi dengan pesan dan \textit{cleanup} sumber daya.

        \subsection{Interupsi INT 21h untuk File}
            \begin{itemize}
                \item \textbf{3Ch Create File}: \texttt{AH=3Ch}, \texttt{CX=atribut}, \texttt{DX=\&nama}; keluar: \texttt{AX=handle}.
                \item \textbf{3Dh Open File}: \texttt{AH=3Dh}, \texttt{AL=mode}, \texttt{DX=\&nama}; keluar: \texttt{AX=handle}.
                \item \textbf{3Eh Close File}: \texttt{AH=3Eh}, \texttt{BX=handle}.
                \item \textbf{3Fh Read File}: \texttt{AH=3Fh}, \texttt{BX=handle}, \\
                \texttt{CX=jumlah}, \texttt{DX=\&buffer}; keluaran: \texttt{AX=byte terbaca}.
                \item \textbf{40h Write File}: \texttt{AH=40h}, \texttt{BX=handle}, \\
                \texttt{CX=jumlah}, \texttt{DX=\&buffer}; keluaran: \texttt{AX=byte tertulis}.
                \item \textbf{41h Delete File}: \texttt{AH=41h}, \texttt{DX=\&nama}.
            \end{itemize}
            Perhatikan selalu pemeriksaan CF dan konsistensi \texttt{AX/CF}.

        \subsection{Operasi File Dasar}
            \subsubsection{Membuat dan membuka}
                Gunakan \texttt{3Ch} untuk membuat (menimpa jika ada tergantung sistem); gunakan \texttt{3Dh} untuk membuka yang sudah ada dengan mode sesuai kebutuhan.

            \subsubsection{Menutup}
                Selalu tutup \texttt{handle} dengan \texttt{3Eh} untuk mencegah kebocoran; DOS membatasi jumlah handle terbuka.

            \subsubsection{Membaca dan menulis}
                Gunakan buffer di segmen data. Jumlah byte aktual terbaca/tertulis dikembalikan di \texttt{AX}; periksa ketidaksamaan dengan \texttt{CX} (mis. akhir berkas pada baca).

            \subsubsection{Menghapus}
                \texttt{41h} menghapus berkas; pastikan ditutup sebelumnya dan atribut memungkinkan penghapusan. Materi ini ditempatkan setelah percabangan dan perulangan agar contoh dapat menggunakan kontrol alur yang tepat.

        \subsection{File Handle dan Error Handling}
            \subsubsection{Manajemen handle}
                Simpan \texttt{AX} hasil \texttt{3Ch/3Dh} ke variabel. Hindari penggunaan handle setelah ditutup. Gunakan nilai sentinel (mis. \texttt{FFFFh}) untuk menandai tidak-valid.

            \subsubsection{Kode error dan validasi}
                Tangkap kasus umum: file tidak ditemukan (\texttt{2}), akses ditolak (\texttt{5}), disk penuh, dan lain-lain. Berikan pesan informatif.

            \subsubsection{Resource cleanup}
                Pada error di tengah rangkaian operasi, lakukan penutupan handle yang sudah terbuka sebelum keluar.

            \subsubsection{Kode Error Umum INT 21h}
                Contoh: \texttt{AX=02h} (file tidak ditemukan), \texttt{03h} (path tidak ditemukan), \texttt{05h} (akses ditolak), \texttt{0Dh} (data invalid). Periksa \texttt{CF} dan \texttt{AX} setelah setiap panggilan file I/O untuk menangani kegagalan dengan baik. 
            \subsubsection{EOF dan Jumlah Byte Terbaca}
                Pada operasi baca (\texttt{AH=3Fh}), nilai kembali \texttt{AX} bisa lebih kecil dari \texttt{CX} menandakan akhir berkas atau buffer tidak penuh. Jangan mengasumsikan buffer terisi penuh. 
            \subsubsection{Buffered I/O Sederhana}
                Gunakan buffer menengah untuk menyalin berkas dalam blok (mis. 128/256/512 byte). Hindari alokasi dinamis yang sulit pada DOS; gunakan buffer statik di segmen data. 
            \subsubsection{Format Teks vs Biner}
                Untuk teks, perhatikan terminator baris (\texttt{CR LF} = \texttt{0Dh 0Ah}). Untuk biner, pertahankan ukuran yang tepat dan hindari transformasi karakter. 

        \subsection{Buffer dan Transfer Data}
            \subsubsection{Buffer dan block transfer}
                Baca/tulis dalam blok untuk efisiensi. Untuk berkas teks, pertimbangkan terminator atau pemrosesan baris.

            \subsubsection{Akses sekuensial}
                Baca berulang dari awal ke akhir. Untuk akses acak, gunakan \texttt{INT 21h, 42h} (dibahas pada bab berikutnya).






\end{document}