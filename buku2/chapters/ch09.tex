\documentclass[../main.tex]{subfiles}
\begin{document}
\chapter{I/O Teks Dasar (Mode Teks BIOS)}\label{ch:io-teks}

    \section{Tujuan Pembelajaran}
        Setelah pertemuan ini, mahasiswa mampu:
        \begin{itemize}
            \item Mengatur posisi kursor dan menulis karakter/teks di layar mode teks melalui BIOS \texttt{INT 10h}.
            \item Menampilkan teks dengan atribut warna dan memahami layout atribut karakter.
            \item Menggunakan \texttt{INT 21h, AH=02h/09h} untuk output karakter/string berakhiran \texttt{\$}.
            \item Menangani karakter khusus (CR/LF) dan perbedaan mekanisme TTY (\texttt{AH=0Eh}).
        \end{itemize}

    \section{Pendahuluan}
        Bab ini membahas input/output teks dasar menggunakan layanan BIOS dan DOS pada mode teks \cite{rbil}. Materi ini ditempatkan setelah percabangan/perulangan agar contoh interaktif dapat ditulis dengan alur program yang jelas.

    \section{Output Teks dengan BIOS \texttt{INT 10h}}
        \subsection{Set Posisi Kursor (\texttt{AH=02h})}
            \begin{itemize}
\item Masukan: \texttt{AH=02h}, \texttt{BH=page}, \texttt{DH=row}, \texttt{DL=column}
\item Efek: Memindahkan kursor ke koordinat (row, column) 0-berbasis
            \end{itemize}

        \subsection{Tulis Karakter + Atribut (\texttt{AH=09h})}
            \begin{itemize}
\item Masukan: \texttt{AH=09h}, \texttt{AL=character}, \texttt{BH=page}, \texttt{BL=attribute}, \texttt{CX=count}
\item Menulis karakter dengan atribut warna tanpa TTY-advance otomatis pada semua mode
            \end{itemize}

        \subsection{Tulis Karakter TTY (\texttt{AH=0Eh})}
            \begin{itemize}
\item Masukan: \texttt{AH=0Eh}, \texttt{AL=character}, \texttt{BH=page}, \texttt{BL=color}
\item Menulis seperti teletype: memajukan kursor dan menggulir bila perlu
            \end{itemize}

        \subsection{Contoh Program Sederhana dengan INT 10h}
            \begin{lstlisting}[language={[x86masm]Assembler}, caption=Contoh Output Teks dengan BIOS INT 10h, label={lst:bios-text-output}]
org 100h

; Set posisi kursor ke baris 5, kolom 10
mov ah, 02h          ; Fungsi set cursor position
mov bh, 0            ; Page 0
mov dh, 5            ; Row 5
mov dl, 10           ; Column 10
int 10h              ; Panggil BIOS

; Tulis karakter 'H' dengan atribut warna
mov ah, 09h          ; Fungsi write character with attribute
mov al, 'H'          ; Karakter yang akan ditulis
mov bh, 0            ; Page 0
mov bl, 0Eh          ; Attribute: Yellow on black
mov cx, 1            ; Jumlah karakter
int 10h              ; Panggil BIOS

; Keluar program
int 20h
            \end{lstlisting}

    \section{Output Teks dengan DOS \texttt{INT 21h}}
        \subsection{Tulis Karakter (\texttt{AH=02h})}
            Fungsi ini menulis satu karakter ke output standar (biasanya layar).
            \begin{itemize}
\item Masukan: \texttt{AH=02h}, \texttt{DL=karakter}
\item Efek: Karakter ditampilkan di posisi kursor saat ini
            \end{itemize}

            \begin{lstlisting}[language={[x86masm]Assembler}, caption=Contoh Tulis Karakter dengan INT 21h AH=02h, label={lst:dos-write-char}]
org 100h

; Tulis karakter 'A'
mov ah, 02h          ; Fungsi write character
mov dl, 'A'          ; Karakter yang akan ditulis
int 21h              ; Panggil DOS

; Tulis karakter newline (CR + LF)
mov dl, 0Dh          ; Carriage Return
int 21h
mov dl, 0Ah          ; Line Feed
int 21h

; Keluar program
mov ah, 4Ch
int 21h
            \end{lstlisting}

        \subsection{Tulis String (\texttt{AH=09h})}
            Fungsi ini menulis string yang diakhiri dengan karakter \texttt{\$} ke output standar.
            \begin{itemize}
\item Masukan: \texttt{AH=09h}, \texttt{DX=offset string}
\item String harus diakhiri dengan \texttt{\$}
\item Efek: String ditampilkan di layar
            \end{itemize}

            \begin{lstlisting}[language={[x86masm]Assembler}, caption=Contoh Tulis String dengan INT 21h AH=09h, label={lst:dos-write-string}]
org 100h

; Tulis string ke layar
mov ah, 09h          ; Fungsi write string
mov dx, offset pesan ; DX = alamat string
int 21h              ; Panggil DOS

; Keluar program
mov ah, 4Ch
int 21h

pesan db 'Hello, World!$'
            \end{lstlisting}

    \section{Atribut Warna dan Layout Karakter}
        \subsection{Format Atribut Karakter}
            Setiap karakter pada mode teks memiliki atribut yang menentukan warna foreground dan background. Format atribut adalah 8-bit dengan struktur sebagai berikut:
            
            \begin{table}[H]
                \centering
                \caption{Format Atribut Karakter Mode Teks}
                \begin{tabular}{|p{1.5cm}|p{2cm}|p{7cm}|}
                \hline
                \textbf{Bit} & \textbf{Nilai} & \textbf{Deskripsi} \\
                \hline
                Bit 0-3 & 0-15 & Foreground color (warna teks) \\
                \hline
                Bit 4-6 & 0-7 & Background color (warna latar) \\
                \hline
                Bit 7 & 0/1 & Blinking (0=normal, 1=berkedip) \\
                \hline
                \end{tabular}
            \end{table}

            \subsection{Kode Warna Standar}
                \begin{table}[H]
                    \centering
                    \caption{Kode Warna Mode Teks}
                    \begin{tabular}{|p{1.5cm}|p{2cm}|p{2cm}|p{5cm}|}
                    \hline
                    \textbf{Kode} & \textbf{Nama} & \textbf{Hex} & \textbf{Keterangan} \\
                    \hline
                    0 & Black & 0h & Hitam \\
                    \hline
                    1 & Blue & 1h & Biru \\
                    \hline
                    2 & Green & 2h & Hijau \\
                    \hline
                    3 & Cyan & 3h & Cyan \\
                    \hline
                    4 & Red & 4h & Merah \\
                    \hline
                    5 & Magenta & 5h & Magenta \\
                    \hline
                    6 & Brown & 6h & Coklat \\
                    \hline
                    7 & Light Gray & 7h & Abu-abu terang \\
                    \hline
                    8 & Dark Gray & 8h & Abu-abu gelap \\
                    \hline
                    9 & Light Blue & 9h & Biru terang \\
                    \hline
                    10 & Light Green & Ah & Hijau terang \\
                    \hline
                    11 & Light Cyan & Bh & Cyan terang \\
                    \hline
                    12 & Light Red & Ch & Merah terang \\
                    \hline
                    13 & Light Magenta & Dh & Magenta terang \\
                    \hline
                    14 & Yellow & Eh & Kuning \\
                    \hline
                    15 & White & Fh & Putih \\
                    \hline
                    \end{tabular}
                \end{table}

            \subsection{Contoh Kombinasi Atribut}
                \begin{lstlisting}[language={[x86masm]Assembler}, caption=Contoh Penggunaan Atribut Warna, label={lst:color-attributes}]
; Contoh berbagai kombinasi atribut warna
mov ah, 09h          ; Fungsi write character with attribute
mov al, 'A'          ; Karakter 'A'
mov bh, 0            ; Page 0
mov cx, 1            ; Jumlah karakter

; Kuning pada hitam (0Eh = 14 = Yellow)
mov bl, 0Eh          ; Yellow on black
int 10h

; Putih pada biru (1Fh = 31 = White on Blue)
mov bl, 1Fh          ; White on blue
int 10h

; Merah terang pada hijau dengan blinking (C2h = 194)
mov bl, 0C2h         ; Light red on green with blinking
int 10h
                \end{lstlisting}

    \section{Karakter Khusus dan Kontrol Layar}
        \subsection{Karakter Kontrol Umum}
            Dalam pemrograman teks, terdapat beberapa karakter khusus yang memiliki fungsi kontrol:
            
            \begin{table}[H]
                \centering
                \caption{Karakter Kontrol Layar}
                \begin{tabular}{|p{1.5cm}|p{2cm}|p{2cm}|p{5cm}|}
                \hline
                \textbf{Kode} & \textbf{Nama} & \textbf{ASCII} & \textbf{Fungsi} \\
                \hline
                0Dh & CR & 13 & Carriage Return (kembali ke awal baris) \\
                \hline
                0Ah & LF & 10 & Line Feed (pindah ke baris berikutnya) \\
                \hline
                08h & BS & 8 & Backspace (mundur satu karakter) \\
                \hline
                07h & BEL & 7 & Bell (bunyi beep) \\
                \hline
                09h & TAB & 9 & Tab horizontal \\
                \hline
                1Bh & ESC & 27 & Escape (karakter kontrol) \\
                \hline
                \end{tabular}
            \end{table}

        \subsection{Perbedaan Mekanisme TTY vs Non-TTY}
            \subsubsection{TTY Mode (AH=0Eh)}
                TTY mode meniru perilaku teletypewriter klasik:
                \begin{itemize}
                    \item Kursor otomatis maju setelah menulis karakter
                    \item Otomatis wrap ke baris berikutnya jika mencapai akhir baris
                    \item Otomatis scroll ke atas jika mencapai akhir layar
                    \item Karakter kontrol seperti CR/LF diproses secara khusus
                \end{itemize}

            \subsubsection{Non-TTY Mode (AH=09h)}
                Non-TTY mode memberikan kontrol penuh:
                \begin{itemize}
                    \item Kursor tidak bergerak otomatis
                    \item Tidak ada wrap atau scroll otomatis
                    \item Karakter kontrol ditampilkan sebagai karakter biasa
                    \item Memerlukan kontrol manual posisi kursor
                \end{itemize}

        \subsection{Contoh Perbedaan TTY vs Non-TTY}
            \begin{lstlisting}[language={[x86masm]Assembler}, caption=Perbandingan TTY vs Non-TTY, label={lst:tty-comparison}]
; Contoh TTY Mode (AH=0Eh)
mov ah, 0Eh          ; Fungsi TTY output
mov al, 'H'          ; Karakter 'H'
mov bh, 0            ; Page 0
mov bl, 0Eh          ; Yellow on black
int 10h              ; Kursor otomatis maju

mov al, 'i'          ; Karakter 'i'
int 10h              ; Kursor otomatis maju lagi

mov al, 0Dh          ; Carriage Return
int 10h              ; Kursor kembali ke awal baris
mov al, 0Ah          ; Line Feed
int 10h              ; Kursor pindah ke baris berikutnya

; Contoh Non-TTY Mode (AH=09h)
mov ah, 09h          ; Fungsi write character with attribute
mov al, 'H'          ; Karakter 'H'
mov bh, 0            ; Page 0
mov bl, 0Eh          ; Yellow on black
mov cx, 1            ; Jumlah karakter
int 10h              ; Kursor TIDAK bergerak

; Harus manual set posisi kursor
mov ah, 02h          ; Set cursor position
mov bh, 0            ; Page 0
mov dh, 5            ; Row 5
mov dl, 11           ; Column 11 (posisi berikutnya)
int 10h

mov ah, 09h          ; Tulis karakter berikutnya
mov al, 'i'
mov cx, 1
int 10h              ; Kursor tetap di posisi yang sama
            \end{lstlisting}

    \section{Contoh Program Lengkap}
        \subsection{Program Menampilkan Teks dengan Warna}
            \begin{lstlisting}[language={[x86masm]Assembler}, caption=Program Lengkap Output Teks Berwarna, label={lst:colored-text-program}]
org 100h

start:
    ; Clear screen dengan scrolling
    mov ah, 06h          ; Fungsi scroll up
    mov al, 0            ; Clear entire screen
    mov bh, 07h          ; Attribute: white on black
    mov cx, 0            ; Upper left corner (0,0)
    mov dx, 184Fh        ; Lower right corner (24,79)
    int 10h              ; Panggil BIOS
    
    ; Set posisi kursor ke tengah layar
    mov ah, 02h          ; Fungsi set cursor
    mov bh, 0            ; Page 0
    mov dh, 12           ; Row 12 (tengah vertikal)
    mov dl, 30           ; Column 30
    int 10h
    
    ; Tulis string dengan warna
    mov si, offset pesan ; SI = alamat pesan
    mov bl, 0Eh          ; Attribute: yellow on black
    
tulis_loop:
    lodsb                ; Load byte dari DS:SI ke AL
    cmp al, 0            ; Cek akhir string
    je selesai           ; Jika 0, selesai
    
    mov ah, 0Eh          ; Fungsi TTY output
    int 10h              ; Tampilkan karakter
    jmp tulis_loop       ; Ulangi
    
selesai:
    ; Tunggu keypress
    mov ah, 00h
    int 16h
    
    ; Keluar program
    mov ah, 4Ch
    int 21h

pesan db 'Program Assembly - I/O Teks Dasar', 0
            \end{lstlisting}

        \subsection{Program Demonstrasi Atribut Warna}
            \begin{lstlisting}[language={[x86masm]Assembler}, caption=Program Demonstrasi Atribut Warna, label={lst:color-demo}]
org 100h

start:
    ; Clear screen
    mov ah, 06h
    mov al, 0
    mov bh, 07h
    mov cx, 0
    mov dx, 184Fh
    int 10h
    
    ; Tampilkan berbagai kombinasi warna
    mov dh, 2            ; Mulai dari baris 2
    mov dl, 10           ; Kolom 10
    
demo_loop:
    ; Set posisi kursor
    mov ah, 02h
    mov bh, 0
    int 10h
    
    ; Tulis karakter dengan atribut
    mov ah, 09h
    mov al, 'A'
    mov bh, 0
    mov bl, [offset warna_tabel + si] ; Ambil warna dari tabel
    mov cx, 1
    int 10h
    
    ; Tulis nama warna
    mov ah, 09h
    mov dx, offset nama_warna[si]
    int 21h
    
    ; Pindah ke baris berikutnya
    inc dh
    add si, 2            ; Next warna (2 bytes per entry)
    cmp si, 16           ; 8 warna x 2 bytes
    jl demo_loop
    
    ; Tunggu keypress
    mov ah, 00h
    int 16h
    
    ; Keluar
    mov ah, 4Ch
    int 21h

; Tabel warna dan nama
warna_tabel db 0Eh, 0Ch, 0Ah, 09h, 0Bh, 0Dh, 06h, 0Fh
nama_warna db 'Yellow$', 0Dh, 0Ah
           db 'Light Red$', 0Dh, 0Ah
           db 'Light Green$', 0Dh, 0Ah
           db 'Light Blue$', 0Dh, 0Ah
           db 'Light Cyan$', 0Dh, 0Ah
           db 'Light Magenta$', 0Dh, 0Ah
           db 'Brown$', 0Dh, 0Ah
           db 'White$', 0Dh, 0Ah
            \end{lstlisting}

        \subsection{Program Input Karakter dengan Echo}
            \begin{lstlisting}[language={[x86masm]Assembler}, caption=Program Input Karakter dengan Echo, label={lst:input-echo}]
org 100h

start:
    ; Clear screen
    mov ah, 06h
    mov al, 0
    mov bh, 07h
    mov cx, 0
    mov dx, 184Fh
    int 10h
    
    ; Set posisi awal
    mov dh, 10
    mov dl, 20
    
    ; Tampilkan prompt
    mov ah, 09h
    mov dx, offset prompt
    int 21h
    
input_loop:
    ; Set posisi kursor
    mov ah, 02h
    mov bh, 0
    int 10h
    
    ; Input karakter
    mov ah, 00h
    int 16h              ; Input tanpa echo
    
    ; Cek apakah Enter (0Dh)
    cmp al, 0Dh
    je exit_program
    
    ; Echo karakter dengan warna
    mov ah, 0Eh
    mov bl, 0Eh          ; Yellow
    int 10h
    
    ; Maju kursor
    inc dl
    cmp dl, 60           ; Batas kolom
    jl input_loop
    
    ; Pindah ke baris berikutnya
    mov dl, 20
    inc dh
    jmp input_loop
    
exit_program:
    ; Keluar
    mov ah, 4Ch
    int 21h

prompt db 'Ketik karakter (Enter untuk keluar): $'
            \end{lstlisting}

    \section{Kesimpulan}
        Bab ini telah membahas konsep-konsep fundamental dalam pemrograman I/O teks dasar menggunakan Intel 8086 assembly language. Berikut adalah ringkasan poin-poin penting yang telah dipelajari:

        \subsection{Penguasaan Interrupt BIOS dan DOS}
            Mahasiswa telah memahami perbedaan dan penggunaan dua sistem interrupt utama:
            \begin{itemize}
                \item \textbf{BIOS INT 10h}: Memberikan kontrol langsung terhadap hardware video dengan fungsi-fungsi seperti positioning kursor (\texttt{AH=02h}), menulis karakter dengan atribut (\texttt{AH=09h}), dan output TTY (\texttt{AH=0Eh})
                \item \textbf{DOS INT 21h}: Menyediakan interface tingkat tinggi untuk output karakter (\texttt{AH=02h}) dan string (\texttt{AH=09h}) yang lebih mudah digunakan namun dengan kontrol terbatas
            \end{itemize}

        \subsection{Pemahaman Sistem Atribut Warna}
            Sistem atribut karakter 8-bit telah dipelajari secara mendalam, meliputi:
            \begin{itemize}
                \item Struktur bit untuk foreground color (bit 0-3), background color (bit 4-6), dan blinking (bit 7)
                \item 16 warna standar yang tersedia dalam mode teks, dari hitam (0h) hingga putih (Fh)
                \item Kemampuan untuk membuat kombinasi warna yang menarik dan efek visual seperti blinking
            \end{itemize}

        \subsection{Penguasaan Karakter Kontrol}
            Mahasiswa telah memahami pentingnya karakter kontrol dalam pemrograman teks:
            \begin{itemize}
                \item Karakter CR/LF untuk kontrol baris baru dan positioning
                \item Karakter khusus seperti BS, BEL, TAB, dan ESC untuk berbagai fungsi kontrol
                \item Perbedaan antara TTY mode yang otomatis memproses karakter kontrol dan non-TTY mode yang memerlukan kontrol manual
            \end{itemize}

        \subsection{Pemahaman Mode Operasi}
            Perbedaan fundamental antara TTY dan non-TTY mode telah dipahami:
            \begin{itemize}
                \item \textbf{TTY Mode}: Meniru perilaku teletypewriter dengan kursor otomatis, wrap, dan scroll
                \item \textbf{Non-TTY Mode}: Memberikan kontrol penuh namun memerlukan manajemen manual posisi kursor
                \item Pemilihan mode yang tepat berdasarkan kebutuhan aplikasi
            \end{itemize}

        \subsection{Aplikasi Praktis}
            Melalui contoh-contoh program yang telah dipelajari, mahasiswa mampu:
            \begin{itemize}
                \item Membuat program output teks dengan kontrol posisi dan warna yang tepat
                \item Mengimplementasikan program interaktif dengan input dan echo berwarna
                \item Menggabungkan berbagai konsep untuk membuat aplikasi yang lebih kompleks
            \end{itemize}

        \subsection{Pentingnya Fondasi I/O Teks}
            Pemahaman I/O teks dasar ini menjadi fondasi penting untuk:
            \begin{itemize}
                \item Pengembangan program assembly yang lebih kompleks
                \item Pemahaman konsep input/output dalam sistem komputer
                \item Persiapan untuk mempelajari topik lanjutan seperti grafik dan interface pengguna
            \end{itemize}

        Dengan penguasaan materi ini, mahasiswa telah memiliki kemampuan dasar yang solid untuk mengembangkan program assembly yang dapat berinteraksi dengan pengguna melalui interface teks yang menarik dan fungsional.

    

\end{document}
