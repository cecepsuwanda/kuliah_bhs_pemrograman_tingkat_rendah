\documentclass[../main.tex]{subfiles}
\begin{document}
\chapter{Pemrosesan File Dasar}

\section{Tujuan Pembelajaran}
Mahasiswa mampu:
\begin{itemize}
    \item Menjelaskan konsep berkas (file), sistem berkas, dan \textit{handle}-based I/O pada DOS.
    \item Menggunakan \texttt{INT 21h} untuk membuat, membuka, menutup, membaca, menulis, dan menghapus berkas.
    \item Menerapkan pola \textit{buffered I/O} sederhana dan menangani kesalahan berdasarkan kode error.
    \item Menyusun program utilitas file dasar seperti penyalinan dan pembacaan konten.
\end{itemize}

\section{Pendahuluan}
Pada DOS, layanan \texttt{INT 21h} menyediakan antarmuka sistem berkas berbasiskan \textit{file handle}. Operasi utama meliputi pembuatan/ pembukaan berkas, pembacaan/penulisan data, dan penutupan. Keberhasilan/kegagalan operasi diindikasikan oleh \textit{carry flag} (CF) dan kode error di \texttt{AX} saat CF diset.

\section{Konsep File Handling}
\subsection{Definisi file dan file system}
File adalah sekumpulan byte pada media penyimpanan, diorganisasikan oleh sistem berkas (FAT pada DOS). Identifikasi via nama (8.3) dan atribut (read-only, hidden, system, archive).

\subsubsection{File handle dan mode akses}
Setelah berkas dibuka/dibuat, DOS mengembalikan \textit{handle} (bilangan kecil) untuk digunakan pada operasi lanjutan. Mode akses: baca (\texttt{AL=0}), tulis (\texttt{AL=1}), baca/tulis (\texttt{AL=2}); lampiran (append) dapat disimulasikan dengan memindah \textit{file pointer} ke akhir.

\subsubsection{Atribut dan perizinan}
Atribut ditentukan pada pembuatan atau dapat diubah menggunakan fungsi lain (di luar cakupan dasar). Gunakan atribut normal (\texttt{CX=0}) untuk kebanyakan kasus.

\subsubsection{Error handling}
CF=1 menunjukkan kesalahan; \texttt{AX} memuat kode (mis. file tidak ditemukan, akses ditolak). Tanggapi dengan pesan dan \textit{cleanup} sumber daya.

\subsection{Interupsi INT 21h untuk File}
\begin{itemize}
  \item \textbf{3Ch Create File}: \texttt{AH=3Ch}, \texttt{CX=atribut}, \texttt{DX=\&nama}; keluar: \texttt{AX=handle}.
  \item \textbf{3Dh Open File}: \texttt{AH=3Dh}, \texttt{AL=mode}, \texttt{DX=\&nama}; keluar: \texttt{AX=handle}.
  \item \textbf{3Eh Close File}: \texttt{AH=3Eh}, \texttt{BX=handle}.
  \item \textbf{3Fh Read File}: \texttt{AH=3Fh}, \texttt{BX=handle}, \\
  \texttt{CX=jumlah}, \texttt{DX=\&buffer}; keluaran: \texttt{AX=byte terbaca}.
  \item \textbf{40h Write File}: \texttt{AH=40h}, \texttt{BX=handle}, \\
  \texttt{CX=jumlah}, \texttt{DX=\&buffer}; keluaran: \texttt{AX=byte tertulis}.
  \item \textbf{41h Delete File}: \texttt{AH=41h}, \texttt{DX=\&nama}.
\end{itemize}
Perhatikan selalu pemeriksaan CF dan konsistensi \texttt{AX/CF}.

\subsection{Operasi File Dasar}
\subsubsection{Membuat dan membuka}
Gunakan \texttt{3Ch} untuk membuat (menimpa jika ada tergantung sistem); gunakan \texttt{3Dh} untuk membuka yang sudah ada dengan mode sesuai kebutuhan.

\subsubsection{Menutup}
Selalu tutup \texttt{handle} dengan \texttt{3Eh} untuk mencegah kebocoran; DOS membatasi jumlah handle terbuka.

\subsubsection{Membaca dan menulis}
Gunakan buffer di segmen data. Jumlah byte aktual terbaca/tertulis dikembalikan di \texttt{AX}; periksa ketidaksamaan dengan \texttt{CX} (mis. akhir berkas pada baca).

\subsubsection{Menghapus}
\texttt{41h} menghapus berkas; pastikan ditutup sebelumnya dan atribut memungkinkan penghapusan.

\subsection{File Handle dan Error Handling}
\subsubsection{Manajemen handle}
Simpan \texttt{AX} hasil \texttt{3Ch/3Dh} ke variabel. Hindari penggunaan handle setelah ditutup. Gunakan nilai sentinel (mis. \texttt{FFFFh}) untuk menandai tidak-valid.

\subsubsection{Kode error dan validasi}
Tangkap kasus umum: file tidak ditemukan (\texttt{2}), akses ditolak (\texttt{5}), disk penuh, dan lain-lain. Berikan pesan informatif.

\subsubsection{Resource cleanup}
Pada error di tengah rangkaian operasi, lakukan penutupan handle yang sudah terbuka sebelum keluar.

\subsubsection{Kode Error Umum INT 21h}
Contoh: \texttt{AX=02h} (file tidak ditemukan), \texttt{03h} (path tidak ditemukan), \texttt{05h} (akses ditolak), \texttt{0Dh} (data invalid). Periksa \texttt{CF} dan \texttt{AX} setelah setiap panggilan file I/O untuk menangani kegagalan dengan baik. 
\subsubsection{EOF dan Jumlah Byte Terbaca}
Pada operasi baca (\texttt{AH=3Fh}), nilai kembali \texttt{AX} bisa lebih kecil dari \texttt{CX} menandakan akhir berkas atau buffer tidak penuh. Jangan mengasumsikan buffer terisi penuh. 
\subsubsection{Buffered I/O Sederhana}
Gunakan buffer menengah untuk menyalin berkas dalam blok (mis. 128/256/512 byte). Hindari alokasi dinamis yang sulit pada DOS; gunakan buffer statik di segmen data. 
\subsubsection{Format Teks vs Biner}
Untuk teks, perhatikan terminator baris (\texttt{CR LF} = \texttt{0Dh 0Ah}). Untuk biner, pertahankan ukuran yang tepat dan hindari transformasi karakter. 

\subsection{Buffer dan Transfer Data}
\subsubsection{Buffer dan block transfer}
Baca/tulis dalam blok untuk efisiensi. Untuk berkas teks, pertimbangkan terminator atau pemrosesan baris.

\subsubsection{Akses sekuensial}
Baca berulang dari awal ke akhir. Untuk acak, gunakan \texttt{INT 21h, 42h} (dibahas pertemuan lanjutan).

\section{Praktikum}
\begin{enumerate}
  \item Program membuat berkas baru dan menulis satu baris teks.
  \item Program membaca kembali isi berkas dan menampilkannya.
  \item Program menyalin berkas (baca-blok/tulis-blok hingga akhir berkas).
  \item Program dengan penanganan error: uji nama tidak valid, akses ditolak, dan disk penuh (simulasi).
\end{enumerate}

\section{Contoh Kode}
\begin{verbatim}
; Program demonstrasi operasi file dasar
TITLE Pemrosesan File
.MODEL SMALL
.STACK 100h

.DATA
    nama_file   DB 'test.txt', 0
    data_tulis  DB 'Hello, World!', 13, 10
    data_baca   DB 100 DUP(?)
    file_handle DW ?
    pesan_sukses DB 'File berhasil dibuat!$'
    pesan_error  DB 'Error dalam operasi file!$'
    pesan_baca   DB 'Data dari file: $'

.CODE
START:
    MOV AX, @DATA
    MOV DS, AX
    
    ; Buat file baru
    MOV AH, 3Ch
    MOV CX, 0        ; atribut normal
    MOV DX, OFFSET nama_file
    INT 21h
    JC  error_handler
    MOV file_handle, AX
    
    ; Tampilkan pesan sukses
    MOV AH, 09h
    MOV DX, OFFSET pesan_sukses
    INT 21h
    
    ; Tulis data ke file
    MOV AH, 40h
    MOV BX, file_handle
    MOV CX, 15       ; jumlah byte
    MOV DX, OFFSET data_tulis
    INT 21h
    JC  error_handler
    
    ; Tutup file
    MOV AH, 3Eh
    MOV BX, file_handle
    INT 21h
    JC  error_handler
    
    ; Buka file untuk dibaca
    MOV AH, 3Dh
    MOV AL, 0        ; mode read
    MOV DX, OFFSET nama_file
    INT 21h
    JC  error_handler
    MOV file_handle, AX
    
    ; Baca data dari file
    MOV AH, 3Fh
    MOV BX, file_handle
    MOV CX, 100
    MOV DX, OFFSET data_baca
    INT 21h
    JC  error_handler
    
    ; Tampilkan data yang dibaca
    MOV AH, 09h
    MOV DX, OFFSET pesan_baca
    INT 21h
    
    ; Tambahkan terminator '$' setelah data yang dibaca
    MOV BX, OFFSET data_baca
    ADD BX, AX
    MOV BYTE PTR [BX], '$'
    
    MOV AH, 09h
    MOV DX, OFFSET data_baca
    INT 21h
    
    ; Tutup file
    MOV AH, 3Eh
    MOV BX, file_handle
    INT 21h
    
    JMP selesai

error_handler:
    MOV AH, 09h
    MOV DX, OFFSET pesan_error
    INT 21h

selesai:
    MOV AH, 4Ch
    INT 21h
END START
\end{verbatim}

\section{Latihan}
\begin{enumerate}
  \item Tulis program yang menambahkan (append) baris baru ke berkas yang ada (gunakan pengaturan \textit{file pointer}).
  \item Tulis program yang membaca dan menampilkan berkas besar secara bertahap per blok 64 byte.
  \item Tulis program salin berkas yang memverifikasi ukuran hasil sama dengan sumber.
  \item Tulis program penghitung jumlah karakter alfabet, digit, dan lainnya pada berkas teks.
\end{enumerate}

\section{Tugas}
\begin{itemize}
  \item \textbf{Editor teks sederhana}: Navigasi baris, tambah/hapus, simpan-buka berkas; batasi fitur inti.
  \item \textbf{Konfigurasi}: Simpan/muat pasangan kunci-nilai sederhana ke/dari berkas; parsir format baris ``kunci=nilai''.
  \item \textbf{Backup}: Salin berkas dari daftar nama; tampilkan ringkasan hasil (berhasil/gagal).
  \item \textbf{Dokumentasi}: Daftar kode error umum dari \texttt{INT 21h} yang Anda temui, dan strategi penanganannya.
\end{itemize}


\end{document}