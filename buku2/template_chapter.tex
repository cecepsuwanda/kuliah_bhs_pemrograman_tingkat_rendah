% Template untuk setiap bab dalam buku Assembly
\chapter{Nama Bab}\label{ch:label-bab}

\section{Tujuan Pembelajaran}\label{sec:label-bab-tujuan}
Setelah mengikuti pertemuan ini, mahasiswa diharapkan mampu:
\begin{itemize}
    \item Tujuan pembelajaran 1
    \item Tujuan pembelajaran 2
    \item Tujuan pembelajaran 3
\end{itemize}

\section{Pendahuluan}\label{sec:label-bab-pendahuluan}
Pendahuluan singkat tentang materi yang akan dibahas.

\section{Materi Pembelajaran}\label{sec:label-bab-materi}
\subsection{Nama Sub-materi}\label{subsec:label-bab-submateri}
\subsubsection{Detail spesifik}

\subsubsection{Contoh implementasi}

% Contoh kode dengan syntax highlighting
\begin{lstlisting}[caption=Contoh Kode Assembly, label={code:contoh}]
org 100h
mov ax, 1234h
add ax, 5678h
int 20h
\end{lstlisting}

% Contoh tabel
\begin{table}[h]
\centering
\caption{Contoh Tabel}
\begin{tabular}{lll}
\toprule
\textbf{Kolom 1} & \textbf{Kolom 2} & \textbf{Kolom} \\
\midrule
Data 1 & Data 2 & Data 3 \\
Data 4 & Data 5 & Data 6 \\
\bottomrule
\end{tabular}
\label{tab:contoh}
\end{table}

\subsection{Contoh dengan referensi silang}

Lihat contoh pada Listing~\ref{code:contoh} dan data pada Tabel~\ref{tab:contoh}.

\section{Contoh Soal dan Pembahasan}\label{sec:label-bab-contoh}
\begin{enumerate}
    \item \textbf{Contoh soal 1}\\
    Pembahasan soal 1.

    \item \textbf{Contoh soal 2}\\
    Pembahasan soal 2.
\end{enumerate}

\section{Latihan}\label{sec:label-bab-latihan}
\begin{enumerate}
    \item Latihan praktik 1
    \item Latihan praktik 2
    \item Latihan praktik 3
\end{enumerate}

\section{Tugas}\label{sec:label-bab-tugas}
\begin{itemize}
    \item \textbf{Tugas utama}: Deskripsi tugas utama
    \item \textbf{Tugas pendukung}: Deskripsi tugas pendukung
    \item \textbf{Dokumentasi}: Dokumentasi yang diperlukan
\end{itemize}

\section{Referensi}\label{sec:label-bab-referensi}
\begin{itemize}
    \item \cite{hyde2010art}
    \item \cite{susanto1995belajar}
    \item Referensi tambahan yang relevan
\end{itemize}
