\chapter{Instalasi Turbo Assembler dan Struktur Program}

\section{Tujuan Pembelajaran}
Mahasiswa mampu:
\begin{itemize}
    \item Melakukan instalasi dan konfigurasi Turbo Assembler (TASM) dan alat \\
    pendukung (TLINK, TD).
    \item Menjelaskan perbedaan struktur dan karakteristik program \texttt{.COM} dan \texttt{.EXE}.
    \item Menggunakan direktif dasar (\texttt{ORG}, \texttt{END}, \texttt{TITLE}, \texttt{PAGE}) dalam berkas ASM.
    \item Menulis, merakit, menautkan, dan menjalankan program assembly sederhana.
\end{itemize}

\section{Pendahuluan}
Turbo Assembler (TASM) adalah assembler dari Borland yang populer di lingkungan DOS. Bersama \textit{toolchain} TLINK (linker) dan TD (debugger), TASM memungkinkan pengembangan program tingkat rendah pada arsitektur x86 klasik. Pemahaman struktur program \texttt{.COM} vs \texttt{.EXE} penting untuk pemilihan model memori, organisasi segmen, dan strategi \textit{build}. Melakukan instalasi dan konfigurasi Turbo Assembler (TASM) dan alat pendukung (TLINK, TD) merupakan langkah dasar yang diperlukan untuk memulai pemrograman assembly pada lingkungan DOS.

\section{Instalasi Turbo Assembler}
\subsection{Persyaratan sistem}
Lingkungan DOS (asli atau emulasi seperti DOSBox) atau sistem kompatibel. Ruang penyimpanan kecil, memori konvensional untuk menjalankan tool.

\subsection{Proses instalasi}
\begin{enumerate}
  \item Dapatkan paket TASM (legal, sesuai lisensi) dan ekstrak ke direktori, misal \texttt{C:\\TASM}.
  \item Pastikan berkas \texttt{TASM.EXE}, \texttt{TLINK.EXE}, \texttt{TD.EXE} tersedia.
  \item (Opsional) Tambahkan contoh proyek dan berkas \texttt{INCLUDE} yang menyertakan definisi konvensional (mis. \texttt{DOS.INC}).
\end{enumerate}

\subsubsection{Konfigurasi lingkungan}
Setel variabel lingkungan \texttt{PATH} agar berisi direktori TASM/TLINK/TD untuk akses dari mana saja. Konfigurasi file \texttt{TASM.CFG} atau \texttt{TLINK.CFG} jika diperlukan (mis., jalur pustaka). Pastikan \texttt{INCLUDE} menunjuk direktori header/\textit{include} jika digunakan.

\subsubsection{Pengaturan path dan direktori}
Contoh (DOSBox): \texttt{SET PATH=C:\\TASM;C:\\BIN;\%PATH\%}. Simpan sumber di direktori proyek terpisah untuk kebersihan struktur.

\subsection{Lingkungan Pengembangan}
\begin{itemize}
  \item \textbf{Editor}: Dapat memakai editor bawaan atau editor teks eksternal (mis., \texttt{EDIT}). Fokus pada penyorotan sintaks dan indentasi konsisten.
  \item \textbf{Assembler (TASM)}: Mengubah \texttt{.ASM} menjadi \texttt{.OBJ}. Opsi \texttt{/zi} untuk simbol debug.
  \item \textbf{Linker (TLINK)}: Menggabungkan \texttt{.OBJ} menjadi \texttt{.EXE} atau \texttt{.COM} (dengan langkah khusus).
  \item \textbf{Debugger (TD)}: Menjalankan \textit{step}, memeriksa register/memori, \textit{breakpoint}.
  \item \textbf{Dokumentasi}: Manual TASM/TLINK/TD, serta \texttt{.INC} referensi.
\end{itemize}

\subsection{Struktur Program COM}
\subsubsection{Karakteristik program COM}
\texttt{.COM} adalah format sederhana, tanpa header, berukuran maksimal ~64 KB. Memulai eksekusi pada offset \texttt{0100h} dengan \texttt{CS=DS=ES=SS} (umumnya sama), cocok untuk program kecil.

\subsubsection{Format file COM dan penggunaan memori}
Citra program dimuat sebagai blok kontinu. Stack perlu diatur manual bila diperlukan. Karena ketiadaan header, kontrol lebih langsung tetapi dengan keterbatasan ukuran dan organisasi segmen.

\subsubsection{Contoh struktur program COM}
\begin{verbatim}
org 100h
start:
    mov dx, offset msg
    mov ah, 09h
    int 21h

    mov ah, 4Ch
    int 21h

msg db 'Hello .COM!$'
\end{verbatim}

\subsection{Struktur Program EXE}
\subsubsection{Karakteristik program EXE}
\texttt{.EXE} memiliki header yang menjelaskan tata letak segmen, \textit{relocation}, dan titik masuk. Mendukung segmen terpisah untuk kode, data, \textit{stack}, sehingga lebih fleksibel untuk program besar.

\subsubsection{Header, segmentasi, dan contoh}
Header memuat informasi ukuran, relocation table, dsb. Model memori (SMALL, TINY, dsb.) memengaruhi pengaturan segmen.
\begin{verbatim}
.MODEL SMALL
.STACK 100h
.DATA
  msg db 'Hello .EXE!$'
.CODE
main PROC
  mov ax, @data
  mov ds, ax

  mov dx, offset msg
  mov ah, 09h
  int 21h

  mov ax, 4C00h
  int 21h
main ENDP
END main
\end{verbatim}

\subsection{Direktif Dasar}
\subsubsection{ORG (Origin)}
Menentukan offset awal kode/data. Pada program \texttt{.COM}, \texttt{org 100h} menyesuaikan lokasi titik masuk setelah PSP (Program Segment Prefix).

\subsubsection{END}
Menandai akhir berkas sumber; opsi label (mis., \texttt{END main}) mendefinisikan titik masuk program.

\subsubsection{TITLE, PAGE, dan lainnya}
\texttt{TITLE} dan \texttt{PAGE} memengaruhi keluaran listing. Direktif lain: \texttt{DB}/\texttt{DW} (data), \texttt{SEGMENT}/\texttt{ENDS}, \texttt{ASSUME}, tergantung assembler yang dipakai.

\subsubsection{Penggunaan direktif dalam program}
Pilih direktif sesuai target (\texttt{.COM} vs \texttt{.EXE}) dan model memori. Gunakan \texttt{END} dengan label fungsi utama untuk program \texttt{.EXE}.

\subsubsection{Menjalankan Toolchain di DOSBox}
Gunakan DOSBox untuk mengeksekusi TASM/TLINK/TD pada sistem modern. Contoh konfigurasi dasar:
\begin{verbatim}
MOUNT C ~/dos
C:
CD \TASM
SET PATH=C:\TASM;C:\BIN;%PATH%
TASM /Z /ZI HELLO.ASM    ; rakit -> HELLO.OBJ
TLINK /V HELLO.OBJ       ; taut -> HELLO.EXE
TD HELLO.EXE             ; debug
\end{verbatim}

\subsubsection{Membangun Program .COM vs .EXE}
Untuk berkas \texttt{.COM}, gunakan model memori TINY dan \texttt{ORG 100h}, lalu tautkan dengan opsi \texttt{/t}:
\begin{verbatim}
TASM /M2 HELLO_COM.ASM        ; menghasilkan HELLO_COM.OBJ
TLINK /T HELLO_COM.OBJ        ; menghasilkan HELLO_COM.COM
\end{verbatim}
Untuk \texttt{.EXE}, gunakan model \texttt{SMALL} (atau lain) dan prosedur bertanda \texttt{END main} seperti contoh sebelumnya.

\subsubsection{Alternatif Assembler Modern (NASM)}
Sebagai alternatif modern, NASM dapat merakit kode 8086. Contoh untuk menghasilkan \texttt{.COM} langsung:
\begin{verbatim}
nasm -f bin hello.asm -o hello.com
\end{verbatim}
Catatan: Sintaks NASM berbeda dari TASM/MASM; sesuaikan direktif dan penanganan segmen.

\subsubsection{Debugging Dasar dengan TD}
Gunakan TD untuk langkah per instruksi, memeriksa register/memori, dan breakpoint. Perintah umum: \texttt{R} (register), \texttt{U} (unassemble), \texttt{D} (dump), \texttt{E} (enter), \texttt{T} (trace), \texttt{P} (proceed). Simpan skrip langkah untuk demonstrasi praktikum.

\subsubsection{Kesalahan Umum dan Solusi}
\begin{itemize}
  \item Label tidak terdefinisi: periksa ejaan, urutan definisi, atau kebutuhan \texttt{EXTRN}/\texttt{PUBLIC} (untuk proyek multi-berkas).
  \item \texttt{CS:IP} tidak benar pada program \texttt{.COM}: pastikan \texttt{ORG 100h} didefinisikan dan tidak ada segmen tambahan.
  \item String untuk \texttt{AH=09h}: pastikan terminator karakter \texttt{'}\$\texttt{'} hadir; untuk \texttt{INT 10h} tidak diperlukan.
\end{itemize}

\section{Praktikum}
\begin{enumerate}
  \item Instal TASM dan pastikan \texttt{TASM}, \texttt{TLINK}, \texttt{TD} berjalan dari command line.
  \item Buat program \texttt{.COM} menampilkan sapaan, rakit dengan TASM, uji di DOSBox.
  \item Buat program \texttt{.EXE} dengan segmen data/kode terpisah, tautkan dengan TLINK, jalankan.
  \item Coba \texttt{TD} untuk menelusuri program: periksa register dan memori.
  \item Terapkan \texttt{ORG} dan \texttt{END} secara tepat pada kedua jenis program.
\end{enumerate}

\section{Contoh Kode}
\begin{verbatim}
; Program sederhana menggunakan direktif dasar
TITLE Program Sederhana
PAGE 60,132

.MODEL SMALL
.STACK 100h

.DATA
    pesan DB 'Hello World!$'

.CODE
    ORG 100h
    START:
        MOV AX, @DATA
        MOV DS, AX
        
        MOV DX, OFFSET pesan
        MOV AH, 09h
        INT 21h
        
        MOV AH, 4Ch
        INT 21h
    END START
\end{verbatim}

\section{Contoh Soal dan Pembahasan}
\begin{enumerate}
  \item \textbf{Jelaskan perbedaan utama \texttt{.COM} dan \texttt{.EXE}}: \texttt{.COM}: tanpa header, maksimal ~64 KB, datar, mulai di \texttt{0100h}; \texttt{.EXE}: berheader, mendukung segmentasi dan program besar.
  \item \textbf{Peran \texttt{ORG 100h} pada program \texttt{.COM}}: Menyesuaikan offset titik masuk setelah PSP sehingga label awal selaras dengan lokasi eksekusi sebenarnya.
  \item \textbf{Alur \textit{build} TASM/TLINK}: TASM: \texttt{.ASM -> .OBJ}; TLINK: \texttt{.OBJ -> .EXE}; untuk \texttt{.COM} gunakan \textit{tiny model} atau metode khusus.
  \item \textbf{Gunakan TD untuk apa?}: Debugging: langkah per instruksi, inspeksi register/memori, \textit{breakpoint}.
\end{enumerate}

\section{Latihan}
\begin{enumerate}
  \item Tulis program \texttt{.COM} yang menampilkan dua baris teks menggunakan \texttt{INT 21h, AH=09h}.
  \item Tulis program \texttt{.EXE} yang menginisialisasi \texttt{DS} dengan benar dan menampilkan string.
  \item Buat variasi yang membaca input karakter tunggal (\texttt{INT 21h, AH=01h}) dan mencetaknya kembali.
  \item Bangun kedua program dengan TASM/TLINK dan uji di DOSBox.
\end{enumerate}

\section{Tugas}
\begin{itemize}
  \item \textbf{Dokumentasi instalasi}: Sertakan tangkapan layar pengaturan PATH, keluaran versi TASM/TLINK/TD.
  \item \textbf{Program COM nama pribadi}: Tampilkan nama Anda dan NIM pada satu baris.
  \item \textbf{Esai perbandingan (300--500 kata)}: \texttt{.COM} vs \texttt{.EXE} dari aspek header, ukuran, segmentasi, dan skenario penggunaan.
  \item \textbf{Log debugging}: Jalankan program \texttt{.EXE} di TD, lakukan \textit{step} beberapa instruksi, catat perubahan register.
\end{itemize}

\section{Referensi}\label{sec:instalasi-referensi}
\begin{itemize}
    \item \cite{borland1990tasm}
    \item \cite{nasm_manual}
    \item \cite{dosbox_manual}
\end{itemize}

