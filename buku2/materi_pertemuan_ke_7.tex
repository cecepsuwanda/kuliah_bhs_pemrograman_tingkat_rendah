\chapter{Percabangan, Loop, dan Interupsi Mouse}

\section{Tujuan Pembelajaran}
Mahasiswa mampu:
\begin{itemize}
    \item Menggunakan instruksi \texttt{CMP} untuk perbandingan dan membaca dampaknya pada \textit{flags}.
    \item Menerapkan lompatan bersyarat \texttt{JE}, \texttt{JNE}, \texttt{JL}/\texttt{JG} dan variasinya berdasarkan interpretasi bertanda/tak bertanda.
    \item Menggunakan \texttt{LOOP}, \texttt{LOOPE}/\texttt{LOOPZ}, \texttt{LOOPNE}/\texttt{LOOPNZ} untuk perulangan berbasis \texttt{CX}.
    \item Berinteraksi dengan mouse melalui \texttt{INT 33h}: inisialisasi, tampil/sembunyi kursor, baca posisi/tombol.
    \item Menyusun program yang menggabungkan percabangan, perulangan, dan input mouse.
\end{itemize}

\section{Pendahuluan}
Kontrol alur adalah inti dari setiap program \cite{susanto1995belajar}. Pada 8086, \texttt{CMP} mengatur \textit{flags} yang selanjutnya diinterpretasikan oleh lompat bersyarat. Untuk iterasi, instruksi keluarga \texttt{LOOP} memanfaatkan \texttt{CX} sebagai pencacah \cite{hyde2010art}. Di sisi \textit{I/O}, \texttt{INT 33h} menyediakan API BIOS untuk mouse pada lingkungan DOS, memungkinkan antarmuka interaktif dasar \cite{nopi2003tutorial}.

\section{Instruksi Perbandingan (CMP)}
\subsubsection{Sintaks dan cara kerja}
\texttt{CMP dest, src} menghitung \(dest - src\) tanpa menyimpan hasil, namun memperbarui \textit{flags} (\texttt{ZF}, \texttt{SF}, \texttt{CF}, \texttt{OF}, \texttt{PF}, \texttt{AF}). \textit{Flags} kemudian digunakan oleh lompat bersyarat.
\begin{verbatim}
mov ax, 10
cmp ax, 10      ; ZF=1 -> equal
cmp ax, 12      ; CF=1 (tak bertanda), SF!=OF (bertanda) -> less
\end{verbatim}

\subsubsection{Perbandingan vs operasi aritmatika}
\texttt{CMP} setara dengan \texttt{SUB} yang hasilnya dibuang. Gunakan \texttt{TEST} untuk perbandingan bitwise (seperti \texttt{AND} tanpa menyimpan).

\subsection{Instruksi Percabangan}
\subsubsection{JE/JNE}
\texttt{JE}/\texttt{JZ} melompat bila \texttt{ZF=1}; \texttt{JNE}/\texttt{JNZ} bila \texttt{ZF=0}. Cocok untuk kesetaraan/kesetidaksetaraan umum.

\subsubsection{JL/JG dan varian bertanda vs tak bertanda}
\begin{itemize}
  \item \textbf{Bertanda}: \texttt{JL} (SF != OF), \texttt{JG} (ZF=0 dan SF=OF), \texttt{JLE}, \texttt{JGE}.
  \item \textbf{Tak bertanda}: \texttt{JB}/\texttt{JC} (CF=1), \texttt{JA} (CF=0 dan ZF=0), \texttt{JBE}, \texttt{JAE}/\texttt{JNC}.
\end{itemize}
Pilih sesuai interpretasi data untuk menghindari logika salah.

\subsubsection{Conditional jumps dan jangkauan}
Lompatan bersyarat menggunakan offset relatif pendek (\(\pm 128\) byte pada 8086). Untuk jangkauan lebih jauh, gunakan skema \textit{inverted branch} + \texttt{JMP} jauh.

\section{Pendalaman}\label{sec:branch-mouse-pendalaman}
\subsection{Pola kontrol alur yang umum}
Gunakan pola \textbf{if-else}, \textbf{switch-like} (tabel lompatan/\textit{jump table} berbasis indeks), dan \textbf{loop counted} (\texttt{CX}) vs \textbf{loop sentinel} (berhenti pada kondisi). Tabel lompatan dapat diimplementasikan dengan daftar alamat dan \texttt{JMP [BX+SI]}. \cite{intel2019manual32}

\subsection{Optimasi cabang}
Cabang pendek (\texttt{Jcc} 8-bit) lebih padat; restrukturisasi kondisi untuk memanfaatkan jangkauan pendek saat mungkin. Hindari memodifikasi \texttt{CX} tak sengaja saat memakai \texttt{LOOP}. \cite{hyde2010art}

\subsection{INT 33h detail dan koordinat}
Skala koordinat mouse bergantung mode tampilan. Pada mode teks, koordinat biasanya dalam sel karakter; di mode grafik dapat berupa piksel dengan rentang yang dapat diatur (fungsi pembatasan). Selalu periksa nilai kembali inisialisasi driver (\texttt{AX=FFFFh}) sebelum memakai layanan mouse. \cite{rbil}

\subsection{Debounce klik dan drag sederhana}
Catat transisi status tombol (tekan/lepas) untuk membedakan klik tunggal vs tahan/drag. Simpan posisi terakhir untuk menghitung delta pergerakan dan ambang jarak. \cite{osdev_wiki}


\subsection{Instruksi Perulangan (LOOP)}
\subsubsection{LOOP}
Menurunkan \texttt{CX} dan melompat jika \texttt{CX} \(\neq 0\). Perhatikan bahwa \texttt{CX} berkurang terlebih dahulu, lalu diuji.
\begin{verbatim}
mov cx, 5
ulang:
  ; body
  loop ulang
\end{verbatim}

\subsubsection{LOOPZ/LOOPE dan LOOPNZ/LOOPNE}
Melompat jika \texttt{CX} \(\neq 0\) \textbf{dan} \texttt{ZF} sesuai (0 atau 1). Umum untuk mengulang pencocokan hingga gagal/berhasil.

\subsubsection{Kontrol perulangan}
Hindari mengubah \texttt{CX} di dalam body kecuali disengaja. Untuk iterasi kompleks, kombinasi \texttt{DEC/JNZ} sering lebih fleksibel.

\subsection{Interupsi Mouse INT 33h}
\subsubsection{Fungsi 00h: Initialize Mouse}
\begin{itemize}
  \item \textbf{Masukan}: \texttt{AX=0000h}
  \item \textbf{Keluaran}: \texttt{AX=FFFFh} jika driver ada; selain itu 0.
\end{itemize}

\subsubsection{Fungsi 01h/02h: Show/Hide Cursor}
\begin{itemize}
  \item \texttt{AX=0001h} menampilkan; \texttt{AX=0002h} menyembunyikan.
\end{itemize}

\subsubsection{Fungsi 03h: Get Position and Buttons}
\begin{itemize}
  \item \textbf{Masukan}: \texttt{AX=0003h}
  \item \textbf{Keluaran}: \texttt{BX}=status tombol, \texttt{CX}=X, \texttt{DX}=Y (unit tergantung mode layar).
\end{itemize}

\subsubsection{Fungsi 04h: Set Position}
\begin{itemize}
  \item \textbf{Masukan}: \texttt{AX=0004h}, \texttt{CX}=X, \texttt{DX}=Y.
\end{itemize}

\subsubsection{Fungsi 05h: Get Button Press Info}
Mengembalikan informasi penekanan tombol terakhir (kode tombol, koordinat saat ditekan).

\section{Praktikum}
\begin{enumerate}
  \item Program percabangan sederhana: bandingkan dua nilai, tampilkan pesan sesuai hasil (JE/JNE).
  \item Program perulangan: gunakan \texttt{LOOP} untuk mencetak pola atau menghitung deret sederhana.
  \item Kombinasi: cari nilai minimum/maksimum dalam array menggunakan \texttt{CMP} + lompatan.
  \item Interaksi mouse: inisialisasi, tampilkan kursor, baca posisi dan tombol, tampilkan di layar.
  \item Mini-game: gerakkan kursor teks mengikuti posisi mouse (mode teks) dan reaksi saat klik.
\end{enumerate}

\section{Contoh Kode}
\begin{verbatim}
; Program percabangan, loop, dan mouse
TITLE Percabangan dan Loop
.MODEL SMALL
.STACK 100h

.DATA
    pesan1 DB 'Bilangan sama$'
    pesan2 DB 'Bilangan berbeda$'
    pesan3 DB 'Loop selesai$'
    bil1 DW 10
    bil2 DW 10

.CODE
START:
    MOV AX, @DATA
    MOV DS, AX
    
    ; Percabangan
    MOV AX, bil1
    CMP AX, bil2
    JE sama
    ; Jika tidak sama
    MOV AH, 09h
    MOV DX, OFFSET pesan2
    INT 21h
    JMP lanjut
    
sama:
    MOV AH, 09h
    MOV DX, OFFSET pesan1
    INT 21h
    
lanjut:
    ; Perulangan
    MOV CX, 5
ulang:
    ; Kode yang diulang
    DEC CX
    JNZ ulang
    
    ; Inisialisasi mouse
    MOV AX, 00h
    INT 33h
    CMP AX, 0
    JE no_mouse
    
    ; Tampilkan cursor mouse
    MOV AX, 01h
    INT 33h
    
    ; Baca posisi mouse
    MOV AX, 03h
    INT 33h
    ; BX = button status, CX = X position, DX = Y position
    
no_mouse:
    MOV AH, 4Ch
    INT 21h
END START
\end{verbatim}

\section{Latihan}
\begin{enumerate}
  \item Buat program yang membandingkan dua bilangan dan menampilkan hasil menggunakan JE/JNE.
  \item Buat program menghitung faktorial (n kecil) menggunakan \texttt{LOOP} atau \texttt{DEC/JNZ}.
  \item Buat program menampilkan angka 1--10 pada baris berbeda menggunakan perulangan.
  \item Buat program yang menampilkan koordinat mouse secara real-time (\textit{polling} \texttt{INT 33h} fungsi 03h).
\end{enumerate}

\section{Tugas}
\begin{itemize}
  \item \textbf{Kalkulator bercabang}: Menu berbasis \textit{branching} untuk memilih operasi aritmatika, validasi input.
  \item \textbf{Sorting sederhana}: Implementasikan \textit{bubble sort} pada array kecil (tak bertanda), tampilkan hasil setiap pass.
  \item \textbf{Game tebak angka + mouse}: Gunakan mouse untuk memilih rentang jawaban, tampilkan umpan balik.
  \item \textbf{Dokumentasi}: Ringkas instruksi lompatan dan kondisi flags yang memicunya; sertakan contoh minimal.
\end{itemize}

\section{Referensi}
% Bibliography is handled by the main document
