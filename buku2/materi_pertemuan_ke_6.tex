\chapter{Input dari Keyboard (INT 16h); Penanganan Keyboard Buffer}

\section{Tujuan Pembelajaran}
Mahasiswa mampu:
\begin{itemize}
  \item Menggunakan layanan BIOS \texttt{INT 16h} untuk membaca input keyboard (blocking/non-blocking) dan memeriksa status.
  \item Menjelaskan konsep keyboard buffer, mekanisme kerja, dan teknik \textit{flush}/penanganan buffer penuh.
  \item Mengembangkan program interaktif berbasis input karakter dan string (dengan/ tanpa echo) dengan validasi.
  \item Menginterpretasikan kode ASCII vs. \textit{scan code} serta menangani tombol khusus (function, arrow, control).
\end{itemize}

\section{Pendahuluan}
Interaksi pengguna pada lingkungan DOS banyak bertumpu pada layanan BIOS untuk perangkat input seperti keyboard. \texttt{INT 16h} menyediakan antarmuka tingkat rendah yang konsisten untuk membaca tombol, memeriksa ketersediaan input, dan mendapatkan status \textit{modifier} (Shift/Ctrl/Alt). Memahami perbedaan operasi \textit{blocking} dan \textit{non-blocking} penting agar program responsif dan tidak \textit{hang} menunggu masukan.

\section{Materi Pembelajaran}
\subsection{Interupsi BIOS INT 16h}
\subsubsection{Fungsi 00h: Read Key (Blocking)}
\begin{itemize}
  \item \textbf{Masukan}: \texttt{AH=00h}
  \item \textbf{Keluaran}: \texttt{AL=ASCII code} (0 jika \textit{non-printable}), \texttt{AH=scan code}
  \item \textbf{Perilaku}: Menunggu hingga tombol ditekan; gunakan untuk input yang memerlukan sinkronisasi.
\end{itemize}

\subsubsection{Fungsi 01h: Check for Key (Non-blocking)}
\begin{itemize}
  \item \textbf{Masukan}: \texttt{AH=01h}
  \item \textbf{Keluaran}: \texttt{ZF=1} jika tidak ada tombol; jika ada, \texttt{AL}/\texttt{AH} berisi kode tanpa menghapus dari buffer.
  \item \textbf{Perilaku}: \textit{Polling} status keyboard; gunakan untuk antarmuka yang tidak boleh menunggu.
\end{itemize}

\subsubsection{Fungsi 02h: Get Keyboard Flags}
\begin{itemize}
  \item \textbf{Masukan}: \texttt{AH=02h}
  \item \textbf{Keluaran}: \texttt{AL} berisi bendera status (Shift, Ctrl, Alt, Caps Lock, Num Lock, Scroll Lock).
  \item Interpretasikan bit-bit sesuai tabel referensi BIOS.
\end{itemize}

\subsubsection{Parameter dan nilai kembali}
Pastikan menyimpan nilai \texttt{AL}/\texttt{AH} segera jika akan digunakan, karena pemanggilan layanan berikutnya dapat menimpanya. Untuk \texttt{01h}, cek \texttt{ZF} sebelum membaca.

\subsubsection{Blocking vs. non-blocking}
\textit{Blocking} menyederhanakan alur namun berisiko mengunci UI. \textit{Non-blocking} memungkinkan pembaruan layar/pekerjaan lain saat menunggu input menggunakan \textit{polling} atau \textit{timeout}.

\subsection{Keyboard Buffer}
\subsubsection{Konsep dan cara kerja}
BIOS menempatkan penekanan tombol pada antrian (buffer). Panggilan baca (00h) mengeluarkan entri dari buffer; pemeriksaan (01h) tidak mengeluarkan.

\subsubsection{Penanganan buffer penuh}
Ketika buffer penuh, penekanan tombol berikut dapat hilang atau menimpa entri lama tergantung implementasi/driver. Program interaktif harus membaca buffer secara reguler.

\subsubsection{Flushing keyboard buffer}
DOS menyediakan \texttt{INT 21h, AH=0Ch} dengan parameter di \texttt{AL} untuk membersihkan buffer dan memproses input keyboard. Gunakan setelah mengubah mode input agar sisa tombol tidak mengganggu.

\subsubsection{Status keyboard buffer}
Gunakan \texttt{01h} untuk memeriksa ketersediaan. Beberapa BIOS menyediakan fungsi lain untuk panjang buffer, tetapi tidak portabel.

\subsection{Input Karakter dan String}
\subsubsection{Karakter tunggal}
Gunakan \texttt{AH=00h} (blocking) untuk membaca satu tombol. Untuk menghindari \textit{hang}, kombinasikan dengan \texttt{01h}.

\subsubsection{String dengan echo}
Gunakan \texttt{INT 21h, AH=0Ah} untuk masukan baris dengan echo; siapkan struktur buffer DOS: byte pertama kapasitas, byte kedua panjang, diikuti penampung.

\subsubsection{String tanpa echo (password)}
Baca per karakter dengan \texttt{INT 16h, AH=00h} dan tampilkan tanda \texttt{'*'} menggunakan \texttt{INT 10h, AH=0Eh}; simpan ke buffer hingga Enter.

\subsubsection{Validasi input dan karakter khusus}
Tolak karakter di luar rentang yang diizinkan; tangani Backspace dengan menghapus dari buffer dan memundurkan kursor; Enter mengakhiri masukan.

\subsection{Karakter Khusus}
\subsubsection{Scan code vs. ASCII}
Tombol \textit{non-printable} menghasilkan \texttt{AL=0} dan \texttt{AH} berisi \textit{scan code}. Tombol huruf/angka memberikan ASCII di \texttt{AL}.

\subsubsection{Function, arrow, control keys}
F1--F12 umumnya menghasilkan \texttt{AL=0} (atau \texttt{E0h} sebagai \textit{prefix} pada beberapa sistem) dan \texttt{AH} menunjukkan kode; panah memerlukan interpretasi \textit{scan code}. \texttt{Ctrl}/\texttt{Alt}/\texttt{Shift} memodifikasi kode dan terekam pada status flags (fungsi 02h).

\section{Praktikum}
\begin{enumerate}
  \item Program pembaca karakter tunggal: tampilkan ASCII dan \textit{scan code} dalam heksadesimal.
  \item Program input string dengan echo menggunakan \texttt{AH=0Ah}; cetak kembali string.
  \item Program input password tanpa echo dengan masking dan dukungan Backspace.
  \item Program \textit{buffer-aware}: baca semua karakter tertunda (\texttt{01h}) dan \texttt{flush} sebelum kembali ke menu.
  \item Program menu interaktif: navigasi dengan panah atas/bawah; Enter untuk memilih; Esc untuk kembali.
\end{enumerate}

\section{Contoh Kode}
\begin{verbatim}
; Program input keyboard dan penanganan buffer
TITLE Input Keyboard
.MODEL SMALL
.STACK 100h

.DATA
    prompt DB 'Masukkan nama: $'
    nama   DB 50 DUP('$')
    pesan  DB 'Halo, $'
    buffer DB 50 DUP(?)

.CODE
START:
    MOV AX, @DATA
    MOV DS, AX
    
    ; Tampilkan prompt
    MOV AH, 09h
    MOV DX, OFFSET prompt
    INT 21h
    
    ; Input string dengan echo (DOS buffered input)
    MOV AH, 0Ah
    MOV DX, OFFSET buffer
    INT 21h
    
    ; Input karakter tunggal (blocking)
    MOV AH, 00h
    INT 16h
    ; AL = ASCII code, AH = Scan code
    
    ; Check for key (non-blocking)
    MOV AH, 01h
    INT 16h
    JZ no_key
    ; Key available, read it
    MOV AH, 00h
    INT 16h
    
no_key:
    ; Flush keyboard buffer (DOS)
    MOV AH, 0Ch
    MOV AL, 00h
    INT 21h
    
    MOV AH, 4Ch
    INT 21h
END START
\end{verbatim}

\section{Latihan}
\begin{enumerate}
  \item Buat program yang menunggu input dari keyboard dan menampilkan nilai ASCII/scan code setiap tombol.
  \item Buat program yang menampilkan \textit{scan code} untuk function keys F1--F4 dan mengeksekusi aksi berbeda.
  \item Buat program input password dengan masking dan validasi panjang (min 6, maks 16), dukung Backspace.
  \item Buat program yang mendeteksi kondisi buffer penuh (simulasikan dengan input cepat), kemudian melakukan \texttt{flush} dan menampilkan pesan peringatan.
\end{enumerate}

\section{Tugas}
\begin{itemize}
  \item \textbf{Login sederhana}: Validasi username dan password yang disimpan di memori; berikan tiga kesempatan; kunci selama 5 detik (simulasikan delay) setelah 3 kegagalan.
  \item \textbf{Menu interaktif}: Navigasi dengan arrow keys, Enter untuk pilih, Esc untuk kembali; sorot pilihan aktif menggunakan warna.
  \item \textbf{Timeout input}: Implementasikan \textit{timeout} 5 detik untuk menekan tombol berikutnya (gunakan \textit{polling} \texttt{01h} dan penundaan berbasis \texttt{INT 1Ah} atau loop terkalibrasi); jika habis, batalkan input.
  \item \textbf{Dokumentasi INT 16h}: Ringkas parameter dan nilai kembali untuk fungsi 00h, 01h, 02h; sertakan tabel contoh keluaran.
\end{itemize}

\section{Referensi}
\begin{itemize}
  \item Hyde, Randall. \textit{The Art of Assembly Language}, 2nd ed., No Starch Press, 2010.
  \item Susanto. \textit{Belajar Pemrograman Bahasa Assembly}, Elex Media Komputindo, 1995.
\end{itemize}
