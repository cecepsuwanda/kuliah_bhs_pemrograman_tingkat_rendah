\chapter{Instruksi Logika dan Output Teks}

\section{Tujuan Pembelajaran}
Mahasiswa mampu:
\begin{itemize}
    \item Menerapkan instruksi logika \texttt{AND}, \texttt{OR}, \texttt{XOR}, \texttt{NOT} pada data 8/16-bit.
    \item Memahami operasi bitwise, masking, set/clear bit, dan kasus penggunaan praktis.
    \item Menggunakan layanan BIOS video \texttt{INT 10h} (fungsi \texttt{02h}, \texttt{09h}, \texttt{0Ah}, \texttt{0Eh}) untuk output teks dan kontrol kursor.
    \item Menampilkan teks di layar dengan kontrol posisi dan atribut warna.
\end{itemize}

\section{Pendahuluan}
Instruksi logika menyediakan manipulasi bit tingkat rendah yang fundamental untuk kontrol perangkat keras, pengolahan bendera (flag), dan optimasi \cite{hyde2010art}. Pada sisi output, layanan BIOS \texttt{INT 10h} memungkinkan interaksi langsung dengan tampilan dalam mode teks/TTY, relevan untuk program DOS dan lingkungan \textit{bare-metal}-like \cite{susanto1995belajar}.

\section{Instruksi Logika}
\subsubsection{AND}
Operasi konjungsi bitwise: bit hasil 1 jika dan hanya jika kedua operand 1. Umum untuk \textit{masking} menghapus bit tertentu.
\begin{verbatim}
mov ax, 0A5Fh
and ax, 00FFh    ; simpan hanya 8 bit rendah -> AX = 005Fh
\end{verbatim}
\textbf{Flags}: memperbarui \texttt{ZF}, \texttt{SF}, \texttt{PF}; \texttt{CF} dan \texttt{OF} direset ke 0.

\subsubsection{OR}
Disjungsi bitwise: bit hasil 1 jika salah satu operand 1. Berguna untuk menyetel bit spesifik.
\begin{verbatim}
mov al, 00110000b
or  al, 00000101b  ; set bit 0 dan 2 -> 00110101b
\end{verbatim}
\textbf{Flags}: \texttt{ZF}, \texttt{SF}, \texttt{PF} diperbarui; \texttt{CF}, \texttt{OF} = 0.

\subsubsection{XOR}
Eksklusif-or: 1 jika operand berbeda; \textit{involutif} (\texttt{x xor k xor k = x}). Biasa untuk enkripsi sederhana dan pembersihan register cepat.
\begin{verbatim}
xor ax, ax   ; cepat set AX=0, tanpa memuat immediate
\end{verbatim}
\textbf{Flags}: mirip \texttt{AND}/\texttt{OR}; \texttt{CF}, \texttt{OF} = 0.

\subsubsection{NOT}
Komplemen bitwise (unary). Sering dipakai untuk inversi \textit{mask}.
\begin{verbatim}
mov bl, 11110000b
not bl                 ; BL = 00001111b
\end{verbatim}
\textbf{Flags}: tidak terpengaruh (kecuali implementasi spesifik; pada 8086, flags tidak terdefinisi diubah untuk NOT—anggap tidak relevan).

\subsubsection{Operasi bitwise dan aplikasinya}
\begin{itemize}
  \item \textbf{Masking}: \texttt{AND} untuk \textit{clear}, \texttt{OR} untuk \textit{set}, \texttt{XOR} untuk \textit{toggle}.
  \item \textbf{Packing/unpacking} field bit dalam satu byte/word.
  \item \textbf{Uji bit}: kombinasikan \texttt{TEST} (seperti \texttt{AND} tanpa menyimpan hasil) dan percabangan pada \texttt{ZF}.
\end{itemize}
\begin{verbatim}
; uji bit 3 (hitung dari 0)
mov al, value
mov ah, al
and ah, 00001000b
jz  bit3_clear
; bit3_set ...
\end{verbatim}

\subsection{Aplikasi Instruksi Logika}
\subsubsection{Manipulasi bit dan masking}
\begin{verbatim}
; clear 2 bit terendah
and al, 11111100b
; set bit 7
or  al, 10000000b
; toggle bit 4
xor al, 00010000b
\end{verbatim}

\subsubsection{Enkripsi sederhana (XOR)}
\begin{verbatim}
; enkripsi/dekripsi XOR dengan kunci tunggal
mov al, [si]
xor al, key
mov [di], al
\end{verbatim}
\textit{Catatan}: XOR cipher lemah dan hanya untuk demonstrasi.

\subsubsection{Operasi set dan clear bit, perbandingan bit}
Gunakan \texttt{BT}/\texttt{BTS}/\texttt{BTR} pada prosesor lebih baru; pada 8086, gunakan \texttt{AND}/\texttt{OR}/\texttt{XOR} manual. Perbandingan bit dilakukan dengan \texttt{TEST}.

\subsection{Interupsi BIOS INT 10h}
\subsubsection{Fungsi 02h: Set Cursor Position}
\begin{itemize}
  \item \texttt{AH=02h}, \texttt{BH=page}, \texttt{DH=row}, \texttt{DL=column}; \texttt{INT 10h}.
  \item Koordinat 0-berbasis; pastikan dalam batas mode teks (mis. 25x80).
\end{itemize}
\subsubsection{Fungsi 09h: Write Character and Attribute}
\begin{itemize}
  \item \texttt{AH=09h}, \texttt{AL=char}, \texttt{BH=page}, \texttt{BL=attribute}, \texttt{CX=count}; \texttt{INT 10h}.
  \item Menulis dengan atribut warna; tidak memajukan kursor secara \textit{TTY-like} untuk semua mode.
\end{itemize}
\subsubsection{Fungsi 0Ah: Write Character Only}
\begin{itemize}
  \item \texttt{AH=0Ah}, parameter mirip, menggunakan atribut saat ini.
\end{itemize}
\subsubsection{Fungsi 0Eh: Write Character in TTY Mode}
\begin{itemize}
  \item \texttt{AH=0Eh}, \texttt{AL=char}, \texttt{BH=page}, \texttt{BL=color}; menulis seperti \texttt{teletype}, memajukan kursor dan menggulir bila perlu.
\end{itemize}

\subsection{Output Teks ke Layar}
\subsubsection{Kontrol posisi kursor}
Pindahkan kursor ke (baris, kolom) dengan \texttt{AH=02h}. Tulis teks karakter-per-karakter (\texttt{0Eh}) atau \texttt{09h} untuk pengulangan.

\subsubsection{Atribut karakter (warna)}
Atribut 8-bit: 4 bit latar depan + 3 bit latar belakang + 1 bit berkedip. Contoh: \texttt{1Eh} (kuning di biru).

\subsubsection{Mode teks dan grafik}
Materi ini berfokus pada mode teks; mode grafik memerlukan fungsi \texttt{INT 10h} lain (dibahas pada pertemuan grafik).

\subsubsection{Penanganan karakter khusus}
Untuk baris baru, gunakan pasangan \texttt{$\backslash$r} dan \texttt{$\backslash$n} atau \texttt{TTY 0Eh}. Simbol \texttt{'\$'} digunakan oleh \texttt{INT 21h, AH=09h}, bukan \texttt{INT 10h}.

\section{Contoh Soal dan Pembahasan}
\begin{enumerate}
  \item \textbf{Masking 8 bit terakhir}.\\ Gunakan \texttt{AND AX, 00FFh} untuk mengekstrak byte rendah dari word.
  \item \textbf{Enkripsi XOR}.\\ Tunjukkan bahwa \texttt{x xor k xor k = x} untuk pembuktian dekripsi.
  \item \textbf{Menampilkan teks di posisi tertentu}.\\ Pindahkan kursor (\texttt{AH=02h}) lalu panggil \texttt{AH=09h} atau \texttt{0Eh}.
  \item \textbf{Teks berwarna}.\\ Tentukan atribut warna pada \texttt{BL} untuk \texttt{AH=09h}.
\end{enumerate}

\section{Praktikum}
\begin{enumerate}
  \item Implementasi \textit{bit playground}: \texttt{AND}/\texttt{OR}/\texttt{XOR}/\texttt{NOT} terhadap nilai contoh, tampilkan hasil dalam heksadesimal.
  \item Implementasi masking dan uji bit spesifik menggunakan \texttt{TEST} + cabang.
  \item Implementasi enkripsi XOR pada buffer kecil di memori (hard-coded); verifikasi dekripsi.
  \item Implementasi output teks: atur kursor ke beberapa posisi dan cetak karakter/teks dengan atribut berbeda.
\end{enumerate}

\section{Contoh Kode}
\begin{verbatim}
; Program operasi logika dan output teks
TITLE Operasi Logika dan Output
.MODEL SMALL
.STACK 100h

.DATA
    pesan DB 'Hello Assembly!$'
    nilai1 DW 0F0F0h
    nilai2 DW 0FF00h

.CODE
START:
    MOV AX, @DATA
    MOV DS, AX
    
    ; Operasi logika
    MOV AX, nilai1
    AND AX, nilai2    ; Masking
    OR AX, 000Fh      ; Set bit terakhir
    XOR AX, 0FFFFh    ; Inversi
    NOT AX            ; Komplemen
    
    ; Output teks ke layar
    MOV AH, 02h       ; Set cursor position
    MOV BH, 00h       ; Page 0
    MOV DH, 10        ; Row 10
    MOV DL, 20        ; Column 20
    INT 10h
    
    ; Tampilkan karakter
    MOV AH, 09h       ; Write character
    MOV AL, 'A'       ; Character 'A'
    MOV BH, 00h       ; Page 0
    MOV BL, 1Eh       ; Yellow on blue
    MOV CX, 1         ; Count
    INT 10h
    
    MOV AH, 4Ch
    INT 21h
END START
\end{verbatim}

\section{Latihan}
\begin{enumerate}
  \item Buat program yang melakukan masking pada 8 bit terakhir untuk sejumlah data dalam array; simpan hasil ke array tujuan.
  \item Buat program enkripsi/dekripsi XOR untuk string dengan kunci 1 byte; tampilkan hasil terenkripsi dalam heksadesimal.
  \item Buat program yang menampilkan teks pada empat sudut layar (0,0), (0,79), (24,0), (24,79).
  \item Buat program yang menampilkan teks dengan semua kombinasi warna latar depan standar pada latar belakang biru.
\end{enumerate}

\section{Tugas}
\begin{itemize}
  \item \textbf{Manipulasi data}: Implementasikan rutin \texttt{set\_bit}, \texttt{clear\_bit}, \texttt{toggle\_bit} untuk sebuah byte di memori; sertakan pengujian.
  \item \textbf{Menu teks}: Buat menu sederhana dengan kontrol posisi kursor, garis pemisah menggunakan karakter ASCII, dan pilihan bernomor; sorot pilihan aktif dengan atribut warna.
  \item \textbf{XOR cipher}: Implementasikan enkripsi/dekripsi string berbasis XOR (1 byte kunci), dokumentasikan kelemahannya dan skenario edukatifnya.
  \item \textbf{Dokumentasi INT 10h}: Buat ringkasan parameter untuk fungsi 02h, 09h, 0Ah, 0Eh dan contoh singkat masing-masing.
\end{itemize}

\section{Referensi}
% Bibliography is handled by the main document
