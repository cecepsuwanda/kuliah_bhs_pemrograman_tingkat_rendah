\chapter{Pemrograman Grafik Dasar}

\section{Tujuan Pembelajaran}
Mahasiswa mampu:
\begin{itemize}
    \item Menjelaskan perbedaan mode teks dan grafik, resolusi, palet warna, dan organisasi memori video.
    \item Menggunakan layanan BIOS \texttt{INT 10h} untuk mengatur mode grafik, menulis/membaca piksel, dan memeriksa mode aktif.
    \item Menggambar piksel dan garis di layar pada koordinat tertentu dengan warna yang dipilih.
    \item Mengimplementasikan algoritma garis (DDA/Bresenham) dan memahami trade-off kinerja/akurasi.
    \item Mengembangkan program grafik sederhana berbasis mode 13h (320x200x256) sebagai studi kasus.
\end{itemize}

\section{Pendahuluan}
Mode grafik menyediakan akses piksel-tingkat ke layar, berbeda dengan mode teks yang berbasis karakter \cite{susanto1995belajar}. Pada lingkungan DOS, BIOS menyediakan fungsi \texttt{INT 10h} untuk mengatur mode, menulis/membaca piksel, dan berbagai operasi tampilan \cite{hyde2010art}. Mode 13h (320x200 dengan 256 warna) populer untuk pembelajaran karena memori video \textit{linear} (satu byte per piksel) yang mudah diakses \cite{nopi2003tutorial}.

\section{Konsep Mode Grafik}
\subsection{Perbedaan mode teks dan grafik}
Mode teks menampilkan karakter dari tabel font di layar, sedangkan mode grafik memetakan memori video langsung ke piksel. Mode teks lebih hemat memori dan sederhana; mode grafik memberi fleksibilitas menggambar bentuk bebas.

\subsubsection{Resolusi layar dan palet}
Resolusi menyatakan jumlah piksel horizontal x vertikal. Pada mode 13h: 320x200 piksel, 256 warna (palet 8-bit). Nilai warna \(0..255\) memetakan ke entri palet yang dapat diatur (di luar cakupan dasar ini).

\subsubsection{Organisasi memori video dan pengalamatan piksel}
Pada mode 13h, buffer video mulai di \texttt{A000:0000}. Offset piksel \((x,y)\) adalah \(y\cdot 320 + x\). Untuk mode lain (planar EGA/VGA), organisasi lebih kompleks.

\subsection{Interupsi INT 10h untuk Grafik}
\subsubsection{Fungsi 00h: Set Video Mode}
\begin{itemize}
  \item \textbf{Masukan}: \texttt{AH=00h}, \texttt{AL=mode} (\texttt{13h} untuk 320x200x256, \texttt{03h} kembali ke teks 80x25).
  \item \textbf{Efek}: Mengatur mode tampilan dan membersihkan layar.
\end{itemize}

\subsubsection{Fungsi 0Ch: Write Pixel}
\begin{itemize}
  \item \textbf{Masukan}: \texttt{AH=0Ch}, \texttt{AL=color}, \texttt{BH=page}, \texttt{CX=x}, \texttt{DX=y}.
  \item \textbf{Efek}: Menggambar piksel pada koordinat (x,y) dengan warna.
\end{itemize}

\subsubsection{Fungsi 0Dh: Read Pixel}
\begin{itemize}
  \item \textbf{Masukan}: \texttt{AH=0Dh}, \texttt{BH=page}, \texttt{CX=x}, \texttt{DX=y}.
  \item \textbf{Keluaran}: \texttt{AL=color} pada piksel (x,y).
\end{itemize}

\subsubsection{Fungsi 0Fh: Get Video Mode}
Mengembalikan mode saat ini, ukuran kolom, dan nomor halaman aktif.

\subsection{Menggambar Piksel}
\subsubsection{Koordinat dan warna}
Koordinat (x,y) berbasis 0. Pastikan berada dalam batas resolusi. Untuk warna, pilih nilai palet \(0..255\).

\subsubsection{Akses memori video langsung}
Selain \texttt{INT 10h, 0Ch}, dapat menulis langsung ke \texttt{A000:0000 + y*320 + x} untuk kinerja. Pastikan \texttt{ES} menunjuk \texttt{A000h} dan gunakan \texttt{MOV} ke \texttt{ES:[DI]}/\texttt{ES:[BX]}.

\section{Pendalaman}\label{sec:grafik-pendalaman}
\subsection{Mode VGA 13h dan palet}
Pada mode 13h, satu byte merepresentasikan indeks palet (0..255). Pengaturan palet (DAC) dilakukan dengan port I/O VGA (di luar cakupan dasar ini), namun penting dipahami bahwa nilai piksel bukan RGB langsung. \cite{abrash_black_book}

\subsection{Akses langsung vs INT 10h}
Pemanggilan \texttt{INT 10h} untuk setiap piksel bersifat mahal. Untuk kinerja tinggi, gunakan akses langsung ke memori video: muat \texttt{ES=A000h}, hitung offset \(y\cdot 320 + x\), lalu simpan byte warna. Gunakan \texttt{REP STOSB} untuk mengisi area. \cite{abrash_black_book}

\subsection{Algoritma garis: DDA vs Bresenham}
DDA menggunakan penambahan pecahan dan pembulatan; sederhana namun lebih lambat. Bresenham memakai hanya integer, cocok untuk perangkat keras lama. Uji di emulator untuk melihat perbedaan jumlah instruksi. \cite{abrash_black_book}

\subsection{Clipping sederhana}
Selalu periksa \(0 \le x < 320\) dan \(0 \le y < 200\) sebelum menulis. Untuk garis, lakukan \textit{trivial reject} ketika seluruh segmen di luar batas; untuk implementasi penuh, terapkan Cohen–Sutherland (lanjutan). \cite{osdev_wiki}

\subsection{Algoritma Menggambar Garis}
\subsubsection{DDA (Digital Differential Analyzer)}
Menginkremen koordinat dengan langkah pecahan. Mudah dipahami namun menggunakan operasi pecahan/float atau perkiraan \textit{fixed-point}; dapat lebih lambat.

\subsubsection{Bresenham}
Menggunakan aritmatika integer dan \textit{error term} untuk menentukan piksel berikutnya. Lebih efisien dan akurat untuk layar raster.

\subsubsection{Implementasi ringkas}
\begin{verbatim}
; Pseudo: Bresenham integer
 dx = abs(x1-x0); dy = -abs(y1-y0)
 sx = x0<x1 ? 1 : -1
 sy = y0<y1 ? 1 : -1
 err = dx + dy
 while true:
   plot(x0,y0)
   if x0==x1 && y0==y1: break
   e2 = 2*err
   if e2 >= dy: err += dy; x0 += sx
   if e2 <= dx: err += dx; y0 += sy
\end{verbatim}

\subsection{Koordinat dan Transformasi}
\subsubsection{Sistem koordinat}
Asal (0,0) di pojok kiri atas; x ke kanan, y ke bawah. Transformasi translasi/skalasi dapat diterapkan sebelum penggambaran.

\subsubsection{Clipping dan viewport}
Batasi penggambaran dalam area tertentu untuk menghindari akses memori di luar layar. Sederhana: \textit{reject} jika di luar; lanjutkan jika di dalam.

\section{Praktikum}
\begin{enumerate}
  \item Atur mode 13h dan gambar beberapa piksel berwarna acak.
  \item Gambar garis horizontal/vertikal dengan perulangan sederhana (inkremen x atau y).
  \item Implementasikan DDA untuk menggambar garis diagonal; bandingkan dengan implementasi sederhana.
  \item Implementasikan Bresenham dan bandingkan jumlah operasi serta kelicinan garis dengan DDA.
  \item Gambar bentuk geometri sederhana (persegi, segitiga) menggunakan fungsi garis.
\end{enumerate}

\section{Contoh Kode}
\begin{verbatim}
; Program demonstrasi grafik dasar (mode 13h)
TITLE Pemrograman Grafik
.MODEL SMALL
.STACK 100h

.DATA
    x1    DW 100
    y1    DW 100
    x2    DW 200
    y2    DW 150
    color DB 15  ; Putih

.CODE
START:
    MOV AX, @DATA
    MOV DS, AX
    
    ; Set mode grafik 320x200, 256 warna
    MOV AH, 00h
    MOV AL, 13h
    INT 10h
    
    ; Gambar piksel tunggal
    MOV AH, 0Ch
    MOV AL, color
    MOV BH, 00h
    MOV CX, 160  ; X coordinate
    MOV DX, 100  ; Y coordinate
    INT 10h
    
    ; Gambar garis horizontal
    MOV CX, 50   ; X start
    MOV DX, 50   ; Y coordinate
    MOV BX, 100  ; X end

gambar_horizontal:
    MOV AH, 0Ch
    MOV AL, color
    MOV BH, 00h
    INT 10h
    INC CX
    CMP CX, BX
    JLE gambar_horizontal
    
    ; Gambar garis vertikal
    MOV CX, 50   ; X coordinate
    MOV DX, 50   ; Y start
    MOV BX, 100  ; Y end

gambar_vertikal:
    MOV AH, 0Ch
    MOV AL, color
    MOV BH, 00h
    INT 10h
    INC DX
    CMP DX, BX
    JLE gambar_vertikal
    
    ; (Implementasi DDA/Bresenham dapat ditambahkan di sini)
    
    ; Tunggu input keyboard
    MOV AH, 00h
    INT 16h
    
    ; Kembali ke mode teks
    MOV AH, 00h
    MOV AL, 03h
    INT 10h
    
    MOV AH, 4Ch
    INT 21h
END START
\end{verbatim}

\section{Latihan}
\begin{enumerate}
  \item Buat program menggambar kotak di posisi dan ukuran tertentu; parameterkan warna.
  \item Buat program menggambar segitiga dari tiga titik; gunakan tiga panggilan fungsi garis.
  \item Buat program menggambar lingkaran sederhana (mis. \textit{midpoint circle}) atau pendekatan parametris dasar.
  \item Buat program menggambar pola grid dan diagonal untuk menguji kerapatan piksel.
\end{enumerate}

\section{Tugas}
\begin{itemize}
  \item \textbf{Bresenham}: Implementasikan penuh algoritma Bresenham untuk garis, dokumentasikan langkah dan bandingkan dengan DDA.
  \item \textbf{Grafik fungsi}: Gambar grafik \(y=f(x)\) (mis. \(\sin x\) terdiskretisasi) pada area layar; skalakan ke piksel.
  \item \textbf{Mini paint}: Buat program kecil yang memungkinkan menggambar garis dengan keyboard/mouse sederhana; simpan layar sementara di memori.
  \item \textbf{Dokumentasi}: Jelaskan perbedaan organisasi memori antar mode dan konsekuensi kinerja untuk write pixel via \texttt{INT 10h} vs akses langsung.
\end{itemize}

\section{Referensi}
% Bibliography is handled by the main document
