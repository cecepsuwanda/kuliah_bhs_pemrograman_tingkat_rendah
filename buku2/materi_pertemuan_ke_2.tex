\chapter{Arsitektur Mikroprosesor Intel 8086: Register, Segmentasi Memori, Mode Pengalamatan; Struktur File ASM}

\section{Tujuan Pembelajaran}
Setelah mengikuti pertemuan ini, mahasiswa mampu:
\begin{itemize}
  \item Menggambarkan arsitektur internal Intel 8086 beserta karakteristik historisnya.
  \item Menjelaskan fungsi register-register utama (umum, indeks, pointer, segmen, IP, FLAGS).
  \item Menjelaskan segmentasi memori, pembentukan alamat fisik, dan batasan mode real.
  \item Menerapkan berbagai mode pengalamatan pada contoh instruksi sederhana.
  \item Menyusun struktur berkas ASM yang benar dengan direktif umum.
\end{itemize}

\section{Pendahuluan}
Intel 8086 (1978) merupakan mikroprosesor 16-bit yang memperkenalkan arsitektur x86. Meskipun modernisasi arsitektur telah berlangsung selama dekade, banyak konsep fundamental (register, segmentasi, mode pengalamatan) tetap relevan untuk memahami pemrograman tingkat rendah dan kompatibilitas ke belakang.

\section{Materi Pembelajaran}
\subsection{Arsitektur Mikroprosesor Intel 8086}
\subsubsection{Sejarah dan karakteristik Intel 8086}
Intel 8086 adalah prosesor 16-bit dengan bus alamat 20-bit, memungkinkan pengalamatan hingga 1 MB memori. Diperkenalkan untuk meningkatkan kemampuan 8080/8085 dan menjadi dasar ekosistem PC awal.

\subsubsection{Arsitektur internal mikroprosesor}
8086 dapat dipandang terdiri dari dua bagian utama: \textit{Bus Interface Unit} (BIU) dan \textit{Execution Unit} (EU). BIU menangani pengambilan instruksi (prefetch) dari memori dan interaksi bus, sedangkan EU mengeksekusi instruksi, mengelola ALU, register, dan flags. Mekanisme \textit{prefetch queue} meningkatkan \textit{throughput} dengan menyiapkan instruksi sebelum dieksekusi.

\subsubsection{Bus sistem dan kontrol}
Bus alamat 20-bit memungkinkan akses 1 MB (\(2^{20}\) byte). Bus data 16-bit memindahkan data per word. Sinyal kontrol (RD, WR, M/IO, INTA, dsb.) mengoordinasikan siklus baca/tulis memori dan I/O. Interupsi dapat memicu transfer kontrol ke \textit{interrupt vector table} di alamat rendah memori.

\subsubsection{Mode operasi (real mode)}
Pada 8086, hanya ada \textit{real mode}. Alamat fisik dihitung sebagai \(\text{segmen} \times 16 + \text{offset}\). Tidak ada proteksi memori atau \textit{paging}. Kode memiliki akses penuh ke seluruh ruang alamat 1 MB, yang menuntut disiplin pengaturan segmen.

\subsection{Register-Register Utama}
\subsubsection{Register umum (AX, BX, CX, DX)}
\begin{itemize}
  \item \textbf{AX} (Accumulator): sering dipakai untuk operasi aritmatika/\textit{I/O}; memiliki bagian 8-bit \texttt{AH}/\texttt{AL}.
  \item \textbf{BX} (Base): kerap sebagai basis pengalamatan memori; bagian \texttt{BH}/\texttt{BL}.
  \item \textbf{CX} (Count): digunakan sebagai pencacah pada instruksi \texttt{LOOP}, \texttt{REP}; bagian \texttt{CH}/\texttt{CL}.
  \item \textbf{DX} (Data): terlibat pada operasi perkalian/pembagian word dan port I/O; bagian \texttt{DH}/\texttt{DL}.
\end{itemize}

\subsubsection{Register indeks (SI, DI)}
\textbf{SI} (Source Index) dan \textbf{DI} (Destination Index) digunakan pada instruksi string (\texttt{MOVS}, \texttt{CMPS}, \texttt{SCAS}, \texttt{LODS}, \texttt{STOS}). Biasanya, \texttt{DS:SI} menunjuk sumber dan \texttt{ES:DI} menunjuk tujuan.

\subsubsection{Register pointer (SP, BP)}
\textbf{SP} (Stack Pointer) menunjuk puncak stack relatif terhadap \texttt{SS}. \textbf{BP} (Base Pointer) membantu \textit{frame} stack untuk parameter lokal (sering digunakan dengan \texttt{SS:BP}).

\subsubsection{Register segmen (CS, DS, SS, ES)}
\textbf{CS}: segmen kode; \textbf{DS}: segmen data; \textbf{SS}: segmen stack; \textbf{ES}: segmen ekstra (mis. tujuan pada instruksi string). Kombinasi pasangan segmen:offset membentuk alamat fisik.

\subsubsection{Register flag (FLAGS)}
Mencerminkan hasil operasi aritmatika/logika dan mengendalikan perilaku instruksi: \textbf{ZF} (Zero), \textbf{CF} (Carry), \textbf{SF} (Sign), \textbf{OF} (Overflow), \textbf{PF} (Parity), \textbf{AF} (Auxiliary Carry), serta \textbf{IF} (Interrupt Enable), \textbf{DF} (Direction), \textbf{TF} (Trap) untuk kontrol eksekusi.

\subsubsection{Register instruksi (IP)}
\textbf{IP} (Instruction Pointer) menyimpan offset instruksi berikutnya dalam segmen kode (\texttt{CS}). Saat instruksi \texttt{CALL}/\texttt{JMP} terjadi, \texttt{IP} diperbarui sesuai target.

\subsection{Segmentasi Memori}
\subsubsection{Konsep segmentasi memori}
Memecah ruang alamat fisik menjadi segmen logis untuk kode, data, dan stack. Ini memudahkan organisasi program dan memungkinkan reposisi segmen dalam ruang 1 MB.

\subsubsection{Register segmen dan offset}
Alamat logis dinyatakan sebagai pasangan \(\text{segment}:\text{offset}\). Contoh: \texttt{DS} berisi \(0x1234\), \texttt{BX} berisi \(0x0020\), maka alamat efektif data bisa \texttt{[BX]} dengan segmen default \texttt{DS}.

\subsubsection{Perhitungan alamat fisik}
Alamat fisik: \(\text{phys} = 16 \times \text{segment} + \text{offset}\). Contoh: \(\text{CS} = 2000_h\), \(\text{IP} = 0100_h\) \(\Rightarrow \text{phys} = 0x2000\times 16 + 0x0100 = 0x20100\).

\subsubsection{Batasan memori dalam mode real}
Offset 16-bit membatasi panjang segmen hingga 64 KB. Tumpang tindih segmen dimungkinkan (segment \texttt{0x1000:0010} dan \texttt{0x1001:0000} menunjuk alamat fisik yang sama). Tidak ada proteksi akses antar segmen.

\subsection{Mode Pengalamatan}
\subsubsection{Pengalamatan register}
Operand berada pada register. Contoh: \texttt{MOV AX, BX}.

\subsubsection{Pengalamatan langsung}
Nilai konstan langsung sebagai operand. Contoh: \texttt{MOV AX, 1234h}.

\subsubsection{Pengalamatan memori langsung}
Alamat memori diberikan eksplisit (offset). Contoh (TASM sintaks): \texttt{MOV AL, [1234h]} (menggunakan segmen default \texttt{DS}).

\subsubsection{Pengalamatan register tidak langsung}
Register berisi offset yang menunjuk memori: \texttt{MOV AL, [BX]}; segmen default \texttt{DS} kecuali untuk beberapa instruksi stack.

\subsubsection{Pengalamatan berbasis indeks}
Kombinasi register basis/indeks dengan displacement: \texttt{MOV AL, [BX+SI+8]} atau \texttt{MOV AX, [BP+DI+16]} (segmen default untuk alamat berbasis \texttt{BP} adalah \texttt{SS}).

\subsubsection{Pengalamatan berbasis register}
Istilah ini sering merujuk pada variasi kombinasi \texttt{BX}, \texttt{BP}, \texttt{SI}, \texttt{DI}, dengan/ tanpa displacement. Penting untuk memahami segmen default: \texttt{DS} untuk \texttt{BX}/\texttt{SI}/\texttt{DI}, \texttt{SS} untuk \texttt{BP}.

\noindent\textbf{Contoh ringkas 8086}
\begin{verbatim}
org 100h
; Inisialisasi data
mov ax, data
mov ds, ax

; Pengalamatan langsung
mov ax, 1234h

; Memori langsung (absolute offset di DS)
mov al, [1234h]

; Tidak langsung via BX
mov bx, 0040h
mov al, [bx]

; Berbasis indeks
mov si, 0010h
mov dl, [bx+si+5]

int 20h

segment data
msg db 'X', 0
\end{verbatim}

\section{Struktur File ASM}
\subsection{Format dasar file assembly}
Program \texttt{.COM} (TASM) lazim menggunakan \texttt{org 100h} dan satu segmen kode datar; program \texttt{.EXE} umumnya mendefinisikan segmen \texttt{.code}, \texttt{.data}, dan mengisi \texttt{END} dengan titik masuk.

\subsection{Direktif assembler}
Contoh umum: \texttt{ORG}, \texttt{DB}/\texttt{DW}/\texttt{DD} (pendefinisian data), \texttt{SEGMENT}/\texttt{ENDS}, \texttt{ASSUME}, \texttt{END}. Makna tepatnya tergantung assembler (TASM/MASM/NASM) dan model memori.

\subsection{Komentar dalam kode}
Gunakan \texttt{;} untuk komentar baris. Komentar harus menjelaskan maksud, bukan menyalin ulang instruksi.

\subsection{Organisasi kode program}
Pisahkan data dari kode. Gunakan label yang deskriptif, dan pertahankan konvensi penamaan konsisten. Untuk prosedur, dokumentasikan input/output via register atau stack.

\section{Contoh Soal dan Pembahasan}
\begin{enumerate}
  \item \textbf{Fungsi register AX, BX, CX, DX}.\\ AX: akumulator; BX: basis alamat; CX: pencacah; DX: perpanjangan hasil/operand I/O.
  \item \textbf{Hitung alamat fisik jika CS = 2000h dan IP = 0100h}.\\ \(\text{phys} = 0x2000\times 16 + 0x0100 = 0x20100\).
  \item \textbf{Contoh pengalamatan register}.\\ \texttt{MOV AX, BX} menyalin isi \texttt{BX} ke \texttt{AX}.
  \item \textbf{Struktur dasar file ASM}.\\ Untuk \texttt{.COM}: \texttt{org 100h}, kode utama, \texttt{int 20h}, akhir. Untuk \texttt{.EXE}: segmen \texttt{.data}, \texttt{.code}, prosedur \texttt{main}, \texttt{END main}.
\end{enumerate}

\section{Latihan}
\begin{enumerate}
  \item Diberikan \(\text{DS}=1234_h\) dan \(\text{BX}=0020_h\). Tentukan alamat fisik untuk \texttt{MOV AL, [BX+10h]}.
  \item Tunjukkan tiga variasi mode pengalamatan berbeda untuk membaca satu byte dari array pada \texttt{DS}.
  \item Tentukan flags yang terpengaruh oleh instruksi \texttt{ADD}, \texttt{SUB}, dan berikan contoh singkat yang mengubah \texttt{CF} dan \texttt{OF}.
  \item Rancang sketsa struktur berkas ASM untuk program \texttt{.EXE} sederhana dengan satu prosedur dan satu data string.
\end{enumerate}

\section{Tugas}
\begin{itemize}
  \item \textbf{Diagram arsitektur 8086}: Gambar blok BIU/EU, bus, serta aliran prefetch. Beri keterangan fungsi setiap blok.
  \item \textbf{Esai (400--600 kata)}: Bedakan register segmen vs. register umum dari perspektif peran dalam pembentukan alamat dan eksekusi instruksi; sertakan contoh.
  \item \textbf{Studi mode pengalamatan}: Buat tabel minimal 8 contoh instruksi yang mencakup register, langsung, tidak langsung, berbasis indeks, dan kombinasi displacement. Jelaskan segmen default yang digunakan.
  \item \textbf{Template berkas ASM}: Buat \LaTeX{} lampiran berisi \textit{template} program \texttt{.COM} dan \texttt{.EXE} (TASM/MASM), dilengkapi komentar pedoman.
\end{itemize}
\noindent\textbf{Keluaran}: PDF jawaban + berkas sumber, sertakan uji asersi manual untuk perhitungan alamat fisik.

\section{Referensi}
\begin{itemize}
  \item Brey, Barry B. \textit{Mikroprosesor Intel 8086/8088 dsb.}, Penerbit Informatika.
  \item Intel Corporation. \textit{Intel 64 and IA-32 Architectures Software Developer's Manual} (bagian pengantar).
\end{itemize}
