\chapter{Pengenalan Bahasa Rakitan dan Sistem Bilangan}\label{ch:pengenalan}

\section{Tujuan Pembelajaran}\label{sec:pengenalan-tujuan}
Setelah mengikuti pertemuan ini, mahasiswa diharapkan mampu:
\begin{itemize}
    \item Menjelaskan definisi dan ruang lingkup bahasa rakitan (assembly language) serta kaitannya dengan arsitektur komputer.
    \item Menguraikan karakteristik bahasa tingkat rendah dan membandingkannya dengan bahasa tingkat tinggi.
    \item Mengidentifikasi keunggulan dan keterbatasan pemrograman assembly dalam konteks kinerja, portabilitas, dan pemeliharaan.
    \item Menjelaskan sistem bilangan biner dan heksadesimal, termasuk notasi, nilai tempat, dan relasi di antara keduanya.
    \item Melakukan konversi antar sistem bilangan (desimal, biner, heksadesimal) secara tepat dan efisien.
    \item Melakukan operasi aritmatika sederhana dalam sistem biner dan menjelaskan aturan pembawaan (carry) dan pinjaman (borrow).
    \item Mengaitkan konsep representasi data (bit, byte, word, two's complement, endianness) dengan perilaku program assembly.
\end{itemize}

\section{Pendalaman}\label{sec:pengenalan-pendalaman}
\subsection{Notasi bilangan lain dan basis umum}
Selain biner dan heksadesimal, beberapa konteks menggunakan oktal (basis 8) dan desimal (basis 10). Oktal historis dipakai pada arsitektur tertentu dan sistem Unix lama. Transformasi umum: basis 2 \(\leftrightarrow\) basis 8 (kelompok 3 bit) dan basis 2 \(\leftrightarrow\) basis 16 (kelompok 4 bit). Pemahaman konversi manual penting untuk membaca dump memori dan register. \cite{osdev_wiki}

\subsection{Rentang nilai dan sign/zero extension}
Untuk lebar \(n\)-bit, rentang tak bertanda adalah \([0,\,2^n-1]\). Untuk bertanda dengan two's complement, rentang adalah \([-2^{n-1},\,2^{n-1}-1]\). Contoh 8-bit: tak bertanda \([0..255]\), bertanda \([-128..127]\). Operasi perlu menjaga konsistensi perluasan ukuran: \textbf{zero extension} saat memaknai nilai tak bertanda (mis., memindahkan \texttt{AL} ke \texttt{AX} dengan mengosongkan \texttt{AH}), dan \textbf{sign extension} untuk bertanda (mis., \texttt{CBW} menandatangani \texttt{AL} ke \texttt{AX}). \cite{wiki_two_complement}

\subsection{Deteksi overflow: CF vs OF}
Pada operasi penjumlahan/pengurangan, \textbf{Carry Flag (CF)} menandai luapan untuk interpretasi tak bertanda, sedangkan \textbf{Overflow Flag (OF)} menandai luapan untuk interpretasi bertanda. Gunakan cabang bersyarat sesuai interpretasi data: \texttt{JC/JNC}/\texttt{JB/JA} untuk tak bertanda; \texttt{JO/JNO}, \texttt{JL/JG} (bertanda) untuk bertanda. \cite{intel2019manual32}

\subsection{Endianness dan layout memori}
Arsitektur 8086 menggunakan little-endian: byte paling rendah disimpan pada alamat terendah. Konsekuensinya: saat memeriksa dump memori untuk word \(0x1234\), tampilannya adalah \texttt{34 12}. Hal ini juga memengaruhi serialisasi data dan interoperabilitas antar mesin. \cite{wiki_endianness}

\subsection{BCD dan konversi karakter}
Sebagian aplikasi era DOS menggunakan \textit{Binary Coded Decimal} (BCD) untuk operasi numerik-desimal. 8086 menyediakan instruksi bantu seperti \texttt{DAA}/\texttt{DAS} (penyesuaian setelah \texttt{ADD}/\texttt{SUB} BCD) serta \texttt{AAA}/\texttt{AAS} untuk ASCII-adjust. Saat menampilkan angka, konversi biner \(\to\) ASCII lazim dilakukan dengan pembagian berulang basis 10 dan menumpuk sisa digit. \cite{intel2019manual32,rbil}

\subsection{Latihan tambahan}
\begin{enumerate}
  \item Tentukan hasil dan status \texttt{CF}/\texttt{OF}/\texttt{SF}/\texttt{ZF} untuk operasi 8-bit: \texttt{0x7F + 0x01}, \texttt{0xFF + 0x01}, \texttt{0x80 - 0x01}.
  \item Lakukan sign-extension dan zero-extension dari 8-bit ke 16-bit untuk nilai \texttt{0xF0} dan jelaskan perbedaannya dalam desimal bertanda/tak bertanda.
  \item Implementasikan (di assembler) konversi word 16-bit ke string desimal (ASCII) tanpa menggunakan instruksi BCD; tampilkan hasilnya via \texttt{INT 21h, AH=09h}.
\end{enumerate}

\section{Pendahuluan}\label{sec:pengenalan-pendahuluan}
Bahasa rakitan (assembly language) adalah bahasa pemrograman berlevel rendah yang menyediakan antarmuka langsung terhadap instruksi mesin (machine instructions) dari sebuah \textit{Instruction Set Architecture} (ISA), misalnya keluarga x86 (Intel 8086 dan penerusnya) \cite{intel2019manual32}. Berbeda dengan bahasa tingkat tinggi yang menawarkan abstraksi yang kaya (tipe data kompleks, manajemen memori otomatis, pustaka standar), assembly memetakan secara hampir satu-ke-satu ke instruksi prosesor, memberikan kontrol sangat granular atas register, memori, dan \textit{I/O} \cite{hyde2010art}. Konsekuensinya, pemrograman assembly menuntut pemahaman yang kuat tentang arsitektur perangkat keras dan representasi data.

\section{Pengenalan Bahasa Rakitan dan Bahasa Tingkat Rendah}\label{sec:pengenalan-bahasa}
\subsubsection{Definisi bahasa rakitan (assembly language)}
Assembly language adalah representasi mnemonik dari instruksi mesin \cite{susanto1995belajar}. Setiap mnemonik (misal, \texttt{MOV}, \texttt{ADD}, \texttt{JMP}) biasanya berkorelasi dekat dengan \textit{opcode} biner yang dieksekusi CPU. Kode assembly dirakit (\textit{assembled}) oleh \textit{assembler} (misal: TASM, MASM, NASM) menjadi \textit{object code} atau berkas executable \cite{hyde2010art}.

\subsubsection{Karakteristik bahasa tingkat rendah}
\begin{itemize}
    \item \textbf{Spesifik arsitektur}: Kode yang ditulis untuk ISA tertentu (misal 8086) tidak portabel ke ISA lain tanpa penulisan ulang \cite{susanto1995belajar}.
    \item \textbf{Kontrol granular}: Pemrogram memiliki kontrol langsung atas register CPU, \textit{flags}, dan layout memori \cite{hyde2010art}.
    \item \textbf{Kinerja dan jejak memori}: Dapat dioptimasi untuk ukuran kecil dan latensi rendah, relevan untuk \textit{embedded} atau komponen \textit{runtime-critical}.
    \item \textbf{Kurangnya abstraksi tingkat tinggi}: Tidak tersedia struktur kontrol tingkat tinggi, \textit{garbage collector}, atau tipe kompleks secara bawaan \cite{susanto1995belajar}.
\end{itemize}

\subsubsection{Perbandingan dengan bahasa tingkat tinggi}
\begin{center}
\begin{tabular}{p{0.28\textwidth} p{0.31\textwidth} p{0.31\textwidth}}
\hline
\textbf{Aspek} & \textbf{Assembly} & \textbf{Bahasa Tingkat Tinggi} \\
\hline
Abstraksi & Rendah; dekat hardware & Tinggi; jauh dari hardware \\
Portabilitas & Rendah (spesifik ISA) & Lebih tinggi (kompiler/VM) \\
Produktivitas & Rendah & Tinggi \\
Kinerja puncak & Sangat tinggi (jika dioptimasi) & Umumnya baik; optimasi otomatis \\
Pemeliharaan & Sulit & Lebih mudah \\
\hline
\end{tabular}
\end{center}

\subsubsection{Keunggulan dan kelemahan bahasa rakitan}
\begin{itemize}
    \item \textbf{Keunggulan}:
    \begin{itemize}
        \item Kontrol penuh terhadap perangkat keras.
        \item Optimasi mikroskopik untuk kinerja/ukuran.
        \item Pemahaman mendalam atas \textit{runtime} dan \textit{calling convention}.
    \end{itemize}
    \item \textbf{Kelemahan}:
    \begin{itemize}
        \item Pengembangan lambat dan rentan kesalahan.
        \item Sulit dipelihara dan kurang portabel.
        \item Minim dukungan pustaka.
    \end{itemize}
\end{itemize}

\subsubsection{Aplikasi bahasa rakitan dalam pemrograman}
\begin{itemize}
    \item \textbf{Perangkat tertanam (embedded)}: \textit{firmware} mikrokontroler, \textit{device driver}, \textit{interrupt service routine} \cite{hyde2010art}.
    \item \textbf{Sistem rendah tingkat}: \textit{bootloader}, BIOS/UEFI, bagian \textit{kernel} yang sensitif kinerja \cite{intel2019manual32}.
    \item \textbf{Optimasi hotspot}: Bagian kecil dari aplikasi yang memerlukan latensi minimal \cite{susanto1995belajar}.
    \item \textbf{Rekayasa balik dan keamanan}: Analisis \textit{malware}, \textit{exploit development}, \textit{debugging} pada level instruksi \cite{hyde2010art}.
\end{itemize}

\noindent\textbf{Cuplikan 8086 (ilustratif)}
\begin{verbatim}
; Tambah dua nilai 8-bit, hasil di AL
org 100h          ; format .COM
mov al, 25h       ; AL = 0x25 (37 desimal)
add al, 13h       ; AL = AL + 0x13 (19 desimal) -> 0x38 (56)
; ... hasil dapat disimpan ke memori atau dipakai lanjut
int 20h           ; kembali ke DOS
\end{verbatim}

\subsection{Sistem Bilangan}
\subsubsection{Sistem bilangan biner (basis 2)}
Biner menggunakan dua digit (0 dan 1). Nilai tempatnya adalah pangkat 2: \(\ldots, 2^3, 2^2, 2^1, 2^0\). Contoh: \(101010_2 = 1\cdot 2^5 + 0\cdot 2^4 + 1\cdot 2^3 + 0\cdot 2^2 + 1\cdot 2^1 + 0\cdot 2^0 = 42_{10}\).

\subsubsection{Sistem bilangan heksadesimal (basis 16)}
Heksadesimal menggunakan digit 0--9 dan A--F (A=10, B=11, \ldots, F=15). Setiap digit heksadesimal memetakan tepat ke 4 bit (\textit{nibble}). Contoh: \(\mathrm{D6}_{16} = 13\cdot 16^1 + 6\cdot 16^0 = 214_{10}\) dan dalam biner menjadi \(\mathrm{D} = 1101\), \(6 = 0110\) sehingga \(\mathrm{D6}_{16} = 1101\,0110_2\).

\subsubsection{Konversi antar sistem bilangan}
\paragraph{Desimal ke biner} gunakan pembagian berulang oleh 2 dan catat sisa; baca sisa dari belakang. Contoh: \(255_{10} \to 11111111_2\).
\paragraph{Biner ke desimal} gunakan penjumlahan nilai tempat. Contoh: \(11010110_2 = 128+64+16+4+2 = 214_{10}\).
\paragraph{Desimal ke heksadesimal} gunakan pembagian berulang oleh 16. Contoh: \(255_{10} \to \mathrm{FF}_{16}\).
\paragraph{Heksadesimal ke desimal} gunakan nilai tempat basis 16. Contoh: \(\mathrm{A5F}_{16} = 10\cdot 16^2 + 5\cdot 16 + 15 = 2655_{10}\).
\paragraph{Heksadesimal ke biner} peta setiap digit ke 4 bit: \(\mathrm{A5F}_{16} = 1010\,0101\,1111_2\).

\subsubsection{Operasi aritmatika dalam biner}
Penjumlahan biner mengikuti aturan: \(0+0=0\), \(0+1=1\), \(1+1=10_2\) (tulis 0, simpan \textit{carry} 1). Contoh:
\begin{verbatim}
   1011   (11 desimal)
 + 1101   (13 desimal)
 -------
  11000   (24 desimal)
\end{verbatim}

\subsubsection{Representasi data dalam komputer}
\begin{itemize}
    \item \textbf{Bit, byte, word}: 1 byte = 8 bit; pada 8086, word = 16 bit \cite{intel2019manual32}.
    \item \textbf{Two's complement}: Representasi bilangan bulat bertanda di mana \(-x\) diperoleh dengan komplemen satu kemudian tambah satu. Contoh (8-bit): \(-13\) dari \(00001101_2\) menjadi \(11110011_2\) \cite{hyde2010art}.
    \item \textbf{Endianness (8086 = little-endian)}: \textit{least significant byte} disimpan pada alamat memori terendah. Contoh: word \(0x1234\) disimpan sebagai byte \(0x34\) lalu \(0x12\) \cite{intel2019manual32}.
\end{itemize}

\section{Contoh Soal dan Pembahasan}\label{sec:pengenalan-contoh}
\begin{enumerate}
    \item \textbf{Konversikan bilangan desimal 255 ke biner dan heksadesimal.}\\
    \(255_{10} = 11111111_2 = \mathrm{FF}_{16}\).

    \item \textbf{Konversikan bilangan biner \(11010110_2\) ke desimal dan heksadesimal.}\\
    Nilai tempat: \(128+64+16+4+2 = 214\). Kelompok 4-bit: \(1101\,0110_2 = \mathrm{D6}_{16}\).

    \item \textbf{Konversikan bilangan heksadesimal \(\mathrm{A5F}_{16}\) ke desimal dan biner.}\\
    \(10\cdot 256 + 5\cdot 16 + 15 = 2655_{10}\). Biner: \(\mathrm{A}\to 1010\), \(\mathrm{5}\to 0101\), \(\mathrm{F}\to 1111\) sehingga \(1010\,0101\,1111_2\).

    \item \textbf{Lakukan operasi penjumlahan biner: \(1011_2 + 1101_2\).}\\
    Hasil: \(11000_2\) sebagaimana langkah pembawaan pada contoh di atas (\(11+13=24\)).
\end{enumerate}

\section{Latihan}
Kerjakan tanpa kalkulator, tulis langkah konversi/aritmatika secara jelas.
\begin{enumerate}
    \item Ubah ke biner dan heksadesimal: \(37_{10}\), \(1023_{10}\), \(4095_{10}\).
    \item Ubah ke desimal dan heksadesimal: \(10010101_2\), \(11110000_2\), \(101011110101_2\).
    \item Ubah ke desimal dan biner: \(\mathrm{7B}_{16}\), \(\mathrm{1C3}_{16}\), \(\mathrm{FE}_{16}\).
    \item Hitung penjumlahan biner: \(11101_2 + 10111_2\), \(101010_2 + 1111_2\).
    \item Untuk 8-bit two's complement, representasikan \(-5\), \(-27\), dan verifikasi dengan penjumlahan dengan nilai positifnya menghasilkan nol (modulo 256).
\end{enumerate}

\section{Tugas}
\begin{itemize}
    \item \textbf{Tabel konversi 0--15}: Buat tabel berisi tiga kolom: desimal, biner (4-bit), heksadesimal. Gunakan lingkungan \texttt{tabular} LaTeX, pastikan format rapi dan konsisten.
    \item \textbf{Esai singkat (300--500 kata)}: Jelaskan mengapa heksadesimal sering digunakan dalam pemrograman \textit{low-level} (alamat memori, \textit{debugging}, \textit{bitmask}, korespondensi nibble--hex, keterbacaan). Sertakan contoh konkret (mis., merepresentasikan pola \textit{bit} register prosesor).
    \item \textbf{Aplikasi assembly dalam kehidupan sehari-hari}: Berikan minimal dua contoh nyata (mis., \textit{bootloader}, \textit{driver} perangkat, \textit{firmware} mikrokontroler). Jelaskan alur singkat peran assembly dan alasannya digunakan (kinerja, determinisme, ukuran kode).
\end{itemize}
\noindent\textbf{Keluaran yang dikumpulkan}: (i) berkas PDF berisi jawaban dan tabel; (ii) berkas \LaTeX{} sumber; (iii) jika menggunakan kode, lampirkan cuplikan dan komentar.

\section{Referensi}
% Bibliography is handled by the main document
