\chapter{Array dan String}

\section{Tujuan Pembelajaran}
Mahasiswa mampu:
\begin{itemize}
    \item Menjelaskan representasi array dan string dalam memori serta pengalamatan elemennya.
    \item Menggunakan instruksi string \texttt{MOVS}, \texttt{CMPS}, \texttt{SCAS}, \texttt{LODS}, \texttt{STOS} dengan benar.
    \item Mengoptimalkan operasi berulang menggunakan awalan \texttt{REP}/\texttt{REPE}/\texttt{REPNE} dan pengelolaan \texttt{DF}.
    \item Membangun rutinitas manipulasi string/array (salin, banding, cari, panjang, balik, gabung, urut).
\end{itemize}

\section{Pendahuluan}
Array dan string adalah struktur data fundamental. Pada 8086, operasi berulang terhadap data berurutan didukung oleh instruksi string yang menggabungkan pembacaan/penulisan data dengan peningkatan/penurunan indeks otomatis (\texttt{SI}/\texttt{DI}) dan pengurangan pencacah (\texttt{CX}) bila digabung dengan awalan \texttt{REP}. Pengelolaan \texttt{Direction Flag} (\texttt{DF}) menentukan arah iterasi (naik dengan \texttt{CLD}, turun dengan \texttt{STD}).

\section{Konsep Array dan String}
\subsection{Definisi dan penyimpanan}
Array adalah koleksi elemen bertipe homogen yang tersusun kontigu. String dapat dipandang sebagai array karakter dengan konvensi terminasi (mis. \texttt{0} atau \texttt{\$}). Penyimpanan memperhatikan ukuran elemen (byte/word) untuk perhitungan offset.

\subsubsection{Pengalamatan dan perhitungan offset}
Offset elemen ke-\(i\) pada array byte: \(base + i\). Pada array word: \(base + 2\cdot i\). Gunakan \texttt{SI}/\texttt{DI} sebagai indeks sumber/tujuan.

\subsubsection{String berterminasi}
\begin{itemize}
  \item \textbf{Null-terminated (C)}: berakhir dengan \texttt{0}. Fungsi seperti panjang memindai hingga \texttt{0}.
  \item \textbf{DOS \$-terminated}: digunakan oleh \texttt{INT 21h, AH=09h}; berakhir dengan \texttt{\$}.
\end{itemize}
Konsisten gunakan satu konvensi per rutinitas.

\subsection{Penyimpanan Array}
\subsubsection{Satu dimensi}
Akses langsung via indeks. Untuk word, manfaatkan \texttt{XLAT} atau perhitungan manual \(i\times 2\).

\subsubsection{Multi dimensi}
Gunakan perataan baris (row-major). Offset \((r,c)\) pada array word berdimensi \(R\times C\): \(base + 2\cdot (r\cdot C + c)\).

\subsection{Instruksi String}
\subsubsection{MOVS}
Menyalin dari \texttt{DS:SI} ke \texttt{ES:DI}; varian \texttt{MOVSB}/\texttt{MOVSW}. Dengan \texttt{REP}, menyalin \texttt{CX} elemen.
\begin{verbatim}
cld
mov  cx, len
lea  si, src
lea  di, dst
rep  movsb
\end{verbatim}

\subsubsection{CMPS}
Membandingkan \texttt{DS:SI} dengan \texttt{ES:DI}; atur \textit{flags}. Dengan \texttt{REPE}/\texttt{REPZ} berhenti pada perbedaan atau saat \texttt{CX}=0.
\begin{verbatim}
cld
mov  cx, n
lea  si, s1
lea  di, s2
repe cmpsb
jne  berbeda
\end{verbatim}

\subsubsection{SCAS}
Membandingkan \texttt{AL}/\texttt{AX} dengan \texttt{ES:DI}. Umum untuk mencari karakter/word.
\begin{verbatim}
cld
mov  al, ch
mov  cx, n
lea  di, buf
repne scasb
jne  not_found
\end{verbatim}

\subsubsection{LODS}
Memuat dari \texttt{DS:SI} ke \texttt{AL}/\texttt{AX}, menaikkan/menurunkan \texttt{SI}.

\subsubsection{STOS}
Menyimpan dari \texttt{AL}/\texttt{AX} ke \texttt{ES:DI}; dengan \texttt{REP}, mengisi blok dengan nilai yang sama.
\begin{verbatim}
cld
mov  al, 0
mov  cx, 80
lea  di, line
rep  stosb
\end{verbatim}

\subsubsection{Awalan REP dan Direction Flag}
\texttt{REP} mengulang hingga \texttt{CX}=0. \texttt{REPE}/\texttt{REPZ} mengulang selama \texttt{ZF=1}; \texttt{REPNE}/\texttt{REPNZ} selama \texttt{ZF=0}. Gunakan \texttt{CLD} untuk iterasi maju, \texttt{STD} untuk mundur; kembalikan ke \texttt{CLD} demi konsistensi.

\subsection{Register untuk Operasi String}
\begin{itemize}
  \item \textbf{SI/DI}: indeks sumber/tujuan (\texttt{DS:SI}, \texttt{ES:DI}).
  \item \textbf{CX}: pencacah pengulangan untuk \texttt{REP}.
  \item \textbf{AL/AX}: akumulator untuk \texttt{LODS}/\texttt{SCAS}/\texttt{STOS}.
  \item \textbf{DF}: arah iterasi; atur dengan \texttt{CLD}/\texttt{STD}.
  \item \textbf{ES}: segmen tujuan harus diinisialisasi (sering disamakan dengan \texttt{DS} bila menyalin dalam segmen sama).
\end{itemize}

\section{Praktikum}
\begin{enumerate}
  \item Demonstrasi akses elemen array (byte/word) dan perhitungan offset.
  \item Salin string menggunakan \texttt{REP MOVSB} dan verifikasi.
  \item Bandingkan dua string dengan \texttt{REPE CMPSB}; tampilkan hasil sama/berbeda.
  \item Cari karakter dalam string dengan \texttt{REPNE SCASB}; laporkan posisi ditemukan.
  \item Implementasi rutin panjang dan pembalik string menggunakan \texttt{LODS}/\texttt{STOS}.
\end{enumerate}

\section{Contoh Kode}
\begin{verbatim}
; Program demonstrasi array dan string
TITLE Array dan String
.MODEL SMALL
.STACK 100h

.DATA
    array   DW 10, 20, 30, 40, 50
    string1 DB 'Hello World$'
    string2 DB 20 DUP('$')
    string3 DB 'Assembly$'
    string4 DB 'Assembly$'
    karakter DB 'l'
    panjang EQU $ - string1

.CODE
START:
    MOV AX, @DATA
    MOV DS, AX
    MOV ES, AX
    
    ; Akses elemen array (word): indeks ke-2 (basis 0)
    MOV BX, 2
    SHL BX, 1            ; *2 untuk word
    MOV AX, array[BX]    ; AX = 30
    
    ; Copy string menggunakan MOVS
    LEA SI, string1
    LEA DI, string2
    MOV CX, panjang
    CLD
    REP MOVSB
    
    ; Perbandingan string menggunakan CMPS
    LEA SI, string3
    LEA DI, string4
    MOV CX, 8
    CLD
    REPE CMPSB
    JE sama
    JMP lanjut
sama:
    ; String sama
lanjut:
    ; Pencarian karakter menggunakan SCAS
    LEA DI, string1
    MOV AL, karakter
    MOV CX, panjang
    CLD
    REPNE SCASB
    JE ditemukan
    JMP selesai

ditemukan:
    ; Karakter ditemukan (posisi dapat dihitung)
selesai:
    ; Isi blok dengan 'X' menggunakan STOS
    LEA DI, string2
    MOV AL, 'X'
    MOV CX, 5
    CLD
    REP STOSB
    
    MOV AH, 4Ch
    INT 21h
END START
\end{verbatim}

\section{Latihan}
\begin{enumerate}
  \item Akses elemen array dengan indeks dan tampilkan nilainya dalam heksadesimal.
  \item Salin string dari satu lokasi ke lokasi lain menggunakan \texttt{MOVS}; pastikan terminator ikut disalin sesuai konvensi.
  \item Bandingkan dua string dengan panjang berbeda; tentukan hasil leksikografis sederhana.
  \item Cari semua kemunculan karakter dalam string dan catat posisinya ke array hasil.
\end{enumerate}

\section{Tugas}
\begin{itemize}
  \item \textbf{Panjang string}: Implementasikan \texttt{strlen} untuk terminator \texttt{0} \textbf{atau} \texttt{'{\$}'} (pilih salah satu) menggunakan \texttt{SCASB}.
  \item \textbf{Balik string}: Implementasikan pembalik in-place menggunakan dua indeks (awal/akhir) atau \texttt{LODS}/\texttt{STOS} ke buffer sementara.
  \item \textbf{Gabung string}: Implementasikan \texttt{strcat} sederhana; pastikan tidak terjadi \textit{overflow} buffer tujuan.
  \item \textbf{Urutkan array}: Urutkan array word kecil (mis. 10 elemen) dengan \textit{bubble sort}; tampilkan hasil setiap putaran.
\end{itemize}

