\documentclass[a4paper,12pt]{book}
\usepackage[utf8]{inputenc}
\usepackage[indonesian]{babel}
\usepackage{enumitem}
\usepackage{geometry}
\usepackage{fancyhdr}
\usepackage{titlesec}
\usepackage{tocloft}
\usepackage{amsmath}
\usepackage{color}
\usepackage{listings}
\usepackage{hyperref}
\usepackage{booktabs}
\usepackage{array}
\usepackage{longtable}
\usepackage{caption}
\usepackage[style=ieee,backend=biber,sorting=nyt]{biblatex}
\usepackage{csquotes}
\addbibresource{referensi.bib}

% Configure csquotes for Indonesian language
\MakeOuterQuote{"}

% Custom biblatex configuration for Indonesian language
\DefineBibliographyStrings{english}{
  bibliography = {Daftar Pustaka},
  references = {Referensi},
  and = {dan},
  andmore = {dkk\adddot},
  in = {dalam},
  editor = {Editor},
  editors = {Editor},
  translator = {Penerjemah},
  translators = {Penerjemah},
  byeditor = {diedit oleh},
  bytranslator = {diterjemahkan oleh},
  volume = {Jilid},
  volumes = {Jilid},
  number = {Nomor},
  chapter = {Bab},
  pages = {Halaman},
  page = {Halaman},
  andothers = {dkk\adddot},
  edition = {Edisi},
  publisher = {Penerbit},
  location = {Tempat},
  date = {Tanggal},
  url = {URL},
  urlseen = {Diakses},
  urldate = {Tanggal Akses}
}

% Setup automatic numbering
\setcounter{chapter}{0}
\setcounter{secnumdepth}{3}
\setcounter{tocdepth}{3}

% Penomoran otomatis untuk section/subsection
\renewcommand{\thesection}{\arabic{chapter}.\arabic{section}}
\renewcommand{\thesubsection}{\thesection.\arabic{subsection}}
\renewcommand{\thesubsubsection}{\thesubsection.\arabic{subsubsection}}

% Pengaturan halaman
\geometry{margin=2cm, headheight=14.5pt}
\pagestyle{fancy}
\fancyhf{}
\fancyfoot[C]{\thepage}
\fancyhead[L]{Pemrograman Bahasa Tingkat Rendah}
\fancyhead[R]{}

% Pengaturan judul chapter dengan penomoran otomatis
\titleformat{\chapter}[display]
{\normalfont\huge\bfseries}{\chaptertitlename\ \thechapter}{20pt}{\Huge}
\titlespacing*{\chapter}{0pt}{50pt}{40pt}

% Pengaturan section dan subsection
\titleformat{\section}
{\normalfont\Large\bfseries}{\thesection}{1em}{}
\titleformat{\subsection}
{\normalfont\large\bfseries}{\thesubsection}{1em}{}
\titleformat{\subsubsection}
{\normalfont\normalsize\bfseries}{\thesubsubsection}{1em}{}

% Pengaturan daftar isi
\renewcommand{\cftchapleader}{\cftdotfill{\cftdotsep}}
\renewcommand{\cftsecleader}{\cftdotfill{\cftdotsep}}

% Setup hyperref
\hypersetup{
    colorlinks=true,
    linkcolor=blue,
    filecolor=magenta,      
    urlcolor=cyan,
    citecolor=red,
    bookmarksnumbered=true,
    bookmarksopen=true,
    bookmarksopenlevel=2
}

% Setup code listing
\lstset{
    basicstyle=\ttfamily\footnotesize,
    breaklines=true,
    frame=single,
    numbers=left,
    numberstyle=\tiny,
    stepnumber=1,
    numbersep=8pt,
    showstringspaces=false,
    tabsize=2,
    language=[x86masm]Assembler
}

\begin{document}

% Halaman judul
\begin{titlepage}
\begin{center}
\vspace*{2cm}
{\Huge\bfseries Bahasa Pemrograman Tingkat Rendah}\\[0.5cm]
{\Large\bfseries (Assembly Language - Intel 8086)}\\[1cm]
{\large Mata Kuliah Teknik Informatika}\\[2cm]

\vfill

{\large Disusun oleh:}\\
{\large Cecep Suwanda, S.Si., M.Kom.}\\[1cm]

{\large Program Studi Teknik Informatika}\\
{\large Fakultas Teknologi Informasi}\\
{\large Universitas Bale Bandung}\\[1cm]

{\large \today}
\end{center}
\end{titlepage}

% Daftar isi
\tableofcontents
\newpage

% Bagian I: Dasar-dasar Assembly
\part{Dasar-dasar Pemrograman Assembly}\label{part:basics}

\chapter{Pengenalan Bahasa Rakitan dan Bahasa Tingkat Rendah; Sistem Bilangan (Biner, Heksadesimal)}

\section{Tujuan Pembelajaran}
Setelah mengikuti pertemuan ini, mahasiswa diharapkan mampu:
\begin{itemize}
  \item Menjelaskan definisi dan ruang lingkup bahasa rakitan (assembly language) serta kaitannya dengan arsitektur komputer.
  \item Menguraikan karakteristik bahasa tingkat rendah dan membandingkannya dengan bahasa tingkat tinggi.
  \item Mengidentifikasi keunggulan dan keterbatasan pemrograman assembly dalam konteks kinerja, portabilitas, dan pemeliharaan.
  \item Menjelaskan sistem bilangan biner dan heksadesimal, termasuk notasi, nilai tempat, dan relasi di antara keduanya.
  \item Melakukan konversi antar sistem bilangan (desimal, biner, heksadesimal) secara tepat dan efisien.
  \item Melakukan operasi aritmatika sederhana dalam sistem biner dan menjelaskan aturan pembawaan (carry) dan pinjaman (borrow).
  \item Mengaitkan konsep representasi data (bit, byte, word, two's complement, endianness) dengan perilaku program assembly.
\end{itemize}

\section{Pendahuluan}
Bahasa rakitan (assembly language) adalah bahasa pemrograman berlevel rendah yang menyediakan antarmuka langsung terhadap instruksi mesin (machine instructions) dari sebuah \textit{Instruction Set Architecture} (ISA), misalnya keluarga x86 (Intel 8086 dan penerusnya). Berbeda dengan bahasa tingkat tinggi yang menawarkan abstraksi yang kaya (tipe data kompleks, manajemen memori otomatis, pustaka standar), assembly memetakan secara hampir satu-ke-satu ke instruksi prosesor, memberikan kontrol sangat granular atas register, memori, dan \textit{I/O}. Konsekuensinya, pemrograman assembly menuntut pemahaman yang kuat tentang arsitektur perangkat keras dan representasi data.

\section{Materi Pembelajaran}
\subsection{Pengenalan Bahasa Rakitan dan Bahasa Tingkat Rendah}
\subsubsection{Definisi bahasa rakitan (assembly language)}
Assembly language adalah representasi mnemonik dari instruksi mesin. Setiap mnemonik (misal, \texttt{MOV}, \texttt{ADD}, \texttt{JMP}) biasanya berkorelasi dekat dengan \textit{opcode} biner yang dieksekusi CPU. Kode assembly dirakit (\textit{assembled}) oleh \textit{assembler} (misal: TASM, MASM, NASM) menjadi \textit{object code} atau berkas executable.

\subsubsection{Karakteristik bahasa tingkat rendah}
\begin{itemize}
  \item \textbf{Spesifik arsitektur}: Kode yang ditulis untuk ISA tertentu (misal 8086) tidak portabel ke ISA lain tanpa penulisan ulang.
  \item \textbf{Kontrol granular}: Pemrogram memiliki kontrol langsung atas register CPU, \textit{flags}, dan layout memori.
  \item \textbf{Kinerja dan jejak memori}: Dapat dioptimasi untuk ukuran kecil dan latensi rendah, relevan untuk \textit{embedded} atau komponen \textit{runtime-critical}.
  \item \textbf{Kurangnya abstraksi tingkat tinggi}: Tidak tersedia struktur kontrol tingkat tinggi, \textit{garbage collector}, atau tipe kompleks secara bawaan.
\end{itemize}

\subsubsection{Perbandingan dengan bahasa tingkat tinggi}
\begin{center}
\begin{tabular}{p{0.28\textwidth} p{0.31\textwidth} p{0.31\textwidth}}
\hline
\textbf{Aspek} & \textbf{Assembly} & \textbf{Bahasa Tingkat Tinggi} \\
\hline
Abstraksi & Rendah; dekat hardware & Tinggi; jauh dari hardware \\
Portabilitas & Rendah (spesifik ISA) & Lebih tinggi (kompiler/VM) \\
Produktivitas & Rendah & Tinggi \\
Kinerja puncak & Sangat tinggi (jika dioptimasi) & Umumnya baik; optimasi oleh kompiler \\
Pemeliharaan & Sulit & Lebih mudah \\
\hline
\end{tabular}
\end{center}

\subsubsection{Keunggulan dan kelemahan bahasa rakitan}
\paragraph{Keunggulan} (i) kontrol penuh terhadap perangkat keras; (ii) optimasi mikroskopik untuk kinerja/ukuran; (iii) pemahaman mendalam atas \textit{runtime} dan \textit{calling convention}.\newline
\paragraph{Kelemahan} (i) pengembangan lambat dan rentan kesalahan; (ii) sulit dipelihara dan kurang portabel; (iii) minim dukungan pustaka.

\subsubsection{Aplikasi bahasa rakitan dalam pemrograman}
\begin{itemize}
  \item \textbf{Perangkat tertanam (embedded)}: \textit{firmware} mikrokontroler, \textit{device driver}, \textit{interrupt service routine}.
  \item \textbf{Sistem rendah tingkat}: \textit{bootloader}, BIOS/UEFI, bagian \textit{kernel} yang sensitif kinerja.
  \item \textbf{Optimasi hotspot}: Bagian kecil dari aplikasi yang memerlukan latensi minimal.
  \item \textbf{Rekayasa balik dan keamanan}: Analisis \textit{malware}, \textit{exploit development}, \textit{debugging} pada level instruksi.
\end{itemize}

\noindent\textbf{Cuplikan 8086 (ilustratif)}
\begin{verbatim}
; Tambah dua nilai 8-bit, hasil di AL
org 100h          ; format .COM
mov al, 25h       ; AL = 0x25 (37 desimal)
add al, 13h       ; AL = AL + 0x13 (19 desimal) -> 0x38 (56)
; ... hasil dapat disimpan ke memori atau dipakai lanjut
int 20h           ; kembali ke DOS
\end{verbatim}

\subsection{Sistem Bilangan}
\subsubsection{Sistem bilangan biner (basis 2)}
Biner menggunakan dua digit (0 dan 1). Nilai tempatnya adalah pangkat 2: \(\ldots, 2^3, 2^2, 2^1, 2^0\). Contoh: \(101010_2 = 1\cdot 2^5 + 0\cdot 2^4 + 1\cdot 2^3 + 0\cdot 2^2 + 1\cdot 2^1 + 0\cdot 2^0 = 42_{10}\).

\subsubsection{Sistem bilangan heksadesimal (basis 16)}
Heksadesimal menggunakan digit 0--9 dan A--F (A=10, B=11, \ldots, F=15). Setiap digit heksadesimal memetakan tepat ke 4 bit (\textit{nibble}). Contoh: \(\mathrm{D6}_{16} = 13\cdot 16^1 + 6\cdot 16^0 = 214_{10}\) dan dalam biner menjadi \(\mathrm{D} = 1101\), \(6 = 0110\) sehingga \(\mathrm{D6}_{16} = 1101\,0110_2\).

\subsubsection{Konversi antar sistem bilangan}
\paragraph{Desimal ke biner} gunakan pembagian berulang oleh 2 dan catat sisa; baca sisa dari belakang. Contoh: \(255_{10} \to 11111111_2\).
\paragraph{Biner ke desimal} gunakan penjumlahan nilai tempat. Contoh: \(11010110_2 = 128+64+16+4+2 = 214_{10}\).
\paragraph{Desimal ke heksadesimal} gunakan pembagian berulang oleh 16. Contoh: \(255_{10} \to \mathrm{FF}_{16}\).
\paragraph{Heksadesimal ke desimal} gunakan nilai tempat basis 16. Contoh: \(\mathrm{A5F}_{16} = 10\cdot 16^2 + 5\cdot 16 + 15 = 2655_{10}\).
\paragraph{Heksadesimal ke biner} peta setiap digit ke 4 bit: \(\mathrm{A5F}_{16} = 1010\,0101\,1111_2\).

\subsubsection{Operasi aritmatika dalam biner}
Penjumlahan biner mengikuti aturan: \(0+0=0\), \(0+1=1\), \(1+1=10_2\) (tulis 0, simpan \textit{carry} 1). Contoh:
\begin{verbatim}
   1011   (11 desimal)
 + 1101   (13 desimal)
 -------
  11000   (24 desimal)
\end{verbatim}

\subsubsection{Representasi data dalam komputer}
\begin{itemize}
  \item \textbf{Bit, byte, word}: 1 byte = 8 bit; pada 8086, word = 16 bit.
  \item \textbf{Two's complement}: Representasi bilangan bulat bertanda di mana \(-x\) diperoleh dengan komplemen satu kemudian tambah satu. Contoh (8-bit): \(-13\) dari \(00001101_2\) menjadi \(11110011_2\).
  \item \textbf{Endianness (8086 = little-endian)}: \textit{least significant byte} disimpan pada alamat memori terendah. Contoh: word \(0x1234\) disimpan sebagai byte \(0x34\) lalu \(0x12\).
\end{itemize}

\section{Contoh Soal dan Pembahasan}
\begin{enumerate}
  \item \textbf{Konversikan bilangan desimal 255 ke biner dan heksadesimal.}\\
  \(255_{10} = 11111111_2 = \mathrm{FF}_{16}\).

  \item \textbf{Konversikan bilangan biner \(11010110_2\) ke desimal dan heksadesimal.}\\
  Nilai tempat: \(128+64+16+4+2 = 214\). Kelompok 4-bit: \(1101\,0110_2 = \mathrm{D6}_{16}\).

  \item \textbf{Konversikan bilangan heksadesimal \(\mathrm{A5F}_{16}\) ke desimal dan biner.}\\
  \(10\cdot 256 + 5\cdot 16 + 15 = 2655_{10}\). Biner: \(\mathrm{A}\to 1010\), \(\mathrm{5}\to 0101\), \(\mathrm{F}\to 1111\) sehingga \(1010\,0101\,1111_2\).

  \item \textbf{Lakukan operasi penjumlahan biner: \(1011_2 + 1101_2\).}\\
  Hasil: \(11000_2\) sebagaimana langkah pembawaan pada contoh di atas (\(11+13=24\)).
\end{enumerate}

\section{Latihan}
Kerjakan tanpa kalkulator, tulis langkah konversi/aritmatika secara jelas.
\begin{enumerate}
  \item Ubah ke biner dan heksadesimal: \(37_{10}\), \(1023_{10}\), \(4095_{10}\).
  \item Ubah ke desimal dan heksadesimal: \(10010101_2\), \(11110000_2\), \(101011110101_2\).
  \item Ubah ke desimal dan biner: \(\mathrm{7B}_{16}\), \(\mathrm{1C3}_{16}\), \(\mathrm{FE}_{16}\).
  \item Hitung penjumlahan biner: \(11101_2 + 10111_2\), \(101010_2 + 1111_2\).
  \item Untuk 8-bit two's complement, representasikan \(-5\), \(-27\), dan verifikasi dengan penjumlahan dengan nilai positifnya menghasilkan nol (modulo 256).
\end{enumerate}

\section{Tugas}
\begin{itemize}
  \item \textbf{Tabel konversi 0--15}: Buat tabel berisi tiga kolom: desimal, biner (4-bit), heksadesimal. Gunakan lingkungan \texttt{tabular} LaTeX, pastikan format rapi dan konsisten.
  \item \textbf{Esai singkat (300--500 kata)}: Jelaskan mengapa heksadesimal sering digunakan dalam pemrograman \textit{low-level} (alamat memori, \textit{debugging}, \textit{bitmask}, korespondensi nibble--hex, keterbacaan). Sertakan contoh konkret (mis., merepresentasikan pola \textit{bit} register prosesor).
  \item \textbf{Aplikasi assembly dalam kehidupan sehari-hari}: Berikan minimal dua contoh nyata (mis., \textit{bootloader}, \textit{driver} perangkat, \textit{firmware} mikrokontroler). Jelaskan alur singkat peran assembly dan alasannya digunakan (kinerja, determinisme, ukuran kode).
\end{itemize}
\noindent\textbf{Keluaran yang dikumpulkan}: (i) berkas PDF berisi jawaban dan tabel; (ii) berkas \LaTeX{} sumber; (iii) jika menggunakan kode, lampirkan cuplikan dan komentar.

\section{Referensi}
\begin{itemize}
  \item Hyde, Randall. \textit{The Art of Assembly Language}, 2nd ed., No Starch Press, 2010.
  \item Susanto. \textit{Belajar Pemrograman Bahasa Assembly}, Elex Media Komputindo, 1995.
  \item Intel Corporation. \textit{Intel 64 and IA-32 Architectures Software Developer's Manual} (bab pengantar untuk referensi arsitektur).
\end{itemize}

\chapter{Arsitektur Mikroprosesor Intel 8086}

\section{Tujuan Pembelajaran}
Setelah mengikuti pertemuan ini, mahasiswa mampu:
\begin{itemize}
    \item Menggambarkan arsitektur internal Intel 8086 beserta karakteristik historisnya.
    \item Menjelaskan fungsi register-register utama (umum, indeks, pointer, segmen, IP, FLAGS).
    \item Menjelaskan segmentasi memori, pembentukan alamat fisik, dan batasan mode real.
    \item Menerapkan berbagai mode pengalamatan pada contoh instruksi sederhana.
    \item Menyusun struktur berkas ASM yang benar dengan direktif umum.
\end{itemize}

\section{Pendahuluan}
Intel 8086 (1978) merupakan mikroprosesor 16-bit yang memperkenalkan arsitektur x86. Meskipun modernisasi arsitektur telah berlangsung selama dekade, banyak konsep fundamental (register, segmentasi, mode pengalamatan) tetap relevan untuk memahami pemrograman tingkat rendah dan kompatibilitas ke belakang.

\section{Arsitektur Mikroprosesor Intel 8086}
\subsection{Sejarah dan karakteristik Intel 8086}
Intel 8086 adalah prosesor 16-bit dengan bus alamat 20-bit, memungkinkan pengalamatan hingga 1 MB memori. Diperkenalkan untuk meningkatkan kemampuan 8080/8085 dan menjadi dasar ekosistem PC awal.

\subsection{Arsitektur internal mikroprosesor}
8086 dapat dipandang terdiri dari dua bagian utama: \textit{Bus Interface Unit} (BIU) dan \textit{Execution Unit} (EU). BIU menangani pengambilan instruksi (prefetch) dari memori dan interaksi bus, sedangkan EU mengeksekusi instruksi, mengelola ALU, register, dan flags. Mekanisme \textit{prefetch queue} meningkatkan \textit{throughput} dengan menyiapkan instruksi sebelum dieksekusi.

\subsection{Bus sistem dan kontrol}
Bus alamat 20-bit memungkinkan akses 1 MB (\(2^{20}\) byte). Bus data 16-bit memindahkan data per word. Sinyal kontrol (RD, WR, M/IO, INTA, dsb.) mengoordinasikan siklus baca/tulis memori dan I/O. Interupsi dapat memicu transfer kontrol ke \textit{interrupt vector table} di alamat rendah memori.

\subsection{Mode operasi (real mode)}
Pada 8086, hanya ada \textit{real mode}. Alamat fisik dihitung sebagai \(\text{segmen} \times 16 + \text{offset}\). Tidak ada proteksi memori atau \textit{paging}. Kode memiliki akses penuh ke seluruh ruang alamat 1 MB, yang menuntut disiplin pengaturan segmen.

\section{Register-Register Utama}
\subsection{Register umum (AX, BX, CX, DX)}
\begin{itemize}
    \item \textbf{AX} (Accumulator): sering dipakai untuk operasi aritmatika/\textit{I/O}; memiliki bagian 8-bit \texttt{AH}/\texttt{AL}.
    \item \textbf{BX} (Base): kerap sebagai basis pengalamatan memori; bagian \texttt{BH}/\texttt{BL}.
    \item \textbf{CX} (Count): digunakan sebagai pencacah pada instruksi \texttt{LOOP}, \texttt{REP}; bagian \texttt{CH}/\texttt{CL}.
    \item \textbf{DX} (Data): terlibat pada operasi perkalian/pembagian word dan port I/O; bagian \texttt{DH}/\texttt{DL}.
\end{itemize}

\subsection{Register Indeks dan Pointer}\label{subsec:arsitektur-indeks-pointer}
\subsubsection{Register indeks (SI, DI)}
\textbf{SI} (Source Index) dan \textbf{DI} (Destination Index) digunakan pada instruksi string (\texttt{MOVS}, \texttt{CMPS}, \texttt{SCAS}, \texttt{LODS}, \texttt{STOS}). Biasanya, \texttt{DS:SI} menunjuk sumber dan \texttt{ES:DI} menunjuk tujuan.

\subsubsection{Register pointer (SP, BP)}
\textbf{SP} (Stack Pointer) menunjuk puncak stack relatif terhadap \texttt{SS}. \textbf{BP} (Base Pointer) membantu \textit{frame} stack untuk parameter lokal (sering digunakan dengan \texttt{SS:BP}).

\subsection{Register Segmen dan Kontrol}\label{subsec:arsitektur-segmen-kontrol}
\subsubsection{Register segmen (CS, DS, SS, ES)}
\textbf{CS}: segmen kode; \textbf{DS}: segmen data; \textbf{SS}: segmen stack; \textbf{ES}: segmen ekstra (mis. tujuan pada instruksi string). Kombinasi pasangan segmen:offset membentuk alamat fisik.

\subsubsection{Register flag (FLAGS)}
Mencerminkan hasil operasi aritmatika/logika dan mengendalikan perilaku instruksi: \textbf{ZF} (Zero), \textbf{CF} (Carry), \textbf{SF} (Sign), \textbf{OF} (Overflow), \textbf{PF} (Parity), \textbf{AF} (Auxiliary Carry), serta \textbf{IF} (Interrupt Enable), \textbf{DF} (Direction), \textbf{TF} (Trap) untuk kontrol eksekusi.

\subsubsection{Register instruksi (IP)}
\textbf{IP} (Instruction Pointer) menyimpan offset instruksi berikutnya dalam segmen kode (\texttt{CS}). Saat instruksi \texttt{CALL}/\texttt{JMP} terjadi, \texttt{IP} diperbarui sesuai target.

\subsection{Segmentasi Memori}
\subsubsection{Konsep segmentasi memori}
Memecah ruang alamat fisik menjadi segmen logis untuk kode, data, dan stack. Ini memudahkan organisasi program dan memungkinkan reposisi segmen dalam ruang 1 MB.

\subsubsection{Register segmen dan offset}
Alamat logis dinyatakan sebagai pasangan \(\text{segment}:\text{offset}\). Contoh: \texttt{DS} berisi \(0x1234\), \texttt{BX} berisi \(0x0020\), maka alamat efektif data bisa \texttt{[BX]} dengan segmen default \texttt{DS}.

\subsubsection{Perhitungan alamat fisik}
Alamat fisik: \(\text{phys} = 16 \times \text{segment} + \text{offset}\). Contoh: \(\text{CS} = 2000_h\), \(\text{IP} = 0100_h\) \(\Rightarrow \text{phys} = 0x2000\times 16 + 0x0100 = 0x20100\).

\subsubsection{Batasan memori dalam mode real}
Offset 16-bit membatasi panjang segmen hingga 64 KB. Tumpang tindih segmen dimungkinkan (segment \texttt{0x1000:0010} dan \texttt{0x1001:0000} menunjuk alamat fisik yang sama). Tidak ada proteksi akses antar segmen.

\subsection{Mode Pengalamatan}
\subsubsection{Pengalamatan register}
Operand berada pada register. Contoh: \texttt{MOV AX, BX}.

\subsubsection{Pengalamatan langsung}
Nilai konstan langsung sebagai operand. Contoh: \texttt{MOV AX, 1234h}.

\subsubsection{Pengalamatan memori langsung}
Alamat memori diberikan eksplisit (offset). Contoh (TASM sintaks): \texttt{MOV AL, [1234h]} (menggunakan segmen default \texttt{DS}).

\subsubsection{Pengalamatan register tidak langsung}
Register berisi offset yang menunjuk memori: \texttt{MOV AL, [BX]}; segmen default \texttt{DS} kecuali untuk beberapa instruksi stack.

\subsubsection{Pengalamatan berbasis indeks}
Kombinasi register basis/indeks dengan displacement: \\
\texttt{MOV AL, [BX+SI+8]} atau \\
\texttt{MOV AX, [BP+DI+16]} \\
(segmen default untuk alamat berbasis \texttt{BP} adalah \texttt{SS}).

\subsubsection{Pengalamatan berbasis register}
Istilah ini sering merujuk pada variasi kombinasi \texttt{BX}, \texttt{BP}, \texttt{SI}, \texttt{DI}, dengan/ tanpa displacement. Penting untuk memahami segmen default: \texttt{DS} untuk \texttt{BX}/\texttt{SI}/\texttt{DI}, \texttt{SS} untuk \texttt{BP}.

\noindent\textbf{Contoh ringkas 8086}
\begin{verbatim}
org 100h
; Inisialisasi data
mov ax, data
mov ds, ax

; Pengalamatan langsung
mov ax, 1234h

; Memori langsung (absolute offset di DS)
mov al, [1234h]

; Tidak langsung via BX
mov bx, 0040h
mov al, [bx]

; Berbasis indeks
mov si, 0010h
mov dl, [bx+si+5]

int 20h

segment data
msg db 'X', 0
\end{verbatim}
\subsubsection{Detail Register FLAGS (8086)}
Bit penting pada \texttt{FLAGS}: \textbf{CF}(0), \textbf{PF}(2), \textbf{AF}(4), \textbf{ZF}(6), \textbf{SF}(7), \textbf{TF}(8), \textbf{IF}(9), \textbf{DF}(10), \textbf{OF}(11). Bit lain pada 8086 dicadangkan. \texttt{TF} mengaktifkan \textit{single-step interrupt} (\texttt{INT 01h}); \texttt{IF} mengizinkan interupsi maskable (\texttt{INT} perangkat keras). Pahami perbedaan \textbf{CF} vs \textbf{OF} untuk aritmatika tak bertanda/bertanda.

\subsubsection{Prefetch Queue dan Dampak Kinerja}
Unit BIU 8086 menyimpan \textit{prefetch queue} sekitar 6 byte instruksi untuk meningkatkan \textit{throughput}. Instruksi yang mengakses memori (khususnya \texttt{CS:IP} aliran) dapat menyurutkan antrean, sehingga kinerja bergantung pada pola campuran instruksi-komputasi vs akses memori.

\subsubsection{Override Segmen dan Segmen Default}
Segmen default: \texttt{CS} untuk pengambilan instruksi, \texttt{DS} untuk kebanyakan akses data, \texttt{SS} untuk alamat berbasis \texttt{BP}, dan \texttt{ES} sebagai tujuan instruksi string. Gunakan prefiks \texttt{ES:}/\texttt{CS:}/\texttt{SS:}/\texttt{DS:} untuk override jika diperlukan, misalnya menulis ke \texttt{ES:DI} saat menyalin blok.

\subsubsection{Pembentukan Alamat Efektif (EA) dan Batasan}
Kombinasi valid pada 8086 untuk pengalamatan berbasis indeks: \texttt{[BX+SI+disp]}, \texttt{[BX+DI+disp]}, \texttt{[BP+SI+disp]}, \texttt{[BP+DI+disp]}. Variasi satu register dengan/tanpa \texttt{disp} juga diperbolehkan. 

Pastikan ukuran operand eksplisit (\texttt{BYTE PTR}/\texttt{WORD PTR}) bila ambigu.

\subsubsection{Praktik Baik}
Inisialisasi segmen dengan benar (mis., muat \texttt{DS} dari label segmen data). Pisahkan data-kode, gunakan label yang deskriptif, dan dokumentasikan register yang diubah oleh setiap prosedur.

\section{Struktur File ASM}
\subsection{Format dasar file assembly}
Program \texttt{.COM} (TASM) lazim menggunakan \texttt{org 100h} dan satu segmen kode datar; program \texttt{.EXE} umumnya mendefinisikan segmen \texttt{.code}, \texttt{.data}, dan mengisi \texttt{END} dengan titik masuk.

\subsection{Direktif assembler}
Contoh umum: \texttt{ORG}, \texttt{DB}/\texttt{DW}/\texttt{DD} (pendefinisian data), \texttt{SEGMENT}/\texttt{ENDS}, \texttt{ASSUME}, \texttt{END}. Makna tepatnya tergantung assembler (TASM/MASM/NASM) dan model memori.

\subsection{Komentar dalam kode}
Gunakan \texttt{;} untuk komentar baris. Komentar harus menjelaskan maksud, bukan menyalin ulang instruksi.

\subsection{Organisasi kode program}
Pisahkan data dari kode. Gunakan label yang deskriptif, dan pertahankan konvensi penamaan konsisten. Untuk prosedur, dokumentasikan input/output via register atau stack.

\section{Contoh Soal dan Pembahasan}
\begin{enumerate}
  \item \textbf{Fungsi register AX, BX, CX, DX}.\\ AX: akumulator; BX: basis alamat; CX: pencacah; DX: perpanjangan hasil/operand I/O.
  \item \textbf{Hitung alamat fisik jika CS = 2000h dan IP = 0100h}.\\ \(\text{phys} = 0x2000\times 16 + 0x0100 = 0x20100\).
  \item \textbf{Contoh pengalamatan register}.\\ \texttt{MOV AX, BX} menyalin isi \texttt{BX} ke \texttt{AX}.
  \item \textbf{Struktur dasar file ASM}.\\ Untuk \texttt{.COM}: \texttt{org 100h}, kode utama, \texttt{int 20h}, akhir. Untuk \texttt{.EXE}: segmen \texttt{.data}, \texttt{.code}, prosedur \texttt{main}, \texttt{END main}.
\end{enumerate}

\section{Latihan}
\begin{enumerate}
  \item Diberikan \(\text{DS}=1234_h\) dan \(\text{BX}=0020_h\). Tentukan alamat fisik untuk \texttt{MOV AL, [BX+10h]}.
  \item Tunjukkan tiga variasi mode pengalamatan berbeda untuk membaca satu byte dari array pada \texttt{DS}.
  \item Tentukan flags yang terpengaruh oleh instruksi \texttt{ADD}, \texttt{SUB}, dan berikan contoh singkat yang mengubah \texttt{CF} dan \texttt{OF}.
  \item Rancang sketsa struktur berkas ASM untuk program \texttt{.EXE} sederhana dengan satu prosedur dan satu data string.
\end{enumerate}

\section{Tugas}
\begin{itemize}
  \item \textbf{Diagram arsitektur 8086}: Gambar blok BIU/EU, bus, serta aliran prefetch. Beri keterangan fungsi setiap blok.
  \item \textbf{Esai (400--600 kata)}: Bedakan register segmen vs. register umum dari perspektif peran dalam pembentukan alamat dan eksekusi instruksi; sertakan contoh.
  \item \textbf{Studi mode pengalamatan}: Buat tabel minimal 8 contoh instruksi yang mencakup register, langsung, tidak langsung, berbasis indeks, dan kombinasi displacement. Jelaskan segmen default yang digunakan.
  \item \textbf{Template berkas ASM}: Buat \LaTeX{} lampiran berisi \textit{template} program \texttt{.COM} dan \texttt{.EXE} (TASM/MASM), dilengkapi komentar pedoman.
\end{itemize}
\noindent\textbf{Keluaran}: PDF jawaban + berkas sumber, sertakan uji asersi manual untuk perhitungan alamat fisik.

\section{Referensi}\label{sec:arsitektur-referensi}
\begin{itemize}
    \item \cite{brey1986mikroprosesor}
    \item \cite{intel2019manual32}
    \item \cite{wiki_8086}
\end{itemize}


\chapter{Instalasi Turbo Assembler dan Struktur Program}

\section{Tujuan Pembelajaran}
Mahasiswa mampu:
\begin{itemize}
    \item Melakukan instalasi dan konfigurasi Turbo Assembler (TASM) dan alat \\
    pendukung (TLINK, TD).
    \item Menjelaskan perbedaan struktur dan karakteristik program \texttt{.COM} dan \texttt{.EXE}.
    \item Menggunakan direktif dasar (\texttt{ORG}, \texttt{END}, \texttt{TITLE}, \texttt{PAGE}) dalam berkas ASM.
    \item Menulis, merakit, menautkan, dan menjalankan program assembly sederhana.
\end{itemize}

\section{Pendahuluan}
Turbo Assembler (TASM) adalah assembler dari Borland yang populer di lingkungan DOS. Bersama \textit{toolchain} TLINK (linker) dan TD (debugger), TASM memungkinkan pengembangan program tingkat rendah pada arsitektur x86 klasik. Pemahaman struktur program \texttt{.COM} vs \texttt{.EXE} penting untuk pemilihan model memori, organisasi segmen, dan strategi \textit{build}. Melakukan instalasi dan konfigurasi Turbo Assembler (TASM) dan alat pendukung (TLINK, TD) merupakan langkah dasar yang diperlukan untuk memulai pemrograman assembly pada lingkungan DOS.

\section{Instalasi Turbo Assembler}
\subsection{Persyaratan sistem}
Lingkungan DOS (asli atau emulasi seperti DOSBox) atau sistem kompatibel. Ruang penyimpanan kecil, memori konvensional untuk menjalankan tool.

\subsection{Proses instalasi}
\begin{enumerate}
  \item Dapatkan paket TASM (legal, sesuai lisensi) dan ekstrak ke direktori, misal \texttt{C:\\TASM}.
  \item Pastikan berkas \texttt{TASM.EXE}, \texttt{TLINK.EXE}, \texttt{TD.EXE} tersedia.
  \item (Opsional) Tambahkan contoh proyek dan berkas \texttt{INCLUDE} yang menyertakan definisi konvensional (mis. \texttt{DOS.INC}).
\end{enumerate}

\subsubsection{Konfigurasi lingkungan}
Setel variabel lingkungan \texttt{PATH} agar berisi direktori TASM/TLINK/TD untuk akses dari mana saja. Konfigurasi file \texttt{TASM.CFG} atau \texttt{TLINK.CFG} jika diperlukan (mis., jalur pustaka). Pastikan \texttt{INCLUDE} menunjuk direktori header/\textit{include} jika digunakan.

\subsubsection{Pengaturan path dan direktori}
Contoh (DOSBox): \texttt{SET PATH=C:\\TASM;C:\\BIN;\%PATH\%}. Simpan sumber di direktori proyek terpisah untuk kebersihan struktur.

\subsection{Lingkungan Pengembangan}
\begin{itemize}
  \item \textbf{Editor}: Dapat memakai editor bawaan atau editor teks eksternal (mis., \texttt{EDIT}). Fokus pada penyorotan sintaks dan indentasi konsisten.
  \item \textbf{Assembler (TASM)}: Mengubah \texttt{.ASM} menjadi \texttt{.OBJ}. Opsi \texttt{/zi} untuk simbol debug.
  \item \textbf{Linker (TLINK)}: Menggabungkan \texttt{.OBJ} menjadi \texttt{.EXE} atau \texttt{.COM} (dengan langkah khusus).
  \item \textbf{Debugger (TD)}: Menjalankan \textit{step}, memeriksa register/memori, \textit{breakpoint}.
  \item \textbf{Dokumentasi}: Manual TASM/TLINK/TD, serta \texttt{.INC} referensi.
\end{itemize}

\subsection{Struktur Program COM}
\subsubsection{Karakteristik program COM}
\texttt{.COM} adalah format sederhana, tanpa header, berukuran maksimal ~64 KB. Memulai eksekusi pada offset \texttt{0100h} dengan \texttt{CS=DS=ES=SS} (umumnya sama), cocok untuk program kecil.

\subsubsection{Format file COM dan penggunaan memori}
Citra program dimuat sebagai blok kontinu. Stack perlu diatur manual bila diperlukan. Karena ketiadaan header, kontrol lebih langsung tetapi dengan keterbatasan ukuran dan organisasi segmen.

\subsubsection{Contoh struktur program COM}
\begin{verbatim}
org 100h
start:
    mov dx, offset msg
    mov ah, 09h
    int 21h

    mov ah, 4Ch
    int 21h

msg db 'Hello .COM!$'
\end{verbatim}

\subsection{Struktur Program EXE}
\subsubsection{Karakteristik program EXE}
\texttt{.EXE} memiliki header yang menjelaskan tata letak segmen, \textit{relocation}, dan titik masuk. Mendukung segmen terpisah untuk kode, data, \textit{stack}, sehingga lebih fleksibel untuk program besar.

\subsubsection{Header, segmentasi, dan contoh}
Header memuat informasi ukuran, relocation table, dsb. Model memori (SMALL, TINY, dsb.) memengaruhi pengaturan segmen.
\begin{verbatim}
.MODEL SMALL
.STACK 100h
.DATA
  msg db 'Hello .EXE!$'
.CODE
main PROC
  mov ax, @data
  mov ds, ax

  mov dx, offset msg
  mov ah, 09h
  int 21h

  mov ax, 4C00h
  int 21h
main ENDP
END main
\end{verbatim}

\subsection{Direktif Dasar}
\subsubsection{ORG (Origin)}
Menentukan offset awal kode/data. Pada program \texttt{.COM}, \texttt{org 100h} menyesuaikan lokasi titik masuk setelah PSP (Program Segment Prefix).

\subsubsection{END}
Menandai akhir berkas sumber; opsi label (mis., \texttt{END main}) mendefinisikan titik masuk program.

\subsubsection{TITLE, PAGE, dan lainnya}
\texttt{TITLE} dan \texttt{PAGE} memengaruhi keluaran listing. Direktif lain: \texttt{DB}/\texttt{DW} (data), \texttt{SEGMENT}/\texttt{ENDS}, \texttt{ASSUME}, tergantung assembler yang dipakai.

\subsubsection{Penggunaan direktif dalam program}
Pilih direktif sesuai target (\texttt{.COM} vs \texttt{.EXE}) dan model memori. Gunakan \texttt{END} dengan label fungsi utama untuk program \texttt{.EXE}.

\subsubsection{Menjalankan Toolchain di DOSBox}
Gunakan DOSBox untuk mengeksekusi TASM/TLINK/TD pada sistem modern. Contoh konfigurasi dasar:
\begin{verbatim}
MOUNT C ~/dos
C:
CD \TASM
SET PATH=C:\TASM;C:\BIN;%PATH%
TASM /Z /ZI HELLO.ASM    ; rakit -> HELLO.OBJ
TLINK /V HELLO.OBJ       ; taut -> HELLO.EXE
TD HELLO.EXE             ; debug
\end{verbatim}

\subsubsection{Membangun Program .COM vs .EXE}
Untuk berkas \texttt{.COM}, gunakan model memori TINY dan \texttt{ORG 100h}, lalu tautkan dengan opsi \texttt{/t}:
\begin{verbatim}
TASM /M2 HELLO_COM.ASM        ; menghasilkan HELLO_COM.OBJ
TLINK /T HELLO_COM.OBJ        ; menghasilkan HELLO_COM.COM
\end{verbatim}
Untuk \texttt{.EXE}, gunakan model \texttt{SMALL} (atau lain) dan prosedur bertanda \texttt{END main} seperti contoh sebelumnya.

\subsubsection{Alternatif Assembler Modern (NASM)}
Sebagai alternatif modern, NASM dapat merakit kode 8086. Contoh untuk menghasilkan \texttt{.COM} langsung:
\begin{verbatim}
nasm -f bin hello.asm -o hello.com
\end{verbatim}
Catatan: Sintaks NASM berbeda dari TASM/MASM; sesuaikan direktif dan penanganan segmen.

\subsubsection{Debugging Dasar dengan TD}
Gunakan TD untuk langkah per instruksi, memeriksa register/memori, dan breakpoint. Perintah umum: \texttt{R} (register), \texttt{U} (unassemble), \texttt{D} (dump), \texttt{E} (enter), \texttt{T} (trace), \texttt{P} (proceed). Simpan skrip langkah untuk demonstrasi praktikum.

\subsubsection{Kesalahan Umum dan Solusi}
\begin{itemize}
  \item Label tidak terdefinisi: periksa ejaan, urutan definisi, atau kebutuhan \texttt{EXTRN}/\texttt{PUBLIC} (untuk proyek multi-berkas).
  \item \texttt{CS:IP} tidak benar pada program \texttt{.COM}: pastikan \texttt{ORG 100h} didefinisikan dan tidak ada segmen tambahan.
  \item String untuk \texttt{AH=09h}: pastikan terminator karakter \texttt{'}\$\texttt{'} hadir; untuk \texttt{INT 10h} tidak diperlukan.
\end{itemize}

\section{Praktikum}
\begin{enumerate}
  \item Instal TASM dan pastikan \texttt{TASM}, \texttt{TLINK}, \texttt{TD} berjalan dari command line.
  \item Buat program \texttt{.COM} menampilkan sapaan, rakit dengan TASM, uji di DOSBox.
  \item Buat program \texttt{.EXE} dengan segmen data/kode terpisah, tautkan dengan TLINK, jalankan.
  \item Coba \texttt{TD} untuk menelusuri program: periksa register dan memori.
  \item Terapkan \texttt{ORG} dan \texttt{END} secara tepat pada kedua jenis program.
\end{enumerate}

\section{Contoh Kode}
\begin{verbatim}
; Program sederhana menggunakan direktif dasar
TITLE Program Sederhana
PAGE 60,132

.MODEL SMALL
.STACK 100h

.DATA
    pesan DB 'Hello World!$'

.CODE
    ORG 100h
    START:
        MOV AX, @DATA
        MOV DS, AX
        
        MOV DX, OFFSET pesan
        MOV AH, 09h
        INT 21h
        
        MOV AH, 4Ch
        INT 21h
    END START
\end{verbatim}

\section{Contoh Soal dan Pembahasan}
\begin{enumerate}
  \item \textbf{Jelaskan perbedaan utama \texttt{.COM} dan \texttt{.EXE}}: \texttt{.COM}: tanpa header, maksimal ~64 KB, datar, mulai di \texttt{0100h}; \texttt{.EXE}: berheader, mendukung segmentasi dan program besar.
  \item \textbf{Peran \texttt{ORG 100h} pada program \texttt{.COM}}: Menyesuaikan offset titik masuk setelah PSP sehingga label awal selaras dengan lokasi eksekusi sebenarnya.
  \item \textbf{Alur \textit{build} TASM/TLINK}: TASM: \texttt{.ASM -> .OBJ}; TLINK: \texttt{.OBJ -> .EXE}; untuk \texttt{.COM} gunakan \textit{tiny model} atau metode khusus.
  \item \textbf{Gunakan TD untuk apa?}: Debugging: langkah per instruksi, inspeksi register/memori, \textit{breakpoint}.
\end{enumerate}

\section{Latihan}
\begin{enumerate}
  \item Tulis program \texttt{.COM} yang menampilkan dua baris teks menggunakan \texttt{INT 21h, AH=09h}.
  \item Tulis program \texttt{.EXE} yang menginisialisasi \texttt{DS} dengan benar dan menampilkan string.
  \item Buat variasi yang membaca input karakter tunggal (\texttt{INT 21h, AH=01h}) dan mencetaknya kembali.
  \item Bangun kedua program dengan TASM/TLINK dan uji di DOSBox.
\end{enumerate}

\section{Tugas}
\begin{itemize}
  \item \textbf{Dokumentasi instalasi}: Sertakan tangkapan layar pengaturan PATH, keluaran versi TASM/TLINK/TD.
  \item \textbf{Program COM nama pribadi}: Tampilkan nama Anda dan NIM pada satu baris.
  \item \textbf{Esai perbandingan (300--500 kata)}: \texttt{.COM} vs \texttt{.EXE} dari aspek header, ukuran, segmentasi, dan skenario penggunaan.
  \item \textbf{Log debugging}: Jalankan program \texttt{.EXE} di TD, lakukan \textit{step} beberapa instruksi, catat perubahan register.
\end{itemize}

\section{Referensi}\label{sec:instalasi-referensi}
\begin{itemize}
    \item \cite{borland1990tasm}
    \item \cite{nasm_manual}
    \item \cite{dosbox_manual}
\end{itemize}



% Bagian II: Pemrograman Assembly Fundamental  
\part{Pemrograman Assembly Fundamental}\label{part:fundamental}

\chapter{Instruksi Dasar: Perpindahan Data dan Aritmatika}

\section{Tujuan Pembelajaran}
Setelah pertemuan ini, mahasiswa mampu:
\begin{itemize}
    \item Menggunakan instruksi perpindahan data \texttt{MOV} dengan benar pada variasi operand register dan memori.
    \item Melakukan operasi aritmatika dasar (\texttt{ADD}, \texttt{SUB}, \texttt{MUL}, \texttt{DIV}) pada data 8/16-bit.
    \item Menjelaskan pengaruh instruksi aritmatika terhadap \textit{flag} prosesor (\texttt{CF}, \texttt{ZF}, \texttt{SF}, \texttt{OF}, \texttt{PF}, \texttt{AF}).
    \item Menulis program aritmatika sederhana yang aman dan memperhatikan kasus tepi (overflow, pembagian nol).
\end{itemize}

\section{Pendahuluan}
Instruksi perpindahan dan aritmatika merupakan fondasi pemrograman assembly \cite{susanto1995belajar}. Pemahaman mendalam mengenai operand, mode pengalamatan, dan efek \textit{flag} akan menentukan ketepatan logika program. Pada 8086, variasi operand dan pembatasan ukuran data harus dipatuhi agar instruksi valid \cite{hyde2010art}.

\section{Instruksi Perpindahan Data (MOV)}
\subsection{Sintaks dan aturan penggunaan}
Bentuk umum: \texttt{MOV destination, source} \cite{hyde2010art}. Aturan pokok: ukuran operand harus sama; perpindahan memori-ke-memori langsung tidak diperbolehkan (harus melalui register); beberapa register khusus (\texttt{CS}) tidak dapat ditulis langsung dengan \texttt{MOV} biasa \cite{susanto1995belajar}.

\subsubsection{Antar register}
\begin{verbatim}
mov ax, bx      ; salin BX -> AX (16-bit)
mov al, ah      ; salin AH -> AL (8-bit)
\end{verbatim}

\section{Pendalaman}\label{sec:aritmatika-pendalaman}
\subsection{Instruksi aritmatika tambahan}
Selain \texttt{ADD}/\texttt{SUB}/\texttt{MUL}/\texttt{DIV}, 8086 mendukung: \texttt{ADC}/\texttt{SBB} (operasi dengan carry/borrow untuk angka multi-word), \texttt{INC}/\texttt{DEC} (tanpa memengaruhi \texttt{CF}), \texttt{NEG} (dua komplemen), \texttt{CMP} (seperti \texttt{SUB} tanpa menyimpan hasil), serta \texttt{DAA}/\texttt{DAS}/\texttt{AAA}/\texttt{AAS} untuk penyesuaian BCD. \cite{intel2019manual32,rbil}

\subsection{Perluasan tanda dan pembagian aman}
Gunakan \texttt{CBW} (convert byte to word) untuk menandatangani \texttt{AL} ke \texttt{AX} sebelum \texttt{IDIV}/\texttt{IMUL} (pada 8086 hanya \texttt{MUL}/\texttt{DIV} tersedia; varian bertanda formal hadir pada generasi lebih baru namun konsep sign-extension tetap relevan). Untuk pembagian 16-bit, pastikan \texttt{DX} berisi bagian tinggi (0 untuk tak bertanda; sign-extension untuk bertanda). Cek penyebut nol untuk menghindari exception. \cite{intel2019manual32}

\subsection{Aritmatika multi-precision}
Untuk menjumlahkan dua bilangan 32-bit tersimpan sebagai \texttt{DX:AX} dan \texttt{BX:CX} (high:low), gunakan \texttt{ADD} pada low word lalu \texttt{ADC} pada high word:
\begin{verbatim}
add ax, cx   ; low
adc dx, bx   ; high + carry
\end{verbatim}

\subsection{Pemilihan cabang berdasarkan flags}
Untuk data tak bertanda gunakan \texttt{JA}/\texttt{JAE}/\texttt{JB}/\texttt{JBE} (berbasis \texttt{CF}/\texttt{ZF}); untuk bertanda gunakan \texttt{JG}/\texttt{JGE}/\texttt{JL}/\texttt{JLE} (berbasis \texttt{SF}/\texttt{OF}/\texttt{ZF}). Pastikan interpretasi konsisten dari sumber data. \cite{intel2019manual32}

\subsection{Kasus tepi umum}
\begin{itemize}
  \item Overflow bertanda: contoh \texttt{0x7FFF + 0x0001} menghasilkan \texttt{OF=1}, \texttt{CF=0}.
  \item Borrow tak bertanda: \texttt{SUB AX, BX} dengan \texttt{AX<BX} akan menyetel \texttt{CF=1}.
  \item \texttt{INC}/\texttt{DEC}: tidak mengubah \texttt{CF}; hati-hati untuk algoritma yang mengandalkan carry chain.
\end{itemize}

\subsubsection{Dari/ke memori}
\begin{verbatim}
mov ax, [si]        ; baca word dari DS:SI ke AX
mov [bx+4], dl      ; tulis byte DL ke DS:[BX+4]
\end{verbatim}

\subsubsection{Dengan konstanta}
\begin{verbatim}
mov ax, 1234h   ; immediate -> register
mov byte ptr [di], 0 ; immediate -> memori (byte)
\end{verbatim}

\subsubsection{Contoh dan praktik baik}
Gunakan \texttt{BYTE PTR}/\texttt{WORD PTR} untuk memperjelas ukuran operand bila ambigu. Pastikan segmen default (\texttt{DS} untuk data, \texttt{SS} untuk \texttt{BP}) dipahami saat mengakses memori.

\subsection{Operasi Aritmatika}
\subsubsection{ADD dan SUB}
\begin{verbatim}
mov ax, 5
add ax, 3      ; AX = 8, ZF=0, CF=0
sub ax, 10     ; AX = FFFEh (jika 16-bit), SF=1, mungkin CF=1
\end{verbatim}
\textbf{Catatan}: \texttt{ADD}/\texttt{SUB} memutakhirkan \texttt{CF}, \texttt{ZF}, \texttt{SF}, \texttt{OF}, \texttt{AF}, \texttt{PF} sesuai hasil.

\subsubsection{MUL dan DIV}
\texttt{MUL} dan \texttt{DIV} menggunakan register implisit \cite{hyde2010art}:\newline
\textbf{8-bit}: \texttt{MUL r/m8} -> \texttt{AX = AL * r/m8}; \texttt{DIV r/m8} -> \texttt{AL = AX / r/m8}, \texttt{AH = AX mod r/m8}.\newline
\textbf{16-bit}: \texttt{MUL r/m16} -> hasil di \texttt{DX:AX}; \texttt{DIV r/m16} -> \texttt{AX = DX:AX / r/m16}, sisa di \texttt{DX} \cite{susanto1995belajar}.
\begin{verbatim}
; Perkalian 16-bit
mov ax, 1234h
mov bx, 0002h
mul bx           ; DX:AX = AX * BX

; Pembagian 8-bit
mov ax, 0123h    ; AL=23h, AH=01h -> 0x0123 = 291 desimal
mov bl, 0Ah      ; 10 desimal
div bl           ; AL=29h (41 des), AH=01h (sisa 1)
\end{verbatim}
\textbf{Perhatian}: \texttt{DIV} memicu kesalahan pembagian jika penyebut 0 atau hasil tidak muat dalam register hasil.

\subsubsection{Penanganan overflow/underflow}
Pantau \texttt{OF}/\texttt{CF} setelah operasi. Gunakan percabangan untuk menangani kondisi luapan. Untuk operasi bertanda, gunakan instruksi khusus (mis. \texttt{IMUL}/\texttt{IDIV} pada prosesor yang mendukung) dan interpretasi \texttt{SF}/\texttt{OF}.

\subsection{Flag Register dan Operasi Aritmatika}
\begin{itemize}
  \item \textbf{CF (Carry)}: Menandakan carry keluar pada penjumlahan atau borrow pada pengurangan (aritmatika tak bertanda).
  \item \textbf{ZF (Zero)}: Hasil operasi bernilai nol.
  \item \textbf{SF (Sign)}: Bit tertinggi hasil (menunjukkan tanda pada interpretasi bertanda).
  \item \textbf{OF (Overflow)}: Luapan pada aritmatika bertanda (hasil tidak muat dalam rentang bertanda).
  \item \textbf{PF (Parity)}: Paritas bit rendah dari hasil (jumlah bit 1 genap).
  \item \textbf{AF (Auxiliary Carry)}: Carry dari bit 3 ke bit 4 (relevan untuk BCD).
\end{itemize}
\noindent\textbf{Contoh pemeriksaan flag}
\begin{verbatim}
add ax, bx
jo overflow_handler   ; jika overflow bertanda
jc carry_handler      ; jika carry (tak bertanda)
jz zero_handler       ; jika hasil nol
\end{verbatim}

\section{Contoh Soal dan Pembahasan}
\begin{enumerate}
  \item \textbf{Program perpindahan data antar register}.\\ Tunjukkan \texttt{MOV AX,BX}, \texttt{MOV CH,CL}, dan jelaskan ukuran operand.
  \item \textbf{Penjumlahan dua bilangan 16-bit}.\\ Baca dari memori ke register, \texttt{ADD}, simpan kembali. Bahas \texttt{CF}/\texttt{OF}.
  \item \textbf{Pengurangan dengan penanganan flag}.\\ Gunakan \texttt{SBB} bila ada carry sebelumnya; contohkan cabang jika \texttt{CF}=1.
  \item \textbf{Perkalian dan pembagian}.\\ Tunjukkan tata letak hasil \texttt{MUL} (\texttt{DX:AX}) dan \texttt{DIV} (hasil dan sisa).
  \item \textbf{Kalkulator sederhana}.\\ Rangka kerangka input satu digit dan operasi dasar; bahas validasi pembagian dengan nol.
\end{enumerate}

\section{Latihan}
\begin{enumerate}
  \item Buat program menghitung \((A+B) - (C+D)\) untuk empat bilangan 16-bit di memori. Laporkan \texttt{CF}/\texttt{OF}.
  \item Buat program menghitung \((X \times Y) / Z\) (8-bit), lengkapi pemeriksaan penyebut nol dan luapan hasil.
  \item Jelaskan secara tertulis bagaimana \texttt{CF}, \texttt{OF}, \texttt{ZF}, \texttt{SF} berubah pada masing-masing operasi contoh Anda.
  \item Buat program yang menampilkan status flag setelah \texttt{ADD} dan \texttt{SUB} menggunakan instruksi cabang bersyarat dan tulis hasil ke memori.
\end{enumerate}

\section{Praktikum}
\begin{enumerate}
  \item Implementasikan skenario perpindahan data antar register dan antara register-memori; verifikasi dengan debugger.
  \item Implementasikan penjumlahan/pengurangan 16-bit, amati \texttt{CF}/\texttt{OF} di debugger.
  \item Implementasikan \texttt{MUL}/\texttt{DIV} (8-bit dan 16-bit), catat tata letak hasil dan sisa.
  \item Susun \textit{mini-kalkulator} yang menerima dua operand kecil (hard-coded) dan memilih operasi melalui konstanta; tampilkan hasil via \texttt{INT 21h, AH=09h} (konversi numerik sederhana dapat berupa heksadesimal).
\end{enumerate}

\section{Contoh Kode}
\begin{verbatim}
; Program operasi aritmatika dasar
TITLE Operasi Aritmatika
.MODEL SMALL
.STACK 100h

.DATA
    bil1 DW 15
    bil2 DW 7
    hasil DW ?

.CODE
START:
    MOV AX, @DATA
    MOV DS, AX
    
    ; Penjumlahan
    MOV AX, bil1
    ADD AX, bil2
    MOV hasil, AX
    
    ; Pengurangan
    MOV AX, bil1
    SUB AX, bil2
    
    ; Perkalian (16-bit implisit: DX:AX = AX * bil2)
    MOV AX, bil1
    MOV BX, bil2
    MUL BX
    
    ; Pembagian (16-bit implisit: AX = DX:AX / BX, sisa di DX)
    XOR DX, DX
    MOV AX, bil1
    MOV BX, bil2
    ; Pastikan BX != 0 sebelum DIV
    OR BX, BX
    JZ done
    DIV BX

 done:
    MOV AH, 4Ch
    INT 21h
END START
\end{verbatim}

\section{Tugas}
\begin{itemize}
  \item \textbf{Kalkulator sederhana}: Implementasikan operasi \texttt{+}, \texttt{-}, \texttt{*}, \texttt{/} untuk operand 8-bit. Tampilkan hasil dalam heksadesimal; tangani pembagian nol.
  \item \textbf{Rata-rata 5 bilangan}: Baca lima nilai 8-bit dari memori, hitung jumlah 16-bit, bagi dengan 5 menggunakan \texttt{DIV} (perhatikan sisa), tampilkan hasil.
  \item \textbf{Dokumentasi flag}: Buat tabel eksperimen yang menunjukkan nilai \texttt{CF}, \texttt{OF}, \texttt{ZF}, \texttt{SF}, \texttt{PF}, \texttt{AF} untuk beberapa kombinasi operand; jelaskan anomali yang Anda amati.
\end{itemize}

\section{Referensi}
% Bibliography is handled by the main document

\chapter{Instruksi Logika dan Output Teks}

\section{Tujuan Pembelajaran}
Mahasiswa mampu:
\begin{itemize}
    \item Menerapkan instruksi logika \texttt{AND}, \texttt{OR}, \texttt{XOR}, \texttt{NOT} pada data 8/16-bit.
    \item Memahami operasi bitwise, masking, set/clear bit, dan kasus penggunaan praktis.
    \item Menggunakan layanan BIOS video \texttt{INT 10h} (fungsi \texttt{02h}, \texttt{09h}, \texttt{0Ah}, \texttt{0Eh}) untuk output teks dan kontrol kursor.
    \item Menampilkan teks di layar dengan kontrol posisi dan atribut warna.
\end{itemize}

\section{Pendahuluan}
Instruksi logika menyediakan manipulasi bit tingkat rendah yang fundamental untuk kontrol perangkat keras, pengolahan bendera (flag), dan optimasi \cite{hyde2010art}. Pada sisi output, layanan BIOS \texttt{INT 10h} memungkinkan interaksi langsung dengan tampilan dalam mode teks/TTY, relevan untuk program DOS dan lingkungan \textit{bare-metal}-like \cite{susanto1995belajar}.

\section{Instruksi Logika}
\subsubsection{AND}
Operasi konjungsi bitwise: bit hasil 1 jika dan hanya jika kedua operand 1. Umum untuk \textit{masking} menghapus bit tertentu.
\begin{verbatim}
mov ax, 0A5Fh
and ax, 00FFh    ; simpan hanya 8 bit rendah -> AX = 005Fh
\end{verbatim}
\textbf{Flags}: memperbarui \texttt{ZF}, \texttt{SF}, \texttt{PF}; \texttt{CF} dan \texttt{OF} direset ke 0.

\subsubsection{OR}
Disjungsi bitwise: bit hasil 1 jika salah satu operand 1. Berguna untuk menyetel bit spesifik.
\begin{verbatim}
mov al, 00110000b
or  al, 00000101b  ; set bit 0 dan 2 -> 00110101b
\end{verbatim}
\textbf{Flags}: \texttt{ZF}, \texttt{SF}, \texttt{PF} diperbarui; \texttt{CF}, \texttt{OF} = 0.

\subsubsection{XOR}
Eksklusif-or: 1 jika operand berbeda; \textit{involutif} (\texttt{x xor k xor k = x}). Biasa untuk enkripsi sederhana dan pembersihan register cepat.
\begin{verbatim}
xor ax, ax   ; cepat set AX=0, tanpa memuat immediate
\end{verbatim}
\textbf{Flags}: mirip \texttt{AND}/\texttt{OR}; \texttt{CF}, \texttt{OF} = 0.

\subsubsection{NOT}
Komplemen bitwise (unary). Sering dipakai untuk inversi \textit{mask}.
\begin{verbatim}
mov bl, 11110000b
not bl                 ; BL = 00001111b
\end{verbatim}
\textbf{Flags}: tidak terpengaruh (kecuali implementasi spesifik; pada 8086, flags tidak terdefinisi diubah untuk NOT—anggap tidak relevan).

\subsubsection{Operasi bitwise dan aplikasinya}
\begin{itemize}
  \item \textbf{Masking}: \texttt{AND} untuk \textit{clear}, \texttt{OR} untuk \textit{set}, \texttt{XOR} untuk \textit{toggle}.
  \item \textbf{Packing/unpacking} field bit dalam satu byte/word.
  \item \textbf{Uji bit}: kombinasikan \texttt{TEST} (seperti \texttt{AND} tanpa menyimpan hasil) dan percabangan pada \texttt{ZF}.
\end{itemize}
\begin{verbatim}
; uji bit 3 (hitung dari 0)
mov al, value
mov ah, al
and ah, 00001000b
jz  bit3_clear
; bit3_set ...
\end{verbatim}

\subsection{Aplikasi Instruksi Logika}
\subsubsection{Manipulasi bit dan masking}
\begin{verbatim}
; clear 2 bit terendah
and al, 11111100b
; set bit 7
or  al, 10000000b
; toggle bit 4
xor al, 00010000b
\end{verbatim}

\subsubsection{Enkripsi sederhana (XOR)}
\begin{verbatim}
; enkripsi/dekripsi XOR dengan kunci tunggal
mov al, [si]
xor al, key
mov [di], al
\end{verbatim}
\textit{Catatan}: XOR cipher lemah dan hanya untuk demonstrasi.

\subsubsection{Operasi set dan clear bit, perbandingan bit}
Gunakan \texttt{BT}/\texttt{BTS}/\texttt{BTR} pada prosesor lebih baru; pada 8086, gunakan \texttt{AND}/\texttt{OR}/\texttt{XOR} manual. Perbandingan bit dilakukan dengan \texttt{TEST}.

\subsection{Interupsi BIOS INT 10h}
\subsubsection{Fungsi 02h: Set Cursor Position}
\begin{itemize}
  \item \texttt{AH=02h}, \texttt{BH=page}, \texttt{DH=row}, \texttt{DL=column}; \texttt{INT 10h}.
  \item Koordinat 0-berbasis; pastikan dalam batas mode teks (mis. 25x80).
\end{itemize}
\subsubsection{Fungsi 09h: Write Character and Attribute}
\begin{itemize}
  \item \texttt{AH=09h}, \texttt{AL=char}, \texttt{BH=page}, \texttt{BL=attribute}, \texttt{CX=count}; \texttt{INT 10h}.
  \item Menulis dengan atribut warna; tidak memajukan kursor secara \textit{TTY-like} untuk semua mode.
\end{itemize}
\subsubsection{Fungsi 0Ah: Write Character Only}
\begin{itemize}
  \item \texttt{AH=0Ah}, parameter mirip, menggunakan atribut saat ini.
\end{itemize}
\subsubsection{Fungsi 0Eh: Write Character in TTY Mode}
\begin{itemize}
  \item \texttt{AH=0Eh}, \texttt{AL=char}, \texttt{BH=page}, \texttt{BL=color}; menulis seperti \texttt{teletype}, memajukan kursor dan menggulir bila perlu.
\end{itemize}

\subsection{Output Teks ke Layar}
\subsubsection{Kontrol posisi kursor}
Pindahkan kursor ke (baris, kolom) dengan \texttt{AH=02h}. Tulis teks karakter-per-karakter (\texttt{0Eh}) atau \texttt{09h} untuk pengulangan.

\subsubsection{Atribut karakter (warna)}
Atribut 8-bit: 4 bit latar depan + 3 bit latar belakang + 1 bit berkedip. Contoh: \texttt{1Eh} (kuning di biru).

\subsubsection{Mode teks dan grafik}
Materi ini berfokus pada mode teks; mode grafik memerlukan fungsi \texttt{INT 10h} lain (dibahas pada pertemuan grafik).

\subsubsection{Penanganan karakter khusus}
Untuk baris baru, gunakan pasangan \texttt{$\backslash$r} dan \texttt{$\backslash$n} atau \texttt{TTY 0Eh}. Simbol \texttt{'\$'} digunakan oleh \texttt{INT 21h, AH=09h}, bukan \texttt{INT 10h}.

\section{Contoh Soal dan Pembahasan}
\begin{enumerate}
  \item \textbf{Masking 8 bit terakhir}.\\ Gunakan \texttt{AND AX, 00FFh} untuk mengekstrak byte rendah dari word.
  \item \textbf{Enkripsi XOR}.\\ Tunjukkan bahwa \texttt{x xor k xor k = x} untuk pembuktian dekripsi.
  \item \textbf{Menampilkan teks di posisi tertentu}.\\ Pindahkan kursor (\texttt{AH=02h}) lalu panggil \texttt{AH=09h} atau \texttt{0Eh}.
  \item \textbf{Teks berwarna}.\\ Tentukan atribut warna pada \texttt{BL} untuk \texttt{AH=09h}.
\end{enumerate}

\section{Praktikum}
\begin{enumerate}
  \item Implementasi \textit{bit playground}: \texttt{AND}/\texttt{OR}/\texttt{XOR}/\texttt{NOT} terhadap nilai contoh, tampilkan hasil dalam heksadesimal.
  \item Implementasi masking dan uji bit spesifik menggunakan \texttt{TEST} + cabang.
  \item Implementasi enkripsi XOR pada buffer kecil di memori (hard-coded); verifikasi dekripsi.
  \item Implementasi output teks: atur kursor ke beberapa posisi dan cetak karakter/teks dengan atribut berbeda.
\end{enumerate}

\section{Contoh Kode}
\begin{verbatim}
; Program operasi logika dan output teks
TITLE Operasi Logika dan Output
.MODEL SMALL
.STACK 100h

.DATA
    pesan DB 'Hello Assembly!$'
    nilai1 DW 0F0F0h
    nilai2 DW 0FF00h

.CODE
START:
    MOV AX, @DATA
    MOV DS, AX
    
    ; Operasi logika
    MOV AX, nilai1
    AND AX, nilai2    ; Masking
    OR AX, 000Fh      ; Set bit terakhir
    XOR AX, 0FFFFh    ; Inversi
    NOT AX            ; Komplemen
    
    ; Output teks ke layar
    MOV AH, 02h       ; Set cursor position
    MOV BH, 00h       ; Page 0
    MOV DH, 10        ; Row 10
    MOV DL, 20        ; Column 20
    INT 10h
    
    ; Tampilkan karakter
    MOV AH, 09h       ; Write character
    MOV AL, 'A'       ; Character 'A'
    MOV BH, 00h       ; Page 0
    MOV BL, 1Eh       ; Yellow on blue
    MOV CX, 1         ; Count
    INT 10h
    
    MOV AH, 4Ch
    INT 21h
END START
\end{verbatim}

\section{Latihan}
\begin{enumerate}
  \item Buat program yang melakukan masking pada 8 bit terakhir untuk sejumlah data dalam array; simpan hasil ke array tujuan.
  \item Buat program enkripsi/dekripsi XOR untuk string dengan kunci 1 byte; tampilkan hasil terenkripsi dalam heksadesimal.
  \item Buat program yang menampilkan teks pada empat sudut layar (0,0), (0,79), (24,0), (24,79).
  \item Buat program yang menampilkan teks dengan semua kombinasi warna latar depan standar pada latar belakang biru.
\end{enumerate}

\section{Tugas}
\begin{itemize}
  \item \textbf{Manipulasi data}: Implementasikan rutin \texttt{set\_bit}, \texttt{clear\_bit}, \texttt{toggle\_bit} untuk sebuah byte di memori; sertakan pengujian.
  \item \textbf{Menu teks}: Buat menu sederhana dengan kontrol posisi kursor, garis pemisah menggunakan karakter ASCII, dan pilihan bernomor; sorot pilihan aktif dengan atribut warna.
  \item \textbf{XOR cipher}: Implementasikan enkripsi/dekripsi string berbasis XOR (1 byte kunci), dokumentasikan kelemahannya dan skenario edukatifnya.
  \item \textbf{Dokumentasi INT 10h}: Buat ringkasan parameter untuk fungsi 02h, 09h, 0Ah, 0Eh dan contoh singkat masing-masing.
\end{itemize}

\section{Referensi}
% Bibliography is handled by the main document

\chapter{Input dari Keyboard dan Penanganan Buffer}

\section{Tujuan Pembelajaran}
Mahasiswa mampu:
\begin{itemize}
    \item Menggunakan layanan BIOS \texttt{INT 16h} untuk membaca input keyboard (blocking/non-blocking) dan memeriksa status.
    \item Menjelaskan konsep keyboard buffer, mekanisme kerja, dan teknik \textit{flush}/penanganan buffer penuh.
    \item Mengembangkan program interaktif berbasis input karakter dan string (dengan/ tanpa echo) dengan validasi.
    \item Menginterpretasikan kode ASCII vs. \textit{scan code} serta menangani tombol khusus (function, arrow, control).
\end{itemize}

\section{Pendahuluan}
Interaksi pengguna pada lingkungan DOS banyak bertumpu pada layanan BIOS untuk perangkat input seperti keyboard \cite{susanto1995belajar}. \texttt{INT 16h} menyediakan antarmuka tingkat rendah yang konsisten untuk membaca tombol, memeriksa ketersediaan input, dan mendapatkan status \textit{modifier} (Shift/Ctrl/Alt) \cite{hyde2010art}. Memahami perbedaan operasi \textit{blocking} dan \textit{non-blocking} penting agar program responsif dan tidak \textit{hang} menunggu masukan.

\section{Interupsi BIOS INT 16h}
\subsubsection{Fungsi 00h: Read Key (Blocking)}
\begin{itemize}
  \item \textbf{Masukan}: \texttt{AH=00h}
  \item \textbf{Keluaran}: \texttt{AL=ASCII code} (0 jika \textit{non-printable}), \texttt{AH=scan code}
  \item \textbf{Perilaku}: Menunggu hingga tombol ditekan; gunakan untuk input yang memerlukan sinkronisasi.
\end{itemize}

\subsubsection{Fungsi 01h: Check for Key (Non-blocking)}
\begin{itemize}
  \item \textbf{Masukan}: \texttt{AH=01h}
  \item \textbf{Keluaran}: \texttt{ZF=1} jika tidak ada tombol; jika ada, \texttt{AL}/\texttt{AH} berisi kode tanpa menghapus dari buffer.
  \item \textbf{Perilaku}: \textit{Polling} status keyboard; gunakan untuk antarmuka yang tidak boleh menunggu.
\end{itemize}

\subsubsection{Fungsi 02h: Get Keyboard Flags}
\begin{itemize}
  \item \textbf{Masukan}: \texttt{AH=02h}
  \item \textbf{Keluaran}: \texttt{AL} berisi bendera status (Shift, Ctrl, Alt, Caps Lock, Num Lock, Scroll Lock).
  \item Interpretasikan bit-bit sesuai tabel referensi BIOS.
\end{itemize}

\subsubsection{Parameter dan nilai kembali}
Pastikan menyimpan nilai \texttt{AL}/\texttt{AH} segera jika akan digunakan, karena pemanggilan layanan berikutnya dapat menimpanya. Untuk \texttt{01h}, cek \texttt{ZF} sebelum membaca.

\subsubsection{Blocking vs. non-blocking}
\textit{Blocking} menyederhanakan alur namun berisiko mengunci UI. \textit{Non-blocking} memungkinkan pembaruan layar/pekerjaan lain saat menunggu input menggunakan \textit{polling} atau \textit{timeout}.

\subsection{Keyboard Buffer}
\subsubsection{Konsep dan cara kerja}
BIOS menempatkan penekanan tombol pada antrian (buffer). Panggilan baca (00h) mengeluarkan entri dari buffer; pemeriksaan (01h) tidak mengeluarkan.

\subsubsection{Penanganan buffer penuh}
Ketika buffer penuh, penekanan tombol berikut dapat hilang atau menimpa entri lama tergantung implementasi/driver. Program interaktif harus membaca buffer secara reguler.

\subsubsection{Flushing keyboard buffer}
DOS menyediakan \texttt{INT 21h, AH=0Ch} dengan parameter di \texttt{AL} untuk membersihkan buffer dan memproses input keyboard. Gunakan setelah mengubah mode input agar sisa tombol tidak mengganggu.

\subsubsection{Status keyboard buffer}
Gunakan \texttt{01h} untuk memeriksa ketersediaan. Beberapa BIOS menyediakan fungsi lain untuk panjang buffer, tetapi tidak portabel.

\subsection{Input Karakter dan String}
\subsubsection{Karakter tunggal}
Gunakan \texttt{AH=00h} (blocking) untuk membaca satu tombol. Untuk menghindari \textit{hang}, kombinasikan dengan \texttt{01h}.

\subsubsection{String dengan echo}
Gunakan \texttt{INT 21h, AH=0Ah} untuk masukan baris dengan echo; siapkan struktur buffer DOS: byte pertama kapasitas, byte kedua panjang, diikuti penampung.

\subsubsection{String tanpa echo (password)}
Baca per karakter dengan \texttt{INT 16h, AH=00h} dan tampilkan tanda \texttt{'*'} menggunakan \texttt{INT 10h, AH=0Eh}; simpan ke buffer hingga Enter.

\subsubsection{Validasi input dan karakter khusus}
Tolak karakter di luar rentang yang diizinkan; tangani Backspace dengan menghapus dari buffer dan memundurkan kursor; Enter mengakhiri masukan.

\subsection{Karakter Khusus}
\subsubsection{Scan code vs. ASCII}
Tombol \textit{non-printable} menghasilkan \texttt{AL=0} dan \texttt{AH} berisi \textit{scan code}. Tombol huruf/angka memberikan ASCII di \texttt{AL}.

\subsubsection{Function, arrow, control keys}
F1--F12 umumnya menghasilkan \texttt{AL=0} (atau \texttt{E0h} sebagai \textit{prefix} pada beberapa sistem) dan \texttt{AH} menunjukkan kode; panah memerlukan interpretasi \textit{scan code}. \texttt{Ctrl}/\texttt{Alt}/\texttt{Shift} memodifikasi kode dan terekam pada status flags (fungsi 02h).

\section{Praktikum}
\begin{enumerate}
  \item Program pembaca karakter tunggal: tampilkan ASCII dan \textit{scan code} dalam heksadesimal.
  \item Program input string dengan echo menggunakan \texttt{AH=0Ah}; cetak kembali string.
  \item Program input password tanpa echo dengan masking dan dukungan Backspace.
  \item Program \textit{buffer-aware}: baca semua karakter tertunda (\texttt{01h}) dan \texttt{flush} sebelum kembali ke menu.
  \item Program menu interaktif: navigasi dengan panah atas/bawah; Enter untuk memilih; Esc untuk kembali.
\end{enumerate}

\section{Contoh Kode}
\begin{verbatim}
; Program input keyboard dan penanganan buffer
TITLE Input Keyboard
.MODEL SMALL
.STACK 100h

.DATA
    prompt DB 'Masukkan nama: $'
    nama   DB 50 DUP('$')
    pesan  DB 'Halo, $'
    buffer DB 50 DUP(?)

.CODE
START:
    MOV AX, @DATA
    MOV DS, AX
    
    ; Tampilkan prompt
    MOV AH, 09h
    MOV DX, OFFSET prompt
    INT 21h
    
    ; Input string dengan echo (DOS buffered input)
    MOV AH, 0Ah
    MOV DX, OFFSET buffer
    INT 21h
    
    ; Input karakter tunggal (blocking)
    MOV AH, 00h
    INT 16h
    ; AL = ASCII code, AH = Scan code
    
    ; Check for key (non-blocking)
    MOV AH, 01h
    INT 16h
    JZ no_key
    ; Key available, read it
    MOV AH, 00h
    INT 16h
    
no_key:
    ; Flush keyboard buffer (DOS)
    MOV AH, 0Ch
    MOV AL, 00h
    INT 21h
    
    MOV AH, 4Ch
    INT 21h
END START
\end{verbatim}

\section{Latihan}
\begin{enumerate}
  \item Buat program yang menunggu input dari keyboard dan menampilkan nilai ASCII/scan code setiap tombol.
  \item Buat program yang menampilkan \textit{scan code} untuk function keys F1--F4 dan mengeksekusi aksi berbeda.
  \item Buat program input password dengan masking dan validasi panjang (min 6, maks 16), dukung Backspace.
  \item Buat program yang mendeteksi kondisi buffer penuh (simulasikan dengan input cepat), kemudian melakukan \texttt{flush} dan menampilkan pesan peringatan.
\end{enumerate}

\section{Tugas}
\begin{itemize}
  \item \textbf{Login sederhana}: Validasi username dan password yang disimpan di memori; berikan tiga kesempatan; kunci selama 5 detik (simulasikan delay) setelah 3 kegagalan.
  \item \textbf{Menu interaktif}: Navigasi dengan arrow keys, Enter untuk pilih, Esc untuk kembali; sorot pilihan aktif menggunakan warna.
  \item \textbf{Timeout input}: Implementasikan \textit{timeout} 5 detik untuk menekan tombol berikutnya (gunakan \textit{polling} \texttt{01h} dan penundaan berbasis \texttt{INT 1Ah} atau loop terkalibrasi); jika habis, batalkan input.
  \item \textbf{Dokumentasi INT 16h}: Ringkas parameter dan nilai kembali untuk fungsi 00h, 01h, 02h; sertakan tabel contoh keluaran.
\end{itemize}

\section{Referensi}
% Bibliography is handled by the main document

\chapter{Percabangan, Loop, dan Interupsi Mouse}

\section{Tujuan Pembelajaran}
Mahasiswa mampu:
\begin{itemize}
    \item Menggunakan instruksi \texttt{CMP} untuk perbandingan dan membaca dampaknya pada \textit{flags}.
    \item Menerapkan lompatan bersyarat \texttt{JE}, \texttt{JNE}, \texttt{JL}/\texttt{JG} dan variasinya berdasarkan interpretasi bertanda/tak bertanda.
    \item Menggunakan \texttt{LOOP}, \texttt{LOOPE}/\texttt{LOOPZ}, \texttt{LOOPNE}/\texttt{LOOPNZ} untuk perulangan berbasis \texttt{CX}.
    \item Berinteraksi dengan mouse melalui \texttt{INT 33h}: inisialisasi, tampil/sembunyi kursor, baca posisi/tombol.
    \item Menyusun program yang menggabungkan percabangan, perulangan, dan input mouse.
\end{itemize}

\section{Pendahuluan}
Kontrol alur adalah inti dari setiap program \cite{susanto1995belajar}. Pada 8086, \texttt{CMP} mengatur \textit{flags} yang selanjutnya diinterpretasikan oleh lompat bersyarat. Untuk iterasi, instruksi keluarga \texttt{LOOP} memanfaatkan \texttt{CX} sebagai pencacah \cite{hyde2010art}. Di sisi \textit{I/O}, \texttt{INT 33h} menyediakan API BIOS untuk mouse pada lingkungan DOS, memungkinkan antarmuka interaktif dasar \cite{nopi2003tutorial}.

\section{Instruksi Perbandingan (CMP)}
\subsubsection{Sintaks dan cara kerja}
\texttt{CMP dest, src} menghitung \(dest - src\) tanpa menyimpan hasil, namun memperbarui \textit{flags} (\texttt{ZF}, \texttt{SF}, \texttt{CF}, \texttt{OF}, \texttt{PF}, \texttt{AF}). \textit{Flags} kemudian digunakan oleh lompat bersyarat.
\begin{verbatim}
mov ax, 10
cmp ax, 10      ; ZF=1 -> equal
cmp ax, 12      ; CF=1 (tak bertanda), SF!=OF (bertanda) -> less
\end{verbatim}

\subsubsection{Perbandingan vs operasi aritmatika}
\texttt{CMP} setara dengan \texttt{SUB} yang hasilnya dibuang. Gunakan \texttt{TEST} untuk perbandingan bitwise (seperti \texttt{AND} tanpa menyimpan).

\subsection{Instruksi Percabangan}
\subsubsection{JE/JNE}
\texttt{JE}/\texttt{JZ} melompat bila \texttt{ZF=1}; \texttt{JNE}/\texttt{JNZ} bila \texttt{ZF=0}. Cocok untuk kesetaraan/kesetidaksetaraan umum.

\subsubsection{JL/JG dan varian bertanda vs tak bertanda}
\begin{itemize}
  \item \textbf{Bertanda}: \texttt{JL} (SF != OF), \texttt{JG} (ZF=0 dan SF=OF), \texttt{JLE}, \texttt{JGE}.
  \item \textbf{Tak bertanda}: \texttt{JB}/\texttt{JC} (CF=1), \texttt{JA} (CF=0 dan ZF=0), \texttt{JBE}, \texttt{JAE}/\texttt{JNC}.
\end{itemize}
Pilih sesuai interpretasi data untuk menghindari logika salah.

\subsubsection{Conditional jumps dan jangkauan}
Lompatan bersyarat menggunakan offset relatif pendek (\(\pm 128\) byte pada 8086). Untuk jangkauan lebih jauh, gunakan skema \textit{inverted branch} + \texttt{JMP} jauh.

\section{Pendalaman}\label{sec:branch-mouse-pendalaman}
\subsection{Pola kontrol alur yang umum}
Gunakan pola \textbf{if-else}, \textbf{switch-like} (tabel lompatan/\textit{jump table} berbasis indeks), dan \textbf{loop counted} (\texttt{CX}) vs \textbf{loop sentinel} (berhenti pada kondisi). Tabel lompatan dapat diimplementasikan dengan daftar alamat dan \texttt{JMP [BX+SI]}. \cite{intel2019manual32}

\subsection{Optimasi cabang}
Cabang pendek (\texttt{Jcc} 8-bit) lebih padat; restrukturisasi kondisi untuk memanfaatkan jangkauan pendek saat mungkin. Hindari memodifikasi \texttt{CX} tak sengaja saat memakai \texttt{LOOP}. \cite{hyde2010art}

\subsection{INT 33h detail dan koordinat}
Skala koordinat mouse bergantung mode tampilan. Pada mode teks, koordinat biasanya dalam sel karakter; di mode grafik dapat berupa piksel dengan rentang yang dapat diatur (fungsi pembatasan). Selalu periksa nilai kembali inisialisasi driver (\texttt{AX=FFFFh}) sebelum memakai layanan mouse. \cite{rbil}

\subsection{Debounce klik dan drag sederhana}
Catat transisi status tombol (tekan/lepas) untuk membedakan klik tunggal vs tahan/drag. Simpan posisi terakhir untuk menghitung delta pergerakan dan ambang jarak. \cite{osdev_wiki}


\subsection{Instruksi Perulangan (LOOP)}
\subsubsection{LOOP}
Menurunkan \texttt{CX} dan melompat jika \texttt{CX} \(\neq 0\). Perhatikan bahwa \texttt{CX} berkurang terlebih dahulu, lalu diuji.
\begin{verbatim}
mov cx, 5
ulang:
  ; body
  loop ulang
\end{verbatim}

\subsubsection{LOOPZ/LOOPE dan LOOPNZ/LOOPNE}
Melompat jika \texttt{CX} \(\neq 0\) \textbf{dan} \texttt{ZF} sesuai (0 atau 1). Umum untuk mengulang pencocokan hingga gagal/berhasil.

\subsubsection{Kontrol perulangan}
Hindari mengubah \texttt{CX} di dalam body kecuali disengaja. Untuk iterasi kompleks, kombinasi \texttt{DEC/JNZ} sering lebih fleksibel.

\subsection{Interupsi Mouse INT 33h}
\subsubsection{Fungsi 00h: Initialize Mouse}
\begin{itemize}
  \item \textbf{Masukan}: \texttt{AX=0000h}
  \item \textbf{Keluaran}: \texttt{AX=FFFFh} jika driver ada; selain itu 0.
\end{itemize}

\subsubsection{Fungsi 01h/02h: Show/Hide Cursor}
\begin{itemize}
  \item \texttt{AX=0001h} menampilkan; \texttt{AX=0002h} menyembunyikan.
\end{itemize}

\subsubsection{Fungsi 03h: Get Position and Buttons}
\begin{itemize}
  \item \textbf{Masukan}: \texttt{AX=0003h}
  \item \textbf{Keluaran}: \texttt{BX}=status tombol, \texttt{CX}=X, \texttt{DX}=Y (unit tergantung mode layar).
\end{itemize}

\subsubsection{Fungsi 04h: Set Position}
\begin{itemize}
  \item \textbf{Masukan}: \texttt{AX=0004h}, \texttt{CX}=X, \texttt{DX}=Y.
\end{itemize}

\subsubsection{Fungsi 05h: Get Button Press Info}
Mengembalikan informasi penekanan tombol terakhir (kode tombol, koordinat saat ditekan).

\section{Praktikum}
\begin{enumerate}
  \item Program percabangan sederhana: bandingkan dua nilai, tampilkan pesan sesuai hasil (JE/JNE).
  \item Program perulangan: gunakan \texttt{LOOP} untuk mencetak pola atau menghitung deret sederhana.
  \item Kombinasi: cari nilai minimum/maksimum dalam array menggunakan \texttt{CMP} + lompatan.
  \item Interaksi mouse: inisialisasi, tampilkan kursor, baca posisi dan tombol, tampilkan di layar.
  \item Mini-game: gerakkan kursor teks mengikuti posisi mouse (mode teks) dan reaksi saat klik.
\end{enumerate}

\section{Contoh Kode}
\begin{verbatim}
; Program percabangan, loop, dan mouse
TITLE Percabangan dan Loop
.MODEL SMALL
.STACK 100h

.DATA
    pesan1 DB 'Bilangan sama$'
    pesan2 DB 'Bilangan berbeda$'
    pesan3 DB 'Loop selesai$'
    bil1 DW 10
    bil2 DW 10

.CODE
START:
    MOV AX, @DATA
    MOV DS, AX
    
    ; Percabangan
    MOV AX, bil1
    CMP AX, bil2
    JE sama
    ; Jika tidak sama
    MOV AH, 09h
    MOV DX, OFFSET pesan2
    INT 21h
    JMP lanjut
    
sama:
    MOV AH, 09h
    MOV DX, OFFSET pesan1
    INT 21h
    
lanjut:
    ; Perulangan
    MOV CX, 5
ulang:
    ; Kode yang diulang
    DEC CX
    JNZ ulang
    
    ; Inisialisasi mouse
    MOV AX, 00h
    INT 33h
    CMP AX, 0
    JE no_mouse
    
    ; Tampilkan cursor mouse
    MOV AX, 01h
    INT 33h
    
    ; Baca posisi mouse
    MOV AX, 03h
    INT 33h
    ; BX = button status, CX = X position, DX = Y position
    
no_mouse:
    MOV AH, 4Ch
    INT 21h
END START
\end{verbatim}

\section{Latihan}
\begin{enumerate}
  \item Buat program yang membandingkan dua bilangan dan menampilkan hasil menggunakan JE/JNE.
  \item Buat program menghitung faktorial (n kecil) menggunakan \texttt{LOOP} atau \texttt{DEC/JNZ}.
  \item Buat program menampilkan angka 1--10 pada baris berbeda menggunakan perulangan.
  \item Buat program yang menampilkan koordinat mouse secara real-time (\textit{polling} \texttt{INT 33h} fungsi 03h).
\end{enumerate}

\section{Tugas}
\begin{itemize}
  \item \textbf{Kalkulator bercabang}: Menu berbasis \textit{branching} untuk memilih operasi aritmatika, validasi input.
  \item \textbf{Sorting sederhana}: Implementasikan \textit{bubble sort} pada array kecil (tak bertanda), tampilkan hasil setiap pass.
  \item \textbf{Game tebak angka + mouse}: Gunakan mouse untuk memilih rentang jawaban, tampilkan umpan balik.
  \item \textbf{Dokumentasi}: Ringkas instruksi lompatan dan kondisi flags yang memicunya; sertakan contoh minimal.
\end{itemize}

\section{Referensi}
% Bibliography is handled by the main document

\include{materi_pertemuan_ke_8}

% Bagian III: Struktur Data dan Kontrol Alur
\part{Struktur Data dan Kontrol Alur}\label{part:structures}

\chapter{Instruksi Stack: PUSH, POP, CALL, RET; Subrutin dan Parameter Sederhana}

\section{Tujuan Pembelajaran}
Mahasiswa mampu:
\begin{itemize}
  \item Menjelaskan konsep tumpukan (stack), prinsip LIFO, serta peran \texttt{SS:SP}.
  \item Menggunakan instruksi \texttt{PUSH}/\texttt{POP} pada register dan memori dengan aman.
  \item Memanggil dan kembali dari subrutin menggunakan \texttt{CALL}/\texttt{RET} (near/far) dan memahami \textit{return address}.
  \item Mengirim parameter dan menerima nilai balik melalui register, stack, atau memori.
  \item Memodularisasi program menjadi prosedur-prosedur yang terpisah dan teruji.
\end{itemize}

\section{Pendahuluan}
Stack adalah struktur data kritis untuk pengelolaan alur eksekusi dan data sementara. Pada 8086, stack berada pada segmen \texttt{SS} dengan penunjuk \texttt{SP}. Instruksi \texttt{PUSH}/\texttt{POP} mengubah \texttt{SP} dan menyimpan/mengambil data 16-bit (atau 8-bit dipecah) dalam urutan LIFO. Instruksi \texttt{CALL}/\texttt{RET} mengandalkan stack untuk menyimpan alamat kembali ketika berpindah ke subrutin.

\section{Materi Pembelajaran}
\subsection{Konsep Stack}
\subsubsection{Definisi dan prinsip LIFO}
Stack menyimpan item sedemikian rupa sehingga item terakhir yang dimasukkan akan diambil pertama kali. Pada 8086, stack tumbuh ke alamat lebih rendah: \texttt{PUSH} menurunkan \texttt{SP} lalu menyimpan data; \texttt{POP} mengambil data lalu menaikkan \texttt{SP}.

\subsubsection{Register SS:SP dan layout memori}
Pasangan \texttt{SS:SP} membentuk alamat efektif untuk operasi stack. Pastikan \texttt{SS} diinisialisasi benar dan \texttt{SP} memiliki ruang cukup, agar tidak menimpa data lain.

\subsubsection{Operasi stack dalam assembly}
Operasi umum: \texttt{PUSH reg/mem/flags}, \texttt{POP reg/mem}, \texttt{PUSHF}/\texttt{POPF}. Simpan register yang akan dipakai di subrutin dan kembalikan sebelum \texttt{RET}.

\subsection{Instruksi Stack}
\subsubsection{PUSH/POP register}
\begin{verbatim}
push ax
push bx
; ...
pop  bx
pop  ax
\end{verbatim}
Urutan pelepasan terbalik terhadap pemasukan.

\subsubsection{PUSH/POP memori dan immediate}
Beberapa assembler mengizinkan \texttt{PUSH word ptr [addr]} atau \texttt{PUSH imm16}. Untuk \texttt{POP} ke memori, gunakan operand memori word yang valid.

\subsubsection{PUSHF/POPF (flag register)}
\texttt{PUSHF} menyimpan FLAGS ke stack; \texttt{POPF} memulihkannya. Berguna saat memodifikasi \textit{flags} sementara.

\subsection{Instruksi Subrutin}
\subsubsection{CALL dan RET}
\texttt{CALL} (near) menyimpan \texttt{IP} ke stack dan melompat dalam segmen yang sama; \texttt{CALL FAR} menyimpan \texttt{CS:IP} untuk melompat antar segmen. \texttt{RET} memulihkan alamat kembali; \texttt{RETF} untuk far return.

\subsubsection{RET dengan parameter}
Variasi \texttt{RET n} menambah \texttt{SP} sebesar \(n\) byte setelah mengambil alamat kembali, berguna untuk membersihkan parameter stack oleh callee (konvensi \textit{pascal}-like).

\subsubsection{Penanganan return address}
Alamat kembali adalah nilai IP (dan CS untuk far) yang \texttt{CALL} dorong ke stack; jangan menimpa area ini dengan \texttt{PUSH}/\texttt{POP} tak terkendali.

\subsection{Parameter dalam Subrutin}
\subsubsection{Melalui register}
Cepat namun terbatas jumlahnya; dokumentasikan register input/output agar konsisten.

\subsubsection{Melalui stack}
Beri \texttt{PUSH} parameter sebelum \texttt{CALL}. Di dalam prosedur: \texttt{PUSH BP}; \texttt{MOV BP, SP}; akses parameter via \texttt{[BP+offset]}. Setelah selesai, \texttt{POP BP}. Pembersihan parameter oleh caller (\texttt{ADD SP, n}) atau callee (\texttt{RET n}).

\subsubsection{Melalui memori}
Sediakan blok data global/struktur; subrutin membaca/menulis langsung. Kurangi \textit{coupling} dengan mendokumentasikan kontrak data.

\subsubsection{Return value}
Letakkan pada \texttt{AX} (konvensi umum) atau tulis ke lokasi memori yang disepakati.

\subsubsection{Konvensi pemanggilan}
Sepakati skema: siapa yang menyimpan register yang dipakai (caller-saves vs callee-saves), mekanisme pembersihan parameter, dan register pengembalian.

\section{Praktikum}
\begin{enumerate}
  \item Demonstrasi \texttt{PUSH}/\texttt{POP}: dorong beberapa register dan verifikasi urutannya dengan debugger.
  \item Prosedur tanpa parameter: tampilkan string, gunakan \texttt{PUSH}/\texttt{POP} untuk menyimpan register yang dipakai.
  \item Prosedur dengan parameter via stack: jumlahkan dua bilangan 16-bit; kembalikan di \texttt{AX} dan/atau memori.
  \item Prosedur dengan return value: implementasikan \texttt{mul8(a,b)} yang mengembalikan hasil 8-bit di \texttt{AL}.
  \item Program modular: pecah menjadi beberapa prosedur (\texttt{read\_hex}, \texttt{print\_hex}, \texttt{add16}, \texttt{menu}).
\end{enumerate}

\section{Contoh Kode}
\begin{verbatim}
; Program demonstrasi stack dan subrutin
TITLE Stack dan Subrutin
.MODEL SMALL
.STACK 100h

.DATA
    pesan1 DB 'Hello from main!$'
    pesan2 DB 'Hello from subrutin!$'
    nilai1 DW 10
    nilai2 DW 20
    hasil DW ?

.CODE
START:
    MOV AX, @DATA
    MOV DS, AX
    
    ; Demonstrasi stack
    MOV AX, 1234h
    PUSH AX
    MOV BX, 5678h
    PUSH BX
    
    ; Ambil dari stack (urutan terbalik)
    POP CX  ; CX = 5678h
    POP DX  ; DX = 1234h
    
    ; Panggil subrutin
    CALL tampilkan_pesan
    
    ; Subrutin dengan parameter (via stack)
    PUSH nilai1
    PUSH nilai2
    CALL tambahkan
    ADD SP, 4  ; Bersihkan parameter dari stack (caller cleans)
    
    MOV AH, 4Ch
    INT 21h

; Subrutin tanpa parameter
TAMPILKAN_PESAN PROC
    PUSH AX
    PUSH DX
    
    MOV AH, 09h
    MOV DX, OFFSET pesan2
    INT 21h
    
    POP DX
    POP AX
    RET
TAMPILKAN_PESAN ENDP

; Subrutin dengan parameter via stack
TAMBAHKAN PROC
    PUSH BP
    MOV  BP, SP
    
    PUSH AX
    PUSH BX
    
    MOV AX, [BP+6]  ; Parameter kedua
    MOV BX, [BP+4]  ; Parameter pertama
    ADD AX, BX
    MOV hasil, AX   ; atau kembalikan di AX
    
    POP BX
    POP AX
    POP BP
    RET
TAMBAHKAN ENDP

END START
\end{verbatim}

\section{Latihan}
\begin{enumerate}
  \item Gunakan stack untuk menyimpan tiga nilai sementara; ambil kembali dalam urutan yang benar dan verifikasi.
  \item Tulis subrutin \texttt{kuadrat16(x)} yang mengembalikan \(x^2\) (perhatikan luapan). Kembalikan di \texttt{DX:AX} untuk hasil 32-bit.
  \item Tulis subrutin \texttt{swap16(a\_ptr,b\_ptr)} yang menukar dua word di memori menggunakan stack sementara.
  \item Rancang program yang memanggil beberapa subrutin secara berantai dengan parameter berbeda; pastikan stack seimbang.
\end{enumerate}

\section{Tugas}
\begin{itemize}
  \item \textbf{Subrutin aritmatika}: Implementasikan \texttt{add16}, \texttt{sub16}, \texttt{mul16}, \texttt{div16} dengan kontrak parameter/return yang jelas.
  \item \textbf{Kalkulator modular}: Antarmuka teks yang memanggil subrutin operasi berdasarkan pilihan pengguna; validasi input dan tangani kesalahan (mis. pembagi nol).
  \item \textbf{Manipulasi string}: Buat subrutin \texttt{strlen}, \texttt{strcpy}, \texttt{strcmp} sederhana untuk string berakhiran \texttt{'$'} atau \texttt{0} (pilih satu dan konsisten).
  \item \textbf{Dokumentasi}: Buat diagram stack frame untuk pemanggilan bertingkat (nested calls) dan jelaskan peran \texttt{BP}, \texttt{SP}, serta urutan penyimpanan register.
\end{itemize}

\section{Referensi}
\begin{itemize}
  \item Hyde, Randall. \textit{The Art of Assembly Language}, 2nd ed., No Starch Press, 2010.
  \item Susanto. \textit{Belajar Pemrograman Bahasa Assembly}, Elex Media Komputindo, 1995.
\end{itemize}

\chapter{Array dan String}

\section{Tujuan Pembelajaran}
Mahasiswa mampu:
\begin{itemize}
    \item Menjelaskan representasi array dan string dalam memori serta pengalamatan elemennya.
    \item Menggunakan instruksi string \texttt{MOVS}, \texttt{CMPS}, \texttt{SCAS}, \texttt{LODS}, \texttt{STOS} dengan benar.
    \item Mengoptimalkan operasi berulang menggunakan awalan \texttt{REP}/\texttt{REPE}/\texttt{REPNE} dan pengelolaan \texttt{DF}.
    \item Membangun rutinitas manipulasi string/array (salin, banding, cari, panjang, balik, gabung, urut).
\end{itemize}

\section{Pendahuluan}
Array dan string adalah struktur data fundamental \cite{susanto1995belajar}. Pada 8086, operasi berulang terhadap data berurutan didukung oleh instruksi string yang menggabungkan pembacaan/penulisan data dengan peningkatan/penurunan indeks otomatis (\texttt{SI}/\texttt{DI}) dan pengurangan pencacah (\texttt{CX}) bila digabung dengan awalan \texttt{REP} \cite{hyde2010art}. Pengelolaan \texttt{Direction Flag} (\texttt{DF}) menentukan arah iterasi (naik dengan \texttt{CLD}, turun dengan \texttt{STD}) \cite{nopi2003tutorial}.

\section{Konsep Array dan String}
\subsection{Definisi dan penyimpanan}
Array adalah koleksi elemen bertipe homogen yang tersusun kontigu. String dapat dipandang sebagai array karakter dengan konvensi terminasi (mis. \texttt{0} atau \texttt{\$}). Penyimpanan memperhatikan ukuran elemen (byte/word) untuk perhitungan offset.

\subsubsection{Pengalamatan dan perhitungan offset}
Offset elemen ke-\(i\) pada array byte: \(base + i\). Pada array word: \(base + 2\cdot i\). Gunakan \texttt{SI}/\texttt{DI} sebagai indeks sumber/tujuan.

\subsubsection{String berterminasi}
\begin{itemize}
  \item \textbf{Null-terminated (C)}: berakhir dengan \texttt{0}. Fungsi seperti panjang memindai hingga \texttt{0}.
  \item \textbf{DOS \$-terminated}: digunakan oleh \texttt{INT 21h, AH=09h}; berakhir dengan \texttt{\$}.
\end{itemize}
Konsisten gunakan satu konvensi per rutinitas.

\subsection{Penyimpanan Array}
\subsubsection{Satu dimensi}
Akses langsung via indeks. Untuk word, manfaatkan \texttt{XLAT} atau perhitungan manual \(i\times 2\).

\subsubsection{Multi dimensi}
Gunakan perataan baris (row-major). Offset \((r,c)\) pada array word berdimensi \(R\times C\): \(base + 2\cdot (r\cdot C + c)\).

\subsection{Instruksi String}
\subsubsection{MOVS}
Menyalin dari \texttt{DS:SI} ke \texttt{ES:DI}; varian \texttt{MOVSB}/\texttt{MOVSW} \cite{hyde2010art}. Dengan \texttt{REP}, menyalin \texttt{CX} elemen \cite{susanto1995belajar}.
\begin{verbatim}
cld
mov  cx, len
lea  si, src
lea  di, dst
rep  movsb
\end{verbatim}

\subsubsection{CMPS}
Membandingkan \texttt{DS:SI} dengan \texttt{ES:DI}; atur \textit{flags}. Dengan \texttt{REPE}/\texttt{REPZ} berhenti pada perbedaan atau saat \texttt{CX}=0.
\begin{verbatim}
cld
mov  cx, n
lea  si, s1
lea  di, s2
repe cmpsb
jne  berbeda
\end{verbatim}

\subsubsection{SCAS}
Membandingkan \texttt{AL}/\texttt{AX} dengan \texttt{ES:DI}. Umum untuk mencari karakter/word.
\begin{verbatim}
cld
mov  al, ch
mov  cx, n
lea  di, buf
repne scasb
jne  not_found
\end{verbatim}

\subsubsection{LODS}
Memuat dari \texttt{DS:SI} ke \texttt{AL}/\texttt{AX}, menaikkan/menurunkan \texttt{SI}.

\subsubsection{STOS}
Menyimpan dari \texttt{AL}/\texttt{AX} ke \texttt{ES:DI}; dengan \texttt{REP}, mengisi blok dengan nilai yang sama.
\begin{verbatim}
cld
mov  al, 0
mov  cx, 80
lea  di, line
rep  stosb
\end{verbatim}

\subsubsection{Awalan REP dan Direction Flag}
\texttt{REP} mengulang hingga \texttt{CX}=0. \texttt{REPE}/\texttt{REPZ} mengulang selama \texttt{ZF=1}; \texttt{REPNE}/\texttt{REPNZ} selama \texttt{ZF=0}. Gunakan \texttt{CLD} untuk iterasi maju, \texttt{STD} untuk mundur; kembalikan ke \texttt{CLD} demi konsistensi.

\subsection{Register untuk Operasi String}
\begin{itemize}
  \item \textbf{SI/DI}: indeks sumber/tujuan (\texttt{DS:SI}, \texttt{ES:DI}).
  \item \textbf{CX}: pencacah pengulangan untuk \texttt{REP}.
  \item \textbf{AL/AX}: akumulator untuk \texttt{LODS}/\texttt{SCAS}/\texttt{STOS}.
  \item \textbf{DF}: arah iterasi; atur dengan \texttt{CLD}/\texttt{STD}.
  \item \textbf{ES}: segmen tujuan harus diinisialisasi (sering disamakan dengan \texttt{DS} bila menyalin dalam segmen sama).
\end{itemize}

\section{Praktikum}
\begin{enumerate}
  \item Demonstrasi akses elemen array (byte/word) dan perhitungan offset.
  \item Salin string menggunakan \texttt{REP MOVSB} dan verifikasi.
  \item Bandingkan dua string dengan \texttt{REPE CMPSB}; tampilkan hasil sama/berbeda.
  \item Cari karakter dalam string dengan \texttt{REPNE SCASB}; laporkan posisi ditemukan.
  \item Implementasi rutin panjang dan pembalik string menggunakan \texttt{LODS}/\texttt{STOS}.
\end{enumerate}

\section{Contoh Kode}
\begin{verbatim}
; Program demonstrasi array dan string
TITLE Array dan String
.MODEL SMALL
.STACK 100h

.DATA
    array   DW 10, 20, 30, 40, 50
    string1 DB 'Hello World$'
    string2 DB 20 DUP('$')
    string3 DB 'Assembly$'
    string4 DB 'Assembly$'
    karakter DB 'l'
    panjang EQU $ - string1

.CODE
START:
    MOV AX, @DATA
    MOV DS, AX
    MOV ES, AX
    
    ; Akses elemen array (word): indeks ke-2 (basis 0)
    MOV BX, 2
    SHL BX, 1            ; *2 untuk word
    MOV AX, array[BX]    ; AX = 30
    
    ; Copy string menggunakan MOVS
    LEA SI, string1
    LEA DI, string2
    MOV CX, panjang
    CLD
    REP MOVSB
    
    ; Perbandingan string menggunakan CMPS
    LEA SI, string3
    LEA DI, string4
    MOV CX, 8
    CLD
    REPE CMPSB
    JE sama
    JMP lanjut
sama:
    ; String sama
lanjut:
    ; Pencarian karakter menggunakan SCAS
    LEA DI, string1
    MOV AL, karakter
    MOV CX, panjang
    CLD
    REPNE SCASB
    JE ditemukan
    JMP selesai

ditemukan:
    ; Karakter ditemukan (posisi dapat dihitung)
selesai:
    ; Isi blok dengan 'X' menggunakan STOS
    LEA DI, string2
    MOV AL, 'X'
    MOV CX, 5
    CLD
    REP STOSB
    
    MOV AH, 4Ch
    INT 21h
END START
\end{verbatim}

\section{Latihan}
\begin{enumerate}
  \item Akses elemen array dengan indeks dan tampilkan nilainya dalam heksadesimal.
  \item Salin string dari satu lokasi ke lokasi lain menggunakan \texttt{MOVS}; pastikan terminator ikut disalin sesuai konvensi.
  \item Bandingkan dua string dengan panjang berbeda; tentukan hasil leksikografis sederhana.
  \item Cari semua kemunculan karakter dalam string dan catat posisinya ke array hasil.
\end{enumerate}

\section{Tugas}
\begin{itemize}
  \item \textbf{Panjang string}: Implementasikan \texttt{strlen} untuk terminator \texttt{0} \textbf{atau} \texttt{'{\$}'} (pilih salah satu) menggunakan \texttt{SCASB}.
  \item \textbf{Balik string}: Implementasikan pembalik in-place menggunakan dua indeks (awal/akhir) atau \texttt{LODS}/\texttt{STOS} ke buffer sementara.
  \item \textbf{Gabung string}: Implementasikan \texttt{strcat} sederhana; pastikan tidak terjadi \textit{overflow} buffer tujuan.
  \item \textbf{Urutkan array}: Urutkan array word kecil (mis. 10 elemen) dengan \textit{bubble sort}; tampilkan hasil setiap putaran.
\end{itemize}

\section{Referensi}
% Bibliography is handled by the main document

\chapter{Pemrograman Modular dan Makro}

\section{Tujuan Pembelajaran}
Mahasiswa mampu:
\begin{itemize}
    \item Menjelaskan prinsip pemrograman modular serta manfaatnya terhadap kualitas perangkat lunak.
    \item Menggunakan \texttt{PROC}/\texttt{ENDP} untuk membangun prosedur (dengan/ tanpa parameter dan nilai balik) dengan disiplin penyimpanan register dan \textit{stack frame}.
    \item Membuat dan memanfaatkan makro (\texttt{MACRO}/\texttt{ENDM}) dengan parameter, termasuk bentuk bersyarat sederhana.
    \item Membedakan penggunaan prosedur vs. makro berdasarkan kebutuhan \textit{runtime}, ukuran kode, dan pemeliharaan.
    \item Mengorganisasi program multi-berkas dengan \texttt{INCLUDE}, modul, dan pustaka prosedur.
\end{itemize}

\section{Pendahuluan}
Pemrograman modular mendorong pemisahan program menjadi komponen-komponen kecil dengan tanggung jawab jelas \cite{susanto1995belajar}. Dalam assembly, modularitas dicapai melalui prosedur (\texttt{PROC}) dan dapat diperkuat dengan makro untuk mengurangi pengulangan pola kode \cite{hyde2010art}. Perbedaan fundamental: prosedur dieksekusi di \textit{runtime} (overhead \texttt{CALL}/\texttt{RET}), sedangkan makro diperluas saat perakitan (\textit{compile-time}), menghasilkan duplikasi kode yang dapat menambah ukuran biner \cite{nopi2003tutorial}.

\section{Konsep Pemrograman Modular}
\subsection{Definisi dan keuntungan}
Modularitas meningkatkan \textit{readability}, \textit{testability}, dan \textit{reusability}. Komponen yang terdefinisi baik mempercepat \textit{debugging} dan memudahkan perawatan.

\subsubsection{Prinsip modularitas}
\begin{itemize}
  \item \textbf{Cohesion}: setiap modul fokus pada satu tanggung jawab.
  \item \textbf{Coupling}: minimalkan ketergantungan antar modul.
  \item \textbf{Kontrak}: dokumentasikan antarmuka (parameter, return, register yang dimodifikasi).
\end{itemize}

\subsubsection{Organisasi kode}
Pisahkan deklarasi data, prosedur utilitas, dan logika utama. Gunakan penamaan konsisten dan komentar yang menerangkan tujuan.

\subsection{Prosedur dengan PROC/ENDP}
\subsubsection{Sintaks dan deklarasi}
\begin{verbatim}
NamaProsedur PROC [NEAR|FAR]
  ; body
  RET [n]
NamaProsedur ENDP
\end{verbatim}
Gunakan \texttt{NEAR} untuk segmen yang sama; \texttt{FAR} bila antar segmen.

\subsubsection{Parameter dan local variables}
Konvensi umum: parameter lewat stack; bentuk \textit{prologue/epilogue} standar:
\begin{verbatim}
push bp
mov  bp, sp
sub  sp, local_size   ; alokasikan lokal (opsional)
; akses param: [bp+4], [bp+6], ...
; akses lokal: [bp-2], [bp-4], ...
mov  sp, bp
pop  bp
ret  n     ; atau ret, caller cleans
\end{verbatim}

\subsubsection{Return value dan konvensi pemanggilan}
Kembalikan nilai di \texttt{AX} (umum) atau di memori. Tetapkan siapa yang menyimpan register: \textit{caller-saves} (penelepon menyimpan) atau \textit{callee-saves} (terpanggil menyimpan). Dokumentasikan jelas.

\subsection{Makro (MACRO)}
\subsubsection{Definisi dan sintaks}
\begin{verbatim}
NamaMakro MACRO [param1[, param2, ...]]
  ; ekspansi
ENDM
\end{verbatim}
Makro memperluas teks pada saat perakitan. Parameter menggantikan token dalam ekspansi.

\subsubsection{Makro berparameter dan bersyarat}
Gunakan arahan perakitan bersyarat (bergantung assembler) untuk variasi ekspansi. Perhatikan bahwa makro tidak memiliki \textit{runtime context} seperti stack kecuali disimulasikan.

\subsubsection{Perbedaan makro vs. prosedur}
Makro: tanpa overhead \texttt{CALL}/\texttt{RET}, namun memperbesar ukuran biner jika sering digunakan. Prosedur: lebih kecil di biner jika dipanggil berulang, namun memiliki overhead panggilan.

\section{Pendalaman}\label{sec:modular-pendalaman}
\subsection{Organisasi proyek multi-berkas}
Pisahkan modul berdasarkan domain (mis. \texttt{io.asm}, \texttt{math.asm}, \texttt{ui.asm}) dan satu berkas utama \texttt{main.asm}. Gunakan \texttt{PUBLIC}/\texttt{EXTRN} (TASM/MASM) untuk mengekspor/mengimpor simbol lintas-berkas. Pastikan kesesuaian prototipe (ukuran parameter) dan konvensi pemanggilan. \cite{borland1990tasm}

\subsection{Makro tingkat lanjut}
Makro dengan parameter mendukung substitusi token dan kondisi perakitan (bergantung assembler). Hindari makro yang mengubah konteks tak terduga (mis., register) tanpa mendokumentasikan efek samping. Untuk debug, sediakan makro \texttt{TRACE} yang dapat diaktif/nonaktifkan \textit{compile-time}. \cite{nasm_manual}

\subsection{Pedoman API prosedur}
Dokumentasikan: parameter (urutan, ukuran, interpretasi bertanda/tak bertanda), return value, register yang dimodifikasi, dan prasyarat segmen (\texttt{DS}/\texttt{ES}). Sertakan kasus tepi dan perilaku error (mis., mengembalikan kode di \texttt{AX}). \cite{osdev_wiki}

\subsection{Strategi pengujian}
Bangun program uji kecil per modul yang memverifikasi kontrak rutinitas. Untuk rutinitas murni (tanpa I/O), validasi tabel-kasus. Gunakan debugger (TD) untuk memeriksa state sebelum/sesudah panggilan. \cite{borland1990tasm}

\subsection{Organisasi Program Modular}
\subsubsection{Struktur file dan include}
Kumpulkan makro umum dalam berkas \texttt{.INC}; sertakan dengan \texttt{INCLUDE}. Simpan prosedur utilitas dalam modul terpisah untuk \textit{link}-time reuse.

\subsubsection{Library routines dan dependencies}
Buat pustaka prosedur (sekumpulan \texttt{.OBJ}) dan tautkan seperlunya. Dokumentasikan ketergantungan dan urutan inisialisasi.

\section{Praktikum}
\begin{enumerate}
  \item Bangun prosedur \texttt{print\_str} (\texttt{AH=09h}) dengan prologue/epilogue rapi.
  \item Prosedur berparameter: \texttt{add16(a,b)} mengembalikan hasil di \texttt{AX}; \texttt{RET 4} (callee cleans) vs \texttt{ADD SP,4} (caller cleans).
  \item Makro sederhana: \texttt{PRINT{\$} pesan} yang memanggil \texttt{INT 21h} untuk string berakhiran \texttt{\$}.
  \item Makro berparameter: \texttt{SWAP var1, var2} untuk menukar dua word.
  \item Susun program modular yang memanfaatkan makro dan prosedur pada beberapa berkas.
\end{enumerate}

\section{Contoh Kode}
\begin{verbatim}
; Program demonstrasi prosedur dan makro
TITLE Pemrograman Modular
.MODEL SMALL
.STACK 100h

.DATA
    pesan1 DB 'Hello from procedure!$'
    pesan2 DB 'Hello from macro!$'
    nilai1 DW 15
    nilai2 DW 25
    hasil  DW ?

; Makro untuk menampilkan pesan
TAMPILKAN_PESAN MACRO pesan
    PUSH AX
    PUSH DX
    MOV  AH, 09h
    MOV  DX, OFFSET pesan
    INT  21h
    POP  DX
    POP  AX
ENDM

; Makro untuk pertukaran nilai
TUKAR_NILAI MACRO var1, var2
    PUSH AX
    MOV  AX, var1
    XCHG AX, var2
    MOV  var1, AX
    POP  AX
ENDM

.CODE
START:
    MOV AX, @DATA
    MOV DS, AX
    
    ; Panggil prosedur
    CALL TAMPILKAN_HELLO
    
    ; Gunakan makro
    TAMPILKAN_PESAN pesan2
    
    ; Pertukaran nilai menggunakan makro
    TUKAR_NILAI nilai1, nilai2
    
    ; Panggil prosedur dengan parameter
    PUSH nilai1
    PUSH nilai2
    CALL TAMBAHKAN
    ADD  SP, 4
    
    MOV AH, 4Ch
    INT 21h

; Prosedur tanpa parameter
TAMPILKAN_HELLO PROC
    PUSH AX
    PUSH DX
    MOV  AH, 09h
    MOV  DX, OFFSET pesan1
    INT  21h
    POP  DX
    POP  AX
    RET
TAMPILKAN_HELLO ENDP

; Prosedur dengan parameter via stack
TAMBAHKAN PROC
    PUSH BP
    MOV  BP, SP
    PUSH AX
    PUSH BX
    MOV  AX, [BP+6]  ; Parameter kedua
    MOV  BX, [BP+4]  ; Parameter pertama
    ADD  AX, BX
    MOV  hasil, AX
    POP  BX
    POP  AX
    POP  BP
    RET
TAMBAHKAN ENDP

END START
\end{verbatim}

\section{Latihan}
\begin{enumerate}
  \item Buat prosedur \texttt{mul8(a,b)} yang mengembalikan hasil di \texttt{AX} dan perhatikan flags.
  \item Buat makro \texttt{PRINTC ch} untuk mencetak satu karakter via \texttt{AH=02h} dan \texttt{DL}.
  \item Buat prosedur \texttt{sort\_word\_asc(ptr,n)} (bubble/selection) dengan kontrak parameter jelas.
  \item Buat makro \texttt{ASSERT cond, msg} yang mencetak pesan jika kondisi gagal (gunakan \texttt{IF} assembler jika tersedia).
\end{enumerate}

\section{Tugas}
\begin{itemize}
  \item \textbf{Library string}: Implementasikan \texttt{strlen}, \texttt{strcpy}, \texttt{strcmp} untuk satu konvensi terminator; gabungkan menjadi pustaka.
  \item \textbf{Makro debugging}: Makro untuk cetak register (AX,BX,CX,DX) dalam heksadesimal; gunakan pada titik-titik penting program.
  \item \textbf{Kalkulator modular}: Gunakan prosedur operasi dan makro UI untuk membangun kalkulator teks dengan menu.
  \item \textbf{Dokumentasi}: Tulis panduan singkat kapan memilih prosedur vs. makro pada proyek Anda.
\end{itemize}

\section{Referensi}
% Bibliography is handled by the main document


% Bagian IV: Pemrograman Modular dan Lanjutan
\part{Pemrograman Modular dan Lanjutan}\label{part:advanced}

\include{materi_pertemuan_ke_12}
\chapter{Pemrosesan File Dasar: Membuka, Menutup, Membaca, Menulis File (INT 21h fungsi 3Ch/3Fh/40h)}

\section{Tujuan Pembelajaran}
Mahasiswa mampu:
\begin{itemize}
  \item Menjelaskan konsep berkas (file), sistem berkas, dan \textit{handle}-based I/O pada DOS.
  \item Menggunakan \texttt{INT 21h} untuk membuat, membuka, menutup, membaca, menulis, dan menghapus berkas.
  \item Menerapkan pola \textit{buffered I/O} sederhana dan menangani kesalahan berdasarkan kode error.
  \item Menyusun program utilitas file dasar seperti penyalinan dan pembacaan konten.
\end{itemize}

\section{Pendahuluan}
Pada DOS, layanan \texttt{INT 21h} menyediakan antarmuka sistem berkas berbasiskan \textit{file handle}. Operasi utama meliputi pembuatan/ pembukaan berkas, pembacaan/penulisan data, dan penutupan. Keberhasilan/kegagalan operasi diindikasikan oleh \textit{carry flag} (CF) dan kode error di \texttt{AX} saat CF diset.

\section{Materi Pembelajaran}
\subsection{Konsep File Handling}
\subsubsection{Definisi file dan file system}
File adalah sekumpulan byte pada media penyimpanan, diorganisasikan oleh sistem berkas (FAT pada DOS). Identifikasi via nama (8.3) dan atribut (read-only, hidden, system, archive).

\subsubsection{File handle dan mode akses}
Setelah berkas dibuka/dibuat, DOS mengembalikan \textit{handle} (bilangan kecil) untuk digunakan pada operasi lanjutan. Mode akses: baca (\texttt{AL=0}), tulis (\texttt{AL=1}), baca/tulis (\texttt{AL=2}); lampiran (append) dapat disimulasikan dengan memindah \textit{file pointer} ke akhir.

\subsubsection{Atribut dan perizinan}
Atribut ditentukan pada pembuatan atau dapat diubah menggunakan fungsi lain (di luar cakupan dasar). Gunakan atribut normal (\texttt{CX=0}) untuk kebanyakan kasus.

\subsubsection{Error handling}
CF=1 menunjukkan kesalahan; \texttt{AX} memuat kode (mis. file tidak ditemukan, akses ditolak). Tanggapi dengan pesan dan \textit{cleanup} sumber daya.

\subsection{Interupsi INT 21h untuk File}
\begin{itemize}
  \item \textbf{3Ch Create File}: \texttt{AH=3Ch}, \texttt{CX=atribut}, \texttt{DX=\&nama}; keluar: \texttt{AX=handle}.
  \item \textbf{3Dh Open File}: \texttt{AH=3Dh}, \texttt{AL=mode}, \texttt{DX=\&nama}; keluar: \texttt{AX=handle}.
  \item \textbf{3Eh Close File}: \texttt{AH=3Eh}, \texttt{BX=handle}.
  \item \textbf{3Fh Read File}: \texttt{AH=3Fh}, \texttt{BX=handle}, \texttt{CX=jumlah}, \texttt{DX=\&buffer}; keluar: \texttt{AX=byte terbaca}.
  \item \textbf{40h Write File}: \texttt{AH=40h}, \texttt{BX=handle}, \texttt{CX=jumlah}, \texttt{DX=\&buffer}; keluar: \texttt{AX=byte tertulis}.
  \item \textbf{41h Delete File}: \texttt{AH=41h}, \texttt{DX=\&nama}.
\end{itemize}
Perhatikan selalu pemeriksaan CF dan konsistensi \texttt{AX/CF}.

\subsection{Operasi File Dasar}
\subsubsection{Membuat dan membuka}
Gunakan \texttt{3Ch} untuk membuat (menimpa jika ada tergantung sistem); gunakan \texttt{3Dh} untuk membuka yang sudah ada dengan mode sesuai kebutuhan.

\subsubsection{Menutup}
Selalu tutup \texttt{handle} dengan \texttt{3Eh} untuk mencegah kebocoran; DOS membatasi jumlah handle terbuka.

\subsubsection{Membaca dan menulis}
Gunakan buffer di segmen data. Jumlah byte aktual terbaca/tertulis dikembalikan di \texttt{AX}; periksa ketidaksamaan dengan \texttt{CX} (mis. akhir berkas pada baca).

\subsubsection{Menghapus}
\texttt{41h} menghapus berkas; pastikan ditutup sebelumnya dan atribut memungkinkan penghapusan.

\subsection{File Handle dan Error Handling}
\subsubsection{Manajemen handle}
Simpan \texttt{AX} hasil \texttt{3Ch/3Dh} ke variabel. Hindari penggunaan handle setelah ditutup. Gunakan nilai sentinel (mis. \texttt{FFFFh}) untuk menandai tidak-valid.

\subsubsection{Kode error dan validasi}
Tangkap kasus umum: file tidak ditemukan (\texttt{2}), akses ditolak (\texttt{5}), disk penuh, dan lain-lain. Berikan pesan informatif.

\subsubsection{Resource cleanup}
Pada error di tengah rangkaian operasi, lakukan penutupan handle yang sudah terbuka sebelum keluar.

\subsection{Buffer dan Transfer Data}
\subsubsection{Buffer dan block transfer}
Baca/tulis dalam blok untuk efisiensi. Untuk berkas teks, pertimbangkan terminator atau pemrosesan baris.

\subsubsection{Akses sekuensial}
Baca berulang dari awal ke akhir. Untuk acak, gunakan \texttt{INT 21h, 42h} (dibahas pertemuan lanjutan).

\section{Praktikum}
\begin{enumerate}
  \item Program membuat berkas baru dan menulis satu baris teks.
  \item Program membaca kembali isi berkas dan menampilkannya.
  \item Program menyalin berkas (baca-blok/tulis-blok hingga akhir berkas).
  \item Program dengan penanganan error: uji nama tidak valid, akses ditolak, dan disk penuh (simulasi).
\end{enumerate}

\section{Contoh Kode}
\begin{verbatim}
; Program demonstrasi operasi file dasar
TITLE Pemrosesan File
.MODEL SMALL
.STACK 100h

.DATA
    nama_file   DB 'test.txt', 0
    data_tulis  DB 'Hello, World!', 13, 10
    data_baca   DB 100 DUP(?)
    file_handle DW ?
    pesan_sukses DB 'File berhasil dibuat!$'
    pesan_error  DB 'Error dalam operasi file!$'
    pesan_baca   DB 'Data dari file: $'

.CODE
START:
    MOV AX, @DATA
    MOV DS, AX
    
    ; Buat file baru
    MOV AH, 3Ch
    MOV CX, 0        ; atribut normal
    MOV DX, OFFSET nama_file
    INT 21h
    JC  error_handler
    MOV file_handle, AX
    
    ; Tampilkan pesan sukses
    MOV AH, 09h
    MOV DX, OFFSET pesan_sukses
    INT 21h
    
    ; Tulis data ke file
    MOV AH, 40h
    MOV BX, file_handle
    MOV CX, 15       ; jumlah byte
    MOV DX, OFFSET data_tulis
    INT 21h
    JC  error_handler
    
    ; Tutup file
    MOV AH, 3Eh
    MOV BX, file_handle
    INT 21h
    JC  error_handler
    
    ; Buka file untuk dibaca
    MOV AH, 3Dh
    MOV AL, 0        ; mode read
    MOV DX, OFFSET nama_file
    INT 21h
    JC  error_handler
    MOV file_handle, AX
    
    ; Baca data dari file
    MOV AH, 3Fh
    MOV BX, file_handle
    MOV CX, 100
    MOV DX, OFFSET data_baca
    INT 21h
    JC  error_handler
    
    ; Tampilkan data yang dibaca
    MOV AH, 09h
    MOV DX, OFFSET pesan_baca
    INT 21h
    
    ; Tambahkan terminator '$' setelah data yang dibaca
    MOV BX, OFFSET data_baca
    ADD BX, AX
    MOV BYTE PTR [BX], '$'
    
    MOV AH, 09h
    MOV DX, OFFSET data_baca
    INT 21h
    
    ; Tutup file
    MOV AH, 3Eh
    MOV BX, file_handle
    INT 21h
    
    JMP selesai

error_handler:
    MOV AH, 09h
    MOV DX, OFFSET pesan_error
    INT 21h

selesai:
    MOV AH, 4Ch
    INT 21h
END START
\end{verbatim}

\section{Latihan}
\begin{enumerate}
  \item Tulis program yang menambahkan (append) baris baru ke berkas yang ada (gunakan pengaturan \textit{file pointer}).
  \item Tulis program yang membaca dan menampilkan berkas besar secara bertahap per blok 64 byte.
  \item Tulis program salin berkas yang memverifikasi ukuran hasil sama dengan sumber.
  \item Tulis program penghitung jumlah karakter alfabet, digit, dan lainnya pada berkas teks.
\end{enumerate}

\section{Tugas}
\begin{itemize}
  \item \textbf{Editor teks sederhana}: Navigasi baris, tambah/hapus, simpan-buka berkas; batasi fitur inti.
  \item \textbf{Konfigurasi}: Simpan/muat pasangan kunci-nilai sederhana ke/dari berkas; parsir format baris ``kunci=nilai''.
  \item \textbf{Backup}: Salin berkas dari daftar nama; tampilkan ringkasan hasil (berhasil/gagal).
  \item \textbf{Dokumentasi}: Daftar kode error umum dari \texttt{INT 21h} yang Anda temui, dan strategi penanganannya.
\end{itemize}

\section{Referensi}
\begin{itemize}
  \item Hyde, Randall. \textit{The Art of Assembly Language}, 2nd ed., No Starch Press, 2010.
  \item Susanto. \textit{Belajar Pemrograman Bahasa Assembly}, Elex Media Komputindo, 1995.
\end{itemize}


% Bagian V: Interaksi Sistem dan Grafik
\part{Interaksi Sistem dan Grafik}\label{part:system}

\chapter{Pemrosesan File Lanjutan: Manipulasi Pointer (INT 21h fungsi 42h), Penyimpanan Data Terformat}

\section{Tujuan Pembelajaran}
Mahasiswa mampu:
\begin{itemize}
  \item Menggunakan \texttt{INT 21h, AH=42h} (LSEEK) untuk memosisikan \textit{file pointer} relatif terhadap awal, posisi kini, atau akhir.
  \item Menerapkan akses acak (random access) pada berkas dengan struktur rekaman (record) tetap/variabel.
  \item Merancang penyimpanan data terformat (record-based) lengkap dengan validasi dan pembaruan in-place.
  \item Menyusun aplikasi berkas yang modular (address book/inventory) dengan operasi CRUD.
\end{itemize}

\section{Pendahuluan}
Operasi berkas lanjutan meliputi pemindahan \textit{file pointer} untuk membaca/menulis pada posisi tertentu. Dengan \texttt{LSEEK} (\texttt{AH=42h}), program dapat mengakses bagian mana pun dari berkas secara efisien tanpa membaca dari awal. Pengorganisasian data ke dalam rekaman tetap (\textit{fixed-length}) atau variabel (\textit{variable-length}) memudahkan pengelolaan dan mempercepat pencarian serta pembaruan.

\section{Materi Pembelajaran}
\subsection{Manipulasi File Pointer}
\subsubsection{Konsep pointer dan posisi}
\textit{File pointer} adalah offset dari awal berkas yang menentukan lokasi baca/tulis berikutnya. Semua operasi baca/tulis memulai dari pointer ini dan memajukannya sesuai jumlah byte yang dioperasikan.

\subsubsection{Fungsi 42h: LSEEK}
\begin{itemize}
  \item \textbf{Masukan}: \texttt{AH=42h}, \texttt{AL=origin} (0=awal, 1=kini, 2=akhir), \texttt{BX=handle}, \texttt{CX:DX=offset 32-bit}.
  \item \textbf{Keluaran}: \texttt{DX:AX} berisi posisi baru; CF=1 saat error.
\end{itemize}
Gunakan nilai 32-bit untuk berkas besar. Hati-hati terhadap tanda (gunakan offset tak bertanda untuk maju; untuk mundur, gunakan representasi dua komplemen).

\subsubsection{Random access}
Dengan menghitung offset rekaman ke-\(i\) (mis., \(i \cdot \text{record\_size}\)) dan memanggil \texttt{LSEEK}, kita dapat langsung membaca/menulis rekaman tersebut tanpa memproses rekaman sebelumnya.

\subsection{Data Terformat}
\subsubsection{Konsep dan pilihan desain}
\begin{itemize}
  \item \textbf{Fixed-length}: setiap rekaman memiliki ukuran konstan; akses mudah via aritmetika offset.
  \item \textbf{Variable-length}: hemat ruang untuk data bervariasi; memerlukan indeks atau delimiter.
  \item \textbf{Serialization}: representasi \textit{in-memory} menjadi urutan byte yang konsisten, memperhatikan endianness dan \textit{alignment}.
\end{itemize}

\subsubsection{Record-based data}
Contoh rekaman: nama (15 byte, \textit{padded space}), umur (1 byte), gaji (2 byte, little-endian). Tentukan \texttt{record\_size} dan \texttt{schema} secara eksplisit.

\subsection{Database Sederhana}
\subsubsection{Struktur record dan indeks}
Sediakan berkas data dan opsional berkas indeks (kunci -> nomor rekaman). Operasi CRUD: buat (append), baca (seek + read), ubah (seek + write), hapus (mark as deleted atau kompaksi).

\subsubsection{File locking dan validasi}
Pada DOS single-tasking, penguncian sering tidak diperlukan, namun tetap validasikan input (panjang nama, rentang umur/gaji) dan tangani kondisi balapan pada akses bersamaan (jika ada).

\subsection{Optimasi Akses Berkas}
\subsubsection{Buffering dan caching}
Baca/tulis dalam blok yang lebih besar untuk mengurangi \textit{overhead} panggilan sistem. Cache rekaman sering diakses untuk mempercepat operasi berulang.

\subsubsection{Batch operations dan pemulihan error}
Kelompokkan pembaruan; pada error, lakukan \textit{rollback} sederhana atau catat log status untuk pemulihan manual.

\section{Praktikum}
\begin{enumerate}
  \item Demonstrasi \texttt{LSEEK}: lompat ke posisi tertentu dan tulis tanda penanda; verifikasi dengan membaca kembali.
  \item Random access: baca rekaman ke-\(n\) dan tampilkan.
  \item Data terformat: simpan beberapa rekaman tetap, lalu ubah satu rekaman di tempat (in-place update).
  \item DB sederhana: operasi CRUD pada berkas data; tampilkan daftar seluruh rekaman.
  \item Manajemen berkas: hapus (mark) rekaman lalu lakukan kompaksi untuk menghilangkan celah.
\end{enumerate}

\section{Contoh Kode}
\begin{verbatim}
; Program demonstrasi manipulasi file pointer dan data terformat (ringkas)
TITLE Pemrosesan File Lanjutan
.MODEL SMALL
.STACK 100h

.DATA
    nama_file   DB 'data.dat', 0
    file_handle DW ?

    record_size EQU 20
    ; Data contoh (fixed-length)
    data1 DB 'John Doe       ', 25, 1000
    data2 DB 'Jane Smith     ', 30, 1500
    data3 DB 'Bob Johnson    ', 35, 2000

    buffer DB record_size DUP(?)
    pesan_error DB 'Error dalam operasi!$'

.CODE
START:
    MOV AX, @DATA
    MOV DS, AX

    ; Buat dan tulis 3 rekaman
    MOV AH, 3Ch
    MOV CX, 0
    MOV DX, OFFSET nama_file
    INT 21h
    JC  error
    MOV file_handle, AX

    ; Tulis data1..data3 (masing2 record_size byte)
    MOV AH, 40h
    MOV BX, file_handle
    MOV CX, record_size
    MOV DX, OFFSET data1
    INT 21h
    JC  error
    MOV DX, OFFSET data2
    INT 21h
    JC  error
    MOV DX, OFFSET data3
    INT 21h
    JC  error

    ; Baca rekaman ke-2 via LSEEK (offset = 1*record_size)
    MOV AH, 42h
    MOV AL, 0          ; from beginning
    MOV BX, file_handle
    MOV CX, 0
    MOV DX, record_size
    INT 21h
    JC  error

    ; Baca ke buffer
    MOV AH, 3Fh
    MOV BX, file_handle
    MOV CX, record_size
    MOV DX, OFFSET buffer
    INT 21h
    JC  error

    ; Update rekaman ke-2 in-place: LSEEK mundur record_size
    MOV AH, 42h
    MOV AL, 1          ; from current
    MOV BX, file_handle
    MOV CX, 0
    MOV DX, -record_size
    INT 21h
    JC  error

    ; (Modifikasi buffer terlebih dahulu, kemudian tulis kembali)
    ; Contoh: ubah umur (offset 15) dan gaji (offset 16..17)
    MOV BYTE PTR buffer[15], 31
    MOV WORD PTR buffer[16], 1600

    ; Tulis balik
    MOV AH, 40h
    MOV BX, file_handle
    MOV CX, record_size
    MOV DX, OFFSET buffer
    INT 21h
    JC  error

    ; Tutup dan selesai
    MOV AH, 3Eh
    MOV BX, file_handle
    INT 21h
    JMP done

error:
    MOV AH, 09h
    MOV DX, OFFSET pesan_error
    INT 21h

done:
    MOV AH, 4Ch
    INT 21h
END START
\end{verbatim}

\section{Latihan}
\begin{enumerate}
  \item Baca rekaman ke-\(k\) dari berkas dan tampilkan seluruh field.
  \item Ubah field tertentu pada rekaman ke-\(k\) lalu tulis kembali (uji beberapa nilai k).
  \item Tandai rekaman sebagai terhapus (byte status) dan buat utilitas kompaksi yang menuliskan ulang hanya rekaman aktif.
  \item Tambahkan utilitas pencarian berdasarkan sebagian nama (\textit{prefix search}).
\end{enumerate}

\section{Tugas}
\begin{itemize}
  \item \textbf{Address book}: Rekaman berisi nama (padded), telepon, dan email; implementasikan CRUD lengkap dan simpan dalam berkas.
  \item \textbf{Inventory sederhana}: Rekaman berisi id, nama barang, jumlah, dan harga; dukung pembaruan stok dan laporan.
  \item \textbf{Student database}: Rekaman berisi NIM, nama, nilai; sediakan pencarian berdasarkan NIM dan penghitungan rata-rata kelas.
  \item \textbf{Optimasi}: Dokumentasikan strategi buffering dan batch write yang Anda terapkan serta pengaruhnya terhadap kinerja.
\end{itemize}

\section{Referensi}
\begin{itemize}
  \item Hyde, Randall. \textit{The Art of Assembly Language}, 2nd ed., No Starch Press, 2010.
  \item Partoharsojo, Hartono. \textit{Tuntunan Praktis Pemrograman Assembly}, Penerbit Informatika.
\end{itemize}

\chapter{Program Residen (TSR): Konsep Terminate-and-Stay-Resident; Contoh Implementasi Sederhana}

\section{Tujuan Pembelajaran}
Mahasiswa mampu:
\begin{itemize}
  \item Menjelaskan konsep \textit{Terminate-and-Stay-Resident} (TSR), arsitektur, dan pola penggunaannya di DOS.
  \item Mengimplementasikan TSR sederhana: inisialisasi, pemasangan handler interupsi, dan terminasi-residen.
  \item Menggunakan \texttt{INT 21h} fungsi \texttt{31h} (TSR), \texttt{25h}/\texttt{35h} (set/get interrupt vector) untuk komunikasi dan pemasangan.
  \item Mendesain mekanisme komunikasi (hotkey/custom interrupt) dan strategi manajemen memori TSR.
\end{itemize}

\section{Pendahuluan}
Program TSR adalah program DOS yang, setelah selesai inisialisasi, tidak sepenuhnya keluar: sebagian memorinya dibiarkan \textit{resident} untuk memberikan layanan latar belakang (mis. hotkey, jam layar, makro). TSR lazim sebelum hadirnya sistem multitugas umum. Kunci teknis TSR adalah pemasangan handler interupsi/\textit{hook} dan pemilihan ukuran memori yang disisakan agar cukup namun tidak boros.

\section{Materi Pembelajaran}
\subsection{Konsep Program TSR}
\subsubsection{Definisi dan perbedaan}
Berbeda dengan program biasa yang mengembalikan seluruh memorinya saat \texttt{INT 21h, 4Ch}, TSR mengeksekusi \texttt{INT 21h, 31h} untuk mengakhiri eksekusi namun menyisakan bagian memori dan vektor interupsi yang terpasang.

\subsubsection{Keuntungan dan kelemahan}
\textbf{Keuntungan}: menyediakan layanan latar belakang, akses cepat (hotkey). \textbf{Kelemahan}: konflik vektor interupsi, fragmentasi memori, kompatibilitas antar TSR.

\subsubsection{Aplikasi TSR}
Jam layar, pengganti \textit{keyboard handler}, \textit{clipboard} sederhana, peluncur cepat, atau \textit{logging}.

\subsubsection{Manajemen memori}
Minimalkan memori resident dengan memindahkan data temporer ke tumpukan non-resident, lepaskan segmen yang tak diperlukan sebelum \texttt{INT 21h,31h}, dan hitung ukuran paragraf (16-byte) yang tepat.

\subsection{Arsitektur Program TSR}
\subsubsection{Inisialisasi}
Deteksi apakah TSR sudah terpasang (misalnya melalui vector marker/custom call). Simpan vektor lama, pasang vektor baru, siapkan state minimal.

\subsubsection{Resident routine dan interrupt handler}
Rutinitas resident dijalankan oleh interupsi yang di-hook (mis. keyboard \texttt{INT 09h}, layanan \texttt{INT 21h}, timer \texttt{INT 1Ch}). Pastikan handler cepat, \textit{reentrant-safe} (seperlunya), dan memanggil handler lama bila tidak memproses sendiri.

\subsubsection{Terminasi-residen}
Panggil \texttt{INT 21h, AH=31h} dengan \texttt{DX} berisi ukuran memori yang dibiarkan resident (dalam paragraf). \texttt{AL} dapat mengembalikan kode keluar.

\subsection{Interupsi untuk TSR}
\subsubsection{Set/Get Interrupt Vector}
\texttt{INT 21h, AH=35h} mendapatkan vektor (ke \texttt{ES:BX}); \texttt{AH=25h} memasang vektor baru dari \texttt{DS:DX}. Simpan vektor lama untuk pemulihan atau pemanggilan berantai (\textit{chain}).

\subsubsection{Custom interrupt}
TSR dapat mengekspos layanan melalui nomor interupsi khusus (jarang) atau \texttt{AH} khusus pada \texttt{INT 21h}. Penanda sederhana: jika \texttt{AH=FFh}, handler mengembalikan tanda di \texttt{AL}.

\section{Implementasi TSR}
\subsection{Struktur dan langkah}
\begin{enumerate}
  \item Cek instalasi: panggil layanan khusus; jika sudah, tawarkan uninstall atau keluar.
  \item Simpan vektor lama; pasang handler baru (\texttt{25h}).
  \item Siapkan data resident (variabel global minimal) dan lepaskan yang tidak diperlukan.
  \item Panggil \texttt{31h} dengan estimasi ukuran memori resident.
\end{enumerate}

\subsection{Contoh Handler}
\begin{verbatim}
; Handler INT 21h kustom: AH=FFh -> tanda aktif
NEW_INT21 PROC FAR
    CMP AH, 0FFh
    JNE CALL_OLD
    MOV AL, 0AAh
    IRET
CALL_OLD:
    JMP DWORD PTR old_int21
NEW_INT21 ENDP
\end{verbatim}

\subsection{Uninstall (opsional)}
Uninstall yang bersih memerlukan restorasi vektor lama dan pembebasan memori (yang tidak mudah pada DOS standar). Banyak TSR tidak mendukung uninstall kecuali didesain khusus.

\section{Praktikum}
\begin{enumerate}
  \item TSR sederhana: pasang handler \texttt{INT 21h} yang merespon \texttt{AH=FFh} dengan tanda aktif.
  \item TSR hotkey: hook \texttt{INT 09h} (keyboard) untuk menangkap kombinasi tombol lalu memanggil layanan tertentu.
  \item TSR timer: hook \texttt{INT 1Ch} untuk tugas periodik (mis. memperbarui jam di sudut layar mode teks).
  \item Komunikasi: sediakan fungsi \texttt{AH=khas} untuk membaca status internal TSR.
  \item Eksperimen uninstall: kembalikan vektor ke lama (jika aman) dan uji konflik.
\end{enumerate}

\section{Contoh Kode}
\begin{verbatim}
; Program TSR sederhana (ringkas)
TITLE Program TSR
.MODEL SMALL
.STACK 100h

.DATA
    pesan_install   DB 'TSR berhasil diinstall!$'
    old_int21       DD ?

.CODE
START:
    MOV AX, @DATA
    MOV DS, AX

    ; Cek instalasi
    MOV AH, 0FFh
    INT 21h
    CMP AL, 0AAh
    JE  already

    ; Simpan vektor lama INT 21h
    MOV AH, 35h
    MOV AL, 21h
    INT 21h
    MOV WORD PTR old_int21, BX
    MOV WORD PTR old_int21[2], ES

    ; Pasang vektor baru INT 21h
    MOV AH, 25h
    MOV AL, 21h
    MOV DX, OFFSET NEW_INT21
    INT 21h

    ; Tampilkan pesan
    MOV AH, 09h
    MOV DX, OFFSET pesan_install
    INT 21h

    ; Terminate and Stay Resident (sisa ~4KB contoh)
    MOV AH, 31h
    MOV AL, 0
    MOV DX, 0100h   ; 256 paragraf (4096 byte) contoh, sesuaikan
    INT 21h

already:
    ; Sudah terpasang: cukup keluar normal
    MOV AH, 4Ch
    INT 21h

; Handler baru INT 21h
NEW_INT21 PROC FAR
    CMP AH, 0FFh
    JNE CALL_OLD
    MOV AL, 0AAh
    IRET
CALL_OLD:
    JMP DWORD PTR old_int21
NEW_INT21 ENDP

END START
\end{verbatim}

\section{Latihan}
\begin{enumerate}
  \item Buat TSR yang menampilkan waktu di pojok layar setiap detik (hook \texttt{INT 1Ch}).
  \item Buat TSR yang menangani \textit{hotkey} (mis. Alt+H) untuk menampilkan bantuan singkat.
  \item Buat TSR pemantau keyboard yang mencatat tombol ke buffer sirkular (hindari menulis berkas langsung dari handler).
  \item Buat TSR yang merespon \texttt{AH=FEh} dengan mengembalikan penghitung internal (mis. jumlah \textit{ticks}).
\end{enumerate}

\section{Tugas}
\begin{itemize}
  \item \textbf{System monitor}: TSR yang menampilkan penggunaan memori DOS (perkiraan), jam, dan indikator Caps/Num/Scroll.
  \item \textbf{Keyboard macro}: TSR yang merekam dan memutar ulang urutan tombol pada hotkey tertentu.
  \item \textbf{File watcher}: TSR yang mengawasi pemanggilan fungsi file \texttt{INT 21h} umum dan menampilkan indikator aktivitas.
  \item \textbf{Dokumentasi}: Uraikan strategi memori TSR Anda, ukuran resident, dan cara menghindari konflik vektor.
\end{itemize}

\section{Referensi}
\begin{itemize}
  \item Hyde, Randall. \textit{The Art of Assembly Language}, 2nd ed., No Starch Press, 2010.
  \item Susanto. \textit{Belajar Pemrograman Bahasa Assembly}, Elex Media Komputindo, 1995.
\end{itemize}


% Daftar Pustaka
\printbibliography[title=Daftar Pustaka]

\end{document}
